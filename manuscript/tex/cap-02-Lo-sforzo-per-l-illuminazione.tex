\chapter{Lo sforzo per l'Illuminazione}

\narrator{Primo narratore.} Il racconto della Rinuncia offerto nei Piṭaka è, nella
sua nuda semplicità, suggestivo. In questa più antica versione, gli
elaborati dettagli di quelle successive sono assenti, come lo sono
quelli della nascita e dei primi anni. Ecco il racconto tratto da vari
discorsi pronunciati per diverse persone.


\voice{Prima voce.} «Prima della mia Illuminazione, quando ero ancora solo un
Bodhisatta non illuminato, essendo io stesso ancora soggetto a nascita,
vecchiaia, malattia, morte, dolore e contaminazioni, cercavo quel che
era pure soggetto a queste cose. Allora pensai: “Perché, essendo io
stesso soggetto a nascita, vecchiaia, malattia, morte, dolore e
contaminazioni, cerco quel che è pure soggetto a queste cose? E se,
essendo io stesso soggetto a queste cose e vedendo in esse il pericolo,
cercassi invece la suprema e incontaminata cessazione della schiavitù,
il Nibbāna, privo di nascita, privo di vecchiaia, privo di malattia,
privo di morte e privo di dolore?”».


\suttaRef{M. 26}


«Prima della mia Illuminazione, quando ero ancora solo un Bodhisatta non
illuminato, pensai: “La vita in famiglia è affollata e polverosa,
l’abbandono di essa comporta spaziose aperture. Vivendo in famiglia non
è facile condurre una santa vita assolutamente perfetta e immacolata
come una conchiglia ben lucidata. E se mi rasassi i capelli e la barba,
indossassi l’abito ocra, e rinunciassi alla vita in famiglia per una
senza dimora?”».


\suttaRef{M. 36, 100}


«In seguito, quando ero ancora giovane, un ragazzo dai capelli neri
benedetto dalla giovinezza, mi rasai i capelli e la barba e – benché mia
madre e mio padre desiderassero altro per me e si addolorassero con il
volto pieno di lacrime – indossai l’abito ocra e rinunciai alla vita in
famiglia per una senza dimora».


\suttaRef{M. 26, 36, 85, 100}


\begin{quote}
\cantor{Cantore}


Ora racconterò la rinuncia alla vita in famiglia, \\
come egli, il possente Veggente, lasciò la sua casa, \\
cosa gli fu chiesto e come descrisse \\
la ragione della sua rinuncia. \\
La vita affollata vissuta in una casa \\
esala un’atmosfera polverosa, \\
ma il suo abbandono comporta spaziose aperture: \\
questo egli vide, e scelse l’abbandono della vita in famiglia. \\
Nel farlo rifiutò \\
ogni cattiva azione del corpo, \\
respinse ogni genere di errata parola \\
e rese inoltre retti i suoi mezzi di sostentamento. \\
Andò nella città di Rājagaha, \\
nel castello di Magadha, \\
là, egli – il Buddha – fece la questua, \\
con più di un marchio d’eccellenza. \\
Il re Bimbisāra \\
lo vide passare dal suo palazzo, \\
e quando egli vide l’eccellenza \\
di tutti i suoi marchi, egli disse: \\
«Guardate, signori, quanto è bello quell’uomo, \\
quanto è maestoso, quanto pura e perfetta è la sua condotta, \\
con gli occhi bassi e consapevole, guarda \\
alla sola distanza d’un giogo d’aratro davanti a lui, \\
non è di basso lignaggio. \\
Mandate subito dei messi reali \\
che seguano la strada percorsa dal bhikkhu». \\
I messi vennero subito inviati \\
e seguirono la sua scia dappresso: \\
«Che strada prenderà il bhikkhu? \\
Qual è il luogo che ha scelto per dimora? \\
Vaga di casa in casa, \\
custodendo le porte dei sensi con vero contenimento, \\
pienamente consapevole e cosciente. \\
Ha subito colmato la sua ciotola per l’elemosina, \\
ora ha terminato la sua questua. \\
Il Saggio s’incammina e lascia la città, \\
prendendo la via per Paṇḍava: \\
deve vivere sull’altura di Paṇḍava».


Quando egli raggiunse la sua dimora \\
i messi lo raggiunsero, \\
ma uno di loro tornò indietro \\
per portare al re la risposta alla sua domanda:


«Il bhikkhu, sire, come una tigre, \\
come un toro o come un leone, \\
abita nella caverna di un monte \\
sul versante orientale di Paṇḍava».


Il guerriero ascoltò il racconto del messaggero, \\
prendendo poi una carrozza reale \\
in fretta uscì dalla città \\
verso l’altura di Paṇḍava. \\
Con il carro andò più lontano che poté, \\
e poi ne discese, \\
percorse a piedi la breve distanza che restava \\
finché arrivò vicino al Saggio.


Il re sedette, scambiò saluti, \\
e gli chiese delle sue condizioni fisiche. \\
Quando questo scambio di cortesie \\
fu terminato, il re gli disse queste parole: \\
«Sei piuttosto giovane, un ragazzo, \\
un uomo nella prima fase della vita. \\
Hai il bell’aspetto d’un uomo \\
d’alto e nobile lignaggio guerriero, \\
uno adatto ad adornare un esercito di prim’ordine \\
per guidare le truppe di elefanti. \\
Ti offro una fortuna: afferrala. \\
Quali i tuoi natali? Dimmelo».


«C’è una terra prosperosa, sire, \\
e forte, proprio di fronte alle pendici dell’Himalaya, \\
abitata dai Kosala \\
la cui stirpe prende il nome dal Sole, \\
il lignaggio è quello dei Sakya. \\
Non ho però lasciato la vita in famiglia per cercare il piacere dei
sensi. \\
Avendo visto il pericolo in essi, sono andato via per sforzarmi, \\
e per cercare un sicuro rifugio nella rinuncia. \\
Questo è il desiderio del mio cuore».
\end{quote}

\suttaRef{Sn. 3:1}


\voice{Prima voce.} «Ho abbandonato la vita in famiglia per una senza dimora per
cercare ciò che è buono,\footnote{Kusala: salutare, vantaggioso} per cercare
il supremo stato della pace sublime. Per questo andai da Āḷāra Kālāma e gli dissi: “Amico
Kālāma, voglio condurre la santa vita in questo Dhamma e in questa
Disciplina”».


«Quando questo fu detto, Āḷāra Kālāma mi rispose: “Il venerabile può
restare qui. Quest’insegnamento è fatto in modo tale che in un tempo non
lungo un uomo saggio può entrare in esso e dimorarvi, realizzando lui
stesso per mezzo della conoscenza diretta quello che il suo stesso
maestro conosce”».


«Imparai presto l’insegnamento. Ritenni che, per quanto concerne la
recitazione e la ripetizione del suo insegnamento, potevo parlare con
conoscenza e certezza, e che perciò sapevo e vedevo, e c’erano altri che
facevano altrettanto».


«Pensai: “Non è per sola fede che Āḷāra Kālāma dichiara il suo
insegnamento, è perché egli è entrato in esso e vi dimora, realizzandolo
lui stesso per mezzo della conoscenza diretta. È certo che egli dimora
in questo insegnamento, conoscendo e vedendo”».


«Allora andai da Āḷāra Kālāma, e gli dissi: “Amico Kālāma, fino a che
punto dichiari di essere entrato in questo insegnamento, realizzandolo
tu stesso per mezzo della conoscenza diretta?”».


«Quando questo fu detto, egli dichiarò la dimensione del nulla-è. Mi
venne in mente: “Āḷāra Kālāma non è il solo ad avere fede, energia,
consapevolezza, concentrazione e comprensione, anch’io ho queste
facoltà. E se io mi sforzassi di realizzare l’insegnamento nel quale
egli dichiara di entrare e di dimorare, realizzandolo io stesso per
mezzo della conoscenza diretta?”».


«Presto ci riuscii. Allora andai da Āḷāra Kālāma, e gli dissi: “Amico
Kālāma, è fino a questo punto che dichiari di essere entrato e di
dimorare in questo insegnamento, realizzandolo tu stesso per mezzo della
conoscenza diretta?”. Egli mi rispose che era così».


«Anch’io, amico, fino a questo punto sono entrato e dimoro in questo
insegnamento, realizzandolo io stesso per mezzo della conoscenza
diretta».


«Siamo fortunati, amico, siamo davvero fortunati, di aver trovato un
uomo così venerabile come nostro compagno nella santa vita. Così
nell’insegnamento nel quale io dichiaro di essere entrato, realizzandolo
io stesso per mezzo della conoscenza diretta, vi sei entrato e vi dimori
anche tu, realizzandolo tu stesso per mezzo della conoscenza diretta. E
l’insegnamento nel quale sei entrato e dimori, realizzandolo tu stesso
per mezzo della conoscenza diretta, è lo stesso nel quale io dichiaro di
essere entrato, realizzandolo io stesso per mezzo della conoscenza
diretta. Allora, tu conosci l’insegnamento che io conosco, io conosco
l’insegnamento che tu conosci. Come sono io, così sei tu. Vieni, amico,
guidiamo insieme questa comunità”. Così Āḷāra Kālāma, il mio maestro, mi
mise alla pari con lui, concedendomi il più alto onore».


«Pensai: “Questo insegnamento non conduce al disincanto, al
dissolvimento della brama, alla cessazione, alla pace, alla conoscenza
diretta, all’Illuminazione, al Nibbāna, ma solo alla dimensione del
nulla-è”. Questo insegnamento non mi soddisfaceva. Lo lasciai per
proseguire la mia ricerca».


«Ancora alla ricerca di ciò che è buono, alla ricerca del supremo stato
della pace sublime, andai da Uddaka Rāmaputta, e gli dissi: “Amico,
voglio condurre la santa vita in questo Dhamma e in questa Disciplina”».


\suttaRef{M. 26, 36, 85, 100}


\narrator{Primo narratore.} La sua esperienza sotto la guida di Uddaka Rāmaputta è
narrata esattamente con le stesse parole, con la differenza che egli
imparò da lui l’ancora più alta fruizione della dimensione della
né-percezione-né-non-percezione, e che Uddaka Rāmaputta gli offrì di
essere da solo l’unica guida della comunità. La conclusione, però, fu la
stessa.


\voice{Prima voce.} «Pensai: “Questo insegnamento non conduce al disincanto, al
dissolvimento della brama, alla cessazione, alla pace, alla conoscenza
diretta, all’Illuminazione, al Nibbāna, ma solo alla dimensione della
né-percezione-né-non-percezione”. Questo insegnamento non mi
soddisfaceva. Lo lasciai per proseguire la mia ricerca».


«Ancora alla ricerca di ciò che è buono, alla ricerca del supremo stato
della pace sublime, vagai facendo varie tappe attraverso il regno di
Magadha e infine arrivai a Senānigāma nei pressi di Uruvelā. Là vidi un
piacevole appezzamento di terra, un delizioso boschetto, un fiume che
scorreva limpido con sponde piane e gradevoli e, nei pressi, un
villaggio adatto per la questua. Pensai: “Questo sarà utile per lo
sforzo di un uomo di rango che cerca un tale sforzo.”».


\suttaRef{M. 26, 36, 85, 100}


«Prima della mia Illuminazione, quando ero ancora solo un Bodhisatta non
illuminato, pensai: “È difficile sopportare di dimorare in remote
boscaglie della foresta, la solitudine è difficile da vivere, è
difficile dilettarsi dell’isolamento. Si potrebbe pensare che la foresta
può rubare la mente a un bhikkhu privo di concentrazione».


«Pensai: “Supponiamo che un monaco o un brāhmaṇa sia impuro nella
condotta del corpo, della parola o della mente, oppure nei suoi mezzi di
sussistenza, che sia avido o molto sensibile alla bramosia per i
desideri sensoriali, o malevolo, con pensieri di odio, oppure preda del
torpore e della sonnolenza, o che sia nervoso e agitato di mente; che
sia incline a vantarsi e a denigrare gli altri; che sia soggetto alla
paura e all’orrore, che desideri guadagni, onore e fama; che sia pigro e
privo di energia, smemorato e non pienamente consapevole, non
concentrato e confuso di mente, privo di comprensione e fanfarone –
quando un monaco o un brāhmaṇa così dimora in una remota boscaglia della
foresta, allora a causa di questi difetti egli evoca paura e terrore non
salutari.\footnote{Akusala: originariamente qui tradotto con “infruttuose” (Nyp.)} Io però non dimoro in una remota boscaglia
della foresta come uno di quelli. Io non ho nessuno di questi difetti.
Io dimoro in una remota boscaglia della foresta come uno degli Esseri
Nobili, che sono liberi da questi difetti”. Vedendo in me stesso tale
libertà da questi difetti, provo grande consolazione a vivere nella
foresta».


«Pensai: “Ci sono però le notti particolarmente sacre della luna piena e
della luna nuova, della quattordicesima e quindicesima notte, e della
mezza luna, dell’ottava notte. E se io trascorressi queste notti in
dimore che incutono timore come templi fatti di boschi, templi fatti di
foreste, templi fatti di alberi, che fanno rizzare i capelli –
incontrerei forse quella paura e quel terrore?”».


«E più tardi, io trascorsi queste notti particolarmente sacre della luna
piena e della luna nuova, della quattordicesima e quindicesima notte, e
della mezza luna, dell’ottava notte, in dimore che incutono timore come
templi fatti di boschi, templi fatti di foreste, templi fatti di alberi,
che fanno rizzare i capelli. Quando dimorai lì, mi si avvicinò un cervo,
o un pavone ruppe un ramo, o il vento fece frusciare le foglie. Pensai:
“Certamente sono quella paura e quel terrore che arrivano”».


«Pensai: “Perché dimoro in constante attesa della paura e del terrore?
Perché non domino quella paura e quel terrore mantenendo la postura
nella quale mi trovo quando vengono da me?”».


«E mentre camminavo, la paura e il terrore vennero da me, ma io non
rimasi fermo in piedi, né mi misi seduto o disteso finché non dominai
quella paura e quel terrore. Mentre stavo in piedi, la paura e il
terrore vennero da me, ma io non camminai, né mi misi seduto o disteso
finché non dominai quella paura e quel terrore. Mentre stavo seduto, la
paura e il terrore vennero da me, ma io non camminai, né mi misi in
piedi o disteso finché non dominai quella paura e quel terrore. Mentre
ero disteso, la paura e il terrore vennero da me, ma io non camminai, né
mi misi in piedi o seduto finché non dominai quella paura e quel
terrore».


\suttaRef{M. 4}


«Mi vennero allora in mente tre similitudini, in modo spontaneo, mai
udite prima».


«Supponiamo che un pezzo di legno bagnato e ricco di linfa sia
nell’acqua, e che un uomo arrivi con un bastoncino di legno per
accendere il fuoco, pensando: “Accenderò un fuoco, produrrò calore”. Che
cosa pensi, quell’uomo potrebbe accendere un fuoco e produrre calore
prendendo il bastoncino di legno e sfregandolo sul pezzo di legno
bagnato e ricco di linfa che è nell’acqua?». – «No, Signore». – «Perché
no? Perché è un pezzo di legno bagnato e ricco di linfa, e per di più è
nell’acqua. Perciò, quell’uomo raccoglierà stanchezza e delusione». –
«Allo stesso modo, se un monaco o un brāhmaṇa vive ancora con il corpo e
con la mente non appartati dai piaceri sensoriali, e se la sua bramosia,
affezione, passione, sete e febbre per i piaceri sensoriali non sono
ancora del tutto abbandonate e placate dentro di lui, allora, se il buon
monaco o brāhmaṇa prova sensazioni dolorose, laceranti, penetranti
imposte dallo sforzo, o se non le prova, in entrambi i casi egli è
incapace di ottenere la conoscenza, la visione profonda e la suprema
Illuminazione. Questa fu la prima similitudine che mi venne in mente in
modo spontaneo, mai udita prima».


«Ancora, supponiamo che un pezzo di legno bagnato e ricco di linfa sia
sulla terraferma, lontano dall’acqua, e che un uomo arrivi con un
bastoncino di legno per accendere il fuoco, pensando: “Accenderò un
fuoco, produrrò calore”. Che cosa pensi, quell’uomo potrebbe accendere
un fuoco e produrre calore prendendo il bastoncino di legno e
sfregandolo sul pezzo di legno bagnato e ricco di linfa che è sulla
terraferma, lontano dall’acqua?». – «No, Signore». – «Perché no? Perché è
un pezzo di legno bagnato e ricco di linfa, benché sia sulla terraferma,
lontano dall’acqua. Perciò, quell’uomo raccoglierà stanchezza e
delusione». – «Allo stesso modo, se un monaco o un brāhmaṇa vive ancora
appartato solo con il corpo dai piaceri sensoriali, e se la sua
bramosia, affezione, passione, sete e febbre per i piaceri sensoriali
non sono ancora del tutto abbandonate e placate dentro di lui, allora,
se il buon monaco o brāhmaṇa prova sensazioni dolorose, laceranti,
penetranti imposte dallo sforzo, o se non le prova, in entrambi i casi
egli è incapace di ottenere la conoscenza, la visione profonda e la
suprema Illuminazione. Questa fu la seconda similitudine che mi venne in
mente in modo spontaneo, mai udita prima».


«Ancora, supponiamo che un pezzo di legno secco e privo di linfa sia
sulla terraferma, lontano dall’acqua, e che un uomo arrivi con un
bastoncino di legno per accendere il fuoco, pensando: “Accenderò un
fuoco, produrrò calore”. Che cosa pensi, quell’uomo potrebbe accendere
un fuoco e produrre calore prendendo il bastoncino di legno e
sfregandolo sul pezzo di legno secco e privo di linfa che è sulla
terraferma, lontano dall’acqua?». – «Sì, Signore». – Perché sì? Perché è
un pezzo di legno secco e privo di linfa, e per di più è sulla
terraferma, lontano dall’acqua». – «Allo stesso modo, se un monaco o un
brāhmaṇa vive con il corpo e con la mente appartati dai piaceri
sensoriali, e se la sua bramosia, affezione, passione, sete e febbre per
i piaceri sensoriali sono del tutto abbandonate e placate dentro di lui,
allora, se il buon monaco o brāhmaṇa prova sensazioni dolorose,
laceranti, penetranti imposte dallo sforzo, o se non le prova, in
entrambi i casi egli è capace di ottenere la conoscenza, la visione
profonda e la suprema Illuminazione. Questa fu la terza similitudine che
mi venne in mente in modo spontaneo, mai udita prima».


«Pensai: “E se, con i denti serrati e la lingua premuta contro il
palato, abbattessi, costringessi e schiacciassi la mia mente con la
mente?”. Allora, come un uomo forte potrebbe afferrarne uno più debole
per la testa o per le spalle e abbatterlo, costringerlo e schiacciarlo,
così con i denti serrati e la lingua premuta contro il palato, io
abbattei, costrinsi e schiacciai la mia mente con la mente. Il sudore
scorreva dalle mie ascelle mentre lo facevo».


«Benché in me fosse sorta un’instancabile energia e si fosse instaurata
un’incessante consapevolezza, tuttavia il mio corpo era affaticato e
agitato perché ero esausto per lo sforzo doloroso. Quando però in me
sorsero queste sensazioni dolorose, esse non ebbero potere sulla mia
mente».


«Pensai: “E se io praticassi la meditazione senza respirare?”. Bloccai
le inspirazioni e le espirazioni nella bocca e nel naso. Quando lo feci,
un forte suono di venti provenne dai fori dei miei orecchi, come il
forte suono che si produce quando vengono gonfiati i mantici di un
fabbro».


«Bloccai le inspirazioni e le espirazioni nella bocca e nel naso. Quando
lo feci, venti violenti torturarono la mia testa, come se un uomo forte
mi stesse spaccando la testa con una spada affilata. E allora nella mia
testa ci furono violenti dolori, come se un uomo forte stesse stringendo
una spessa striscia di cuoio attorno alla testa, come una fascia per la
testa. E allora venti violenti mi lacerarono il ventre, come quando un
abile macellaio o il suo apprendista lacerano il ventre di un bue con un
coltello affilato. Poi nel mio ventre v’era un violento bruciore, come
se due uomini forti avessero afferrato un uomo più debole con entrambe
le braccia e lo arrostissero su una fossa di carboni ardenti».


«E ogni volta, benché in me fosse sorta un’instancabile energia e si
fosse instaurata un’incessante consapevolezza, tuttavia il mio corpo era
affaticato e agitato perché ero esausto per lo sforzo doloroso. Quando
però in me sorsero queste sensazioni dolorose, esse non ebbero potere
sulla mia mente».


«Quando le divinità mi videro, dissero: “Il monaco Gotama è morto”.
Altre divinità dissero: “Il monaco Gotama non è morto, sta morendo”.
Altre divinità ancora dissero: “Il monaco Gotama non è morto né sta
morendo, il monaco Gotama è un Arahant, un santo, perché questa è la
strada dei santi”».


«Pensai: “E se eliminassi il cibo del tutto?”. Allora delle divinità
vennero da me e dissero: “Caro Signore, non eliminare il cibo del tutto.
Se lo fai, noi ti inietteremo del cibo divino nei pori e tu vivrai di
questo”. Pensai: “Se affermo di digiunare completamente, e queste
divinità mi iniettano del cibo divino nei pori e io vivo di questo,
allora mentirò”. Le congedai dicendo: “Non ve n’è bisogno”».


«Pensai: “E se assumessi pochissimo cibo, una manciata ogni tanto,
diciamo, che si tratti di zuppa di fagioli o di zuppa di lenticchie o di
zuppa di piselli?”. Così feci. Quando lo feci, il mio corpo si ridusse
in uno stato estremamente emaciato, a causa dello scarsissimo cibo i
miei arti divennero come degli steli congiunti di vite o di bambù. Le
mie natiche divennero come gli zoccoli d’un cammello, le sporgenze della
mia colonna vertebrale si spinsero in fuori come perle infilate, le mie
costole divennero prominenti come le false travi di un vecchio fienile
senza tetto, il luccichio dei miei occhi affossati nelle orbite sembrava
il luccichio dell’acqua nel fondo di un pozzo profondo, il mio cuoio
capelluto divenne striminzito e avvizzito come una zucca verde
striminzisce e avvizzisce al vento e al sole. Se toccavo la pelle del
mio ventre, incontravo la mia colonna vertebrale e, se toccavo la mia
colonna vertebrale, incontravo la pelle del mio ventre, perché la pelle
del mio ventre s’era attaccata alla mia colonna vertebrale. Se urinavo o
evacuavo il mio intestino, vi cadevo sopra con il viso. Se cercavo di
dare sollievo al mio corpo strofinandomi gli arti con le mani, i peli,
decompostisi alla radice a causa dello scarsissimo cibo, cadevano dal
mio corpo mentre strofinavo».


«Quando gli esseri umani mi videro, dissero: “Il monaco Gotama è un uomo
di pelle scura”. Altri esseri umani dissero: “Il monaco Gotama non è un
uomo di pelle scura, è un uomo di pelle semi-scura”. Altri esseri umani
ancora dissero: “Il monaco Gotama non è un uomo di pelle scura, né un
uomo di pelle semiscura, è di pelle chiara”. Il colore della mia pelle
si era deteriorato fino a questo punto a causa dello scarsissimo cibo».


\suttaRef{M. 36, 85, 100}


\begin{quote}
\cantor{Cantore}


Quando mi sforzavo per vincere me stesso, \\
accanto al vasto Nerañjarā, \\
risolutamente assorbito per ottenere \\
la vera cessazione della schiavitù, \\
Namucī arrivò e mi parlò \\
con parole adorne di compassione, così: \\
«Oh, sei emaciato e pallido, \\
e sei pure al cospetto della morte, \\
mille parti di te sono promesse alla morte, \\
ma una parte di te possiede ancora la vita. \\
Vivi, Signore! La vita è la cosa migliore, \\
se vivi puoi ottenere meriti.


Vieni, vivi la santa vita e riversa \\
libagioni sui santi fuochi, \\
e così otterrai un mondo di meriti. \\
Che cosa puoi mai fare ora con i tuoi sforzi? \\
Il sentiero dello sforzo è aspro \\
e difficile e duro da sopportare». \\
Mentre Māra pronunciava questi versi \\
si appressò fino a venirgli vicino. \\
Il Beato gli rispose così: \\
«O Malvagio, \\
o cugino del Negligente, \\
sei venuto fino qui per i tuoi fini. \\
Non ho affatto bisogno di meriti, \\
che Māra parli di meriti \\
a chi di essi ha bisogno. \\
Perché io ho fiducia ed energia, \\
e anche comprensione. \\
Così, mentre io soggiogo me stesso \\
perché mi parli della vita? \\
C’è questo vento che soffia e che può asciugare \\
perfino la corrente dei fiumi che scorre: \\
così, mentre io soggiogo me stesso \\
perché non dovrebbe disseccare il mio sangue? \\
E quando il sangue si dissecca, la bile \\
e il flegma si asciugano, la carne che si consuma \\
acquieta la mente: io avrò più \\
Consapevolezza, Comprensione, \\
avrò maggiore Concentrazione. \\
Perché vivendo in questo modo giungerò a conoscere \\
i limiti della sensazione. \\
La mia mente non guarda ai desideri sensoriali: \\
tu vedi la purezza di un essere. \\
Il tuo primo squadrone è Desiderio Sensoriale, \\
il secondo è chiamato Noia, poi \\
Fame e Sete compongono il terzo, e \\
Bramosia è il quarto della serie, \\
il quinto è Torpore e Accidia, \\
mentre la Codardia si allinea come sesto, \\
Incertezza è il settimo, l’ottavo è \\
Malizia congiunta a Ostinazione, \\
Guadagno, Onore e Fama inoltre, e \\
Notorietà malamente conquistata, \\
Lode di Se Stessi e Denigrazione degli Altri. \\
Questi sono i tuoi squadroni, Namucī, \\
questi sono gli squadroni armati dell’Oscuro, \\
nessuno, solo il coraggioso li sconfiggerà e \\
otterrà la beatitudine della vittoria. \\
Io agito lo stendardo che rifiuta ogni ritirata. \\
Miserevole è qui la vita, io affermo. \\
Meglio morire adesso in battaglia \\
piuttosto che scegliere di vivere nella sconfitta. \\
Ci sono qui asceti e brāhmaṇa \\
che si sono arresi e non \\
si sono visti più: non conoscono \\
i sentieri percorsi dal pellegrino. \\
Così, vedendo ora gli squadroni di Māra \\
schierati con elefanti tutt’intorno, \\
io esco di gran carriera per combattere, per \\
non essere scacciato dal mio presidio. \\
Tu hai schierato degli squadroni che il mondo \\
con tutte le sue divinità non può sconfiggere, \\
ma io li abbatterò con la Comprensione, \\
come una pietra un vaso d’argilla cruda».\footnote{Ciò a cui gli ultimi versi (qui omessi ma inclusi nel \hyperlink{cap-04-La-diffusione-del-Dhamma#pag70A}{}) di questo canto fanno riferimento è collocato dai \emph{Commentari} un anno più tardi rispetto al resto.}
\end{quote}

\suttaRef{Sn. 3:2}


«Pensai: “Ogni volta che un monaco o un brāhmaṇa ha provato in passato,
prova adesso o proverà in futuro sensazioni dolorose, laceranti e
penetranti imposte dallo sforzo, è possibile che queste siano uguali ad
esse, ma non più forti. Da questa faticosa penitenza, però, non ho
ottenuto alcuna caratteristica superiore alla condizione umana, degna
della conoscenza e della visione degli Esseri Nobili. Può esserci
un’altra via per l’Illuminazione?”».


«Pensai al tempo in cui mio padre, il Sakya, era al lavoro e io sedevo
alla frescura, all’ombra d’un albero di melarosa, del tutto discosto dai
desideri sensoriali, e discosto da cose non salutari entrai e dimorai
nel primo jhāna,\footnote{NDT. Assorbimento mentale (jhāna), uno stato di forte concentrazione focalizzata su una singola sensazione fisica (che conduce a un \emph{rūpajjhāna}), oppure su di una nozione mentale (che conduce a un \emph{arūpajjhāna}). I quattro jhāna sono descritti \hyperlink{pag27}{}; si veda anche la narrazione dell’ottenimento del Nibbāna finale (\emph{Parinibbāna}) del Buddha (\hyperlink{cap-15-L-ultimo-anno#pag364}{}).} che è accompagnato dal pensiero e
dall’esplorazione uniti alla felicità e al piacere nati
dall’isolamento.\footnote{NDT. Qui come in seguito, allorché questo passo si ripete, i due termini “pensiero” ed “esplorazione” – nel testo inglese si legge «by thinking and exploring» – si riferiscono ai vocaboli in pāli \emph{vitakka} e \emph{vicāra}, i quali sono talora tradotti in altro modo sia in inglese sia in italiano. Si è comunque preferito restare più vicini alle scelte di Bhikkhu Ñāṇamoli.} Pensai: “Che sia questa la via per
l’Illuminazione?”».


«Allora pensai: “Perché temo questo piacere? È un piacere che non ha
nulla a che vedere con i piaceri sensoriali e con le cose non salutari”.
Poi pensai: “Non temo questo piacere perché non ha nulla a che vedere
con i piaceri sensoriali e con le cose non salutari».


«Pensai: “Non è possibile giungere a tale piacere con un corpo
eccessivamente emaciato. E se mangiassi un po’ di cibo solido, del riso
bollito e del pane?”».\footnote{Il dizionario della Pāli Text Society ha “junket” (giuncata) per \emph{kummāsa}, che i \emph{Commentari} dicono essere tuttavia fatta di farina (\emph{yava}).}


«In quel tempo i cinque bhikkhu che erano al mio servizio pensavano: “Se
il monaco Gotama perverrà a qualche conoscenza, ci informerà”. Appena
mangiai del cibo solido, il riso bollito e il pane, i cinque bhikkhu se
ne andarono disgustati pensando: “Il monaco Gotama è diventato
autoindulgente, ha rinunciato allo sforzo ed è tornato alla lussuria”».


\suttaRef{M. 36, 85, 100}


\narrator{Primo narratore.} A questo punto il Bodhisatta fece cinque sogni.


\narrator{Secondo narratore.} Avvenne nella notte precedente l’Illuminazione, e
questi sogni erano una premonizione del fatto che stava per raggiungere
il suo obiettivo.


\voice{Prima voce.} «Appena prima di conseguire l’Illuminazione, il Perfetto,
realizzato e completamente illuminato, fece cinque sogni importanti.
Quali cinque? Quando era ancora solo un Bodhisatta non illuminato, la
Grande Terra era il suo letto. L’Himalaya, il re delle montagne, era il
suo cuscino. La sua mano sinistra stava nell’Oceano Orientale, la sua
mano destra stava nell’Oceano Occidentale, i suoi piedi stavano
nell’Oceano Meridionale. Questo fu il suo primo sogno, ed esso premoniva
la sua scoperta della piena e suprema Illuminazione. Quando era ancora
solo un Bodhisatta non illuminato, una pianta rampicante crebbe dal suo
ombelico e giunse a toccare le nuvole. Questo fu il suo secondo sogno,
ed esso premoniva la sua scoperta del Nobile Ottuplice Sentiero. Quando
era ancora solo un Bodhisatta non illuminato, dei bruchi bianchi con la
testa nera si arrampicarono sui suoi piedi e risalirono le sue ginocchia
fino a ricoprirlo completamente. Questo fu il suo terzo sogno, ed esso
premoniva che molti laici vestiti di bianco avrebbero scelto il Perfetto
come rifugio durante la sua vita. Quando era ancora solo un Bodhisatta
non illuminato, quattro uccelli di diverso colore giunsero dai quattro
punti cardinali e, quando si posarono ai suoi piedi, divennero tutti
bianchi. Questo fu il suo quarto sogno, ed esso premoniva che le quattro
caste – i nobili guerrieri, i sacerdoti brāhmaṇa, i commercianti e
artigiani, i servi – avrebbero realizzato la suprema liberazione
allorché il Dhamma e la Disciplina sarebbero state proclamate dal
Perfetto. Quando era ancora solo un Bodhisatta non illuminato, egli
camminava su un’enorme montagna di sporcizia senza essere contaminato
dal sudiciume. Questo fu il suo quinto sogno, ed esso premoniva che il
Perfetto avrebbe ottenuto i generi di prima necessità – abito, cibo
ricevuto in elemosina, dimora e medicinali – e tuttavia li avrebbe usati
senza bramosia né illusioni o attaccamento, percependone i pericoli e
comprendendone gli scopi».


\suttaRef{A. 5:196}


\narrator{Primo narratore.} L’Illuminazione stessa è descritta in vari discorsi e
da diverse numerose angolazioni, come se un albero dovesse essere
descritto dall’alto, dal basso e da vari lati, o un viaggio per terra,
per acqua e per aria.\footnote{I diversi modi nei quali i discorsi descrivono l’Illuminazione sono: in termini di genesi interdipendente (originazione interdipendente o coproduzione condizionata) (S. 12:10, 65; cf. D. 14); di tre vere conoscenze o scienze (M. 4, 100); gratificazione, inadeguatezza (pericolo) e fuga nel caso dei cinque aggregati (S. 22:26), degli elementi (S. 14:31), dei desideri sensoriali (S. 35:117; M. 14), della sensazione (S. 36:24), del mondo (A. 3:101); in termini di quattro imprese (A. 5:68), di quattro fondamenti della consapevolezza (S. 47:31), di quattro basi per il successo spirituale (S. 51:9), dell’abbandono dei cattivi pensieri (M. 19), ecc.}


\narrator{Secondo narratore.} Vi è una descrizione dell’Illuminazione come
conquista delle tre vere conoscenze raccontata nel modo seguente, sulla
base dello sviluppo della meditazione. Vi sono poi descrizioni di essa
in termini d’una scoperta della struttura della condizionalità
nell’impermanente processo dell’esistenza, e in termini di ricerca di
un’interpretazione non ingannevole, di una vera scala di valori, nel
mondo problematico delle idee, delle azioni e delle cose, delle
probabilità e delle certezze. Questa è la descrizione in termini di
meditazione che conduce alla scoperta delle Quattro Nobili Verità.


\voice{Prima voce.} (FIXME label pag27)«Dopo aver mangiato cibo solido e aver riacquistato le
forze, allora, del tutto discosto dai desideri sensoriali, discosto da
cose non salutari entrai e dimorai nel primo jhāna, che è accompagnato
dal pensiero e dall’esplorazione uniti alla felicità e al piacere nati
dall’isolamento. Quando però in me sorse questa sensazione piacevole,
non le consentii d’impossessarsi della mia mente. Con l’acquietarsi del
pensiero e dell’esplorazione entrai e dimorai nel secondo jhāna, che,
privo di pensiero ed esplorazione, è accompagnato da fiducia interiore e
unificazione della mente unite alla felicità e al piacere nati dalla
concentrazione. Quando però in me sorse questa sensazione piacevole, non
le consentii d’impossessarsi della mia mente. Con lo svanire anche di
questa felicità, mentre provavo ancora piacere nel corpo, dimorai
nell’equanimità contemplativa, consapevole e pienamente presente entrai
e dimorai nel terzo jhāna, in relazione al quale gli Esseri Nobili
affermano: “Dimora piacevolmente osservando con equanimità e
consapevolezza”. Quando però in me sorse questa sensazione piacevole,
non le consentii d’impossessarsi della mia mente. Con l’abbandono del
piacere e del dolore del corpo, e con la precedente scomparsa della
gioia e dell’afflizione mentale, entrai e dimorai nel quarto jhāna, nel
quale non c’è né piacere né dolore e la purezza della consapevolezza è
dovuta all’equanimità contemplativa. Quando però in me sorse questo
piacere, non gli consentii d’impossessarsi della mia mente».


(FIXME label pag27b)«Quando la mia mente fu così concentrata, purificata, luminosa,
immacolata e priva di imperfezioni, allorché divenne malleabile,
duttile, stabile e imperturbabile, la indirizzai e la rivolsi alla
conoscenza del ricordo delle vite precedenti, ricordai la molteplicità
delle mie vite passate, vale a dire una nascita, due, tre, quattro,
cinque nascite, dieci, venti, trenta, quaranta, cinquanta nascite, un
centinaio di nascite, un migliaio di nascite, centomila nascite, molte
età di contrazione del mondo, molte età di espansione del mondo, molte
età di contrazione e di espansione del mondo: “Là ero chiamato in tal
modo, ero di quella razza, con tale aspetto, tale il cibo, tale
esperienza di piacere e dolore, tale durata della vita. E morto là,
ricomparivo da qualche altra parte e lì ero chiamato in tal modo, ero di
quella razza, con tale aspetto, tale esperienza di piacere e dolore,
tale durata della vita. E morto lì, ricomparvi qui”. Così, in modo
dettagliato e particolareggiato ricordai la molteplicità delle mie vite
passate. Questa fu la prima vera conoscenza da me conseguita nella prima
veglia notturna. L’ignoranza fu bandita e sorse la vera conoscenza,
l’oscurità fu bandita e sorse la luce, come avviene in chi è diligente,
ardente e dotato di autocontrollo. Quando però in me sorse questa
sensazione piacevole, non le consentii d’impossessarsi della mia mente».


(FIXME label pag28)«Quando la mia mente fu così concentrata …​ la indirizzai e la rivolsi
alla conoscenza della morte e della rinascita degli esseri. Con l’occhio
divino, che è purificato e supera quello umano, vidi gli esseri morire e
rinascere, inferiori e superiori, belli e brutti, felici e infelici
nelle loro destinazioni. Compresi come gli esseri scompaiono e
ricompaiono in accordo con le loro azioni: “Questi esseri meritevoli
della loro sorte che ebbero una cattiva condotta con il corpo, con la
parola e con la mente, che oltraggiarono gli Esseri Nobili, con errate
visioni, che diedero seguito all’errata visione nelle loro azioni, alla
dissoluzione del corpo, dopo la morte, sono riapparsi in una condizione
di privazione, in una destinazione infelice, nella perdizione, perfino
all’inferno. Ma questi esseri meritevoli della loro sorte che ebbero una
buona condotta con il corpo, con la parola e con la mente, che non
oltraggiarono gli Esseri Nobili, con rette visioni, che diedero seguito
alla retta visione nelle loro azioni, alla dissoluzione del corpo, dopo
la morte, sono riapparsi in una destinazione felice, perfino in un
paradiso celeste”. Così, con l’occhio divino, che è purificato e supera
quello umano, vidi gli esseri morire e rinascere, inferiori e superiori,
belli e brutti, felici e infelici nelle loro destinazioni. Compresi come
gli esseri scompaiono e ricompaiono in accordo con le loro azioni.
Questa fu la seconda vera conoscenza da me conseguita nella seconda
veglia notturna. L’ignoranza fu bandita e sorse la vera conoscenza,
l’oscurità fu bandita e sorse la luce, come avviene in chi è diligente,
ardente e dotato di autocontrollo. Quando però in me sorse questa
sensazione piacevole, non le consentii d’impossessarsi della mia mente».


«Quando la mia mente fu così concentrata …​ la indirizzai e la rivolsi
alla conoscenza dell’esaurimento delle contaminazioni. Ebbi la diretta
conoscenza, come invero è, che “Questa è la sofferenza”, che “Questa è
l’origine della sofferenza”, che “Questa è la cessazione della
sofferenza” e che “Questo è il Sentiero che conduce alla cessazione
della sofferenza”. Ebbi la diretta conoscenza, come invero è, che
“Queste sono contaminazioni”, che “Questa è l’origine delle
contaminazioni”, che “Questa è la cessazione delle contaminazioni” e che
“Questo è il Sentiero che conduce alla cessazione delle contaminazioni”.
Conoscendo e vedendo in questo modo, il mio cuore fu liberato dalla
contaminazione del desiderio sensoriale, dalla contaminazione
dell’essere e dalla contaminazione dell’ignoranza. Quando il mio cuore
fu liberato, giunse la conoscenza: “È liberato”. Ebbi la diretta
conoscenza: “La nascita è distrutta, la santa vita è stata vissuta, quel
che doveva essere fatto è stato fatto, non ci sarà altra rinascita”.
Questa fu la terza vera conoscenza da me conseguita nella terza veglia
notturna. L’ignoranza fu bandita e sorse la vera conoscenza, l’oscurità
fu bandita e sorse la luce, come avviene in chi è diligente, ardente e
dotato di autocontrollo. Quando però in me sorse questa sensazione
piacevole, non le consentii d’impossessarsi della mia mente».


\suttaRef{M. 36}


\narrator{Secondo narratore.} Questa è la descrizione nei termini della struttura
della condizionalità, in altre parole della genesi
interdipendente.\footnote{Per la genesi interdipendente (originazione interdipendente o coproduzione condizionata), si veda il capitolo 12.} Dovremo tornare in seguito su questo
argomento.


\voice{Prima voce.} «Prima della mia Illuminazione, quando ero ancora solo un
Bodhisatta non illuminato, pensai: “Questo mondo è caduto in un pantano
perché è nato, invecchia e muore, scompare e riappare, e tuttavia non
conosce una via d’uscita da questa sofferenza. Quando sarà individuata
una via d’uscita da questa sofferenza?”».


«Pensai: “Che cos’è che fa giungere all’esistenza l’invecchiamento e la
morte? Quali sono le condizioni di cui necessitano?”. Allora mediante
un’appropriata attenzione\footnote{Oppure: approfondita considerazione, saggia riflessione (\emph{yoniso manasikāra}) (Nyp.).} riuscii a capire: “La
vecchiaia e la morte giungono all’esistenza quando c’è la nascita, la
nascita è la condizione di cui necessitano”».


«Pensai: “Che cos’è che fa giungere all’esistenza la nascita? Qual è la
condizione di cui necessita?”. Allora mediante un’appropriata attenzione
riuscii a capire: “La nascita giunge all’esistenza quando c’è divenire,
il divenire è la condizione di cui necessita”».


«Pensai: “Che cos’è che fa giungere all’esistenza il divenire? Qual è la
condizione di cui necessita?”. Allora mediante un’appropriata attenzione
riuscii a capire: “Il divenire giunge all’esistenza quando c’è
l’attaccamento, l’attaccamento è la condizione di cui necessita”».


«…​ L’attaccamento giunge all’esistenza quando c’è la brama …​».


«…​ La brama giunge all’esistenza quando c’è la sensazione (piacevole,
dolorosa o neutra) …​».


«…​ La sensazione giunge all’esistenza quando c’è il contatto …​».


«…​ Il contatto giunge all’esistenza quando c’è la sestuplice base per
il contatto …​».


«Pensai: “Che cos’è che fa giungere all’esistenza la sestuplice base per
il contatto? Qual è la condizione di cui necessita?”. Allora mediante
un’appropriata attenzione riuscii a capire: “La sestuplice base per il
contatto giunge all’esistenza quando c’è nome-e-forma, nome-e-forma è la
condizione di cui necessita”».


«Pensai: “Che cos’è che fa giungere all’esistenza nome-e-forma? Qual è
la condizione di cui necessita?”. Allora mediante un’appropriata
attenzione riuscii a capire: “Nome-e-forma giunge all’esistenza quando
c’è la coscienza, la coscienza è la condizione di cui necessita”».


«Pensai: “Che cos’è che fa giungere all’esistenza la coscienza? Qual è
la condizione di cui necessita?”. Allora mediante un’appropriata
attenzione riuscii a capire: “La coscienza giunge all’esistenza quando
c’è nome-e-forma, nome-e-forma è la condizione di cui necessita”».


«Pensai: “Questa coscienza gira su se stessa, non va al di là di
nome-e-forma. Ed è questo che succede quando si nasce, si invecchia e si
muore, si scompare o si riappare. Vale a dire: nome-e-forma è la
condizione per l’esistenza della coscienza; la coscienza, per
nome-e-forma; nome-e-forma, per la sestuplice base per il contatto; il
contatto, per la sensazione; la sensazione, per la brama; la brama, per
l’attaccamento; l’attaccamento, per il divenire; il divenire, per la
nascita; la nascita, per l’invecchiamento e la morte, e anche per
l’afflizione, il lamento, il dolore, il dispiacere e la disperazione.
Così ha origine tutto questo aggregato di sofferenza”. L’origine,
l’origine: questa fu l’intuizione, la conoscenza, la comprensione, la
visione, la luce che sorse in me per cose mai udite prima».


«Pensai: “Che cos’è che non fa giungere all’esistenza l’invecchiamento e
la morte? Che cosa deve cessare perché cessino l’invecchiamento e la
morte?”. Allora mediante un’appropriata attenzione riuscii a capire:
“Quando non c’è nascita, non giunge all’esistenza l’invecchiamento e la
morte, con la cessazione della nascita c’è la cessazione
dell’invecchiamento e della morte”».


«…​ Quando non c’è il divenire, non giunge all’esistenza la nascita
…​».


«…​ Quando non c’è l’attaccamento, non giunge all’esistenza il divenire
…​».


«…​ Quando non c’è la brama, non giunge all’esistenza l’attaccamento
…​».


«…​ Quando non c’è la sensazione, non giunge all’esistenza la brama
…​».


«…​ Quando non c’è il contatto, non giunge all’esistenza la sensazione
…​».


«…​ Quando non c’è la sestuplice base per il contatto, non giunge
all’esistenza il contatto …​».


«…​ Quando non c’è nome-e-forma, non giunge all’esistenza la sestuplice
base per il contatto …​».


«…​ Quando non c’è la coscienza, non giunge all’esistenza nome-e-forma
…​».


«Pensai: “Che cos’è che non fa giungere all’esistenza la coscienza? Che
cosa deve cessare perché cessi la coscienza?”. Allora mediante
un’appropriata attenzione riuscii a capire: “Quando non c’è
nome-e-forma, non giunge a esistere la coscienza, con la cessazione di
nome-e-forma c’è la cessazione della coscienza”».


«Pensai: “Questo è il Sentiero per l’Illuminazione che ora ho raggiunto,
vale a dire: con la cessazione di nome-e-forma, c’è la cessazione della
coscienza; con la cessazione della coscienza, la cessazione di
nome-e-forma; con la cessazione di nome-e-forma, la cessazione della
sestuplice base; con la cessazione della sestuplice base, la cessazione
del contatto; con la cessazione del contatto, la cessazione della
sensazione; con la cessazione della sensazione, la cessazione della
brama; con la cessazione della brama, la cessazione dell’attaccamento;
con la cessazione dell’attaccamento, la cessazione del divenire; con la
cessazione del divenire, la cessazione della nascita; con la cessazione
della nascita, cessano l’invecchiamento e la morte, e anche
l’afflizione, il lamento, il dolore, il dispiacere e la disperazione.
Così c’è la cessazione di tutto questo aggregato di sofferenza”. La
cessazione, la cessazione: questa fu l’intuizione, la conoscenza, la
comprensione, la visione, la luce che sorse in me per cose mai udite
prima».


«Supponiamo che vagando in una foresta selvaggia una persona trovi un
antico sentiero, un antico percorso, usato dagli uomini di un tempo, che
lo segua e che, facendolo, scopra un’antica città, un’antica capitale di
un regno, dove avevano vissuto gli uomini di un tempo, con parchi e
boschetti e laghi, circondata da mura e bella a vedersi. Così anche io
ho trovato l’antico sentiero, l’antico percorso, usato dagli Esseri
Completamente Illuminati di un tempo».


«E qual era quell’antico sentiero, quell’antico percorso? Era questo
Nobile Ottuplice Sentiero, vale a dire: retta visione, retta intenzione,
retta parola, retta azione, retto modo di vivere, retto sforzo, retta
consapevolezza, retta concentrazione».


«Lo seguii. Facendolo, conobbi direttamente l’invecchiamento e la morte,
la loro origine, la loro cessazione e la via che conduce alla loro
cessazione. Conobbi direttamente il divenire …​ l’attaccamento …​ la
brama …​ la sensazione …​ il contatto …​ la sestuplice base …​
nome-e-forma …​ la coscienza …​ Conobbi direttamente le formazioni
mentali, la loro origine, la loro cessazione e la via che conduce alla
loro cessazione».


\suttaRef{S. 12:65; cf. D. 14}


\narrator{Secondo narratore.} Ecco infine la descrizione in termini d’un retto
giudizio del mondo degli atti e delle idee condizionate, classificati in
questo discorso nei cinque aggregati, all’interno dei quali tutta
l’esperienza dei fenomeni condizionati può, allorché essa viene
analizzata, rientrare.


\voice{Prima voce.} «Prima della mia Illuminazione, quando ero ancora solo un
Bodhisatta non illuminato, pensai: “Nel caso della forma materiale,
della sensazione (piacevole, dolorosa o neutra), della percezione, delle
formazioni mentali, della coscienza, qual è la gratificazione, qual è il
pericolo, qual è la via d’uscita?”. Allora pensai: “Nel caso di ognuna
di esse la gratificazione è rappresentata dal piacere corporeo e dalla
gioia mentale che sorge in dipendenza da queste cose (i cinque
aggregati). Il fatto che queste cose sono tutte impermanenti, dolorose e
soggette al cambiamento è il pericolo. Il disciplinamento, l’abbandono
del desiderio e della bramosia per essi sono la via d’uscita».


Fino a quando non conobbi per mezzo di una conoscenza diretta, così
com’è in realtà, che quella era la gratificazione, quello il pericolo e
quella la via d’uscita, nel caso dei cinque aggregati affetti
dall’attaccamento, fino ad allora non affermai di aver scoperto la
Suprema Illuminazione nel mondo con i suoi deva, con i suoi Māra e con
le sue divinità, in questa generazione con i suoi monaci e brāhmaṇa, con
i suoi principi e uomini. Però, appena conobbi per mezzo di una
conoscenza diretta, così com’è in realtà, che quella è la
gratificazione, quello il pericolo e quella la via d’uscita, nel caso
dei cinque aggregati affetti dall’attaccamento, allora affermai di aver
scoperto la Suprema Illuminazione nel mondo con i suoi deva, con i suoi
Māra e con le sue divinità, in questa generazione con i suoi monaci e
brāhmaṇa, con i suoi principi e uomini».


\suttaRef{S. 22:26}


«Essendo io stesso soggetto a nascita, invecchiamento, malattia, morte,
dolore e contaminazioni, vedendo il pericolo in quel che è soggetto a
queste cose, e cercando la suprema cessazione della schiavitù, ciò che
non nasce, non invecchia, non si ammala, non muore, ciò che è senza
dolore e senza contaminazioni, il Nibbāna, lo ottenni. La conoscenza e
la visione sorsero in me: “La mia Liberazione è certa, questa è la mia
ultima nascita, ora non ci saranno più rinascite in vite future”».


\suttaRef{M. 26}


\narrator{Secondo narratore.} L’Illuminazione è stata ora raggiunta. E la
tradizione afferma che le prime parole pronunciate dal Buddha – non più
Bodhisatta – furono queste.


\begin{quote}
\cantor{Cantore}


Cercando il costruttore della casa, ma senza trovarlo, \\
in tondo ho viaggiato per innumerevoli vite.


Oh! è doloroso nascere ancora e poi ancora. \\
Costruttore della casa, ora ti ho visto, \\
non costruirai di nuovo la casa. \\
Le tue assi sono state rimosse, \\
anche la tua trave di colmo è stata spezzata.


La mia mente ha raggiunto l’increato Nibbāna \\
e la fine di ogni genere di brama.
\end{quote}

\suttaRef{Dh. 153-54}


\narrator{Secondo narratore.} Se queste furono le prime parole pronunciate
dall’Illuminato, esse secondo la tradizione non lo furono tuttavia ad
alta voce. Le prime parole pronunciate ad alta voce furono quelle
contenute nella prima delle tre strofe che cominciano: «Quando le cose
sono del tutto manifeste …​» (si veda l’inizio del prossimo capitolo).


