\chapter{La vecchiaia}

\voice{Prima voce.} Così ho udito. Una volta, quando il Beato risiedeva a
Sāvatthī, il re Pasenadi di Kosala andò a trovarlo a mezzogiorno. Il Beato gli
chiese: «Da dove vieni, a mezzogiorno, gran re?».

«Signore, mi sono molto occupato dell’amministrazione di quelle cose che devono
essere fatte da sovrani consacrati guerrieri, ubriachi d’autorità e ossessionati
dalla brama per i piaceri sensoriali, che hanno reso stabili i loro territori e
sono sopravvissuti alla conquista di un’ampia distesa di terra».

«Cosa ne pensi, gran re? Se un uomo fidato e affidabile venisse da te giungendo
dall’est e dicesse: “Sappiate, sire, che io vengo dall’est. Là ho visto
un’imponente montagna che, alta fino ai cieli, avanza e schiaccia ogni essere
vivente. Fai quel che dovresti fare, sire”. E poi un uomo giungesse dall’ovest,
un altro dal nord e un altro ancora dal sud, e ognuno dicesse la stessa cosa.
Ora, con una minaccia così poderosa che incombe, quella di un’impietosa
distruzione dell’umanità, dell’impossibilità di conservare l’esistenza umana,
che cosa faresti?».

«In una circostanza come quella, Signore, che cos’altro potrei fare se non
camminare nel Dhamma, camminare nella rettitudine, coltivare quel che è salutare
e ottenere meriti?».

«Ti dico, gran re, ti dichiaro: la vecchiaia e la morte incombono su di te. Con
la vecchiaia e la morte che incombono su di te, gran re, che cosa faresti?».

«Con la vecchiaia e la morte che incombono su di me, Signore, che cos’altro
potrei fare se non camminare nel Dhamma, camminare nella rettitudine, coltivare
quel che è salutare e ottenere meriti? Per quanto riguarda quelle cose che
possono essere fatte da sovrani consacrati guerrieri, ubriachi d’autorità e
ossessionati dalla brama per i piaceri sensoriali, che hanno reso stabili i loro
territori e sono sopravvissuti alla conquista di un’ampia distesa di terra –
intendo battagliare con elefanti, cavalli, carri e fanteria – esse sono prive di
scopo e di utilità quando la vecchiaia e la morte incombono su di me. Nella mia
corte ci sono ministri esperti in incanti capaci di confondere l’avanzare dei
nemici, ma essi sono privi di scopo e di utilità quando la vecchiaia e la morte
incombono su di me. Nella mia corte ci sono oro e lingotti immagazzinati
sottoterra e riposti in magazzini per comprare con il denaro i nemici che
avanzano, ma essi sono privi di scopo e di utilità quando la vecchiaia e la
morte incombono su di me. Quando la vecchiaia e la morte incombono su di me,
Signore, che cos’altro potrei fare se non camminare nel Dhamma, camminare nella
rettitudine, coltivare quel che è salutare e ottenere meriti?».

«È così, gran re, è così. Quando la vecchiaia e la morte incombono su di te, che
cos’altro puoi fare se non camminare nel Dhamma, camminare nella rettitudine,
coltivare quel che è salutare e ottenere meriti?».

\suttaRef{S. 3:25}

Una volta, quando il Beato soggiornava a Sāvatthī nel Monastero Orientale, il
Palazzo della Madre di Migāra, egli si era alzato dal ritiro al crepuscolo e
stava seduto a scaldarsi la schiena ai raggi del sole che tramontava. Il
venerabile Ānanda andò da lui e gli rese omaggio. Mentre sfregava gli arti del
Beato, disse: «È meraviglioso, Signore, è magnifico! Ora il colore della pelle
del Beato non è più chiaro e luminoso. Tutte le sue membra sono flaccide e
grinzose, il suo corpo è piegato in avanti e sembra che ci sia un mutamento
nelle facoltà sensoriali dei suoi occhi, dei suoi orecchi, del suo naso, della
sua lingua e delle sue sensazioni corporee».

«È così, Ānanda, è così. La giovinezza deve invecchiare, la salute deve
ammalarsi, la vita deve morire. Ora il colore della mia pelle non è più chiaro e
luminoso. Tutte le mie membra sono flaccide e grinzose, il mio corpo è piegato
in avanti e sembra che ci sia un mutamento nelle facoltà sensoriali dei miei
occhi, dei miei orecchi, del mio naso, della mia lingua e delle mie sensazioni
corporee».

Così disse il Beato. Quando il Sublime ebbe detto questo, il Maestro disse
ancora:

\begin{quote}
Vergognati, sordida Età, \\
artefice di bruttezza! \\
L’Età ha ora calpestato \\
quella forma che un tempo aveva grazia. \\
Per vivere cento anni \\
non si può ingannare il Decadimento \\
nel quale nessuno alloggia [a lungo] \\
ma schiaccia ogni cosa.
\end{quote}

\suttaRef{S. 48:41}

Una volta il Beato soggiornava a Sāmagāna, nel territorio dei Sakya, subito dopo
che Nigaṇṭha Nāthaputta era morto a Pāvā. Dopo la sua morte i Nigaṇṭha si erano
divisi in due fazioni, e litigavano, bisticciavano, disputavano e si ferivano a
vicenda con frecce fatte di parole: «Tu non conosci questo Dhamma e Disciplina.
Come puoi comprendere questo Dhamma e Disciplina? Sei in torto. Io ho ragione.
Io sono coerente. Tu non sei coerente. Quel che avrebbe dovuto essere detto
all’inizio tu l’hai detto alla fine. Quel che avrebbe dovuto essere detto alla
fine tu l’hai detto all’inizio. Quel che hai ideato è stato capovolto. Il tuo
insegnamento è stato smentito. Sei stato sconfitto. Va e impara meglio, o
districatene da solo se puoi». Sembrava che ci fosse una conflittualità interna
tra gli allievi di Nigaṇṭha Nāthaputta. E i suoi discepoli vestiti di bianco
erano delusi, costernati e disgustati dai suoi allievi, come se costoro
dimorassero nel suo mal dichiarato Dhamma e Disciplina, che era difficile da
penetrare, che non conduceva da nessuna parte, che non era in grado di favorire
la pace, [che era stato] proclamato da chi non era completamente illuminato, [un
Dhamma e una Disciplina] il cui luogo sacro era spezzato e lasciato privo di un
rifugio.

Allora il novizio Cunda, che aveva trascorso la stagione delle piogge a Pāvā,
andò dal venerabile Ānanda e gli disse quel che era successo. Loro andarono
insieme dal Beato e il venerabile Ānanda lo informò di quello che il novizio
Cunda aveva detto. Egli aggiunse: «Signore, io penso: “Che non ci siano
controversie quando il Beato se ne sarà andato. Le controversie conducono alla
sventura e all’infelicità di molti, al danno, alla sventura e alla sofferenza di
divinità e uomini”».

«Ānanda, cosa ne pensi? Questi insegnamenti che io ho appreso per conoscenza
diretta e vi ho insegnato – ossia i quattro fondamenti della consapevolezza, i
quattro retti sforzi, le quattro basi per il successo [spirituale], le cinque
facoltà spirituali, i cinque poteri, i sette fattori dell’Illuminazione e il
Nobile Ottuplice Sentiero – vedi anche due soli bhikkhu che li descrivano in
modo discorde?».

«No, Signore, ma ci sono persone che sono ora remissive nei riguardi del Beato,
ma che potrebbero, quando egli se ne sarà andato, generare nel Saṅgha
controversie sui mezzi di sostentamento e sul Codice della Disciplina Monastica.
Queste controversie condurrebbero alla sventura e all’infelicità di molti».

«Le controversie sui mezzi di sostentamento o sul Codice della Disciplina
Monastica sono marginali, Ānanda. Se però nel Saṅgha dovessero sorgere
controversie sul Sentiero o sul modo di praticare, controversie come queste
condurrebbero davvero alla sventura e all’infelicità di molti».

\suttaRef{M. 104}

Una volta il Beato soggiornava a Vesālī, nel boschetto a nord della città.
Allora Sunakkhatta, un figlio dei Licchavi, aveva appena lasciato questo Dhamma
e Disciplina e durante le riunioni, a Vesālī, affermava questo: «Il monaco
Gotama non ha ottenuto alcuna condizione sovrumana, alcuna distinzione nella
conoscenza e nella visione che sia degna di un Essere Nobile. Il monaco Gotama
insegna un Dhamma elaborato solo mediante il pensiero, seguendo il suo proprio
modo di indagare, così come a lui è venuto in mente, e a chiunque quel Dhamma
venga insegnato per il suo beneficio, esso, se praticato, conduce solo alla
completa cessazione della sofferenza (ma a nient’altro)».

Il venerabile Sāriputta sentì queste cose e le raccontò al Beato. «Sāriputta,
Sunakkhatta, uomo fuorviato, è un uomo in collera ed è con collera che queste
parole sono pronunciate. Pensando di screditare il Perfetto, in realtà egli lo
loda. Perché significa lodare il Perfetto, quando si dice: “E a chiunque quel
Dhamma venga insegnato per il suo beneficio, esso, se praticato, conduce solo
alla completa cessazione della sofferenza”».

«Ora, come uno che ha vissuto [le cose di cui parla], ho avuto esperienza
diretta di questo genere di santa vita conosciuta come quella dotata di quattro
fattori: io ho praticato gli estremi dell’ascetismo, della rozzezza, della
scrupolosità e dell’isolamento».

«Questo era il mio ascetismo.\footnote{Le austerità qui descritte sono
  principalmente quelle raccomandate dalla religione jainista.} Andavo nudo,
rifiutavo le convenzioni, mi leccavo le mani. Non andavo quando ero chiamato,
non mi fermavo quando me lo si chiedeva. Non accettavo una cosa quando mi veniva
portata né una cosa appositamente fatta [per me], oppure un invito. Non
accettavo nulla da una pentola, da una ciotola, attraverso la soglia [di una
porta], attraverso un bastone, attraverso un pestello, da due persone che
mangiavano insieme, da una donna con un bambino, da una donna che allattava, da
un luogo in cui una donna giaceva con un uomo, dal quale veniva distribuito del
cibo, in cui c’era un cane che aspettava, nel quale c’erano mosche che
ronzavano. Non accettavo né pesce né carne, non bevevo alcolici, vino oppure
liquori fermentati. Mi attenevo a un boccone ogni casa, a due bocconi ogni due
case … a sette bocconi ogni sette case. Vivevo di un piattino [di cibo], di due
piattini … di sette piattini ogni giorno. Mangiavo una volta al giorno, una
volta ogni due giorni … una volta ogni sette giorni, e così una volta ogni due
settimane, dimoravo seguendo la pratica di mangiare a intervalli prestabiliti.
Mi cibavo di ortaggi oppure di miglio, di riso selvatico, bucce, muschio, crusca
di riso, sciacquature, farina di sesamo, erba, sterco di mucca. Vivevo delle
radici degli alberi e di frutti, come uno che si ciba di frutti fatti cadere dal
vento. Mi vestivo di canapa, in misto canapa, di sudari, di stracci scartati, di
corteccia d’albero, di pelle d’antilope, di tessuto d’erba \emph{kusa}, di
tessuto di corteccia, di tessuto di legno, di lana fatta di capelli, di lana
d’animale, di penne di gufo. Ero uno che s’estirpava capelli e barba, attuando
la pratica dell’estirparsi capelli e barba. Ero uno che stava in piedi in
continuazione, rifiutavo di mettermi a sedere. Ero uno che stava accovacciato in
continuazione, mi dedicavo a mantenere la posizione accovacciata. Ero uno che
usava giacigli di spine, usai come letto un giaciglio di spine. Dimorai nella
pratica di balnearmi per la terza volta al calar della notte. Nei fatti, dimorai
nella pratica dei supplizi e della tortura del corpo nei suoi vari aspetti. Così
era il mio ascetismo».

«Questa era la mia rozzezza. Proprio come un albero \emph{tinduka} con il tronco
che nel corso degli anni accumula strati su strati, s’incrosta e poi si scrosta,
così anche sul mio corpo nel corso degli anni la polvere e lo sporco
s’accumulavano, s’incrostavano e poi si scrostavano. Non mi veniva in mente:
“Oh, che io strofini via questa polvere e questo sporco con la mano, o che
qualcun altro strofini via questa polvere e questo sporco con la sua mano”, non
mi venne mai in mente. Così era la mia rozzezza».

«Questa era la mia scrupolosità. Ero sempre consapevole quando facevo un passo
in avanti o indietro, fino al punto che ero colmo di compassione perfino per una
goccia d’acqua: “Che io non faccia del male alle minuscole creature che stanno
nelle fenditure del terreno”. Così era la mia scrupolosità».

«Questo era il mio isolamento. Me ne andavo in una qualche foresta e là restavo.
Proprio come un cervo cresciuto nella foresta quando vede degli esseri umani
fugge da un bosco all’altro, da una boscaglia all’altra, da una cavità
all’altra, da un poggio all’altro, anche io quando vedevo un bovaro, un pastore
o qualcuno che raccoglieva erba o rami, oppure un boscaiolo, fuggivo da un bosco
all’altro, da una boscaglia all’altra, da una cavità all’altra, da un poggio
all’altro. Perché? Così loro non avrebbero visto me o io non avrei visto loro.
Così era il mio isolamento».

«Andavo a carponi per le stalle quando il bestiame era uscito e il bovaro se
n’era andato, e mi nutrivo degli escrementi dei vitelli da latte. Fin quando
duravano i miei stessi escrementi e la mia stessa urina, mi nutrivo dei miei
stessi escrementi e della mia stessa urina. Così grande era la mia deformazione
nel nutrirmi».

«Andavo in qualche bosco che ispirava timore e là vivevo, in un bosco che
ispirava un tale timore che normalmente a un uomo che non fosse stato libero
dalla brama gli si sarebbero drizzati i capelli. Di notte dimoravo all’aperto e
di giorno nel bosco, quando quelle fredde notti invernali arrivavano durante gli
“otto giorni di ghiaccio”. Di giorno dimoravo all’aperto e di notte nel bosco
nell’ultimo mese della stagione calda. E là mi venne in mente in modo spontaneo
questa strofa, mai udita prima:

\begin{quote}
Congelato di notte e bruciato di giorno, \\
da solo in boschi che ispirano timore, \\
nudo, senza un fuoco vicino al quale sedersi, \\
l’eremita ancora persegue la sua ricerca».
\end{quote}

«Dormivo in un carnaio con le ossa di un morto come cuscino. E giovani pastori
si avvicinavano e mi sputavano addosso, mi urinavano addosso, mi gettavano
sporcizia e mi conficcavano bastoncini negli orecchi. Io però non nutrii rancore
per loro. Così era la mia equanimità».

«Ci sono alcuni monaci e brāhmaṇa che sostengono e credono che la purificazione
provenga dal cibo. E dicono: “Viviamo di noci di cola”. E mangiano noci di cola,
mangiano polvere di noci di cola, bevono acqua di noci di cola e fanno molte
preparazioni con le noci di cola. Ho fatto esperienza di mangiare una sola noce
di cola al giorno. Potresti però pensare, Sāriputta, che allora una noce di cola
fosse più grande, ma non dovresti pensare così. La noce di cola allora aveva
grosso modo la stessa grandezza che ha ora. Mangiando una sola noce di cola al
giorno, il mio corpo divenne estremamente emaciato … Ancora, ci sono alcuni
monaci e brāhmaṇa che sostengono e credono che la purificazione provenga dal
cibo. E dicono: “Viviamo di fagioli” … Dicono: “Viviamo di sesamo” … Dicono:
“Viviamo di riso” … Ho fatto esperienza di mangiare un solo fagiolo … un solo
seme di sesamo … un solo grano di riso al giorno … Mediante quel rito,
quell’osservanza, quella pratica di difficili imprese, però, non ho ottenuto
alcuna condizione sovrumana, alcuna distinzione nella conoscenza e nella visione
che sia degna di un Essere Nobile. Perché no? Perché non ho acquisito la nobile
comprensione che, se acquisita, conduce alla completa cessazione della
sofferenza in chi la pratica, in quanto essa appartiene a una nobile condizione
e conduce al di là (del mondo)».

«Ci sono alcuni monaci e brāhmaṇa che sostengono e credono che la purificazione
provenga da un particolare ciclo di rinascite. È però impossibile rinvenire un
ciclo di rinascite che io non abbia già attraversato in questo lungo viaggio, ad
eccezione delle Pure Dimore,\footnote{Le “Pure Dimore” sono parte dell’alto
  mondo di Brahmā (\emph{brahmaloka}), abitate solo da chi ha raggiunto la
  condizione di Chi è Senza Ritorno (si veda il
  \hyperlink{cap-12-La-Dottrina#pag263}{}), che sono là rinati alla morte e là
  vivono senza tornare in nessun altro mondo finché non raggiungono il Nibbāna
  definitivo.} perché se io fossi nato nelle Pure Dimore non sarei mai dovuto
tornare in questo mondo».

«Ci sono alcuni monaci e brāhmaṇa che sostengono e credono che la purificazione
provenga da pratiche sacrificali. È però impossibile rinvenire un tipo di
sacrificio che non sia stato da me offerto in questo lungo viaggio, quale
sovrano consacrato e guerriero o quale ricco membro della casta brāhmaṇa».

«Ci sono alcuni monaci e brāhmaṇa che sostengono e credono che la purificazione
provenga dall’adorazione del fuoco. È però impossibile rinvenire quel tipo di
fuoco che non sia già stato da me venerato in questo lungo viaggio, quale
sovrano consacrato e guerriero o quale ricco membro della casta brāhmaṇa».

«Ci sono alcuni monaci e brāhmaṇa che sostengono e credono questo: “Per tutto il
tempo che questo buon uomo è ancora giovane, un ragazzo dai capelli neri,
benedetto dalla gioventù, nella prima fase della vita, altrettanto a lungo egli
sarà perfetto per lucida comprensione. Quando però questo buon uomo sarà
anziano, vecchio, appesantito dagli anni, avanti nella vita e giunto allo stadio
finale, avendo ottanta, novanta o cento anni, allora la lucidità della sua
comprensione sarà perduta”. Non si dovrebbe pensare così. Ora io sono anziano,
vecchio, appesantito dagli anni, avanti nella vita e giunto allo stadio finale:
i miei anni hanno superato gli ottanta. Supponiamo che io abbia quattro
discepoli che possano giungere ai cento anni, la cui vita possa durare cento
anni, perfetti in consapevolezza, attenzione, memoria e lucidità di comprensione
– proprio come un arciere ben dotato, addestrato, esperto e provato, può con
facilità scagliare una freccia attraverso l’ombra di una palma, supponiamo che
loro siano fino a tal punto perfetti in consapevolezza, attenzione, memoria e
lucidità di comprensione. E supponiamo che loro mi facciano continuamente
domande sui quattro fondamenti della consapevolezza, e io risponda, e loro
ricordino ogni risposta, finché non abbiano più domande supplementari, e che non
facciano pause se non per mangiare, bere, masticare e assaporare, per orinare e
defecare e per riposare per vincere la sonnolenza: e l’esposizione del Dhamma
del Perfetto, le sue spiegazioni dei fattori del Dhamma e le sue risposte alle
domande non siano ancora terminate. E che nel frattempo, però, quei quattro miei
discepoli che possono giungere ai cento anni, la cui vita può durare cento anni,
siano morti alla fine di quei cento anni. Sāriputta, anche se tu dovessi
portarmi in giro su un letto, non ci sarà alcun cambiamento nella lucidità della
comprensione del Perfetto».

\suttaRef{M. 12}

\narrator{Primo narratore.} Negli ultimi anni del Buddha si verificarono un
certo numero di vessazioni – eventi, ossia, che sarebbero stati vessatorî
secondo il comune buon senso. Poco sopra si è detto come Sunakkhatta, che in
precedenza era stato un bhikkhu (oltre che attendente personale del Buddha),
lasciò il Buddha, parlò pubblicamente contro di lui e sminuì i suoi poteri
sovrannaturali, per cui il Buddha fece il suo “ruggito del leone” , dichiarando
che non c’era automortificazione che non avesse praticato e metodo di
autopurificazione che non avesse provato. Egli avrebbe poi presto perso i suoi
due discepoli eminenti. Nel frattempo, il re Pasenadi di Kosala, suo devoto
sostenitore per più di quarant’anni, fu sempre più infastidito da problemi
politici.

\narrator{Secondo narratore.} Il re Pasenadi aveva la stessa età del Buddha e,
perciò, aveva superato gli ottant’anni. Era stato travagliato da guerre
occasionali e prive di risultati con suo nipote, re Ajātasattu di Magadha, e da
sommosse politiche all’interno del suo stesso regno. Di conseguenza a un intrigo
di palazzo, il suo comandante in armi, il generale Bandhula, fu accusato di aver
complottato contro di lui e messo a morte. Solo in seguito egli apprese che era
innocente. Il rimorso lo assediava. Forse al fine di fare ammenda, promosse a
quella stessa carica il nipote del generale, Dīgha Kārāyana.

\suttaRef{Commentario a M. 89 e D. 16}

\narrator{Primo narratore.} Il re Pasenadi andò dal Buddha per chiedergli
consiglio. Quando morì la sua devota consorte, la regina Mallikā, si recò
profondamente sconsolato dal Buddha, che allora si trovava a Sāvatthī, per
cercare consolazione.

\suttaRef{Si veda A. 5:49}

\narrator{Secondo narratore.} Il palazzo del re e la sua smagliante capitale non
gli recavano più alcun piacere. Li lasciò per un po’ per errare di luogo in
luogo con un grande seguito, ma senza alcuna meta.

\narrator{Primo narratore.} Durante questo vagare nostalgico e sconsolato,
quando la strada dell’anziano sovrano incrociava quella del Buddha, il re andava
a trovarlo. La sua morte non è registrata nel Tipiṭaka. Un discorso [del Buddha]
è però collocato dal Commentario tra gli eventi che immediatamente la
precedettero. Questo è il racconto del loro ultimo incontro.

\voice{Prima voce.} Così ho udito. Una volta il Buddha soggiornava nel
territorio dei Sakya, in una città dei Sakya chiamata Medaḷumpa. In quella
circostanza il re Pasenadi di Kosala giunse a Nāgaraka per alcuni affari e altre
cose ancora. Egli disse allora a Dīgha Kārāyana: «Amico mio, chiama a raduno le
carrozze di corte. Andiamo nel parco degli svaghi per vedere un piacevole
panorama».

«E sia, sire», rispose Dīgha Kārāyana. Quando le carrozze furono pronte, egli
informò il re: «Sire, le carrozze di corte sono pronte. È tempo ora, gran re, di
fare quel che ritenete opportuno».

Così il re Pasenadi montò su una carrozza di corte e, con tutto il seguito
regio, si diresse verso il parco. Procedette finché la strada lo consentì alle
carrozze e poi scese e continuò a piedi. Mentre stava camminando e vagando per
fare un po’ d’esercizio, osservò le radici degli alberi che infondevano in lui
fiducia e sicurezza. Erano calme, le voci non le disturbavano, con un’aria
distaccata, su di esse si sarebbe potuto giacere nascosti dalla gente,
favorevoli al ritiro. Vederle gli ricordò il Beato. Allora egli disse: «Dīgha
Kārāyana, amico mio, queste radici degli alberi sono come quelle … quando noi
eravamo soliti prestare omaggio al Beato, realizzato e completamente illuminato.
Dove vive ora il Beato, realizzato e completamente illuminato?».

«C’è una città dei Sakya, sire, chiamata Medaḷumpa. Il Beato, realizzato e
completamente illuminato vive lì ora».

«Quanto dista Nāgaraka da Medaḷumpa?».

«Non dista molto, sire. Circa tre leghe. C’è abbastanza luce del giorno per
arrivare fin là».

«Allora prepara le carrozze, amico mio. Andiamo a trovare il Beato, realizzato e
completamente illuminato».

«E sia, sire», rispose Dīgha Kārāyana. Così il re condusse [il suo seguito] da
Nāgaraka fino alla città dei Sakya, Medaḷumpa, arrivando lì quando era ancora
giorno. Attraversò il parco, andando avanti finché la strada consentì alle
carrozze di procedere, e poi scese e continuò a piedi.

In quell’occasione numerosi bhikkhu stavano facendo la meditazione camminata
all’aperto. Il re si avvicinò a loro e chiese: «Venerabili signori, dove sta
vivendo ora il Beato, realizzato e completamente illuminato? Vorremmo vedere il
Beato, realizzato e completamente illuminato».

«Quella è la sua dimora, gran re, quella con la porta chiusa. Vai fino lì
silenziosamente, fino al porticato, senza affrettarti, poi tossisci e bussa. Il
Beato ti aprirà».

Il re Pasenadi lì e allora si tolse la spada e il turbante reale, e le consegnò
a Dīgha Kārāyana. Dīgha Kārāyana pensò: «Il re sta ora recandosi a un incontro
privato: devo ora aspettare qui da solo?».

Seguendo le indicazioni, il re si recò fino alla porta. Quando egli bussò, il
Beato aprì la porta. Il re entrò nella dimora e si prostrò ai piedi del Beato.
Ricoprì di baci i piedi del Beato, e li accarezzò con le sue mani pronunciando
il suo nome in questo modo: «Signore, io sono re Pasenadi di Kosala, Signore, io
sono re Pasenadi di Kosala».

«Gran re, quale beneficio vedi nel prestare un così estremo omaggio a questo
corpo e a mostrare una tale amicizia?».

«Signore, io credo che in relazione al Beato queste cose siano vere: il Beato è
completamente illuminato, il Dhamma è ben proclamato dal Beato, il Saṅgha dei
discepoli del Beato è sulla buona strada. Ora, Signore, io vedo alcuni monaci e
brāhmaṇa che conducono la santa vita per dieci, venti, trenta, quarant’anni, e
poi li vedo godere di tutti e cinque i tipi di piacere sensoriale e indulgere a
essi. Qui, invece, vedo i bhikkhu condurre la santa vita in tutte le sue
perfezioni per tutta la vita e finché dura il loro respiro. Infatti, Signore, da
nessun’altra parte vedo una santa vita così perfetta come qui. Ecco perché credo
che in relazione al Beato queste cose siano vere: il Beato è completamente
illuminato, il Dhamma è ben proclamato dal Beato, il Saṅgha dei discepoli del
Beato è sulla buona strada».

«Ancora una volta, Signore, i sovrani litigano con i sovrani, i nobili guerrieri
con i nobili guerrieri, i brāhmaṇa con i brāhmaṇa, i capifamiglia con i
capifamiglia, la madre con il figlio, il figlio con la madre, il padre con il
figlio, il figlio con il padre, il fratello con il fratello, il fratello con la
sorella, la sorella con il fratello, l’amico con l’amico. Qui, però, vedo i
bhikkhu dilettarsi nella concordia, vivere senza contrasti come il latte con
l’acqua e guardarsi l’un l’altro con occhi gentili. In verità, Signore, non ho
visto in alcun altro posto un gruppo di persone vivere così armoniosamente
insieme. Anche per questa ragione credo che in relazione al Beato queste cose
siano vere».

«Ancora, Signore, ho camminato e vagato di parco in parco, di giardino in
giardino, e ho visto alcuni monaci e brāhmaṇa scarni, infelici, sgradevoli,
itterici, con le vene sporgenti sulle loro membra, che a malapena si potrebbe
pensare che uno possa volerli guardare una seconda volta. Pensai: “Certamente
questi venerabili stanno conducendo la santa vita insoddisfatti, oppure hanno
commesso un qualche crimine e lo stanno nascondendo, per questo sono così”.
Andai da loro e chiesi perché fossero così, e mi risposero: “Siamo scontenti,
gran re”. Qui, invece, vedo i bhikkhu sorridenti e allegri, sinceramente
gioiosi, chiaramente deliziati, le loro facoltà sono fresche, non agitate,
imperturbate, e che vivono di ciò che gli altri donano loro, dimorando con menti
simili a quelle di un cerbiatto libero. Pensai: “Se sono così, questi venerabili
certamente percepiscono che vi siano delle qualità straordinarie e distintive
nella Dispensazione del Beato”. Anche per questa ragione credo che in relazione
al Beato queste cose siano vere».

«Ancora, Signore, in quanto nobile sovrano consacrato e guerriero, sono in grado
di far giustiziare coloro che devono essere giustiziati, di multare coloro che
devono essere multati e di esiliare coloro che devono essere esiliati. Quando
però sono seduto in consiglio, [mentre parlo] mi interrompono, benché io dica:
“Buoni uomini, quando sono seduto in consiglio [mentre parlo] non
interrompetemi, aspettate che io finisca di parlare”. Qui vedo però un uditorio
di molte centinaia di bhikkhu e, mentre il Beato espone il Dhamma, tra i suoi
discepoli non si sente neanche uno starnuto o qualcuno che si schiarisce la
gola. Una volta, mentre il Beato esponeva il Dhamma a un uditorio di molte
centinaia [di bhikkhu], un discepolo del Beato si schiarì la gola. Allora uno
dei suoi compagni nella santa vita lo toccò con il ginocchio, dicendogli:
“Silenzio, venerabile signore, non fare rumori. Il Maestro sta esponendo il
Dhamma”. Pensai: “È meraviglioso, è stupefacente come un uditorio possa essere
così ben disciplinato senza né punizioni né armi”. Infatti, Signore, non ho
visto in alcun altro posto un uditorio così ben disciplinato. Anche per questa
ragione credo che in relazione al Beato queste cose siano vere».

«Ancora, Signore, ho visto alcuni allievi di nobili guerrieri, alcuni allievi di
brāhmaṇa, alcuni allievi di capifamiglia, alcuni allievi di monaci, che erano
intelligenti e conoscevano le teorie altrui come un provetto tiratore scelto sa
tirare con l’arco: uno avrebbe pensato che sarebbero certamente riusciti a
demolire falsi punti di vista con la conoscenza che possedevano. Loro sentirono
dire: “Il monaco Gotama visiterà questo villaggio o questa città”. E formulavano
una domanda: “Se egli sarà interrogato in questo modo, egli risponderà in questo
modo, e noi refuteremo la sua teoria in questo modo. E se sarà interrogato in
quest’altro modo, egli risponderà in quest’altro modo, e noi refuteremo la sua
teoria in quest’altro modo”. Loro sentirono dire: “Il monaco Gotama è venuto a
visitare questo villaggio o questa città”. Loro andarono dal monaco Gotama. Il
monaco Gotama li istruì, esortò, risvegliò e incoraggiò con un discorso di
Dhamma. Dopo di questo non gli posero neanche la domanda: come avrebbero potuto
refutare la sua teoria? Nei fatti divennero suoi discepoli. Anche per questa
ragione credo che in relazione al Beato queste cose siano vere».

«Ancora, Signore, ci sono Isidatta e Purāṇa, due miei carpentieri, che da me
ricevono cibo e sono mantenuti, per i quali io sono colui che provvede ai loro
mezzi di sostentamento e latore della loro buona fama. Nonostante ciò loro hanno
più rispetto del Beato che di me. Una volta, quando ero uscito con un esercito
per delle manovre e stavo mettendo alla prova questi carpentieri, avvenne che
trovammo ospizio in una dimora molto stretta. Allora questi due carpentieri
trascorsero la maggior parte della notte parlando di Dhamma, dopo di che si
misero a giacere rivolgendo la loro testa verso il luogo nel quale avevano
sentito che si trovava il Beato e i loro piedi verso di me. Pensai: “È
meraviglioso, è stupefacente! Certamente questi buoni uomini percepiscono che vi
siano delle qualità straordinarie e distintive nella Dispensazione del Beato”.
Anche per questa ragione credo che in relazione al Beato queste cose siano vere:
il Beato è completamente illuminato, il Dhamma è ben proclamato dal Beato, il
Saṅgha dei discepoli del Beato è sulla buona strada».

«Ancora, Signore, il Beato è un nobile guerriero e io sono un nobile guerriero,
il Beato è un Kosala e io sono un Kosala, il Beato ha ottant’anni e io ho
ottant’anni. Queste sono le ragioni per cui penso sia opportuno porgere un così
estremo omaggio al Beato e mostrare una tale amicizia. E ora, Signore, noi
andiamo, siamo impegnati e abbiamo molto da fare».

«È tempo ora, gran re, di fare quel che ritieni opportuno».

Allora il re Pasenadi di Kosala si alzò dal posto in cui sedeva e, dopo aver
prestato omaggio al Beato, se ne andò, girandogli a destra.

Subito dopo che se ne fu andato, il Beato si rivolse ai bhikkhu con queste
parole: «Bhikkhu, questo re Pasenadi ha proferito parole monumentali sul Dhamma.
Ricordatele, perché esse favoriscono il benessere e riguardano i fondamenti
della santa vita».

Questo è ciò che il Beato disse. I bhikkhu si sentirono soddisfatti, e si
deliziarono alle sue parole.

\suttaRef{M. 89}

\narrator{Primo narratore.} Quel che avvenne dopo che il re era stato a
colloquio [con il Beato] è raccontato solo nel Commentario.

\narrator{Secondo narratore.} Quando il re entrò nel luogo in cui il Buddha
dimorava, lasciando le insegne reali a Dīgha Kārāyana, quest’ultimo,
risentitosi, divenne sospettoso. Iniziò a fantasticare, pensando che il sovrano
avesse fatto giustiziare suo zio, il generale, dopo un precedente colloquio con
il Buddha, e si chiese se ora non stesse per venire il suo turno. Appena il re
entrò nel luogo in cui il Buddha dimorava, Dīgha Kārāyana se ne andò, prese con
sé le insegne e si recò nell’accampamento. Là intimò al figlio del re, il
principe Viḍūḍabha, di salire al trono immediatamente, minacciando di farlo lui
stesso se non avesse obbedito. Il principe acconsentì. Lasciando là solo un
cavallo, una spada e una donna del gineceo, Dīgha Kārāyana e il resto del
seguito partirono molto rapidamente per Sāvatthī, dopo aver detto alla donna che
aspettava di avvertire il re di non seguirli, se alla sua vita ci teneva. Quando
il re uscì dalla dimora dal luogo in cui il Buddha dimorava, vedendo che lì non
c’era nessuno, andò ove prima si trovava l’accampamento. La donna che
l’attendeva gli disse quel che era avvenuto.

Egli decise di chiedere aiuto a suo nipote, re Ajātasattu. Durante il lungo
viaggio per Rājagaha mangiò del cibo grossolano, di un genere del quale non era
solito cibarsi e bevette molta acqua. Quando arrivò a Rājagaha era tardi e le
porte della città erano chiuse, così che fu costretto a trascorrere la notte in
un pubblico ricovero. Durante la notte fu attaccato da una violenta malattia e
verso l’alba spirò. La donna che l’aveva atteso, tra le braccia della quale il
re era spirato, lamentò ad alta voce: «Il mio signore, il re di Kosala, che
governò due regni, è morto come un povero e ora giace in un pubblico ricovero
fuori dalle mura d’una città straniera!». La notizia giunse al re Ajātasattu,
che subito ordinò un regio funerale. In seguito fece mostra d’indignazione,
ordinando un attacco punitivo contro suo cugino, ora re Viḍūḍabha, ma venne
subito persuaso dai suoi ministri che, siccome l’anziano sovrano era morto, un
tentativo del genere sarebbe stato inutile ed egli puntualmente riconobbe la
successione di suo cugino.

