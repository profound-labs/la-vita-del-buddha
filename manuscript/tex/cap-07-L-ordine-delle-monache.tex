\chapter{L'Ordine delle monache}

\narrator{Primo narratore.} Il racconto appena offerto ha mostrato come il Buddha
acconsentì a trascorrere la stagione delle piogge a
Sāvatthī.\footnote{Il \emph{Mālālaṅkāravatthu} dice che questa stagione delle piogge fu trascorsa a Rājagaha, nel Boschetto di Bambù, ma ciò è errato.} Perciò,
se il computo tradizionale delle
prime tre piogge successive all’Illuminazione è esatto, la quarta
stagione delle piogge fu trascorsa nel Boschetto di Jeta. Questo è un
episodio che potrebbe appartenere a quel periodo.


\voice{Prima voce.} Così ho udito. Una volta, quando il Beato soggiornava a
Sāvatthī, nel Boschetto di Jeta, nel Parco di Anāthapiṇḍika, il
venerabile Nanda, il figlio della zia materna del Beato, disse a un
certo numero di bhikkhu: «Amici, sto conducendo la santa vita
insoddisfatto. Non posso continuare la santa vita. Rinuncerò
all’addestramento e tornerò a ciò che ho abbandonato».


Allora quei bhikkhu andarono dal Beato e glielo raccontarono. Il Beato
disse a un bhikkhu: «Bhikkhu, vai a dire al bhikkhu Nanda queste parole
a nome mio: “Il Maestro ti chiama, amico”».


«E sia, Signore», rispose il bhikkhu. Ed egli si recò dal venerabile
Nanda e gli portò il messaggio. Il venerabile andò dal Beato, che gli
chiese: «Nanda, è vero, come sembra, che tu stai conducendo la santa
vita insoddisfatto, che tu non puoi continuare la santa vita, e che tu
rinuncerai all’addestramento e tornerai a ciò che hai abbandonato?».


«Sì, Signore».


«Perché, Nanda?».


«Signore, quando sono andato via per rinunciare alla vita famigliare, la
bellissima Sakya Janapadakalyāṇī, con i suoi capelli un po’ tirati
indietro, mi seguì con lo sguardo e disse: “Torna presto, principe”.
Quando penso a questo, conduco la santa vita insoddisfatto».


Allora il Beato prese il venerabile Nanda per un braccio e, con la
stessa velocità con cui un uomo forte distende il suo braccio piegato o
piega il suo braccio disteso, scomparvero entrambi dal Boschetto di Jeta
e apparvero nel paradiso delle Trentatré Divinità. In quel momento
cinquecento ninfe dai piedi di colomba erano giunte per prendersi cura
di Sakka, Sovrano degli dèi. Il Beato chiese al venerabile Nanda:
«Nanda, vedi queste cinquecento ninfe dai piedi di colomba?».


«Sì, Signore».


«Cosa ne pensi, Nanda, cosa è più adorabile, più bello, più
affascinante, la bellezza Sakya di Janapadakalyāṇī oppure queste
cinquecento ninfe dai piedi di colomba?».


«Signore, la bellezza Sakya di Janapadakalyāṇī è quella di una scimmia
ustionata con il naso e le orecchie mozzate, se paragonata a queste
cinquecento ninfe dai piedi di colomba. Lei non conta nulla, non è
affatto come loro, non è possibile alcun confronto. Queste cinquecento
ninfe sono infinitamente più adorabili, belle e affascinanti».


«Prova diletto nella santa vita, Nanda, dilettati in essa e ti
garantisco che avrai cinquecento ninfe dai piedi di colomba».


«Signore, se il Beato mi garantisce che le avrò, allora proverò diletto
nella santa vita».


Allora il Beato prese il venerabile Nanda per un braccio e, come prima,
scomparvero entrambi dal paradiso delle Trentatré Divinità e riapparvero
nel Boschetto di Jeta.


I bhikkhu ascoltarono: «Sembra che il venerabile Nanda stia conducendo
la santa vita per amore delle ninfe, sembra che il Beato gli abbia
garantito che avrà cinquecento ninfe dai piedi di colomba». Allora
coloro che tra i bhikkhu erano suoi amici lo trattarono come un
mercenario che aveva venduto se stesso: «Sembra che il venerabile Nanda
sia un mercenario, poiché conduce la santa vita per amore delle ninfe.
Sembra che il Beato gli abbia garantito che avrà cinquecento ninfe dai
piedi di colomba».


Egli si vergognò, si sentì umiliato e costernato quando udì queste
parole dai suoi compagni. Andò così a dimorare in solitudine, appartato,
diligente, ardente e dotato di autocontrollo, fino a che lui stesso
realizzò la conoscenza diretta, e qui e ora entrò e dimorò in quella
suprema meta della santa vita per la quale gli uomini di famiglia
giustamente lasciano la loro casa per una vita priva di fissa dimora.


Egli ne ebbe la conoscenza diretta: «La nascita è distrutta, la santa vita è stata vissuta, quel che doveva
essere fatto è stato fatto, non ci sarà altra rinascita». E il
venerabile Nanda divenne uno degli Arahant.


Quando la notte fu ben avanzata, una divinità di straordinaria bellezza
e che illuminava tutto il Boschetto di Jeta, andò dal Beato e, dopo
avergli prestato omaggio, si mise in piedi da un lato.
La divinità disse:


«Signore, il venerabile Nanda, il fratellastro del Beato, figlio della
sorella di sua madre, ha lui stesso realizzato la conoscenza diretta, e
qui e ora è entrato e dimora nella liberazione della mente e nella
liberazione per mezzo della comprensione, che è priva delle
contaminazioni per l’esaurimento delle contaminazioni». E anche il Beato
conosceva questo fatto.


Alla fine della notte il venerabile Nanda andò dal Beato e disse:
«Signore, benché il Beato mi abbia garantito che avrei ottenuto
cinquecento ninfe dai piedi di colomba, io lo libero da quella
promessa».


«Avevo già letto la tua mente con la mia mente, Nanda. Anche delle
divinità me l’hanno detto. Così, quando il tuo cuore fu liberato dalle
contaminazioni, io ero già libero dalla mia promessa». E conoscendo il
significato di ciò, il Beato esclamò queste parole:


\begin{quote}
Quando un bhikkhu ha superato la palude \\
e frantumato la spina dei desideri sensoriali \\
e raggiunto la distruzione dell’illusione, \\
piaceri e dolori non lo scuoteranno più.
\end{quote}

\emph{Ud. 3:2}


\narrator{Primo narratore.} La successiva stagione delle piogge, la quinta, fu
trascorsa a Vesālī, la capitale del Videha, un territorio collocato a
sud-est di Kosala e sulla riva occidentale del Gange. Era una
confederazione retta da un’oligarchia, non da una monarchia.


\narrator{Secondo narratore.} Nei mesi che seguirono, il re Suddhodana cadde malato
e morì come Arahant. Il Buddha visitò nuovamente la sua città natale.


\voice{Seconda voce.} Avvenne questo. Il Buddha, il Beato, stava vivendo tra i
Sakya nel Parco di Nigrodha a Kapilavatthu. Mahāpajāpatī Gotamī si recò
da lui. Gli prestò omaggio e si mise in piedi da un lato. Allora lei
disse: «Signore, sarebbe cosa buona se le donne potessero ottenere di
abbandonare la casa e la vita famigliare per la vita religiosa nel
Dhamma e nella Disciplina dichiarate dal Perfetto».


«Basta così, Gotamī, non chiedere che le donne ottengano di abbandonare
la casa e la vita famigliare per la vita religiosa nel Dhamma e nella
Disciplina dichiarate dal Perfetto».


Lei lo chiese per una seconda e per una terza volta, ma ottenne un
rifiuto. Allora pensò: «Il Beato non lo consente». Divenne triste e
infelice. Prestò omaggio al Beato e andò via, girandogli a destra.


Ora, allorché il Beato era rimasto a Kapilavatthu per tutto il tempo che
volle, si mise in viaggio per tappe verso Vesālī. Quando infine vi
giunse, andò a vivere nel Salone con il Tetto Aguzzo nella Grande
Foresta.


Nel frattempo Mahāpajāpatī Gotamī si era tagliata i capelli e aveva
indossato la veste ocra. Con un certo numero di donne Sakya partì per
Vesālī. All’arrivo, andò nel Salone con il Tetto Aguzzo nella Grande
Foresta e si mise in piedi fuori dalla veranda. I suoi piedi erano
gonfi, le sue membra coperte di polvere ed era triste e infelice,
singhiozzava e le lacrime le scendevano sul volto. Mentre stava in piedi
in questo modo, il venerabile Ānanda la vide. Le chiese: «Gotamī, perché
stai così, fuori dalla veranda?».


«Venerabile Ānanda, è perché il Beato non consente che le donne
ottengano di abbandonare la casa e la vita famigliare per la vita
religiosa nel Dhamma e nella Disciplina dichiarate dal Perfetto».


«Gotamī, aspetta qui fino a quando io stesso non lo avrò chiesto al
Beato». Il venerabile Ānanda andò dal Beato, gli raccontò l’accaduto e
poi disse: «Signore, sarebbe cosa buona se le donne potessero ottenere
di abbandonare la casa e la vita famigliare per la vita religiosa nel
Dhamma e nella Disciplina dichiarate dal Perfetto».


«Basta così, Ānanda, non chiedere che le donne ottengano di abbandonare
la casa e la vita famigliare per la vita religiosa nel Dhamma e nella
Disciplina dichiarate dal Perfetto».


Lui lo chiese per una seconda e per una terza volta, ma ottenne un
rifiuto. Allora pensò: «Il Beato non lo consente. E se io lo chiedessi
al Beato in un altro modo?». Allora disse: «Signore, le donne sono in
grado, dopo aver abbandonato la casa e la vita famigliare per la vita
religiosa nel Dhamma e nella Disciplina dichiarate dal Perfetto, di
realizzare il frutto di Chi è Entrato nella Corrente o di Chi Torna una
Sola Volta o di Chi è Senza Ritorno o la condizione di Arahant?».


«Lo sono, Ānanda».\footnote{Almeno due dei discorsi più esoterici del \emph{Sutta Piṭaka} (M. 44 and S. 44:1) furono pronunciati da bhikkhunī. Un certo numero di donne si distinsero per particolari virtù (A. 1:14) e c’è una collezione di versi pronunciati da loro allorché raggiunsero la condizione di Arahant. - NDT. Si veda ora \emph{Therīgāthā. Canti spirituali della monache buddhiste}, a cura di A.S. COMBA, Raleigh 2016; per una scelta di brani di P.FILIPPANI-RONCONI, cf. \emph{Canone buddhista. Discorsi brevi}, Torino 2004, pp. 695-724.}


«Se è così, Signore, poiché Mahāpajāpatī Gotamī è stata di grande aiuto
al Beato allorché, come sorella della madre, gli è stata nutrice, madre
adottiva, gli ha dato il latte, ha allattato il Beato quando sua madre
morì. Dal momento che è così, Signore, sarebbe cosa buona se le donne
potessero ottenere di abbandonare la casa e la vita famigliare per la
vita religiosa».


«Ānanda, se Mahāpajāpatī Gotamī accetta otto punti capitali, questo
significherà la sua piena ammissione. Questi sono gli otto punti. Una
bhikkhuṇī che è stata ammessa anche da cento anni deve porgere omaggio,
alzarsi, rendere onore reverenziale e salutare in modo rispettoso un
bhikkhu che è stato ammesso in quello stesso giorno. Una bhikkhuṇī non
deve trascorrere una stagione delle piogge in un luogo in cui non ci
siano bhikkhu. Ogni mezzo mese una bhikkhuṇī deve attendersi due cose
dal Saṅgha dei bhikkhu: l’incontro per il giorno dell’osservanza,
l’\emph{Uposatha}, ogni mezzo mese, e una visita per l’esortazione. Alla fine
delle piogge una bhikkhuṇī deve prestarsi a essere ripresa da entrambi i
Saṅgha a riguardo di tre argomenti, ossia se qualcosa d’improprio nel
suo comportamento è stato visto, sentito o sospettato. Quando una
bhikkhuṇī si è resa responsabile di una grave offesa, deve fare
penitenza di fronte a entrambi i Saṅgha. Una persona in prova che chiede
l’ammissione, deve chiederla a entrambi i Saṅgha dopo essersi addestrata
nei sei precedenti punti per due anni. Per nessuna ragione una bhikkhuṇī
deve trovare difetti o maltrattare un bhikkhu. Da oggi in poi non è
consentito alle bhikkhuṇī di fare discorsi ai bhikkhu, mentre è
consentito ai bhikkhu di fare discorsi alle bhikkhuṇī. Queste otto cose
devono essere onorate, rispettate, riverite e venerate e non si deve a
esse trasgredire per tutto il tempo che dura la vita. Se Mahāpajāpatī
Gotamī accetta questi otto punti capitali, questo significherà la sua
piena ammissione».


Quando il venerabile Ānanda ebbe imparato questi otto punti capitali dal
Beato, andò da Mahāpajāpatī Gotamī e le comunicò quel che il Beato aveva
detto.


«Venerabile Ānanda, se una donna – o un uomo – giovane, giovanile,
appassionata di ornamenti, con la testa lavata, ottenesse una ghirlanda
di fiori di loto, di gelsomini o di rose, la accetterebbe con entrambe
le mani e se la metterebbe sul capo. Allo stesso modo, io accetto questi
otto punti capitali per non trasgredirli finché dura la mia vita».


Allora il venerabile Ānanda tornò dal Beato e gli disse: «Signore,
Mahāpajāpatī Gotamī ha accettato gli otto punti capitali. Ora ha la
piena ammissione».


«Ānanda, se le donne non avessero ottenuto di abbandonare la casa e la
vita famigliare per la vita religiosa nel Dhamma e nella Disciplina
dichiarate dal Perfetto, la santa vita sarebbe durata a lungo, la santa
vita sarebbe durata un migliaio di anni. Ora, però, poiché le donne
l’hanno ottenuto, la santa vita non durerà a lungo, la santa vita durerà
solo cinquecento anni».


«Proprio come le stirpi con molte donne e pochi uomini vanno facilmente
in rovina a causa di ladri e banditi, allo stesso modo il Dhamma e la
Disciplina nei quali le donne ottengono di abbandonare la casa per la
vita religiosa non durano a lungo. Proprio come quando quella piaga
chiamata muffa grigia cade su un campo di riso in maturazione e quel
campo di riso in maturazione non dura a lungo, proprio come quando
quella piaga chiamata ruggine rossa cade su un campo di canne da
zucchero in maturazione e quel campo di canne da zucchero in maturazione
non dura a lungo, allo stesso modo il Dhamma e la Disciplina nei quali
le donne ottengono di abbandonare la casa per la vita religiosa non
durano a lungo. Così come un uomo costruisce in anticipo un argine per
far sì che l’acqua di un grande bacino non causi un’inondazione, io ho
resi noti in anticipo questi otto punti capitali che le bhikkhuṇī non
devono trasgredire finché dura la loro vita».


\emph{Vin. Cv. 10:1; A. 8:51}


\narrator{Secondo narratore.} Quando lei in seguito chiese istruzioni per le donne
Sakya che l’avevano accompagnata, il Buddha ordinò che i bhikkhu
avrebbero dovuto dare loro la piena ammissione come bhikkhuṇī. Le
bhikkhuṇī, ottenuta la piena ammissione, reclamarono allora che, a
differenza di loro, Mahāpajāpatī non aveva ottenuto la piena ammissione.
Mediante l’Anziano Ānanda lei si appellò al Buddha, che risolse la
controversia ripetendo che nel suo caso l’accettazione degli otto punti
rappresentava la piena ammissione. Poi si recò ancora dall’Anziano
Ānanda chiedendogli che il Buddha consentisse a bhikkhu e bhikkhuṇī di
prestare omaggio agli anziani indipendentemente da quale delle due
comunità appartenessero. Il Buddha rispose che nessun bhikkhu doveva
prestare omaggio a una bhikkhuṇī.


\voice{Seconda voce.} Un’altra volta Mahāpajāpatī Gotamī andò dal Beato. Dopo
avergli prestato omaggio si mise in piedi da un lato e disse: «Signore,
sarebbe bene che il Beato m’istruisse brevemente, in modo che dopo aver
ascoltato il Dhamma dal Beato, io possa dimorare sola, ritirata,
diligente, ardente e dotata di autocontrollo».


«Gotamī, quelle cose di cui tu sai: “Queste cose conducono alla
passione, non alla diminuzione della passione. All’attaccamento, non
all’assenza di attaccamento. All’accumulo di \emph{kamma} per la rinascita,
non all’assenza di accumulo. All’ambizione, non alla modestia. A
sentirsi scontenti, non a sentirsi appagati. A voler stare in compagnia,
non alla solitudine. All’indolenza, non all’energico vigore. Alla
lussuria, non alla frugalità”. A proposito di queste cose puoi
certamente dire: “Questo non è il Dhamma, questo non è la Disciplina,
questo non è l’insegnamento del Maestro”. Però, quelle cose di cui tu
sai: “Queste cose conducono alla diminuzione della passione, non alla
passione. All’assenza di attaccamento, non all’attaccamento. All’assenza
di accumulo di \emph{kamma} per la rinascita, non all’accumulo. Alla
modestia, non all’ambizione. A sentirsi appagati, non a sentirsi
scontenti. Alla solitudine, non a voler stare in compagnia. All’energico
vigore, non all’indolenza. Alla frugalità, non alla lussuria”. A
proposito di queste cose puoi certamente dire: “Questo è il Dhamma,
questo è la Disciplina, questo è l’insegnamento del Maestro”».


\emph{Vin. Cv. 10:5; A. 8:53}


