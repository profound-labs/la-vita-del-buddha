\chapter{Dopo l’Illuminazione}

\voice{Prima voce.} Così ho udito. Una volta, quando il Beato aveva appena
ottenuto l’Illuminazione, si trovava a Uruvelā presso le sponde del
fiume Nerañjarā ai piedi dell’albero della bodhi, l’albero
dell’Illuminazione. Allora il Beato si mise a sedere ai piedi
dell’albero della bodhi per sette giorni senza interruzione, provando la
beatitudine della Liberazione.


Al termine dei sette giorni emerse da quella concentrazione, e nella
prima veglia della notte la sua mente fu occupata dalla genesi
interdipendente in ordine diretto, in questo modo: «Quello giunge
all’esistenza quando c’è questo; quello sorge con il sorgere di questo.
Vale a dire: l’ignoranza è la condizione che fa giungere all’esistenza
le formazioni mentali; con le formazioni mentali quale condizione, la
coscienza; con la coscienza quale condizione, nome-e-forma; con
nome-e-forma quale condizione, la sestuplice base; con la sestuplice
base quale condizione, il contatto; con il contatto quale condizione, la
sensazione; con la sensazione quale condizione, la brama; con la brama
quale condizione, l’attaccamento; con l’attaccamento quale condizione,
il divenire; con il divenire quale condizione, la nascita; con la
nascita quale condizione giungono all’esistenza l’invecchiamento e la
morte, e anche l’afflizione, il lamento, il dolore, il dispiacere e la
disperazione. Così ha origine tutto questo aggregato di sofferenza».


Conoscendo il significato di ciò, il Beato esclamò queste parole:


\begin{quote}
Quando le cose sono del tutto manifeste \\
all’ardente brāhmaṇa\footnote{Vi è un costante gioco di parole, ammesso che quest’espressione sia adatta, tra i termini “brāhmaṇa” (casta divina, un eremita, un divino sacerdote), \emph{brahma} (divino, celeste, perfetto) e \emph{Brahmā} (divinità, Alta Divinità, o divinità al di là degli dèi dei sei paradisi sensoriali). Il sacerdozio dei brāhmaṇa proviene da questa stessa casta, per la quale si rivendica un legame particolare con \emph{Brahmā}, ed è questo che può giustificare la traduzione “divino”. Di solito la parola non viene tradotta. Altri termini che riecheggiano questi significati sono le Divine Dimore (\emph{brahmāvihara}: \hyperlink{cap-10-Il-periodo-di-mezzo#pag200}{}) della compassione amorevole, ecc., la “santa” o “divina vita” (\emph{brahmacariya}) o “pura condotta”, che è tale in virtù della “divina” caratteristica della castità, il “divino veicolo” (\emph{brahmayāna}: \hyperlink{cap-12-La-Dottrina#pag281}{}), e così via.} che pratica la meditazione \\
tutti i suoi dubbi svaniscono, perché egli sa \\
che ogni cosa deve avere la sua causa.
\end{quote}

Nella seconda veglia della notte la sua mente fu occupata dalla genesi
interdipendente in ordine inverso, in questo modo: «Quello non giunge
all’esistenza quando non c’è questo; quello cessa con la cessazione di
questo. Vale a dire: con la cessazione dell’ignoranza c’è la cessazione
delle formazioni mentali; con la cessazione delle formazioni mentali, la
cessazione della coscienza; con la cessazione della coscienza, la
cessazione di nome-e-forma; con la cessazione di nome-e-forma, la
cessazione della sestuplice base; con la cessazione della sestuplice
base, la cessazione del contatto; con la cessazione del contatto, la
cessazione della sensazione; con la cessazione della sensazione, la
cessazione della brama; con la cessazione della brama, la cessazione
dell’attaccamento; con la cessazione dell’attaccamento, la cessazione
del divenire; con la cessazione del divenire, la cessazione della
nascita; con la cessazione della nascita, la cessazione
dell’invecchiamento e della morte, e anche dell’afflizione, del lamento,
del dolore, del dispiacere e della disperazione. Così c’è la cessazione
di tutto questo aggregato di sofferenza».


Conoscendo il significato di ciò, il Beato esclamò queste parole:


\begin{quote}
Quando le cose sono del tutto manifeste \\
all’ardente brāhmaṇa che pratica la meditazione \\
tutti i suoi dubbi svaniscono, perché egli sa \\
come giungono alla fine i fenomeni condizionati.
\end{quote}

Nella terza veglia della notte la sua mente fu occupata dalla genesi
interdipendente in ordine diretto e inverso, in questo modo: «Quello
giunge all’esistenza quando c’è questo; quello sorge con il sorgere di
questo. Vale a dire: l’ignoranza è la condizione che fa giungere
all’esistenza le formazioni mentali; con le formazioni mentali quale
condizione, la coscienza; …​ con la nascita quale condizione giungono
all’esistenza l’invecchiamento e la morte, e anche l’afflizione, il
lamento, il dolore, il dispiacere e la disperazione. Così ha origine
tutto questo aggregato di sofferenza. Con la cessazione dell’ignoranza
c’è la cessazione delle formazioni mentali; con la cessazione delle
formazioni mentali, la cessazione della coscienza; …​ con la cessazione
della nascita, la cessazione dell’invecchiamento e della morte, e anche
dell’afflizione, del lamento, del dolore, del dispiacere e della
disperazione. Così c’è la cessazione di tutto questo aggregato di
sofferenza».


Conoscendo il significato di ciò, il Beato esclamò queste parole:


\begin{quote}
Quando le cose sono del tutto manifeste \\
all’ardente brāhmaṇa che pratica la meditazione \\
ecco, come il sole che illumina il cielo \\
egli respinge le orde di Māra.
\end{quote}

\suttaRef{Ud. 1:1-3; cf. Vin. Mv. 1:1}


Al termine dei sette giorni,\footnote{La collocazione di questo e del successivo episodio in tale punto è indicata dagli stessi testi. Anche il \emph{Mālālaṅkāravatthu} inserisce qui la tentazione delle figlie di Māra. Ācariya Buddhaghosa, però, ne parla in relazione al primo anno dopo l’Illuminazione (si veda il \hyperlink{cap-04-La-diffusione-del-Dhamma#pag70}{}). Un altro episodio, qui non incluso, di alcuni brāhmaṇa che rimproverano il Buddha per non aver prestato loro omaggio (\hyperlink{cap-09-La-fine-del-primo-ventennio#pag137}{}), è correlato ad A. 4:22.} dopo essere emerso da
quella concentrazione, il Beato esaminò il mondo con l’occhio di un
Illuminato. Quando lo fece, vide gli esseri che bruciavano per molti
generi di fiamme ed erano consumati da molte febbri generate dalla
brama, dall’odio e dall’illusione.


Conoscendo il significato di ciò, egli esclamò queste parole:


\begin{quote}
Il mondo è angosciato in quanto esposto al contatto, \\
perfino quel che il mondo chiama “io” è nei fatti nocivo, \\
e non importa che cosa concepisca (i concetti dell’io) \\
perché la cosa è sempre diversa da quello (che esso concepisce). \\
Il mondo, i cui esseri in altro si stanno trasformando, \\
è destinato all’esistenza, esposto all’esistenza, \\
apprezza solo l’esistenza, \\
tuttavia ciò che apprezza reca solo paura, \\
e quel che si teme è dolore. \\
Questa santa vita è vissuta per abbandonare la
sofferenza.\footnote{Questi sono due versi difficili. È davvero necessario tradurre la parola \emph{bhava} più o meno coerentemente con “esistenza” piuttosto che con “divenire”. Gli “essenziali dell’esistenza” sono altrove spiegati come riferiti a tutte le componenti dell’esistenza, dai possessi personali oggettivi alle bramosie e attitudini soggettive.}
\end{quote}

«Qualsiasi monaco o brāhmaṇa che abbia descritto la liberazione
dall’esistenza come avvenuta per mezzo dell’amore dell’esistenza,
nessuno di loro, questo dico, si è liberato dall’esistenza. E qualsiasi
monaco o brāhmaṇa che abbia descritto la via d’uscita dall’esistenza
come avvenuta per mezzo dell’amore della non-esistenza, nessuno di loro,
questo dico, si è liberato dall’esistenza. Per mezzo di quel che è
essenziale all’esistenza, sorge la sofferenza. Quando tutti gli
attaccamenti sono esauriti, la sofferenza non c’è più».


\begin{quote}
Guarda questo grande mondo: \\
gli esseri esposti all’ignoranza lo assaporano, \\
non si liberano mai dell’esistenza. \\
Quale che sia il tipo di esistenza, in ogni modo, in qualsiasi luogo, \\
tutto è impermanente, \\
infestato dal dolore e soggetto al cambiamento. \\
Perciò, un uomo che lo vede così com’è \\
abbandona la brama per l’esistenza, \\
non è attratto dalla non-esistenza. \\
Il dissolversi senza residuo, il cessare,
l’Estinzione\footnote{“Estinzione” e “Nibbāna” sono ovunque utilizzati in modo intercambiabile. “Estinzione” deve essere intesa come estinzione del fuoco (S. 35:28, cit. nel \hyperlink{cap-04-La-diffusione-del-Dhamma#pag73}{}) della bramosia, dell’odio e dell’illusione, e delle loro conseguenze. Non deve essere intesa per significare l’“estinzione di una persona vivente” (si veda il \hyperlink{cap-11-La-persona#pag226}{}). La moderna etimologia fa derivare la parola \emph{nibbāna} (sanscrito: \emph{nirvāṇa}) dal prefisso negativo \emph{ni}(r) cui si aggiunge la radice vā (soffiare), con il senso di “cessazione del soffio vitale”. Il significato originario fu probabilmente estinzione di un fuoco per la cessazione del soffio di un mantice, ad esempio il fuoco di un fabbro. Pare che tale significato sia poi stato applicato all’estinzione del fuoco mediante qualsiasi mezzo, ad esempio l’esaurimento della fiamma di una lampada (\emph{nibbāyati}: M. 140; \emph{nibbanti}: Sn. 2:1, v. 14). Per via analogica ciò fu esteso all’estinzione della brama e al riposo, del tutto raggiunti da un Arahant durante la vita. Alla sua morte fisica la processualità legata ai cinque aggregati si dissolverà senza essere rinnovata. Nibbāna è un termine erroneamente identificato come “estinzione di un sé esistente” e, allo stesso modo, della perpetuazione del sé (\hyperlink{cap-12-La-Dottrina#pag254}{}).} \\
con la fine assoluta di ogni brama. \\
Quando un bhikkhu raggiunge il Nibbāna \\
mediante il non-attaccamento, \\
per lui non ci sarà altra esistenza. \\
Māra è sconfitto, vinta è la battaglia, \\
quando uno come lui ha oltrepassato ogni esistenza.
\end{quote}

\suttaRef{Ud. 3:10}


\voice{Seconda voce.} Avvenne anche questo allorché al termine dei sette giorni
il Beato emerse da quella concentrazione e dai piedi dell’albero della
bodhi si recò ai piedi dell’albero \emph{ajapāla nigrodha}, il baniano del
guardiano delle greggi di capre. Si mise a sedere ai piedi dell’\emph{ajapāla
nigrodha} per sette giorni senza interruzione, provando la beatitudine
della Liberazione.


Allora uno della casta dei brāhmaṇa, dell’altezzosa discendenza degli
Huhuṅka, andò dal Beato e scambiò con lui dei saluti. Quando furono
terminati i formali doveri di reciproca cortesia, si mise in piedi da un
lato e disse: «Chi è un brāhmaṇa, Maestro Gotama? Quali sono le cose che
fanno un brāhmaṇa?».


Conoscendo il significato di ciò, il Beato esclamò queste parole:


\begin{quote}
Il brāhmaṇa che ha vinto le cose malvagie, \\
non altezzoso, privo di contaminazioni e dotato di autocotrollo, \\
con perfetta conoscenza e che vive la vita brāhmaṇica, \\
può a ragione utilizzare la parola “brāhmaṇa”, \\
se non è fiero di nulla che sta nel mondo.
\end{quote}

\suttaRef{Vin. Mv. 1:2; cf. Ud. 1:4}


Inoltre, avvenne pure questo allorché al termine dei sette giorni il
Beato emerse da quella concentrazione e dai piedi dell’\emph{ajapāla
nigrodha} si recò ai piedi dell’albero di Mucalinda.


In quell’occasione ci fu una grande tempesta fuori stagione, con sette
giorni di pioggia, di venti freddi e oscurità. Allora Mucalinda, il
nāga, il reale serpente, uscì dal suo regno. Avvolse sette volte il
corpo del Beato nelle sue spire, e rimase lì, con il suo grande
cappuccio allargato sopra il capo del Beato, pensando: «Facciamo in modo
che il Beato non senta né il freddo né il caldo, e neanche il tocco dei
tafani, delle zanzare, del vento, del sole e delle creature
striscianti».


Al termine dei sette giorni Mucalinda vide che il cielo era luminoso e
senza nuvole. Egli svolse le sue spire dal corpo del Beato. Fece allora
svanire la sua stessa forma, assunse quella di un giovane brāhmaṇa e si
mise in piedi davanti al Beato con le mani giunte in alto in segno di
reverenza.


Conoscendo il significato di ciò, il Beato esclamò queste parole:


\begin{quote}
L’isolamento è felicità per chi è appagato, \\
per chi ha imparato il Dhamma, e ha visto. \\
La cordialità nei riguardi del mondo è felicità \\
per lui, che è paziente con gli esseri viventi. \\
Disinteresse per il mondo è felicità \\
per lui che ha superato il desiderio sensoriale. \\
Vincere però l’orgoglio dell’“io sono” \\
questa è la felicità più grande di tutte.
\end{quote}

\suttaRef{Vin. Mv. 1:3; cf. Ud. 2:1}


Una volta, quando il Beato emerse da quella concentrazione dai piedi
dell’albero di Mucalinda si recò ai piedi dell’albero \emph{rājāyatana} per
sette giorni senza interruzione, provando la beatitudine della
Liberazione.


\label{pag41}In quell’occasione due mercanti, Tapussa e Bhalluka, stavano viaggiando
sulla strada che viene da Ukkalā. Una divinità, che in una vita passata
era stata una loro parente, disse loro: «Signori, c’è questo Beato che,
da poco illuminatosi, vive alle radici dell’albero \emph{rājāyatana}. Andate
a prestargli omaggio e offritegli un dolce di riso e del miele. Questo
vi porterà benessere e felicità».


Così, costoro portarono un dolce di riso e del miele al Beato, e, dopo
avergli prestato omaggio, si misero in piedi da un lato. Poi dissero:
«Signore, che il Beato accetti questo dolce di riso e questo miele, così
che ciò possa portarci benessere e felicità».


Il Beato pensò: «Gli Esseri Perfetti non accettano cibo direttamente
nelle loro mani. In qual modo potrei accettare questo dolce di riso e
questo miele?». Allora i Quattro Divini Sovrani, consapevoli nelle loro
menti del pensiero del Beato, portarono quattro ciotole di cristallo dai
quattro punti cardinali: «Signore, che il Beato accetti il dolce di riso
e il miele in queste ciotole».


Il Beato accettò il dolce di riso e il miele in una delle nuove ciotole
di cristallo e, dopo averlo fatto, mangiò. Allora i mercanti, Tapussa e
Bhalluka, dissero: «Noi prendiamo rifugio nel Beato e nel Dhamma. Da
oggi che il Beato ci consideri suoi seguaci che hanno preso rifugio in
lui per tutto il tempo che durerà il loro respiro».


Poiché costoro furono i primi seguaci al mondo, essi presero solo due
rifugi.


\suttaRef{Vin. Mv. 1:4}


\voice{Seconda voce.} Una volta, inoltre, alla fine dei sette giorni il Beato
emerse da quella concentrazione e dai piedi dell’albero \emph{rājāyatana} si
recò all’\emph{ajapāla nigrodha}, l’albero di baniano del guardiano di capre.


\voice{Prima voce.} Mentre il Beato era in ritiro da solo sorse in lui questo
pensiero: «Ci sono cinque facoltà spirituali che, se mantenute in essere
e sviluppate, sfociano in Ciò Che Non Muore, raggiungono Ciò Che Non
Muore e terminano in Ciò Che Non Muore. Quali cinque? Sono le facoltà
della fede, dell’energia, della consapevolezza, della concentrazione e
della comprensione».


\label{pag45}Allora Brahmā Sahampati nella sua mente fu consapevole del pensiero
sorto nella mente del Beato, e con la stessa velocità con cui un uomo
forte distende il suo braccio piegato o piega il suo braccio disteso,
scomparve dal mondo di Brahmā e apparve di fronte a lui. Sistemò la
veste superiore su una spalla e, alzando le mani giunte verso il Beato,
disse: «Così è, Beato, così è, Sublime. Quando queste cinque facoltà
sono mantenute in essere e sviluppate, sfociano in Ciò Che Non Muore,
raggiungono Ciò Che Non Muore e terminano in Ciò Che Non Muore. Un
tempo, Signore, vivevo la santa vita sotto il Buddha Kassapa. Allora ero
conosciuto come il bhikkhu Sahaka. Fu mantenendo in essere e sviluppando
queste cinque facoltà che la mia bramosia per i desideri sensoriali
svanì e che alla dissoluzione del corpo, dopo la morte, ricomparvi in
una destinazione felice, nel mondo di Brahmā. Là sono noto come Brahmā
Sahampati. Così è, Beato, cosi è, Sublime. Conosco e capisco come queste
cinque facoltà, quando sono mantenute in essere e sviluppate, sfociano
in Ciò Che Non Muore, raggiungono Ciò Che Non Muore e terminano in Ciò
Che Non Muore».


\suttaRef{S. 48:57}


Ora, mentre il Beato era in ritiro da solo sorse in lui questo pensiero:
«Questo sentiero, ossia i quattro fondamenti della consapevolezza, è un
sentiero che va verso una sola direzione:\footnote{Invece di «che va verso una sola direzione», il termine composto \emph{ekāyana} è di solito tradotto con «l’unica via»; si veda però l’uso di tale termine in M. 12.} verso la
purificazione degli esseri, verso il superamento dell’afflizione e del
lamento, verso la scomparsa del dolore e del dispiacere, verso
l’ottenimento del vero scopo, verso la realizzazione del Nibbāna. Quali
quattro? Un bhikkhu dovrebbe dimorare contemplando il corpo come corpo,
ardente, pienamente presente e consapevole, avendo messo da parte
bramosia e afflizione per il mondo. Oppure dovrebbe dimorare
contemplando le sensazioni come sensazioni, ardente, pienamente presente
e consapevole, avendo messo da parte bramosia e afflizione per il mondo.
Oppure dovrebbe dimorare contemplando la coscienza come coscienza,
ardente, pienamente presente e consapevole, avendo messo da parte
bramosia e afflizione per il mondo. Oppure dovrebbe dimorare
contemplando gli oggetti mentali come oggetti mentali, ardente,
pienamente presente e consapevole, avendo messo da parte bramosia e
afflizione per il mondo».


Giunse allora Brahmā Sahampati, che espresse la sua approvazione come
prima.


\suttaRef{S. 47:18, 43}


Ora, mentre il Beato era in ritiro da solo sorse in lui questo pensiero:
«Sono libero da quella penitenza, sono del tutto libero da quell’inutile
penitenza. Assolutamente certo e consapevole, ho ottenuto
l’Illuminazione».


Allora Māra il Malvagio nella sua mente fu consapevole del pensiero
sorto nella mente del Beato, andò da lui e pronunciò queste strofe:


\begin{quote}
«Tu hai abbandonato il sentiero dell’ascetismo \\
mediante il quale gli uomini purificano se stessi, \\
tu non sei puro, tu immagini di essere puro. \\
Il sentiero della purezza è lontano da te».
\end{quote}

Il Beato riconobbe Māra il Malvagio, e gli rispose con queste strofe:


\begin{quote}
«Conosco queste penitenze per ottenere Ciò Che Non Muore, \\
quale che sia il loro genere, sono vane \\
come i remi e il timone di una barca sulla terra ferma. \\
Ma è a causa dello sviluppo \\
di virtù, concentrazione, comprensione, \\
che ho raggiunto l’Illuminazione; e tu, \\
Sterminatore, ora sei stato sconfitto».
\end{quote}

Allora Māra il Malvagio seppe: «Il Beato mi conosce, il Sublime mi
conosce». Triste e deluso, subito sparì.


\suttaRef{S. 4:1}


Ora, mentre il Beato era in ritiro da solo sorse in lui questo pensiero:
«Chi non ha nulla da venerare e nessuno al quale obbedire vive infelice.
Dov’è qui però un monaco o un brāhmaṇa sotto il quale posso vivere,
onorandolo e rispettandolo?».


Allora pensò: «Potrei vivere sotto un altro monaco o brāhmaṇa e
rispettarlo per perfezionare un imperfetto codice di virtù, o un codice
di concentrazione, o un codice di comprensione, o un codice di
liberazione, o un codice di conoscenza e visione della liberazione. Non
vedo però in questo mondo con i suoi deva, con i suoi Māra e con le sue
divinità, in questa generazione con i suoi monaci e brāhmaṇa, con i suoi
principi e uomini, nessun monaco o brāhmaṇa in cui queste cose siano più
perfette che in me, sotto il quale potrei vivere, onorandolo e
rispettandolo. C’è però questo Dhamma scoperto da me. E se io vivessi
sotto questo Dhamma, onorandolo e rispettandolo?».


Allora Brahmā Sahampati nella sua mente fu consapevole del pensiero
sorto nella mente del Beato. Egli apparve di fronte al Beato: «Questo è
bene, Signore. I Beati dei tempi passati, realizzati e completamente
illuminati, vivevano sotto il Dhamma, onorandolo e rispettandolo. E
anche in futuro faranno nello stesso modo».


\suttaRef{S. 6:2; A. 4:21}


\voice{Seconda voce.} Ora, mentre il Beato era in ritiro da solo sorse in lui
questo pensiero: «Questo Dhamma che io ho conseguito è profondo e
difficile da vedere, difficile da scoprire. È la meta più serena,
superiore a tutte le altre, non raggiungibile con il solo raziocinio,
sottile, il saggio lo deve sperimentare personalmente. Questa
generazione però confida nell’attaccamento, apprezza l’attaccamento, si
delizia nell’attaccamento. Per una generazione come questa è difficile
vedere la verità, ossia la condizionalità specifica, la genesi
interdipendente. Ed è difficile vedere questa verità, ossia
l’acquietarsi di tutte le formazioni, la rinuncia agli essenziali
dell’esistenza, l’esaurimento della brama, il dissolversi dell’avidità,
la cessazione, il Nibbāna. Se io insegnassi il Dhamma, gli altri non mi
capirebbero, e questo sarebbe per me pesante e fastidioso».


A quel punto gli vennero in modo spontaneo in mente queste strofe, mai
udite prima:


\begin{quote}
Basta con l’insegnamento del Dhamma \\
che anche per me è stato difficile da raggiungere, \\
perché non sarà mai compreso \\
da coloro che vivono nella brama e nell’odio. \\
Gli uomini sono intrisi di bramosia, e chi è avvolto \\
da una nube di oscurità non vedrà mai \\
ciò che va contro la corrente, che è sottile, \\
profondo e difficile da vedere, astruso.
\end{quote}

Pensando questo, la sua mente inclinò verso l’inattività, il non
insegnamento del Dhamma.


Allora Brahmā Sahampati, che nella sua mente fu consapevole del pensiero
sorto nella mente del Beato, pensò: «Il mondo sarà perduto, il mondo
sarà del tutto perduto, perché la mente del Beato, realizzato e
completamente illuminato, inclina verso l’inattività, verso il non
insegnamento del Dhamma».


Così, con la stessa velocità con cui un uomo forte distende il suo
braccio piegato o piega il suo braccio disteso, Brahmā Sahampati
scomparve dal mondo di Brahmā e apparve di fronte al Beato. Sistemò la
veste su una spalla e, mettendo il ginocchio destro a terra e alzando le
mani giunte verso il Beato, disse: «Signore, che il Beato insegni il
Dhamma, che il Sublime insegni il Dhamma. Ci sono esseri che hanno solo
poca polvere negli occhi, saranno perduti se non ascoltano il Dhamma.
Alcuni di loro otterranno la conoscenza finale del Dhamma».


Dopo aver detto questo, Brahmā Sahampati aggiunse:


\begin{quote}
A Magadha fino ad ora è apparso \\
\emph{dhamma} impuro insegnato da uomini impuri: \\
apri i Cancelli di Ciò Che Non Muore: consenti loro di ascoltare \\
il Dhamma Immacolato. \\
Ascendi, o Saggio, la torre del Dhamma, \\
e, come vede la gente tutt’intorno \\
chi sta in piedi su una solida colonna di pietra, \\
sonda, o Saggio Privo di Dolore e Che Tutto Vede, \\
questa razza umana inghiottita da quel dolore \\
che nascita e vecchiaia portano con sé. \\
Sorgi, o Eroe, Vittorioso, Portatore di Conoscenza, \\
Libero da Ogni Debito, e vai per il mondo.


Proclama il Dhamma, perché alcuni, \\
o Beato, capiranno.
\end{quote}

Il Beato ascoltò la supplica di Brahmā Sahampati. Per compassione verso
gli esseri egli sondò il mondo con l’occhio di un Buddha. Come in uno
stagno di fiori di loto blu, rossi o bianchi, alcuni fiori di loto che
sono nati e cresciuti nell’acqua prosperano immersi nell’acqua senza
uscirne fuori, e altri che sono nati e cresciuti nell’acqua poggiano
sulla superficie dell’acqua, e altri ancora che sono nati e cresciuti
nell’acqua ne escono fuori e stanno ritti e puliti, senza essere bagnati
da essa, allo stesso modo egli vide esseri con poca polvere negli occhi
e con molta polvere negli occhi, con facoltà intense e facoltà spente,
con buone qualità e cattive qualità, ai quali è facile insegnare e
difficile insegnare, e altri che dimoravano vedendo paura e biasimo
nell’altro mondo. Quando ebbe visto questo, rispose:


\begin{quote}
Spalancati sono i portali di Ciò Che Non Muore. \\
Che abbiano fede coloro che ascoltano.\footnote{«Che abbiano fede coloro che ascoltano» (\emph{ye sotavanto pamuñcantu saddhaṃ}) è un passo molto controverso. Di solito viene reso con «Che coloro che ascoltano rinuncino alla loro fede». Questo significato, però, stride con lo spirito dell’insegnamento. Esso dipende anche dall’interpretazione della parola \emph{vissajjentu} (che il Commentario glossa con \emph{pamuñcantu}) come «fate che rinuncino», ma questa parola può anche significare «che loro diano» o «che loro impieghino». Così \emph{pamuñcantu}: «che loro mostrino, che loro mettano in evidenza». Che il Commentario intenda il passo in questo modo è confermato da quanto si legge alla fine del relativo paragrafo: «Lasciate che ognuno proponga la sua fede»: Comm. a M. 26), nel quale \emph{upanetu} parafrasa \emph{pamuñcantu}. L’espressione idiomatica ricorre in Sn. 1146, dove sfortunatamente è stata talvolta confusa con un’altra espressione idiomatica, \emph{saddhā-vimutto}: “liberazione mediante la fede”.} Se pensavo di \\
non insegnare il sublime Dhamma che conosco, \\
era perché m’importunava pensare all’insegnamento.
\end{quote}

Allora Brahmā Sahampati pensò: «Ho reso possibile che il Dhamma sia
insegnato dal Beato». E dopo avergli prestato omaggio, girandogli a
destra, subito scomparve.


\suttaRef{Vin. Mv. 1:5; cf. M. 26 e 85; S. 6:1}


Il Beato pensò: «A chi per primo insegnerò il Dhamma? Chi comprenderà
subito questo Dhamma?». Poi pensò: «Āḷāra Kālāma è saggio, sapiente e
acuto. Da lungo tempo ha poca polvere negli occhi. E se per primo
insegnassi il Dhamma a lui? Lo comprenderà subito».


Allora delle invisibili divinità dissero al Beato: «Signore, Āḷāra
Kālāma è morto sette giorni fa». E la conoscenza e la visione sorsero in
lui: «Āḷāra Kālāma è morto sette giorni fa». Pensò: «Quel che Āḷāra
Kālāma ha perduto è molto. Se avesse ascoltato questo Dhamma, lo avrebbe
subito compreso».


Il Beato pensò: «Uddaka Rāmaputta è saggio, sapiente e acuto. Da lungo
tempo ha poca polvere negli occhi. E se per primo insegnassi il Dhamma a
lui? Lo comprenderà subito».


Allora delle invisibili divinità dissero al Beato: «Signore, Uddaka
Rāmaputta è morto la scorsa notte». E la conoscenza e la visione sorsero
in lui: «Uddaka Rāmaputta è morto la scorsa notte». Pensò: «Quel che
Uddaka Rāmaputta ha perduto è molto. Se avesse ascoltato questo Dhamma,
lo avrebbe subito compreso».


Il Beato pensò: «A chi per primo insegnerò il Dhamma? Chi comprenderà
subito questo Dhamma?». Poi pensò: «I bhikkhu del gruppo dei cinque che
mi assistevano nel mio sforzo erano molto servizievoli. E se per primi
insegnassi il Dhamma a loro?». Pensò inoltre: «Dove vivono adesso i
bhikkhu del gruppo dei cinque?». E con l’occhio divino, che è purificato
e supera quello umano, vide che stavano vivendo a Benares, nel Parco
delle Gazzelle a Isipatana, nella Località dei Veggenti.


Il Beato restò a Uruvelā per tutto il tempo che volle, e poi partì per
recarsi a Benares per tappe.


Tra il luogo dell’Illuminazione e Gayā, il monaco Upaka lo vide per
strada. Disse: «Le tue facoltà sono rasserenate, amico. Il colore della
tua pelle è chiaro e luminoso. Sotto chi hai praticato la vita
religiosa? Chi è il tuo maestro? Quale Dhamma professi?».


Quando ciò fu detto, il Beato si rivolse al monaco Upaka in strofe:


\begin{quote}
Io sono Chi Tutto Trascende,\footnote{“Chi Tutto Trascende” (\emph{sabbābhibhū}): un derivato della radice \emph{bhū} (essere), nel senso di “al di là dell’esistenza” o “chi ha superato ogni esistenza”. \emph{Abhibhū}, che incontreremo di nuovo più avanti, è parafrasato da alcuni traduttori con “maestria” (come in \emph{abhibhāyatana}) o Conquistatore come epiteto di Mahā-brahmā. Può essere ritenuto come uno degli esempi dell’uso di un termine corrente da parte del Buddha, ma in un contesto che ne trasforma il significato.} un Onnisciente, \\
incontaminato dalle cose, rinunciando a tutto, \\
mediante la libertà della cessazione della brama. Ciò lo devo \\
alla mia stessa saggezza. A chi altri dovrei attribuire tutto questo? \\
Non ho alcun maestro, e uno simile a me \\
non esiste da nessuna parte in tutto il mondo \\
con tutti i suoi dèi, perché \\
persona a me omologa non c’è. \\
Io nel mondo sono il Maestro \\
senza pari, finanche realizzato, \\
e io solo sono completamente illuminato, \\
spento, i cui fuochi sono tutti estinti. \\
Io ora vado nella città di Kāsi \\
per mettere la Ruota del Dhamma \\
in moto: in un mondo bendato \\
io vado a rullare il Tamburo di Ciò Che Non Muore.


Secondo quel che dici, amico, tu sei un Vittorioso Universale.


I vittoriosi come me, Upaka, \\
sono coloro le cui contaminazioni sono del tutto esaurite. \\
Ho riportato la vittoria su ogni stato del male: \\
è per questo che io sono un Vittorioso.
\end{quote}

Quando ciò fu detto, il monaco Upaka commentò: «Così sia, amico».
Scrollando il capo, prese un sentiero secondario e se ne andò.


Viaggiando per tappe, il Beato giunse infine a Benares, nel Parco delle
Gazzelle a Isipatana, dove si trovavano i bhikkhu del gruppo dei cinque.
Da lontano videro che arrivava. Si misero allora d’accordo: «Amici, sta
arrivando il monaco Gotama, che è diventato autoindulgente, ha
rinunciato allo sforzo ed è tornato alla lussuria. Non dobbiamo
prestargli omaggio né alzarci in piedi per lui, e neanche ricevere la
sua ciotola e la veste superiore. Gli possiamo lasciare un posto a
sedere. Che sieda, se vuole».


Però, non appena il Beato si avvicinò, furono incapaci di prestare fede
al loro accordo. Uno gli andò incontro e prese la ciotola e la veste
superiore, un altro preparò un posto a sedere, un altro preparò
dell’acqua, uno sgabello e un asciugamano. Il Beato si mise a sedere nel
posto preparatogli e si lavò i piedi. Loro si rivolsero a lui
chiamandolo per nome e “amico”.


Quando ciò fu detto, lui disse loro: «Bhikkhu, non rivolgetevi al
Perfetto chiamandolo per nome e “amico”: il Perfetto è realizzato e
completamente illuminato. Ascoltate, bhikkhu, Ciò Che Non Muore è stato
raggiunto. Vi istruirò. Vi insegnerò il Dhamma. Praticando dopo essere
stati istruiti, realizzandolo voi stessi qui e ora per mezzo della
conoscenza diretta, entrerete e dimorerete in quella suprema meta della
santa vita per la quale gli uomini di famiglia giustamente lasciano la
loro casa per una vita priva di fissa dimora».


Allora i bhikkhu del gruppo dei cinque dissero: «Amico Gotama, quando
praticavi con disagi, privazioni e mortificazioni non hai ottenuto
alcuna caratteristica superiore alla condizione umana, degna della
conoscenza e della visione degli Esseri Nobili. Ora che sei
autoindulgente, hai rinunciato allo sforzo e sei tornato alla lussuria,
come puoi aver ottenuto tali caratteristiche?».


Allora il Beato disse al gruppo dei cinque: «Il Perfetto non è
autoindulgente, non ha rinunciato allo sforzo, non è tornato alla
lussuria. Il Perfetto è realizzato e completamente illuminato.
Ascoltate, bhikkhu, Ciò Che Non Muore è stato raggiunto. Vi istruirò. Vi
insegnerò il Dhamma. Praticando dopo essere stati istruiti,
realizzandolo voi stessi qui e ora per mezzo della conoscenza diretta,
entrerete e dimorerete in quella suprema meta della santa vita per la
quale gli uomini di famiglia giustamente lasciano la loro casa per una
vita priva di fissa dimora».


Una seconda volta i bhikkhu del gruppo dei cinque gli dissero la stessa
cosa, e una seconda volta egli rispose loro nella stessa maniera. Una
terza volta loro dissero la stessa cosa. Quando ciò fu detto, egli
chiese loro: «Bhikkhu, mi avete mai sentito parlare in questo modo in
precedenza?». «No, Signore».


«Il Perfetto è realizzato e completamente illuminato. Ascoltate,
bhikkhu, Ciò Che Non Muore è stato raggiunto. Vi istruirò. Vi insegnerò
il Dhamma. Praticando dopo essere stati istruiti, realizzandolo voi
stessi qui e ora per mezzo della conoscenza diretta, entrerete e
dimorerete in quella suprema meta della santa vita per la quale gli
uomini di famiglia giustamente lasciano la loro casa per una vita priva
di fissa dimora».


\suttaRef{Vin. Mv.1:6; cf. M. 26 e 85}


Il Beato riuscì a convincerli. Loro intesero il Beato, ascoltarono e
aprirono i loro cuori alla conoscenza. Allora il Beato si rivolse ai
bhikkhu del gruppo dei cinque in questo modo:


(\emph{La Messa in Moto della Ruota del Dhamma})


«Bhikkhu, ci sono questi due estremi che non dovrebbero essere coltivati
da chi lascia la propria casa. Quali due? La dedizione alla ricerca dei
desideri sensoriali, che è cosa bassa, grossolana, volgare, ignobile e
dannosa, e la dedizione all’automortificazione, che è dolorosa,
ignobile e dannosa. La Via di Mezzo scoperta dal Perfetto evita entrambi
questi estremi, dà la visione, dà la conoscenza e conduce alla pace,
alla conoscenza diretta, all’Illuminazione, al Nibbāna. E qual è questa
Via di Mezzo? È questo Nobile Ottuplice Sentiero, vale a dire: retta
visione, retta intenzione, retta parola, retta azione, retto modo di
vivere, retto sforzo, retta consapevolezza e retta concentrazione.
Questa è la Via di Mezzo scoperta dal Perfetto, che dà la visione, dà la
conoscenza, e conduce alla pace, alla conoscenza diretta,
all’Illuminazione, al Nibbāna».


«C’è questa nobile verità della sofferenza: la nascita è sofferenza,
l’invecchiamento è sofferenza, la malattia è sofferenza, la morte è
sofferenza, afflizione e lamento, dolore, dispiacere e disperazione sono
sofferenza, associarsi con quel che si detesta è sofferenza, separarsi
da quel che si ama è sofferenza, non ottenere quel che si vuole è
sofferenza. In breve, i cinque aggregati affetti
dall’attaccamento\footnote{Degli “aggregati affetti dall’attaccamento” (\emph{upādānakkhandha}) si tratta nel cap. 12.} sono sofferenza».


«C’è questa nobile verità dell’origine della sofferenza: è la brama, che
produce rinnovate esistenze, è accompagnata da diletto e lussuria,
diletto per questo e per quello. In altre parole, brama per desideri
sensoriali, brama di essere, brama di non-essere».


«C’è questa nobile verità della cessazione della sofferenza: è il
dissolversi e il cessare senza residuo, la rinuncia, l’abbandono, il
lasciar andare e il rifiuto di questa stessa brama».


«C’è questa nobile verità della via che conduce alla cessazione della
sofferenza: è questo Nobile Ottuplice Sentiero, vale a dire: retta
visione, retta intenzione, retta parola, retta azione, retto modo di
vivere, retto sforzo, retta consapevolezza e retta concentrazione».


«“C’è questa nobile verità della sofferenza”: questa fu l’intuizione, la
conoscenza, la comprensione, la visione, la luce che sorse in me su cose
mai udite prima. “Questa nobile verità deve essere penetrata conoscendo
pienamente la sofferenza”: questa fu l’intuizione, la conoscenza, la
comprensione, la visione, la luce che sorse in me su cose mai udite
prima. “Questa nobile verità è stata penetrata conoscendo pienamente la
sofferenza”: questa fu l’intuizione, la conoscenza, la comprensione, la
visione, la luce che sorse in me su cose mai udite prima».


«“C’è questa nobile verità dell’origine della sofferenza”: questa fu
l’intuizione, la conoscenza, la comprensione, la visione, la luce che
sorse in me su cose mai udite prima. “Questa nobile verità deve essere
penetrata abbandonando l’origine della sofferenza”: questa fu
l’intuizione, la conoscenza, la comprensione, la visione, la luce che
sorse in me su cose mai udite prima. “Questa nobile verità è stata
penetrata abbandonando l’origine della sofferenza”: questa fu
l’intuizione, la conoscenza, la comprensione, la visione, la luce che
sorse in me su cose mai udite prima».


«“C’è questa nobile verità della cessazione della sofferenza”: questa fu
l’intuizione, la conoscenza, la comprensione, la visione, la luce che
sorse in me su cose mai udite prima. “Questa nobile verità deve essere
penetrata realizzando la cessazione della sofferenza”: questa fu
l’intuizione, la conoscenza, la comprensione, la visione, la luce che
sorse in me su cose mai udite prima. “Questa nobile verità è stata
penetrata realizzando la cessazione della sofferenza”: questa fu
l’intuizione, la conoscenza, la comprensione, la visione, la luce che
sorse in me su cose mai udite prima».


«“C’è questa nobile verità della via che conduce alla cessazione della
sofferenza”: questa fu l’intuizione, la conoscenza, la comprensione, la
visione, la luce che sorse in me su cose mai udite prima. “Questa nobile
verità deve essere penetrata mantenendo in essere\footnote{\emph{bhāvetabbaṃ}: “deve essere coltivata, sviluppata” (Nyp.).} la
via che conduce alla cessazione della sofferenza”: questa fu
l’intuizione, la conoscenza, la comprensione, la visione, la luce che
sorse in me su cose mai udite prima. “Questa nobile verità è stata
penetrata mantenendo in essere la via che conduce alla cessazione della
sofferenza”: questa fu l’intuizione, la conoscenza, la comprensione, la
visione, la luce che sorse in me su cose mai udite prima».


«Finché la mia corretta conoscenza e visione di questi dodici aspetti –
in queste tre fasi di penetrazione di ognuna delle Quattro Nobili Verità
– non fu del tutto pura, non affermai di aver ottenuto la piena
Illuminazione in questo mondo con i suoi deva, con i suoi Māra e con le
sue divinità, in questa generazione con i suoi monaci e brāhmaṇa, con i
suoi principi e uomini. Però, appena la mia corretta conoscenza e
visione di questi dodici aspetti – in queste tre fasi di penetrazione di
ognuna delle Quattro Nobili Verità – fu del tutto pura, allora affermai
di aver ottenuto la piena illuminazione in questo mondo con i suoi deva,
con i suoi Māra e con le sue divinità, in questa generazione con i suoi
monaci e brāhmaṇa, con i suoi principi e uomini».


«La conoscenza e visione sorsero in me: “La liberazione del mio cuore è
certa, questa è l’ultima nascita, non ci saranno più rinnovate
esistenze”».


\suttaRef{Vin. Mv. 1:6; S. 56:11}


Ora, mentre questo discorso veniva pronunciato, la pura, immacolata
visione del Dhamma sorse nel venerabile Kondañña in questo modo: tutto
quel che sorge deve cessare.


E quando la Ruota del Dhamma fu messa in moto dal Beato, le divinità
della Terra esclamarono: «A Benares, nel Parco delle Gazzelle a
Isipatana, il Perfetto, realizzato e completamente illuminato, ha messo
in moto l’incomparabile Ruota del Dhamma, che non può essere fermata da
monaci o brāhmaṇa, da divinità, da Māra o da chiunque altro nel mondo».
Sentendo l’esclamazione delle divinità della Terra, le divinità del
paradiso dei Quattro Sovrani esclamarono: «A Benares …​». Le divinità
Tāvatiṃsa (le Trentatré Divinità) …​ le divinità Tusita (i Gioiosi) …​
le divinità Yāma (i Beati) …​ le divinità Nimmānarati (Coloro che si
deliziano nel creare) …​ le divinità Paranimmitavasavatti (Coloro che
detengono il potere sulle creazioni altrui) …​ le divinità del Seguito
di Brahmā esclamarono: «A Benares …​».


In quel minuto, in quel momento, in quell’istante, la notizia si propagò
fino al mondo di Brahmā. E questo sistema di diecimila mondi si scosse,
tremò e vacillò, mentre una luce grande e incommensurabile che superava
per splendore quella degli dèi apparve nel mondo.


Il Beato esclamò: «Kondañña conosce, Kondañña conosce!» E fu così che
quel venerabile ottenne il nome Aññāta Kondañña, Kondañña che conosce.


Allora Aññāta Kondañña, che aveva visto e raggiunto e trovato e
penetrato il Dhamma, che si era lasciato alle spalle ogni incertezza e i
cui dubbi erano svaniti, che aveva ottenuto una perfetta fiducia ed era
divenuto indipendente dagli altri nella Dispensazione del Maestro, disse
al Beato: «Signore, desidero abbracciare la vita religiosa e ricevere la
piena ammissione dal Beato».


«Vieni bhikkhu», disse il Beato. E aggiunse: «Il Dhamma è ben
proclamato. Vivi la santa vita per completare la fine della sofferenza».
E questa fu la piena ammissione.


Allora il Beato insegnò agli altri bhikkhu e li istruì con un discorso
di Dhamma. Quando lo fece, nel venerabile Vappa e nel venerabile
Bhaddiya sorse la pura, immacolata visione del Dhamma: tutto quel che
sorge deve cessare. Anche loro chiesero e ricevettero la piena
ammissione.


Così, vivendo del cibo portato che loro gli portavano, il Beato insegnò
agli altri bhikkhu e li istruì con un discorso di Dhamma. Tutti e sei
vissero del cibo che veniva portato da tre di loro. Allora nel
venerabile Mahānāma e nel venerabile Assaji sorse la pura, immacolata
visione del Dhamma, e anche loro chiesero e ricevettero la piena
ammissione.


Allora il Beato si rivolse ai bhikkhu in questo modo:


\suttaRef{Vin. Mv. 1:6}


(\emph{Il Discorso della Caratteristica del Non-Sé})


«Bhikkhu, la forma materiale è non-sé. Se la forma materiale fosse un
sé, questa forma materiale non condurrebbe all’afflizione, e si potrebbe
a essa ingiungere: “Che la mia forma materiale sia così, che la mia
forma materiale non sia così”. E siccome la forma materiale è non-sé,
essa conduce all’afflizione, e a essa non si può ingiungere: “Che la mia
forma materiale sia così, che la mia forma materiale non sia così”».


«La sensazione è non-sé …​».


«La percezione è non-sé …​».


«Le formazioni mentali sono non-sé …​».


«La coscienza è non-sé. Se la coscienza fosse un sé, questa coscienza
non condurrebbe all’afflizione, e si potrebbe a essa ingiungere: “Che la
mia coscienza sia così, che la mia coscienza non sia così”. E siccome la
coscienza è non-sé, essa conduce all’afflizione, e a essa non si può
ingiungere: “Che la mia coscienza sia così, che la mia coscienza non sia
così”».


«Che cosa ne pensate, bhikkhu, la forma materiale è permanente o
impermanente?». «Impermanente, Signore». «Ciò che è impermanente è
spiacevole o piacevole?». «Spiacevole, Signore». «A riguardo di ciò che
è impermanente, spiacevole e soggetto al cambiamento, è giusto dire:
“Questo è mio, questo è quel che io sono, questo è il mio sé?”». «No,
Signore».


«Che cosa ne pensate, bhikkhu, la sensazione è permanente o
impermanente? …​». «Che cosa ne pensate, bhikkhu, la percezione è
permanente o impermanente? …​». «Che cosa ne pensate, bhikkhu, le
formazioni mentali sono permanenti o impermanenti? …​».


«Che cosa ne pensate, bhikkhu, la coscienza è permanente o
impermanente?». «Impermanente, Signore». «Ciò che è impermanente è
spiacevole o piacevole?». «Spiacevole, Signore». «A riguardo di ciò che
è impermanente, spiacevole e soggetto al cambiamento, è giusto dire:
“Questo è mio, questo è quel che io sono, questo è il mio sé?”». «No,
Signore».


«Per questa ragione, bhikkhu, qualsiasi forma materiale, passata, futura
o presente, interna o esterna, grossolana o sottile, inferiore o
superiore, lontana o vicina, dovrebbe essere considerata come realmente
è per mezzo della retta comprensione in questo modo: “Questo non è mio,
questo non è quel che io sono, questo non è il mio sé”».


«Qualsiasi sensazione …​».


«Qualsiasi percezione …​».


«Qualsiasi formazione mentale …​».


«Qualsiasi coscienza, passata, futura o presente, interna o esterna,
grossolana o sottile, inferiore o superiore, lontana o vicina, dovrebbe
essere considerata come realmente è per mezzo della retta comprensione
in questo modo: “Questo non è mio, questo non è quel che io sono, questo
non è il mio sé”».


«Con questa comprensione, bhikkhu, un saggio nobile discepolo diventa
disincantato nei riguardi della forma materiale, diventa disincantato
nei riguardi della sensazione, diventa disincantato nei riguardi della
percezione, diventa disincantato nei riguardi delle formazioni mentali,
diventa disincantato nei riguardi della coscienza. Diventando
disincantato, la sua brama svanisce. Con lo svanire della brama, il suo
cuore è liberato. Quando il suo cuore è liberato, giunge la conoscenza:
“È liberato”. Egli comprende: “La nascita è distrutta, la santa vita è
stata vissuta, quel che doveva essere fatto è stato fatto, non ci sarà
altra rinascita”».


Questo è quel che il Beato disse. I bhikkhu del gruppo dei cinque erano
lieti, le sue parole li deliziarono. Ora, mentre questo discorso era
tenuto, i cuori dei bhikkhu del gruppo dei cinque furono liberati dalle
contaminazioni mediante il non-attaccamento. E allora ci furono sei
Arahant, sei esseri realizzati, nel mondo.


\suttaRef{Vin. Mv. 1:6; cf. S. 22:59}


