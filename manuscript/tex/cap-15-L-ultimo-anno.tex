\chapter{L'ultimo anno}

\narrator{Primo narratore.} Gli eventi che seguono avvennero in un solo anno, alla
fine del quale il Buddha ottenne il Nibbāna definitivo. Tutti questi
eventi, ad eccezione della morte dei due discepoli eminenti, sono
contenuti in un brano o sutta. Solamente ora nel Canone ricomincia un
resoconto cronologico degli eventi.


\voice{Prima voce.} Così ho udito. Una volta il Beato stava soggiornando a
Rājagaha sul Picco dell’Avvoltoio. Il re Ajātasattu era allora ansioso
di attaccare i Vajji. Diceva: «Sterminerò questi Vajji, che sono forti e
potenti, li annienterò, li condurrò alla distruzione e alla rovina».


Il re Ajātasattu disse allora al brāhmaṇa Vassakāra, un ministro di
Magadha: «Vieni, brāhmaṇa, andiamo dal Beato e diciamogli: “Signore,
Ajātasattu Vedehiputta, re di Magadha, presta omaggio con il suo capo ai
piedi del Beato, e gli chiede se è libero dall’afflizione e dalla
malattia e se è in salute, se ha forza e felicità”. E dite: “Signore,
Ajātasattu Vedehiputta, re di Magadha, è ansioso di attaccare i Vajji.
Egli dice: ‘Sterminerò questi Vajji, che sono forti e potenti, li
annienterò, li condurrò alla distruzione e alla rovina’ ”. Prestate ben
attenzione alla sua risposta e riportatemela, perché il Beato non mente
mai».


«E sia, sire», rispose Vassakāra. Chiamò poi a raccolta un certo numero
di carrozze reali. Salì su una di esse e uscì da Rājagaha, dirigendosi
verso il Picco dell’Avvoltoio, andando avanti finché la strada consentì
alle carrozze di procedere. Poi scese e continuò a piedi fino al luogo
in cui si trovava il Beato. Lo salutò e si mise a sedere da un lato.
Dopo averlo fatto, egli consegnò il suo messaggio.


Il Beato si rivolse al venerabile Ānanda, che stava dietro di lui e gli
faceva aria con un ventaglio: «Ānanda, hai sentito dire che i Vajji si
riuniscono spesso in assemblea e che le loro assemblee sono ben
frequentate?».


«Sì, Signore, l’ho sentito».


«Fino a quando si comportano così, Ānanda, si possono attendere di
essere prosperi e di non andare incontro al declino. Hai sentito dire
che si riuniscono in assemblea in concordia, si alzano per andar via in
concordia, assolvono ai loro doveri come Vajji in concordia? Che evitano
promulgazioni di cose non promulgate, non aboliscono promulgazioni
esistenti e procedono in accordo con le antiche leggi Vajji in quanto
promulgate? Che onorano, rispettano, riveriscono e venerano gli anziani
Vajji e pensano che debbano essere ascoltati? Che vivono senza stuprare
e rapire le donne e le ragazze dei loro clan? Che onorano, rispettano,
riveriscono e venerano i sacrari Vajji sia in città sia in campagna
senza consentire che vengano dimenticate le legittime oblazioni fino ad
ora date e ricevute? Che la legittima protezione, difesa e custodia è
messa in atto fra i Vajji per gli Arahant, così che gli Arahant che non
sono giunti nel regno possano giungervi e gli Arahant che vi sono giunti
possano vivere felicemente?».


«Sì, Signore, l’ho sentito».


«Fino a quando si comportano così, Ānanda, si possono attendere di
essere prosperi e di non andare incontro al declino».


Allora il Beato si rivolse a Vassakāra: «Una volta, brāhmaṇa, quando
soggiornavo a Vesālī, nel Sacrario di Sārandada, insegnai ai Vajji
queste sette cose che prevengono il declino. Fino a quando queste cose
durano e sono insegnate, i Vajji si possono attendere di essere prosperi
e di non andare incontro al declino».


Quando ciò fu detto, Vassakāra osservò: «Se i Vajji posseggono una sola
di queste cose, Maestro Gotama, possono attendersi di essere prosperi e
di non andare incontro al declino. E che cosa si potrebbe dire, allora,
se le possedessero tutte e sette? Infatti, Maestro Gotama, re Ajātasattu
non avrà mai la meglio sui Vajji combattendo, a meno che egli non li
compri o semini il dissenso tra loro. Ora dobbiamo andare, Maestro
Gotama. Siamo impegnati e abbiamo molto da fare».


«È tempo ora, brāhmaṇa, di fare quel che ritieni opportuno».


Vassakāra fu soddisfatto. Deliziandosi delle parole del Beato, egli si
alzò e, dopo avergli espresso il suo apprezzamento, se ne andò. Appena
se ne fu andato, il Beato disse al venerabile Ānanda: «Ānanda, vai,
raduna tutti i bhikkhu che vivono nelle vicinanze di Rājagaha e falli
incontrare nel salone».


«E sia, Signore», egli rispose. Quando ciò fu fatto, egli informò il
Beato. Allora il Beato si alzò dal luogo in cui sedeva e andò nel
salone, ove si mise a sedere nel posto preparatogli. Poi si rivolse ai
bhikkhu: «Bhikkhu, vi insegnerò sette cose che prevengono il declino.
Ascoltate e prestate attenzione a quello che dirò».


«E sia, Signore», loro risposero.


Il Beato disse: «Fino a quando i bhikkhu si riuniscono spesso in
assemblea e le loro assemblee sono ben frequentate; fino a quando si
riuniscono in assemblea in concordia, si alzano per andar via in
concordia, assolvono ai loro doveri come membri del Saṅgha in concordia;
fino a quando evitano promulgazioni di cose non promulgate, non
aboliscono promulgazioni esistenti e procedono in accordo con i precetti
dell’addestramento in quanto promulgati; fino a quando onorano,
rispettano, riveriscono e venerano i bhikkhu anziani che hanno
esperienza, che da tempo hanno abbracciato la vita religiosa e sono
padri e guide del Saṅgha, e pensano che debbano essere ascoltati; fino a
quando non cadono in potere della brama che conduce alla rinascita; fino
a quando apprezzano le dimore nella foresta; fino a quando mantengono in
se stessi la consapevolezza, di modo che buoni compagni nella santa vita
non ancora arrivati possano arrivare e buoni compagni nella santa vita
che sono giunti possano vivere felicemente; altrettanto a lungo si
possono attendere di essere prosperi e di non andare incontro al
declino».


«Altre sette cose che prevengono il declino: fino a quando i bhikkhu
evitano di deliziarsi, di gioire e prendere piacere nell’essere
affaccendati, nel pettegolezzo, nel dormire e nella compagnia; fino a
quando non hanno desideri malsani ed evitano di cadere in loro potere;
fino a quando non hanno cattivi amici ed evitano di cadere in loro
potere; fino a quando non si fermano a mezza strada con il
raggiungimento delle sole distinzioni più basse e mondane, altrettanto a
lungo si possono attendere di essere prosperi e di non andare incontro
al declino».


«Altre sette cose che prevengono il declino: fino a quando i bhikkhu
hanno fede, coscienza, senso di vergogna, conoscono [il Dhamma], sono
energici, consapevoli e posseggono la comprensione, altrettanto a lungo
si possono attendere di essere prosperi e di non andare incontro al
declino».


«Altre sette cose che prevengono il declino: fino a quando i bhikkhu
mantengono in essere i fattori dell’Illuminazione della consapevolezza,
dell’investigazione degli stati [mentali], dell’energia, della felicità,
della tranquillità, della concentrazione e dell’equanimità, altrettanto
a lungo si possono attendere di essere prosperi e di non andare incontro
al declino».


«Altre sette cose che prevengono il declino: fino a quando i bhikkhu
sviluppano la percezione dell’impermanenza, del non-sé, della ripugnanza
del corpo, del pericolo, dell’abbandono della brama, del distacco dalla
brama e della cessazione della brama, altrettanto a lungo si possono
attendere di essere prosperi e di non andare incontro al declino».


\suttaRef{D. 16; A. 7:20}


«Sei cose che prevengono il declino: fino a quando i bhikkhu sia in
pubblico sia in privato mantengono [in essere] atti corporei, verbali e
mentali di gentilezza amorevole nei riguardi dei loro compagni nella
santa vita; fino a quando sono imparziali e condividono senza
discriminazioni con i loro compagni nella santa vita quel che di
legittimo viene legittimamente acquisito, perfino ciò che è contenuto
nella ciotola; fino a quando i bhikkhu vivono tra i loro compagni nella
santa vita sia in pubblico sia in privato con virtù non lacere, non
discontinue, non macchiate, che favoriscono l’emancipazione, sono
raccomandate dal saggio, non sono fraintese e sostengono la
concentrazione; fino a quando i bhikkhu vivono in possesso della visione
degli Esseri Nobili che conduce fuori (dal ciclo delle rinascite), che
conduce alla completa estinzione della sofferenza per chi la porta a
effetto, altrettanto a lungo si possono attendere di essere prosperi e
di non andare incontro al declino».


«Fino a quando queste cose che prevengono il declino persistono e sono
insegnate tra i bhikkhu, loro si possono attendere di essere prosperi e
di non andare incontro al declino».


E mentre il Beato stava soggiornando a Rājagaha sul Picco
dell’Avvoltoio, egli offrì spesso questo discorso di Dhamma ai bhikkhu:
«Questa è la virtù, questa è la concentrazione, questa è la
comprensione. La concentrazione fortificata dalla virtù reca grandi
benefici e grandi frutti, la comprensione fortificata dalla
concentrazione reca grandi benefici e grandi frutti, il cuore
fortificato dalla comprensione si libera completamente dalle
contaminazioni, dalla contaminazione del desiderio sensoriale, dalla
contaminazione dell’esistenza, dalla contaminazione delle opinioni e
dalla contaminazione dell’ignoranza».


Quando il Beato ebbe soggiornato a Rājagaha per tutto il tempo che
volle, disse al venerabile Ānanda: «Vieni, Ānanda, andiamo ad
Ambalaṭṭhikā».


«E sia, Signore», rispose il venerabile Ānanda. Allora il Beato viaggiò
verso Ambalaṭṭhikā con un grande seguito di bhikkhu. Quando fu là visse
nella Casa del Re di Ambalaṭṭhikā.


E mentre il Beato viveva lì, spesso offrì questo discorso di Dhamma ai
bhikkhu: «Questa è la virtù, questa è la concentrazione, questa è la
comprensione. La concentrazione fortificata dalla virtù reca grandi
benefici e grandi frutti, la comprensione fortificata dalla
concentrazione reca grandi benefici e grandi frutti, il cuore
fortificato dalla comprensione si libera completamente dalle
contaminazioni, dalla contaminazione del desiderio sensoriale, dalla
contaminazione dell’esistenza, dalla contaminazione delle opinioni e
dalla contaminazione dell’ignoranza».


Quando il Beato ebbe soggiornato ad Ambalaṭṭhikā per tutto il tempo che
volle, disse al venerabile Ānanda: «Vieni, Ānanda, andiamo a Nālandā».


«E sia, Signore», rispose il venerabile Ānanda. Allora il Beato viaggiò
verso Nālandā con un grande seguito di bhikkhu. Quando fu là visse nel
Boschetto di Mango di Pāvārika a Nālandā.


\suttaRef{D. 16}


Allora il venerabile Sāriputta andò dal Beato e disse: «Signore, di
questo sono certo: non c’è mai stato, mai ci sarà e non c’è un altro
monaco o brāhmaṇa più distinto per Illuminazione del Beato».


«Questa è un’affermazione grandiosa e ardita, Sāriputta, un ruggito del
leone privo di compromessi. Conosci tu, allora, tutti i Buddha del
passato realizzati e completamente illuminati, leggendo la loro mente
con la tua mente in questo modo: “Questa era la loro virtù, questa era
la loro concentrazione, questa era la loro comprensione, questa era lo
stato [mentale] nel quale dimoravano, questo il modo della loro
liberazione”?».


«No, Signore».


«Conosci tu, allora, tutti i Buddha del futuro, realizzati e
completamente illuminati, leggendo la loro mente in quel modo?».


«No, Signore».


«Conosci me, allora, realizzato e completamente illuminato, leggendo la
mia mente in quel modo?».


«No, Signore».


«Come puoi allora fare quest’affermazione grandiosa e ardita, questo
ruggito del leone privo di compromessi?».


«Signore, non conosco gli Esseri del passato, del futuro e del presente
realizzati e completamente illuminati, leggendo la loro mente con la mia
mente. Non di meno, ho una certezza in relazione al Dhamma. Supponiamo
che un re possegga una città di frontiera con profondi fossati, forti
terrapieni e bastioni, e una sola porta, e abbia un guardiano saggio,
intelligente, sagace che blocca alla porta chi non conosce e fa entrare
solamente chi conosce. E siccome lui stesso ha fatto un giro intorno
alla città e non ha visto varchi nei terrapieni né alcun foro abbastanza
grande per farci passare un gatto, può giungere alla conclusione che
esseri viventi più grandi d’una certa dimensione debbano entrare e
uscire usando la porta, così, Signore, ho una certezza in relazione al
Dhamma. Tutti i Beati del passato, realizzati e completamente
illuminati, hanno la mente ben salda nei quattro fondamenti della
consapevolezza. Dopo aver abbandonato i cinque impedimenti, le
contaminazioni del cuore che indeboliscono la comprensione, hanno
scoperto la suprema e completa Illuminazione mantenendo in essere i
sette fattori dell’Illuminazione. Tutti i Beati del futuro, realizzati e
completamente illuminati, faranno lo stesso. Il Beato ora, realizzato e
completamente illuminato, ha fatto lo stesso».


\suttaRef{D. 16; S. 47:12}


E mentre il Beato stava soggiornando nel Boschetto di Mango di Pāvārika
a Nālandā, egli spesso offrì questo discorso di Dhamma ai bhikkhu:
«Questa è la virtù, questa è la concentrazione, questa è la
comprensione. La concentrazione fortificata dalla virtù reca grandi
benefici e grandi frutti, la comprensione fortificata dalla
concentrazione reca grandi benefici e grandi frutti, il cuore
fortificato dalla comprensione si libera completamente dalle
contaminazioni, dalla contaminazione del desiderio sensoriale, dalla
contaminazione dell’esistenza, dalla contaminazione delle opinioni e
dalla contaminazione dell’ignoranza».


Quando il Beato ebbe soggiornato a Nālandā per tutto il tempo che volle,
disse al venerabile Ānanda: «Vieni, Ānanda, andiamo a Pāṭaligāma».


«E sia, Signore», rispose il venerabile Ānanda. Allora il Beato viaggiò
verso Pāṭaligāma con un grande seguito di bhikkhu.


I seguaci di Pāṭaligāma sentirono dire: «Sembra che il Beato sia
arrivato a Pāṭaligāma». Allora si recarono dal Beato e, dopo avergli
prestato omaggio, si misero a sedere da un lato. Dopo averlo fatto,
dissero: «Che il Beato accetti di venire nel nostro ostello. Il Beato
accettò in silenzio. Vedendo che aveva acconsentito, loro si alzarono
dal posto in cui sedevano e, dopo avergli prestato omaggio, andarono
verso l’ostello girandogli a destra. Distesero ovunque delle stuoie,
prepararono dei posti a sedere e un grande contenitore d’acqua, e
appesero una lampada a olio. Poi dissero al Beato quel che avevano
fatto, aggiungendo: «È tempo ora, Signore, che il Beato faccia quel che
ritiene opportuno».


Allora il Beato si vestì, prese la ciotola e la veste superiore e andò
all’ostello. Dopo essersi lavato i piedi, entrò e si mise a sedere
presso il pilastro centrale, rivolto a est. E, dopo essersi lavati i
piedi, i bhikkhu del Saṅgha entrarono e si misero a sedere presso il
muro a ovest, rivolti a est, con il Beato davanti a loro. E i seguaci di
Pāṭaligāma, dopo essersi lavati i piedi, entrarono nell’ostello e si
misero a sedere presso il muro a est, rivolti a ovest, con il Beato
davanti a loro. Allora il Beato si rivolse ai seguaci di Pāṭaligāma con
queste parole:


«Capifamiglia, l’uomo non virtuoso incorre in questi cinque pericoli per
aver mancato di virtù. Quali cinque? L’uomo non virtuoso che manca di
virtù patisce una gran perdita di ricchezza a causa della negligenza.
Secondo, acquisisce una cattiva nomea. Terzo, a qualsiasi assemblea egli
prenda parte – di nobili guerrieri, brāhmaṇa, capifamiglia o monaci –
non si sente sicuro [di sé], manca di fiducia. Quarto, muore confuso.
Quinto, alla dissoluzione del corpo, dopo la morte, riappare in una
condizione di privazione, in una destinazione infelice, nella
perdizione, perfino all’inferno».


«L’uomo virtuoso, invece, ottiene questi cinque benefici mediante il
perfezionamento della virtù. Quali cinque? L’uomo virtuoso, che ha
perfezionato la virtù, ottiene grande ricchezza a causa della diligenza.
Secondo, acquisisce una buona nomea. Terzo, a qualsiasi assemblea egli
prenda parte – di nobili guerrieri, brāhmaṇa, capifamiglia o monaci – si
sente sicuro [di sé], non manca di fiducia. Quarto, muore non confuso.
Quinto, alla dissoluzione del corpo, dopo la morte, riappare in una
destinazione felice, perfino in un paradiso celeste».


Poi, quando il Beato ebbe istruito, esortato, risvegliato e incoraggiato
i seguaci di Pāṭaligāma per buona parte della notte, li lasciò dicendo:
«Capifamiglia, la notte è quasi trascorsa. È tempo ora, per voi, di fare
quel che ritenete opportuno».


«E sia, Signore», loro risposero, e si alzarono dai posti in cui erano
seduti, prestarono omaggio al Beato e se ne andarono, girandogli a
destra. Subito dopo che se ne furono andati, il Beato si recò in una
stanza vuota.


In quel tempo Sunidha e Vassakāra, ministri di Magadha, stavano
costruendo una città a Pāṭaligāma per tenere a bada i Vajji. Schiere di
divinità, a migliaia, si aggiravano per i campi. Divinità potenti
influenzavano la mente di sovrani e ministri potenti per far costruire
le città nei posti frequentati da loro. Divinità intermedie
influenzavano la mente di sovrani e ministri intermedi per far costruire
le città nei posti frequentati da loro. Divinità minori influenzavano la
mente di sovrani e ministri minori per far costruire le città nei posti
frequentati da loro. Con l’occhio divino, che è purificato e supera
quello umano, il Beato vide queste divinità. Allora, quando la notte si
approssimava all’alba, il Beato, si alzò e chiese al venerabile Ānanda:
«Ānanda, chi sta costruendo una città a Pāṭaligāma?». «La stanno
costruendo Sunidha e Vassakāra, Signore».


«Lo stanno facendo come se fossero stati consigliati dalle Divinità del
paradiso delle Trentatré Divinità», disse il Beato, e raccontò quel che
aveva visto. Egli aggiunse: «Tra tutte le dimore dei nobili e tra tutti
i centri di commercio, Pāṭaliputta\footnote{Il villaggio di Pāṭaligāma cambia qui il suo nome in Pāṭaliputta per l’edificazione della nuova città (oggi chiamata Patna). In seguito diventerà famosa in quanto capitale dell’impero di Asoka, che si sviluppò dal regno di Magadha.} sarà la città più
grande, luogo nel quale le borse dei tesori sono dissigillate. Sarà a
rischio per tre pericoli: il fuoco, l’acqua e il dissenso».


Allora Sunidha e Vassakāra andarono dal Beato e lo invitarono per il
pasto del giorno seguente. Quando il pasto fu terminato, quando il Beato
ebbe finito di mangiare e non teneva più la ciotola in mano, loro
presero sedili più bassi e si misero a sedere da un lato. Allora il
Beato impartì la benedizione con queste strofe:


\begin{quote}
Dove un uomo saggio prende dimora, \\
che lì nutra il virtuoso \\
che vive la buona vita autocontrollato, \\
e faccia offerte alle divinità del luogo. \\
Per quest’onore e rispetto nei loro riguardi, \\
lo ripagheranno nello stesso modo \\
perché il loro amore per lui è come \\
l’amore di una madre per il proprio figlio. \\
E quando un uomo è amato dalle divinità, \\
lo attendono sempre cose di buon auspicio.
\end{quote}

Allora il Beato si alzò dal posto in cui sedeva e andò via. In quella
circostanza, però, Sunidha e Vassakāra seguirono il Beato, pensando: «La
porta [della città] dalla quale il Beato andrà via sarà detta la Porta
di Gotama, il guado mediante il quale attraverserà il Gange sarà detto
Guado del Beato». E la porta dalla quale il Beato andò via fu detta la
Porta di Gotama. Quando però il Beato arrivò al Gange, il fiume era
talmente in piena e così colmo che perfino i corvi vi si potevano
abbeverare. Alcune persone che volevano raggiungere la riva opposta
stavano cercando delle barche, altre stavano cercando qualcosa che
galleggiasse e altre ancora stavano legando assieme delle zattere.
Allora, con la stessa velocità con cui un uomo forte distende il suo
braccio piegato o piega il suo braccio disteso, il Beato con il Saṅgha
dei bhikkhu scomparve dalla riva del Gange e si trovò sull’altra riva.
Egli vide le persone che volevano attraversare cercando delle barche,
cercando qualcosa che galleggiasse e legando assieme delle zattere.
Conoscendo il significato di ciò, egli esclamò queste parole:


\begin{quote}
Mentre coloro che vogliono attraversare la corrente \\
costruiscono ponti ed evitano gli abissi, \\
mentre la gente lega assieme zattere, \\
il saggio è già dall’altra parte.
\end{quote}

\suttaRef{D. 16; Ud. 8:6; Vin. Mv. 6:28}


Allora il Beato disse al venerabile Ānanda: «Vieni, Ānanda, andiamo a
Koṭigāma».


«E sia, Signore», rispose il venerabile Ānanda. Allora il Beato viaggiò
verso Koṭigāma con un grande seguito di bhikkhu. Lì il Beato soggiornò a
Koṭigāma. E lì si rivolse ai bhikkhu con queste parole: «Bhikkhu, è a
causa del non aver scoperto, del non aver penetrato le Quattro Nobili
Verità che sia io sia voi abbiamo dovuto viaggiare e arrancare in questo
lungo cerchio. Quali quattro? Esse sono la Nobile Verità della
Sofferenza, la Nobile Verità dell’Origine della Sofferenza, la Nobile
Verità della Cessazione della Sofferenza, e la Nobile Verità del
Sentiero che conduce alla Cessazione della Sofferenza. Quando però
queste Quattro Nobili Verità sono scoperte e penetrate, la brama per
l’esistenza è eliminata, la brama che conduce all’esistenza è abolita e
non c’è rinnovamento dell’esistenza».


E mentre il Beato stava soggiornando a Koṭigāma, egli spesso offrì
questo discorso di Dhamma ai bhikkhu: «Questa è la virtù, questa è la
concentrazione, questa è la comprensione. La concentrazione fortificata
dalla virtù reca grandi benefici e grandi frutti, la comprensione
fortificata dalla concentrazione reca grandi benefici e grandi frutti,
il cuore fortificato dalla comprensione si libera completamente dalle
contaminazioni, dalla contaminazione del desiderio sensoriale, dalla
contaminazione dell’esistenza, dalla contaminazione delle opinioni e
dalla contaminazione dell’ignoranza».


\suttaRef{D. 16; Vin. Mv. 6:29}


Quando il Beato ebbe soggiornato a Koṭigāma per tutto il tempo che
volle, disse al venerabile Ānanda: «Vieni, Ānanda, andiamo a Nādikā».


«E sia, Signore», rispose il venerabile Ānanda. Allora il Beato viaggiò
verso Nādikā con un grande seguito di bhikkhu. Quando fu là visse nella
Casa dei Mattoni a Nādikā.


Allora il venerabile Ānanda andò dal Beato. Egli disse: «Signore, a
Nādikā è morto il bhikkhu chiamato Sāḷha. Qual è la sua destinazione?
Qual è la sua rinascita? La bhikkhuṇī chiamata Nandā, il seguace laico
chiamato Sudatta, la seguace laica chiamata Sujātā, i seguaci laici
chiamati Kakudha, Kālinga, Nikaṭa, Kaṭissabha, Tuṭṭha, Santuṭṭha, Bhadda
e Subhadda, costoro sono morti a Nādikā. Qual è la loro destinazione?
Qual è la loro rinascita?».\footnote{Il Commentario afferma che il \emph{Janavasabha Sutta} (D. 18) fu pronunciato a questo punto.}


«Il bhikkhu Sāḷha, Ānanda, mediante la realizzazione qui e ora, è
entrato e dimora nella liberazione della mente e nella liberazione per
mezzo della comprensione, che è priva delle contaminazioni per
l’esaurimento delle contaminazioni. La bhikkhuṇī Nandā, mediante la
distruzione delle cinque catene inferiori, riapparirà spontaneamente
altrove, e lì otterrà il Nibbāna senza mai tornare da quel mondo. Il
seguace laico Sudatta, con la distruzione delle tre catene [inferiori] e
con l’attenuazione della brama, dell’odio e dell’illusione, ha ottenuto
la condizione di Chi Torna Una Sola Volta, e tornerà una volta in questo
mondo per porre fine alla sofferenza. La seguace laica Sujātā, con la
distruzione delle tre catene [inferiori] ha ottenuto la condizione di
Chi è Entrato nella Corrente, non è più soggetto alla perdizione, è
certo della rettitudine e destinato all’Illuminazione. I seguaci laici
Kakudha, Kālinga, Nikaṭa, Kaṭissabha, Tuṭṭha, Santuṭṭha, Bhadda e
Subhadda, e altri cinquanta seguaci laici hanno tutti raggiunto la
condizione di Chi è Entrato nella Corrente. Novanta seguaci laici hanno
raggiunto la condizione di Chi è Entrato nella Corrente. Più di
cinquecento seguaci laici hanno raggiunto la condizione di Chi è Entrato
nella Corrente».


«Per gli esseri umani morire è un fatto naturale, ma se tu vieni e mi
poni questa domanda tutte le volte che qualcuno muore, questo tedia il
Beato. Perciò vi offrirò un’esposizione del Dhamma chiamata “lo Specchio
del Dhamma”, conoscendo il quale un nobile discepolo può predire da sé:
“Per me non c’è più inferno, non c’è più nascita animale, non c’è più
regno degli spiriti, non ci sono più stati di privazione, destinazioni
infelici o perdizione. Ho raggiunto la condizione di Chi è Entrato nella
Corrente, non sono più soggetto a perdizione, sono certo della
rettitudine e destinato all’Illuminazione».


«E qual è l’esposizione del Dhamma chiamata “lo Specchio del Dhamma”? Un
nobile discepolo ha fiducia assoluta nel Buddha: “Il Beato è così perché
è realizzato, completamente illuminato, perfetto nella conoscenza e
nella condotta, sublime, conoscitore dei mondi, incomparabile guida
degli uomini che devono essere addestrati, insegnante di dèi e uomini,
illuminato, beato”. Egli ha fiducia assoluta nel Dhamma: “Il Dhamma è
ben proclamato dal Beato, il suo effetto è visibile qui e ora, è senza
tempo (non differito), invita all’investigazione, conduce verso
l’interiorità e può essere direttamente sperimentato dal saggio”. Egli
ha fiducia assoluta nel Saṅgha: “Il Saṅgha dei discepoli del Beato è
sulla buona strada, è entrato nella retta strada, nella vera strada,
nella giusta strada, ossia, [il Saṅgha] delle quattro paia di uomini,
degli otto tipi di persone;\footnote{Le “quattro paia di uomini, gli otto tipi di persone” sono spiegate come chi raggiunge il Sentiero e ne ottiene le fruizioni nel caso di ognuno dei quattro stadi (sentieri) della realizzazione. Si afferma che la “fruizione” segue immediatamente il raggiungimento di ognuno di tali stadi (si veda Sn. 2:1, vv. 5 e 6). Questo è uno dei significati dell’espressione “senza tempo (non differito)” usata per il Dhamma poco sopra, nel senso che il fruttuoso conseguimento del Sentiero non richiede attese, ad esempio fin dopo la morte, per la fruizione di esso.} questo Saṅgha dei discepoli
del Beato degno di doni, ospitalità, offerte e saluti reverenti, in
quanto incomparabile campo di meriti per il mondo”. Egli è perfetto
nelle virtù amate dagli Esseri Nobili, non lacere, non discontinue, non
macchiate, che favoriscono l’emancipazione, sono raccomandate dal
saggio, non sono fraintese e sostengono la concentrazione. Questa è
l’esposizione del Dhamma chiamata “lo Specchio del Dhamma”, conoscendo
il quale un nobile discepolo può predire da sé: “Per me non c’è più
inferno … Ho raggiunto la condizione di Chi è Entrato nella Corrente,
non sono più soggetto a perdizione, sono certo della rettitudine e
destinato all’Illuminazione”».


E mentre il Beato stava soggiornando a Nādikā nella Casa dei Mattoni,
egli spesso offrì questo discorso di Dhamma ai bhikkhu: «Questa è la
virtù, questa è la concentrazione, questa è la comprensione. La
concentrazione fortificata dalla virtù reca grandi benefici e grandi
frutti, la comprensione fortificata dalla concentrazione reca grandi
benefici e grandi frutti, il cuore fortificato dalla comprensione si
libera completamente dalle contaminazioni, dalla contaminazione del
desiderio sensoriale, dalla contaminazione dell’esistenza, dalla
contaminazione delle opinioni e dalla contaminazione dell’ignoranza».


\suttaRef{D. 16}


Quando il Beato ebbe soggiornato a Nādikā per tutto il tempo che volle,
disse al venerabile Ānanda: «Vieni, Ānanda, andiamo a Vesālī».


«E sia, Signore», rispose il venerabile Ānanda. Allora il Beato viaggiò
verso Vesālī con un grande seguito di bhikkhu. Quando fu là visse nel
Boschetto di Ambapālī a Vesālī. Là si rivolse ai bhikkhu con queste
parole: «Bhikkhu, un bhikkhu dovrebbe vivere consapevole e pienamente
presente: questa è la mia istruzione per voi. E com’è che un bhikkhu
dovrebbe vivere consapevole? Un bhikkhu dimora contemplando il corpo
come corpo, ardente, pienamente presente, consapevole, avendo messo da
parte bramosia e afflizione per il mondo. Egli dimora contemplando le
sensazioni come sensazioni, ardente, pienamente presente, consapevole,
avendo messo da parte bramosia e afflizione per il mondo. Egli dimora
contemplando la coscienza come coscienza, ardente, pienamente presente,
consapevole, avendo messo da parte bramosia e afflizione per il mondo.
Egli dimora contemplando gli oggetti mentali come oggetti mentali,
ardente, pienamente presente, consapevole, avendo messo da parte
bramosia e afflizione per il mondo. E com’è un bhikkhu pienamente
presente? Un bhikkhu è pienamente presente quando si muove avanti e
indietro, quando guarda avanti e lontano, quando piega ed estende gli
arti, quando indossa la veste superiore fatta di toppe, la ciotola e le
altre vesti, quando mangia, quando beve, quando mastica, quando
assapora, quando evacua l’intestino e urina, quando cammina, quando sta
in piedi, quando sta seduto, quando va a dormire, quando si sveglia,
parla e mantiene il silenzio. Un bhikkhu dovrebbe vivere consapevole e
pienamente presente: questa è la mia istruzione per voi».


\suttaRef{D. 16; cf. D. 22}


La cortigiana Ambapālī sentì dire che il Beato era giunto a Vesālī e che
stava soggiornando nel suo boschetto di manghi (\emph{amba}). Ella fece
preparare un certo numero di carrozze di corte. Salì su una di esse e la
guidò fuori da Vesālī, verso il suo boschetto di manghi, procedendo
finché la strada lo consentì alle carrozze. Poi scese e continuò a piedi
fino al luogo in cui si trovava il Beato. Gli prestò omaggio e poi si
mise a sedere da un lato. Dopo che l’ebbe fatto, il Beato la istruì,
esortò, risvegliò e incoraggiò con un discorso di Dhamma. Poi lei gli
disse: «Signore, che il Beato con il Saṅgha accetti il pasto di domani
da me».


Il Beato accettò in silenzio. Quando lei vide che egli aveva accettato,
si alzò dal posto in cui sedeva e, dopo avergli prestato omaggio, se ne
andò girandogli a destra.


I Licchavi di Vesālī, però, sentirono anche loro dire che il Beato stava
soggiornando nel boschetto di mango di Ambapālī. Pure loro fecero
preparare un certo numero di carrozze di corte, salirono su di esse e le
guidarono fuori da Vesālī. Alcune erano in blu, dipinte di blu, con
tappezzerie blu e ornamenti blu. Alcune erano in giallo, dipinte di
giallo, con tappezzerie gialle e ornamenti gialli. Alcune erano in
rosso, dipinte di rosso, con tappezzerie rosse e ornamenti rossi. Alcune
erano in bianco, dipinte di bianco, con tappezzerie bianche e ornamenti
bianchi.


La cortigiana Ambapālī si affiancò [con la carrozza] ai giovani
Licchavi, asse ad asse, ruota a ruota, giogo a giogo. Loro allora le
dissero: «Ehi, Ambapālī, perché ti sei affiancata [con la carrozza] ai
giovani Licchavi, asse ad asse, ruota a ruota, giogo a giogo?».


«Signori, ho appena invitato il Saṅgha dei bhikkhu guidato dal Beato per
il pasto di domani».


«Ehi, Ambapālī, cedi a noi quel pasto per centomila monete».


«Signori, non vi cederei il pasto di domani nemmeno se mi deste Vesālī
con tutte le sue terre».


Allora i Licchavi schioccarono le dita: «Oh! La ragazza dei manghi ci ha
battuti, la ragazza dei manghi è stata più astuta di noi!».


Guidarono verso il boschetto di Ambapālī. Il Beato li vide da lontano
che arrivavano. Egli disse ai bhikkhu: «Che i bhikkhu che non hanno mai
visto le Divinità del paradiso delle Trentatré Divinità guardino i
Licchavi, che osservino i Licchavi, che immaginino che le Divinità del
paradiso delle Trentatré Divinità siano come i Licchavi».


I Licchavi procedettero finché la strada lo consentì alle carrozze. Poi
scesero e continuarono a piedi fino al luogo in cui si trovava il Beato.
Gli prestarono omaggio e poi si misero a sedere da un lato. Allora il
Beato li istruì, esortò, risvegliò e incoraggiò con un discorso di
Dhamma. Poi loro gli dissero: «Signore, che il Beato con il Saṅgha
accetti il pasto di domani da noi».


«Ho già accettato il pasto di domani, Licchavi, dalla cortigiana
Ambapālī».


Allora i Licchavi schioccarono le dita: «Oh! La ragazza dei manghi ci
ha battuti, la ragazza dei manghi è stata più astuta di noi!».


Erano tuttavia felici e soddisfatti per le parole del Beato, e si
alzarono dal posto in cui sedevano e se ne andarono, girandogli a
destra.


Così, quando la notte fu trascorsa, la cortigiana Ambapālī, che aveva
fatto preparare vari tipi di buon cibo nel suo parco, annunciò che era
giunto il momento: «È ora, Signore, il pasto è pronto».


Quando il Beato ebbe finito di mangiare e non teneva più la ciotola in
mano, Ambapālī prese un sedile più basso e si mise a sedere da un lato.
Ella disse: «Signore, offro in dono questo boschetto di manghi al Saṅgha
dei bhikkhu guidato dal Beato». Il Beato accettò il parco e, dopo averle
dato istruzioni con un discorso di Dhamma, si alzò dal posto in cui
sedeva e se ne andò.


E mentre il Beato stava soggiornando a Vesālī nel boschetto di Ambapālī,
egli spesso offrì questo discorso di Dhamma ai bhikkhu: «Questa è la
virtù, questa è la concentrazione, questa è la comprensione. La
concentrazione fortificata dalla virtù reca grandi benefici e grandi
frutti, la comprensione fortificata dalla concentrazione reca grandi
benefici e grandi frutti, il cuore fortificato dalla comprensione si
libera completamente dalle contaminazioni, dalla contaminazione del
desiderio sensoriale, dalla contaminazione dell’esistenza, dalla
contaminazione delle opinioni e dalla contaminazione dell’ignoranza».


\suttaRef{D. 16; cf. Vin. Mv. 6:30}


Quando il Beato ebbe soggiornato nel boschetto di Ambapālī per tutto il
tempo che volle, disse al venerabile Ānanda: «Vieni, Ānanda, andiamo a
Beluvagāmaka».


«E sia, Signore», rispose il venerabile Ānanda. Allora il Beato viaggiò
verso Beluvagāmaka con un grande seguito di bhikkhu. Quando fu là visse
a Beluvagāmaka. Là si rivolse ai bhikkhu con queste parole: «Venite,
bhikkhu, per la stagione delle piogge risiedete nei pressi di Vesālī,
ovunque abbiate degli amici, dei compagni o dei conoscenti. Io risiederò
qui a Beluvagāmaka».


«E sia, Signore», loro risposero. E così fecero.


Dopo che il Beato ebbe preso residenza per la stagione delle piogge, una
grave malattia lo aggredì, con violenti e mortali dolori. Egli la
sopportò senza lamentarsi, consapevole e pienamente presente. Allora
egli pensò: «Non è corretto che io ottenga il Nibbāna definitivo senza
aver parlato con i miei attendenti e senza essermi accomiatato dal
Saṅgha dei bhikkhu. E se io in modo forzato eliminassi questa malattia
prolungando la volontà di vivere?». Così fece. E la malattia cessò.


Il Beato guarì da quella malattia. Subito dopo egli uscì dal suo luogo
di ricovero e si mise a sedere nel posto preparatogli sul retro della
dimora. Il venerabile Ānanda andò da lui e disse: «Ero solito vedere il
Beato a suo agio e in salute, Signore. Infatti, durante la malattia del
Beato il mio corpo era come se fosse rigido, non vedevo bene e i miei
pensieri erano tutti poco chiari. Signore, tuttavia mi confortava sapere
che il Beato non avrebbe ottenuto il Nibbāna definitivo senza
pronunciarsi in merito al Saṅgha dei bhikkhu».


«Ānanda, che cosa però si attende da me il Saṅgha? Il Dhamma che ho
insegnato non ha una versione segreta e una pubblica: qui non c’è alcun
“insegnante con il pugno chiuso” per le cose buone. Certamente potrebbe
esserci qualcuno che pensa “Io governerò il Saṅgha” oppure “Il Saṅgha
dipende da me”, che potrebbe pronunciarsi in merito al Saṅgha. Un
Perfetto, però, non pensa in questo modo. Come può allora pronunciarsi
in merito al Saṅgha? Ora sono anziano, Ānanda, i miei anni hanno
superato gli ottanta: proprio come un vecchio carro può andare avanti
con l’aiuto di espedienti, allo stesso modo sento che il corpo del
Perfetto può andare avanti solo con l’aiuto di espedienti. Perché il
corpo del Perfetto è a proprio agio solo mediante la non-attenzione a
tutti i segni e mediante la cessazione di certi tipi di sensazioni, ed
egli entra e dimora nella liberazione del cuore priva di segni. Così,
Ānanda, ognuno di voi deve fare di se stesso la propria
isola,\footnote{La parola \emph{dīpa} può significare sia “isola” sia “lampada”. Il Commentario la spiega con “isola”.} di se stesso
e di nessun altro il proprio
rifugio, ognuno di voi deve fare del Dhamma la propria isola, del Dhamma
e di nient’altro il proprio rifugio. E come lo fa un bhikkhu? Un bhikkhu
dimora contemplando il corpo come corpo, ardente, pienamente presente e
consapevole, avendo messo da parte bramosia e afflizione per il mondo.
Egli dimora contemplando le sensazioni come sensazioni … contemplando la
coscienza come coscienza … contemplando gli oggetti mentali come oggetti
mentali, ardente, pienamente presente e consapevole, avendo messo da
parte bramosia e afflizione per il mondo. Sia ora sia quando me ne sarò
andato, è uno di costoro, chiunque egli sia, di quelli che fanno di se
stessi la propria isola, di se stessi e di nessun altro il proprio
rifugio, che fanno del Dhamma la propria isola, del Dhamma e di
nient’altro il proprio rifugio: costui sarà il più eminente dei miei
bhikkhu, ossia di coloro che vogliono addestrarsi».


\suttaRef{D. 16; S. 47:9}


\narrator{Secondo narratore.} Benché non sia esplicitamente affermato nei Piṭaka, a
questo punto il Buddha pare che sia stato in visita a Sāvatthī, e fu
mentre si trovava lì che la notizia della morte dei suoi due discepoli
eminenti lo raggiunse.


\voice{Prima voce.} Una volta il Beato stava soggiornando a Sāvatthī, nel
Boschetto di Jeta, nel Parco di Anāthapiṇḍika. Allora il venerabile
Sāriputta stava però soggiornando a Nālagāmaka, nel territorio di
Magadha: egli era afflitto, sofferente e gravemente malato. Suo monaco
attendente era il novizio Cunda. Con quella malattia il venerabile
Sāriputta ottenne il Nibbāna definitivo. Allora il novizio Cunda prese
la ciotola e l’abito monastico del venerabile Sāriputta e si recò dal
venerabile Ānanda a Sāvatthī, nel Boschetto di Jeta. Gli prestò omaggio
e disse: «Signore, il venerabile Sāriputta ha ottenuto il Nibbāna
definitivo. Questa è la sua ciotola e questo è il suo abito monastico».


«Amico Cunda, lo dobbiamo dire al Beato per sua informazione, dobbiamo
vedere il Beato e dirgli questo. Andiamo e diciamoglielo».


«E sia, Signore», rispose il novizio Cunda. Andarono insieme dal Beato e
gli prestarono omaggio. Poi si misero a sedere da un lato e il
venerabile Ānanda disse: «Signore, questo novizio, Cunda, mi ha detto
che il venerabile Sāriputta ha ottenuto il Nibbāna definitivo e che
questa è la sua ciotola e questo è il suo abito monastico. In verità,
Signore, quando ho sentito questa cosa, il mio corpo era come se fosse
rigido, non vedevo bene e i miei pensieri erano tutti poco chiari».


«Ānanda, è perché pensi che ottenendo il Nibbāna definitivo egli abbia
portato via il codice della virtù, il codice della concentrazione, il
codice della comprensione, il codice della liberazione o il codice della
conoscenza e visione della liberazione?».


«Non è questo, Signore. Penso, però, quanto egli sia stato d’aiuto per i
suoi compagni nella santa vita, consigliandoli, informandoli,
istruendoli, esortandoli, risvegliandoli e incoraggiandoli, quanto
instancabile egli sia stato nell’insegnare loro il Dhamma. Noi
ricordiamo quanto il venerabile Sāriputta ci abbia nutriti, arricchiti e
aiutati con il Dhamma».


«Ānanda, non ti ho già detto che c’è separazione, distacco e divisione
da tutto quello che ci è caro e che amiamo? Come potrebbe avvenire che
quel che è nato, giunto all’esistenza, formato e soggetto alla decadenza
non decada? Questo non è possibile. È come se il ramo principale di un
grande albero fermo e massiccio sia caduto. Allo stesso modo, Sāriputta
ha ottenuto il Nibbāna definitivo in una grande comunità ferma e
massiccia. Come potrebbe avvenire che quel che è nato, giunto
all’esistenza, formato e soggetto alla decadenza non decada? Questo non
è possibile. Perciò, Ānanda, ognuno di voi deve fare di se stesso la
propria isola, di se stesso e di nessun altro il proprio rifugio, ognuno
di voi deve fare del Dhamma la propria isola, del Dhamma e di
nient’altro il proprio rifugio».


\suttaRef{S. 47:13}


Una volta il Beato stava soggiornando con una grande comunità di bhikkhu
nel territorio dei Vajji, a Ukkācelā, sulla riva del Gange. Era subito
dopo che Sāriputta e Moggallāna avevano ottenuto il Nibbāna definitivo.
In quell’occasione il Beato era seduto all’aperto, circondato dal Saṅgha
dei bhikkhu. Poi, dopo aver osservato il silenzioso Saṅgha dei bhikkhu,
si rivolse a loro con queste parole: «Ora quest’assemblea è come se
fosse vuota. Quest’assemblea è per me vuota ora che Sāriputta e
Moggallāna hanno ottenuto il Nibbāna definitivo. Non c’è luogo verso il
quale si possa guardare e dire: “Sāriputta e Moggallāna vivono là”. I
Beati del passato, realizzati e completamente illuminati, ognuno di loro
aveva una coppia di discepoli uguali a Sāriputta e Moggallāna, e così
avverrà per quelli del futuro. È meraviglioso, è stupefacente come i
discepoli attuino l’insegnamento del Maestro e adempiano ai suoi
consigli, e come siano cari al Saṅgha e amati, rispettati e riveriti dal
Saṅgha! È meraviglioso, è stupefacente che il Perfetto, quando una tale
coppia di discepoli ha ottenuto il Nibbāna, non si addolori né si
lamenti! Come potrebbe avvenire che quel che è nato, giunto
all’esistenza, formato e soggetto alla decadenza non decada? Questo non
è possibile».


\suttaRef{S. 47:14}


Un mattino il Beato si vestì, prese la ciotola e la veste superiore, e
andò a Vesālī per la questua. Quando ebbe fatto il giro per la questua a
Vesālī e fu ritornato dopo il pasto, disse al venerabile Ānanda: «Prendi
una stuoia, Ānanda, andiamo al Sacrario di Cāpāla a trascorrere la
giornata».


«E sia, Signore», rispose il venerabile Ānanda, e prese una stuoia e
seguì il Beato fino al Sacrario di Cāpāla. Là il Beato si mise a sedere
sulla stuoia preparatagli, e il venerabile Ānanda gli prestò omaggio e
si mise a sedere da un lato. Dopo averlo fatto, il Beato disse: «Vesālī
è piacevole, Ānanda, e altrettanto il Sacrario di Udena, il Sacrario di
Gotamaka, il Sacrario di Sattambaka, il Sacrario di Bahuputta, il
Sacrario di Sārananda e il Sacrario di Cāpāla. Chiunque abbia mantenuto
in essere e sviluppato le quattro basi per il successo spirituale, le
abbia rese veicolo, le abbia rese il fondamento, le abbia instaurate,
consolidate e propriamente intraprese, potrebbe, se lo volesse, vivere
per un’era o per quel che rimane di un’era. Ānanda, il Perfetto ha fatto
tutto questo. Egli potrebbe, se lo volesse, vivere per un’era o per quel
che rimane di un’era».


Pure dopo che il Beato ebbe offerto un’allusione così chiara,
un’indicazione così evidente, il venerabile Ānanda non la comprese. Egli
non implorò il Beato: «Signore, che il Beato viva per un’era, che il
Beato viva un’era per il benessere e la felicità di molti, per
compassione nei riguardi del mondo, per il bene, il benessere e la
felicità di divinità e uomini»: fino a questo punto la sua mente era
sotto l’influsso di Māra. Una seconda e una terza volta il Beato disse
la stessa cosa, e la mente del venerabile Ānanda rimase sotto l’influsso
di Māra.\footnote{È opportuno notare che il Buddha decise di insegnare la sua dottrina dietro invito di una Divinità (\hyperlink{cap-03-Dopo-l-Illuminazione#pag45}{}), e che egli abbandonò la sua determinazione di vivere in assenza di un invito a prolungarla a causa dell’intervento di Māra (la “Morte”).} Allora il
Beato disse al venerabile Ānanda:
«Puoi andare, Ānanda, è tempo di fare quel che reputi opportuno».


«E sia, Signore», egli rispose e, alzandosi dal posto in cui sedeva,
prestò omaggio al Beato. Poi, girandogli a destra, andò a sedersi ai
piedi di un albero che stava nelle vicinanze.


Subito dopo che se ne fu andato, Māra il Malvagio andò dal Beato e si
mise in piedi da un lato. Egli disse: «Che il Beato ottenga il Nibbāna
definitivo ora, che il Sublime ottenga il Nibbāna definitivo ora. Ora è
tempo che il Beato ottenga il Nibbāna definitivo». Allora il Beato
pronunciò queste parole: «Otterrò il Nibbāna definitivo, Malvagio,
quando i bhikkhu, le bhikkhuṇī, i seguaci laici e le seguaci laiche,
miei discepoli, saranno saggi, disciplinati, perfettamente fiduciosi e
sapienti, ricorderanno il Dhamma propriamente, praticheranno la via del
Dhamma e, dopo averlo imparato dai loro insegnanti, lo annunceranno,
insegneranno, dichiareranno, istituiranno, riveleranno, esporranno e
spiegheranno, saranno in grado di confutare in modo ragionevole le
teorie degli altri che sorgono e potranno insegnare il Dhamma con i suoi
prodigi». – «Ora, però, tutto questo si è realizzato. Che il Beato
ottenga il Nibbāna definitivo ora». Il Beato pronunciò queste parole:
«Otterrò il Nibbāna definitivo, Malvagio, quando questa santa vita si
sarà affermata, sarà prospera, diffusa e disseminata tra molti, ben
esemplificata dagli uomini». – «Ora, però, tutto questo si è realizzato.
Che il Beato ottenga il Nibbāna definitivo ora».


Quando ciò fu detto, il Beato rispose: «Puoi acquietarti, Malvagio.
Presto avrà luogo l’ottenimento del Nibbāna definitivo del Beato. Fra
tre mesi il Perfetto otterrà il Nibbāna definitivo».


Fu allora che, al Sacrario di Cāpāla, il Beato, consapevole e pienamente
presente, abbandonò la volontà di vivere. Quando lo fece, ci fu un gran
terremoto, pauroso e orripilante, e i tamburi del cielo risuonarono.
Conoscendo il significato di ciò, il Beato esclamò queste parole:


\begin{quote}
Il saggio rinunciò alla volontà di vivere, \\
sia commensurabile sia incommensurabile, \\
e concentrato interiormente e pure felice \\
lasciò cadere il suo autodivenire come una cotta di maglia.
\end{quote}

Il venerabile Ānanda pensò: «È meraviglioso, è stupefacente! Questo è
stato un gran terremoto, un terremoto davvero grande. È stato pauroso e
orripilante, e i tamburi del cielo hanno risuonato. Che cosa l’ha
causato, qual è stata la ragione per la manifestazione di quel gran
terremoto?».


Egli andò dal Beato e, dopo avergli prestato omaggio, si mise a sedere
da un lato. Dopo averlo fatto, egli chiese al Beato del terremoto.


«Ci sono otto cause, Ānanda, otto ragioni per la manifestazione di
grandi terremoti. Quali otto? La grande terra sta nell’acqua, l’acqua
sta nell’aria e l’aria sta nello spazio. Ci sono circostanze in cui
soffiano grandi venti (si muovono grandi forze), i grandi venti soffiano
(le grandi forze si muovono) e fanno tremare l’acqua, e l’acqua che
trema fa tremare la terra. Questa è la prima ragione. Ancora, un monaco
o un brāhmaṇa possiede poteri sovrannaturali e ha raggiunto la
padronanza della mente, oppure delle divinità possono essere forti e
potenti. Chi ha mantenuto in essere la percezione della terra
limitatamente e la percezione dell’acqua smisuratamente può scuotere
questa terra e farla tremare, agitare e scuotere. Questa è la seconda
ragione. Ancora, quando un Bodhisatta, consapevole e pienamente
presente, scompare dal paradiso dei Gioiosi ed entra nel grembo di sua
madre, allora la terra trema, s’agita, freme e si scuote. Questa è la
terza ragione. Ancora, quando un Bodhisatta, consapevole e pienamente
presente, esce dal grembo di sua madre, allora la terra trema … Questa è
la quarta ragione. Ancora, quando un Perfetto scopre la suprema, piena
Illuminazione, allora la terra trema … Questa è la quinta ragione.
Ancora, quando un Perfetto mette in moto l’incomparabile Ruota del
Dhamma, allora la terra trema … Questa è la sesta ragione. Ancora,
quando un Perfetto, consapevole e pienamente presente, abbandona la
volontà di vivere, allora la terra trema … Questa è la settima ragione.
Ancora, quando un Perfetto ottiene il Nibbāna definitivo con l’elemento
Nibbāna privo di residui del passato attaccamento, allora la terra trema
… Questa è l’ottava ragione».\footnote{Nel testo ora segue un resoconto degli otto generi di assemblee, delle otto basi della trascendenza e delle otto liberazioni, qui omesse per ragioni di spazio.}


\suttaRef{D. 16; A. 8:70; Ud. 6:1}


«Una volta, Ānanda, quando da poco ero illuminato, mentre soggiornavo a
Uruvelā, sulla riva del fiume Nerañjarā, ai piedi del baniano del
guardiano delle greggi di capre, Māra il Malvagio venne da me e disse:
“Che il Beato ottenga il Nibbāna definitivo ora”». Allora il Beato
proseguì narrando tutto quel che era avvenuto tra lui e Māra. Poi egli
disse: «E ora, Ānanda, proprio oggi, al Sacrario di Cāpāla, il Beato,
consapevole e pienamente presente, ha abbandonato la volontà di vivere».


Il venerabile Ānanda, quando udì questo, disse: «Signore, che il Beato
viva per un’era, che il Beato viva un’era per il benessere e la felicità
di molti, per compassione nei riguardi del mondo, per il bene, il
benessere e la felicità di divinità e uomini».


«Basta così, Ānanda, non chiedere questo al Beato ora, il tempo per
chiedere questo al Beato è ormai passato».


Una seconda volta il venerabile Ānanda fece la stessa richiesta e
ricevette la stessa risposta. La terza volta il Beato disse:


«Tu riponi la tua fiducia nell’Illuminazione del Perfetto, Ānanda?».


«Sì, Signore».


«Allora perché eserciti pressioni sul Beato per tre volte?».


«Signore, ho udito e imparato questo dalle labbra stesse del Beato:
“Chiunque abbia mantenuto in essere e sviluppato le quattro basi per il
successo spirituale, le abbia rese veicolo, le abbia rese il fondamento,
le abbia instaurate, consolidate e propriamente intraprese, potrebbe, se
lo volesse, vivere per un’era o per quel che rimane di un’era. Ānanda,
il Perfetto ha fatto tutto questo. Egli potrebbe, se lo volesse, vivere
per un’era o per quel che rimane di un’era”».


«Tu hai fiducia, Ānanda?».


«Sì, Signore».


«Allora, Ānanda, tua è la mancanza, tuo è l’errore. Perché pure quando
il Perfetto ti ha offerto un’allusione così chiara, un’indicazione così
evidente, tu non sei stato in grado di comprenderla e non hai implorato
il Perfetto di vivere per un’era per il bene, il benessere e la felicità
di divinità e uomini. Se tu lo avessi fatto, il Perfetto avrebbe
rifiutato due volte e, poi, la terza volta avrebbe acconsentito. Una
volta, quando soggiornavo sul Picco dell’Avvoltoio a Rājagaha, là io ti
dissi: “Rājagaha è piacevole, Ānanda, e altrettanto lo è il Picco
dell’Avvoltoio. Chiunque abbia mantenuto in essere e sviluppato le
quattro basi per il successo spirituale … potrebbe, se lo volesse,
vivere per un’era o per quel che rimane di un’era. Ānanda, il Perfetto
ha fatto tutto questo. Egli potrebbe, se lo volesse, vivere per un’era o
per quel che rimane di un’era”. Però, pure quando il Perfetto ti ha
offerto un’allusione così chiara, un’indicazione così evidente, tu non
sei stato in grado di comprenderla e non hai implorato il Perfetto:
“Signore, che il Beato viva per un’era, che il Beato viva un’era per il
benessere e la felicità di molti, per compassione nei riguardi del
mondo, per il bene, il benessere e la felicità di divinità e uomini”. Se
tu lo avessi fatto, il Perfetto avrebbe rifiutato due volte e, poi, la
terza volta avrebbe acconsentito. Perciò, Ānanda, tua è la mancanza, tuo
è l’errore. Inoltre, una volta, quando soggiornavo nel Parco di Nigrodha
a Rājagaha … sulla Collina dei Rapinatori … sui pendii del Vebhāra …
nella Caverna di Sattapaṇṇi … Sul Picco Nero ai pendii di Isigili …
sotto la Roccia a Strapiombo del Lago dei Serpenti nel Bosco Fresco …
nel Parco della Calda Fonte … nel Boschetto di Bambù, nel Sacrario degli
Scoiattoli … nel Boschetto di Manghi di Jīvaka … nel Parco delle
Gazzelle a Maddakucchi … Inoltre, una volta, quando soggiornavo qui a
Vesālī nel Sacrario Udena … nel Sacrario Gotamaka … nel Sacrario
Sattamba … nel Sacrario Bahuputta … nel Sacrario Sārandada … e anche
ora, qui, oggi nel Sacrario Cāpāla … Non ti ho già detto, Ānanda, che
c’è separazione, distacco e divisione da tutto quello che ci è caro e
che amiamo? Come potrebbe avvenire che quel che è nato, giunto
all’esistenza, formato e soggetto alla decadenza non decada? Questo non
è possibile. Il Perfetto ha rinunciato, lasciato cadere, lasciato
andare, abbandonato, lasciato, ha rinunciato alla volontà di vivere.
Queste parole inequivocabili sono state esclamate dal Perfetto: “Presto
avrà luogo l’ottenimento del Nibbāna definitivo del Perfetto. Fra tre
mesi il Perfetto otterrà il Nibbāna definitivo”. Per il Perfetto è
impossibile tornare indietro su queste parole. Andiamo nel Salone con il
Tetto Aguzzo nella Grande Foresta, Ānanda».


«E sia, Signore», rispose il venerabile Ānanda, e quando si furono
recati là, il Beato si rivolse al venerabile Ānanda: «Ānanda, vai,
raduna tutti i bhikkhu che vivono nelle vicinanze di Vesālī e falli
incontrare nel salone».


Quando ciò fu fatto, egli informò il Beato. Allora il Beato si alzò dal
luogo in cui sedeva e andò nel salone, ove si mise a sedere nel posto
preparatogli, e rivolse ai bhikkhu queste parole: «Bhikkhu, vi ho
insegnato le cose che ho direttamente conosciuto. Queste cose le dovete
imparare a fondo e mantenerle in essere, svilupparle e attuarle
costantemente, così che questa santa vita possa durare a lungo. Dovete
farlo per il benessere e la felicità di molti, per compassione nei
riguardi del mondo, per il bene, il benessere e la felicità di divinità
e uomini. E quali sono queste cose? Esse sono i quattro fondamenti della
consapevolezza, i quattro retti sforzi, le quattro basi per il successo
spirituale, le cinque facoltà spirituali, i cinque poteri spirituali, i
sette fattori dell’Illuminazione e il Nobile Ottuplice Sentiero. Vi ho
insegnato queste cose, avendole direttamente conosciute. Queste cose
dovete impararle a fondo … per il benessere e la felicità di divinità e
uomini».


Poi il Beato rivolse ai bhikkhu queste parole: «Infatti, bhikkhu, questo
vi dichiaro: dissolversi è nella natura di tutte le formazioni.
Raggiungete la perfezione mediante la diligenza. Presto il Beato otterrà
il Nibbāna definitivo». Così disse il Beato. Avendo il Sublime detto
questo, il Maestro aggiunse:


\begin{quote}
Matura è la mia età e poco mi resta da vivere: \\
vi lascio e vado via, il mio rifugio è pronto. \\
Siate diligenti, consapevoli e virtuosi, o bhikkhu, \\
con pensieri ben concentrati \\
continuate a sorvegliare il vostro cuore. \\
Chi vive diligentemente questo Dhamma e Disciplina \\
abbandonerà il ciclo delle rinascite e porrà fine al dolore.
\end{quote}

Quando fu mattino, il Beato si vestì, prese la ciotola e la veste
superiore e si recò a Vesālī per la questua. Dopo aver fatto la questua
a Vesālī e mentre stava tornando dopo il pasto, rivolse lo sguardo a
Vesālī con lo sguardo di un elefante. Allora egli disse al venerabile
Ānanda: «Ānanda, questa è l’ultima volta che il Perfetto vede Vesālī.
Vieni, Ānanda, andiamo a Bhaṇḍagāma».


«E sia, Signore», rispose il venerabile Ānanda. Allora il Beato viaggiò
verso Bhaṇḍagāma con un grande seguito di bhikkhu. Quando fu là visse a
Bhaṇḍagāma. Là si rivolse ai bhikkhu con queste parole: «Bhikkhu, è a
causa del non aver scoperto, del non aver penetrato quattro cose che sia
io sia voi abbiamo dovuto viaggiare e arrancare in questo lungo cerchio.
Quali quattro? Esse sono la virtù degli Esseri Nobili, la concentrazione
degli Esseri Nobili, la comprensione degli Esseri Nobili e la
liberazione degli Esseri Nobili. Quando però queste quattro cose sono
state scoperte e penetrate, la brama per l’esistenza è eliminata, la
brama che conduce all’esistenza è abolita e non c’è rinnovamento
dell’esistenza».


\suttaRef{D. 16; cf. A. 4:1}


E mentre il Beato stava soggiornando a Bhaṇḍagāma, egli spesso offrì
questo discorso di Dhamma ai bhikkhu: «Questa è la virtù, questa è la
concentrazione, questa è la comprensione. La concentrazione fortificata
dalla virtù reca grandi benefici e grandi frutti, la comprensione
fortificata dalla concentrazione reca grandi benefici e grandi frutti,
il cuore fortificato dalla comprensione si libera completamente dalle
contaminazioni, dalla contaminazione del desiderio sensoriale, dalla
contaminazione dell’esistenza, dalla contaminazione delle opinioni e
dalla contaminazione dell’ignoranza».


Quando il Beato ebbe soggiornato a Bhaṇḍagāma per tutto il tempo che
volle, disse al venerabile Ānanda: «Vieni, Ānanda, andiamo a
Hatthigāma».


«E sia, Signore», rispose il venerabile Ānanda. Allora il Beato viaggiò
verso Hatthigāma con un grande seguito di bhikkhu.


E allo stesso modo visitò Ambagāma e Jambugāma. Quando il Beato ebbe
soggiornato a Jambugāma per tutto il tempo che volle, disse al
venerabile Ānanda: «Vieni, Ānanda, andiamo a Bhoganagara».


«E sia, Signore», rispose il venerabile Ānanda. Allora il Beato viaggiò
verso Bhoganagara con un grande seguito di bhikkhu. Quando fu là visse
nel Sacrario di Ānanda a Bhoganagara. E là rivolse ai bhikkhu queste
parole: «Bhikkhu, vi insegnerò le quattro principali autorità. Ascoltate
e prestate attenzione a quello che dirò».


«E sia, Signore», loro risposero. Il Beato disse: «Bhikkhu, un bhikkhu
può dire: “L’ho udito e imparato dalle labbra stesse del Beato, questo è
il Dhamma, questa è la Disciplina, questo è l’insegnamento del Maestro”.
Oppure un bhikkhu può dire: “In un certo luogo dimorano una comunità con
anziani e guide, l’ho udito e imparato dalle labbra di quella comunità,
questo è il Dhamma, questa è la Disciplina, questo è l’insegnamento del
Maestro”. Oppure un bhikkhu può dire: “In un certo luogo dimora un
anziano bhikkhu che è sapiente, esperto di tradizioni, che ha
memorizzato la Disciplina, che ha memorizzato il Codice, l’ho udito e
imparato dalle labbra di quell’anziano, questo è il Dhamma, questa è la
Disciplina, questo è l’insegnamento del Maestro”».


Ora, quest’affermazione di un bhikkhu non dev’essere né approvata né
disapprovata. Senza che sia approvata o disapprovata, queste sue parole
e sillabe devono essere ben imparate e poi verificate nel Vinaya
(Disciplina) o confermate dai sutta (Discorsi). Se si constata che non
sono verificate nel Vinaya né confermate dai sutta, la conclusione cui
giungere è questa: “Certamente questa non è la parola del Beato. Essa è
stata erroneamente imparata da quel bhikkhu o da quella comunità o da
quegli anziani o da quell’anziano”, e voi di conseguenza dovete
rifiutarla. Se tuttavia si constata che sono verificate nel Vinaya e
confermate dai sutta, la conclusione cui giungere è questa: “Certamente
questa è la parola del Beato. Essa è stata giustamente imparata da quel
bhikkhu o da quella comunità o da quegli anziani o da quell’anziano”.
Dovete ricordare queste quattro principali autorità».


\suttaRef{D. 16; cf. A. 4:180}


E mentre il Beato stava soggiornando nel Sacrario di Ānanda a
Bhoganagara, egli spesso offrì questo discorso di Dhamma ai bhikkhu:
«Questa è la virtù, questa è la concentrazione, questa è la
comprensione. La concentrazione fortificata dalla virtù reca grandi
benefici e grandi frutti, la comprensione fortificata dalla
concentrazione reca grandi benefici e grandi frutti, il cuore
fortificato dalla comprensione si libera completamente dalle
contaminazioni, dalla contaminazione del desiderio sensoriale, dalla
contaminazione dell’esistenza, dalla contaminazione delle opinioni e
dalla contaminazione dell’ignoranza».


\suttaRef{D. 16}


Quando il Beato ebbe soggiornato a Bhoganagara per tutto il tempo che
volle, disse al venerabile Ānanda: «Vieni, Ānanda, andiamo a Pāvā».


«E sia, Signore», rispose il venerabile Ānanda. Allora il Beato viaggiò
verso Pāvā con un grande seguito di bhikkhu. Quando fu là visse nel
boschetto di manghi a Pāvā, che apparteneva a Cunda, il figlio
dell’orafo.


Cunda il figlio dell’orafo sentì dire che il Beato soggiornava nel suo
boschetto. Egli allora andò dal Beato e, dopo avergli prestato omaggio,
si mise a sedere da un lato. Allora il Beato lo istruì, esortò,
risvegliò e incoraggiò con un discorso di Dhamma. Successivamente Cunda
disse al Beato: «Signore, che il Beato con il Saṅgha dei bhikkhu accetti
da me il pasto di domani».


Il Beato acconsentì in silenzio. Quando Cunda vide che il Beato aveva
accettato, si alzò dal posto in cui sedeva e, dopo aver prestato
omaggio, andò via girandogli a destra.


Quando la notte fu terminata egli, che aveva fatto preparare buon cibo
di vario genere nella sua casa e carne macinata di
maiale\footnote{“Carne macinata di maiale” (\emph{sūkara-maddava}): su tale espressione si discute da moltissimo tempo. Il Commentario a questo sutta dice: «Si tratta di carne già in vendita in un mercato (si veda Vin. Mv. 6:31), di un maiale \emph{ekajeṭṭhaka}, non troppo giovane né troppo anziano. Sembra che si tratti di un piatto morbido e succulento; significa che era preparato e cotto con cura. (Alcuni dicono però che \emph{sūkara-maddava} indica la ricetta di riso bollito fino a divenire morbido con cinque ingredienti aggiunti, tutti di provenienza vaccina, come se il nome di questa bevanda fosse “bibita di mucca”. Altri ancora dicono che sia un tipo di elisir, che rientrava nella scienza degli elisir, e che Cunda lo preparò pensando “che il Beato non ottenga il Nibbāna finale”. Le divinità dei quattro continenti, però, con le loro duemila isole, infusero in esso un’essenza nutritiva)». Il passo tra parentesi tonde non si trova in tutte le edizioni. Oltre a questo, il Commentario all’\emph{Udāna} afferma: «\emph{Sūkara-maddava}, secondo il Grande Commentario Cingalese (non più esistente), è carne di maiale tenera e succulenta in vendita al mercato. Alcuni dicono tuttavia che non si tratta di carne di maiale ma di germogli di bambù calpestati da maiali. Altri ritengono che sia un genere di funghi che crescono in luoghi calpestati da maiali. Inoltre, altri ancora affermano che sia un elisir, e che l’orafo, avendo sentito dire che quel giorno il Beato stava per ottenere il Nibbāna definitivo, pensò: “Forse dopo averlo consumato vivrà più a lungo” e lo offrì al Maestro con il desiderio di prolungare la sua vita» (Commentario a Ud. 8:5). Mangiare carne era consentito ai monaci dal Buddha a tre condizioni: che non si fosse visto, udito o sospettato che l’animale era stato ucciso per colui che lo avrebbe mangiato (M. 55, Vin. Mv. 6:31, cf. A. 4:44; anche Vin. Cv. 7:4 cit. nel \hyperlink{cap-13-Devadatta#pag298}{}). Probabilmente non riusciremo mai a sapere l’esatto significato. È stato scelto “carne macinata di maiale” perché elusivo e vicino all’espressione originale: \emph{sūkara} = maiale; \emph{maddava} = dolce.}
in abbondanza, annunciò che era giunto il
momento: «È ora, Signore, il pasto è pronto». Allora, essendo mattino,
il Beato si vestì, prese la ciotola e la veste superiore e andò con il
Saṅgha dei bhikkhu da Cunda, il figlio dell’orafo. Egli si mise a sedere
nel posto preparatogli. Poi disse a Cunda: «Servi a me quella carne
macinata di maiale che hai preparato, Cunda, ma servi tutto l’altro cibo
che hai preparato al Saṅgha dei bhikkhu».


«E sia, Signore», rispose Cunda, e così fece. Allora il Beato gli disse:
«Cunda, se ne è rimasta un po’ di carne di maiale macinata, interrala in
una buca. Oltre al Beato non vedo nessuno in questo mondo con i suoi
deva, con i suoi Māra e con le sue divinità, in questa generazione con i
suoi monaci e brāhmaṇa, con i suoi principi e uomini, che sia in grado
di digerirla se la mangia».


«E sia, Signore», rispose Cunda, e interrò in una buca la carne macinata
di maiale rimasta. Allora andò dal Beato e, dopo avergli prestato
omaggio, si mise a sedere da un lato. Allora il Beato lo istruì con un
discorso di Dhamma, dopo il quale si alzò dal posto in cui sedeva e se
ne andò.


Fu dopo che il Beato aveva mangiato il cibo offerto da Cunda, il figlio
dell’orafo, che una grave malattia lo aggredì, con un flusso di sangue
accompagnato da dolori violenti e mortali. Egli la sopportò senza
lamentarsi, consapevole e pienamente presente. Poi egli disse al
venerabile Ānanda: «Vieni, Ānanda, andiamo a Kusinārā».


«E sia, Signore», rispose il venerabile Ānanda.


Durante il viaggio il Beato lasciò la strada e si recò ai piedi di un
albero. Egli disse al venerabile Ānanda: «Per favore, Ānanda, ripiega la
mia veste in quattro e distendila, sono stanco, mi metterò a sedere».


«E sia, Signore», rispose il venerabile Ānanda. Il Beato si mise a
sedere nel posto preparatogli. Quando lo ebbe fatto, disse: «Per favore,
Ānanda, portami dell’acqua. Ho sete e berrò».


Il venerabile Ānanda disse: «Signore, sono appena passati circa
cinquecento carri, l’acqua è stata smossa dalle ruote, scorre poco ed è
densa e torbida. Il fiume Kakutthā, gradevole e con le sponde piane, con
la sua acqua chiara, piacevole e fresca non è molto distante. Il Beato
può bere lì e rinfrescare le sue membra».


Il Beato chiese una seconda volta e ricevette la stessa risposta. Una
terza volta il Beato disse: «Per favore, Ānanda, portami dell’acqua. Ho
sete e berrò».


«E sia, Signore», rispose il venerabile Ānanda. Prese una ciotola e si
recò al ruscello. Allora il ruscello, che era stato smosso dalle ruote,
scorreva poco ed era denso e torbido, ma appena il venerabile Ānanda lo
raggiunse iniziò a scorrere chiaro e limpido. Egli si stupì. Poi prese
dell’acqua nella ciotola, tornò dal Beato e gli raccontò quello che era
avvenuto, aggiungendo: «Signore, che il Beato beva l’acqua, che il Beato
beva l’acqua». E il Beato bevve l’acqua.


\suttaRef{D. 16; Ud. 8:5}


In quel momento un Malla di nome Pukkusa, un discepolo di Āḷāra Kālāma
passò per la strada che andava da Kusinārā a Pāvā. Egli vide il Beato
che sedeva ai piedi dell’albero e andò da lui. Dopo avergli prestato
omaggio si mise a sedere da un lato e disse: «È meraviglioso, Signore, è
magnifico il sereno dimorare che ottengono coloro che abbracciano la
vita religiosa. Una volta, quando Āḷāra Kālāma era in viaggio, lasciò la
strada e si mise a sedere ai piedi di un albero che stava nei pressi per
dimorarvi durante il giorno. Allora circa cinquecento carri gli
passarono molto vicini. In seguito arrivò un uomo che seguiva quella
carovana di carri, ed egli si avvicinò ad Āḷāra Kālāma e gli chiese:
“Signore, hai visto cinquecento carri passare?” – “No, amico, non li ho
visti”. – “Signore, ma non hai sentito il loro rumore?” – “No, amico,
non ho sentito il loro rumore”. – “Signore, ma allora dormivi?” – “No,
amico, non dormivo”. – “Signore, ma eri cosciente?” – “Sì, amico, ero
cosciente”. – “Allora, Signore, eri cosciente e sveglio ma non hai né
visto i cinquecento carri passare vicino a te né sentito il loro rumore,
benché la tua veste superiore sia sporca di fango?” – “Proprio così,
amico”. Allora, Signore, quell’uomo pensò: “È meraviglioso, è magnifico
il sereno dimorare che ottengono coloro che abbracciano la vita
religiosa perché, mentre sono coscienti e svegli, loro non vedono
cinquecento carri passare né sentono il loro rumore!” E, dopo aver
espresso la sua grande fiducia in Āḷāra Kālāma, se ne andò per la sua
strada».


«Cosa ne pensi, Pukkusa? Che cosa è meno probabile e più difficile che
un uomo cosciente e sveglio non veda cinquecento carri che gli passano
molto vicini né senta il loro rumore, oppure che un uomo cosciente e
sveglio mentre c’è una pioggia torrenziale con fulmini che lampeggiano e
tuoni che rombano non veda né senta il rumore?».


«Signore, che cosa vuoi che siano cinquecento, seicento, settecento,
ottocento, novecento carri, o perfino mille carri? È molto meno
probabile e molto più difficile che un uomo cosciente e sveglio mentre
c’è una pioggia torrenziale con fulmini che lampeggiano e tuoni che
rombano non veda né senta il rumore».


«Una volta, Pukkusa, vivevo nei pressi di Ātumā in un ricovero per la
trebbiatura. Allora c’era una pioggia torrenziale con fulmini che
lampeggiavano e tuoni che rombavano, e due aratori, che erano fratelli,
erano stati uccisi, come pure quattro buoi. Una gran folla uscì allora
da Ātumā e si recò dai due fratelli e dai quattro buoi che erano stati
uccisi. Quella volta, però, io ero uscito dal ricovero per la
trebbiatura e stavo facendo la meditazione camminata all’aperto, davanti
all’entrata. Un uomo si separò dalla folla e, dopo avermi prestato
omaggio, si mise in piedi da un lato. Io gli chiesi: “Perché si è
riunita questa gran folla, amico?” – “Signore, c’è stata una pioggia
torrenziale con fulmini che lampeggiavano e tuoni che rombavano, e due
aratori, che erano fratelli, sono stati uccisi, come pure quattro buoi.
Ecco perché qui si è riunita questa gran folla. Tu, però, Signore,
dov’eri?” – “Ero qui, amico”. – “Signore, ma tu hai visto?” – “No,
amico, non ho visto”. – “Signore, ma non hai sentito il rumore?” – “No,
amico, non ho sentito il rumore”. – “Signore, ma allora dormivi?” – “No,
amico, non dormivo”. – “Signore, ma eri cosciente?” – “Sì, amico, ero
cosciente”. – “Allora, Signore, eri cosciente e sveglio mentre c’era una
pioggia torrenziale con fulmini che lampeggiavano e tuoni che rombavano,
ma non hai né visto né sentito il rumore?” – “Proprio così, amico”.
Allora quell’uomo pensò: “È meraviglioso, è magnifico il sereno dimorare
che ottengono coloro che abbracciano la vita religiosa perché, mentre
sono coscienti e svegli quando c’è una pioggia torrenziale con fulmini
che lampeggiano e tuoni che rombano, loro non vedono né sentono il
rumore”. E, dopo aver espresso la sua totale fiducia in me, mi prestò
omaggio e se ne andò, girandomi a destra».


«Signore, la fiducia che avevo in Āḷāra Kālāma è come se fosse stata
spazzata via da un forte vento o portata via da un fiume che scorre
rapido. Magnifico, Signore, magnifico, Signore! … Io prendo rifugio nel
Beato, nel Dhamma e nel Saṅgha. Da oggi che il Beato mi consideri un suo
seguace che ha preso rifugio in lui per tutto il tempo che durerà il suo
respiro».


Allora Pukkusa il Malla disse a un uomo: «Per favore, procurami due
vesti stampate in oro pronte da indossare».


«Sì, Signore», rispose l’uomo, e gliele portò. Allora Pukkusa le porse
al Beato: «Signore, che il Beato accetti da me per compassione queste
due vesti stampate in oro pronte da indossare».


«Allora, Pukkusa, puoi vestire me con una e Ānanda con l’altra».


«Sì, Signore», egli rispose, e lo fece. Allora il Beato istruì, ammonì,
risvegliò e incoraggiò Pukkusa il Malla con un discorso di Dhamma, dopo
il quale Pukkusa si alzò dal posto in cui sedeva, prestò omaggio al
Beato e andò via, girandogli a destra.


Subito dopo che egli se ne fu andato, il venerabile Ānanda mise le due
vesti stampate in oro pronte da indossare sul corpo del Beato. Allora,
però, sembrò che la loro brillantezza si estinguesse. Il venerabile
Ānanda disse: «È meraviglioso, Signore, è magnifico quanto è puro e
luminoso il colore della pelle del Beato! Quando ho messo queste due
vesti stampate in oro pronte da indossare sul corpo del Beato, è
sembrato che la loro brillantezza si estinguesse».


«È così, Ānanda, è così. Due sono le circostanze in cui il colore della
pelle del Perfetto diventa eccezionalmente chiaro e luminoso. Quali due?
Alla vigilia della scoperta della suprema e piena Illuminazione e alla
vigilia del suo ottenimento del Nibbāna definitivo, con l’elemento
Nibbāna privo del residuo del passato attaccamento. Infatti, Ānanda, è
nell’ultima veglia della prossima notte, tra i due alberi sāla gemelli
nel boschetto di alberi \emph{sāla} dei Malla sulla curva dove si svolta
verso Kusinārā, che il Beato otterrà il Nibbāna definitivo».


«E sia, Signore», rispose il venerabile Ānanda.


Allora il Beato si avvicinò al fiume Kakutthā con una grande comunità di
bhikkhu, ed entrò nel fiume, si fece il bagno e bevve, dopo di che ne
uscì e andò in un boschetto di manghi. Là disse al venerabile Cundaka:
«Cundaka, per favore, piega la mia veste superiore in quattro e
distendila. Sono stanco e voglio giacere».


Allora il Beato si mise a giacere sul suo lato destro nella posizione
del leone, con un piede sovrapposto all’altro, consapevole e pienamente
presente, dopo aver deciso il momento in cui si sarebbe svegliato. E il
venerabile Cundaka si mise seduto lì, di fronte al Beato».


\suttaRef{D. 16}


Il Beato disse al venerabile Ānanda: «Ānanda, è possibile che qualcuno
possa far provare rimorso a Cunda, figlio dell’orafo, in questo modo:
“Non è un guadagno per te, è una perdita per te, Cunda, che il Perfetto
abbia ottenuto il Nibbāna definitivo dopo aver ricevuto da te l’ultimo
cibo in elemosina”. Qualsiasi rimorso di tal genere dev’essere
neutralizzato in questo modo: “È un guadagno per te, è un gran guadagno,
Cunda, che il Perfetto abbia ottenuto il Nibbāna definitivo dopo aver
ricevuto da te l’ultimo cibo in elemosina. Ho udito e imparato questo
dalle labbra stesse del Beato, amico Cunda: ‘Questi due tipi di cibo
offerto in elemosina hanno uguale frutto e uguale maturazione, e il loro
frutto e la loro maturazione è molto maggiore di qualsiasi altro. Quali
due? Essi sono il cibo offerto in elemosina dopo aver mangiato il quale
un Perfetto scopre la suprema e piena Illuminazione e il cibo offerto in
elemosina dopo aver mangiato il quale un Perfetto ottiene il Nibbāna
definitivo con l’elemento Nibbāna privo del residuo del passato
attaccamento. Cunda, il figlio dell’orafo ha accumulato un’azione che
condurrà alla longevità, a una buona posizione, alla felicità, alla
buona fama e al paradiso’ ”. Qualsiasi rimorso dev’essere neutralizzato
in questo modo».


Conoscendo il significato di ciò, il Beato esclamò queste parole:


\begin{quote}
Quando un uomo dona, il suo merito crescerà, \\
nessuna ostilità può crescere in chi è contenuto. \\
Chi è abile evita il male, otterrà il Nibbāna \\
ponendo fine alla brama, all’odio e all’illusione.
\end{quote}

\suttaRef{D. 16; Ud. 8:5}


Allora il Beato disse al venerabile Ānanda: «Vieni, Ānanda, andiamo
sull’altra sponda del fiume Hiraññavatī, nel boschetto di alberi \emph{sāla}
dei Malla sulla curva dove si svolta verso Kusinārā».


«E sia, Signore», rispose il venerabile Ānanda. Allora il Beato andò con
una grande comunità di bhikkhu sull’altra sponda del fiume Hiraññavatī,
nel boschetto di alberi \emph{sāla} dei Malla sulla curva dove si svolta
verso Kusinārā. Poi egli disse al venerabile Ānanda: «Ānanda, per
favore, preparami un letto con la testa a nord tra i due alberi \emph{sāla}
gemelli. Sono stanco e voglio giacere».


«E sia, Signore», rispose il venerabile Ānanda, e così fece. Allora il
Beato si mise a giacere sul suo lato destro nella posizione del leone,
con un piede sovrapposto all’altro, consapevole e pienamente presente.


In quell’occasione, i due alberi \emph{sāla} gemelli erano completamente
ricoperti di fiori, benché non fosse la giusta stagione. Si sparsero, si
diffusero e cosparsero il corpo del Beato per venerazione nei suoi
riguardi. E dei celestiali fiori di \emph{mandārava} e della celestiale
polvere di legno di sandalo caddero dal cielo e si sparsero, si
diffusero e cosparsero il corpo del Beato per venerazione nei suoi
riguardi. E della celestiale musica risuonò e delle celestiali canzoni
furono cantate nel cielo per venerazione nei suoi riguardi.


Allora il Beato disse ad Ānanda: «Ānanda, i due alberi \emph{sāla} gemelli
sono completamente ricoperti di fiori, benché non sia la giusta
stagione. Si spargono, si diffondono e cospargono il corpo del Beato per
venerazione nei suoi riguardi. E dei celestiali fiori di mandārava e
della celestiale polvere di legno di sandalo cadono dal cielo e si
spargono, si diffondono e cospargono il corpo del Beato per venerazione
nei suoi riguardi. E della celestiale musica risuona e delle celestiali
canzoni sono cantate nel cielo per venerazione nei suoi riguardi. Non è
però così che si onora, rispetta, ossequia, riverisce e venera un
Perfetto: è piuttosto un bhikkhu o una bhikkhuṇī, un seguace laico o una
seguace laica che vive in accordo con il Dhamma, che entra nella giusta
strada, che cammina nel Dhamma, che onora, rispetta, ossequia, riverisce
e venera un Perfetto con la maggiore venerazione possibile. Perciò,
Ānanda, addestratevi in questo modo: “Noi vivremo nella via del Dhamma,
entreremo nella giusta strada e cammineremo nel Dhamma”».


Proprio in quel momento il venerabile Upavāna si trovava in piedi di
fronte al Beato, facendogli aria con un ventaglio. Allora il Beato lo
congedò, dicendo: «Vai, bhikkhu, non stare di fronte a me».


Il venerabile Ānanda pensò: «Il venerabile Upavāna per lungo tempo è
stato attendente del Beato, gli è stato vicino e lo ha accompagnato da
vicino. All’ultimo momento, tuttavia, il Beato lo congeda, dicendo:
“Vai, bhikkhu, non stare di fronte a me”. Qual è la ragione?». Egli fece
questa domanda al Beato, che rispose: «Ānanda, la maggior parte delle
divinità provenienti da dieci sistemi di mondi sono giunte per vedere il
Beato. Per dodici leghe tutt’intorno al boschetto di alberi \emph{sāla} non
c’è posto della grandezza corrispondente alla punta d’un crine di
cavallo che non sia occupato da divinità. Stanno protestando: “Siamo
giunti da lontano per vedere il Perfetto. Di tanto in tanto Esseri
Perfetti sorgono nel mondo, realizzati e completamente illuminati.
Questa notte, nell’ultima veglia, avrà luogo l’ottenimento del Nibbāna
definitivo da parte del Perfetto. E questo eminente bhikkhu sta di
fronte al Beato e ci ostacola la vista, così che all’ultimo momento non
saremo in grado di vedere il Perfetto”. Le divinità stanno protestando,
Ānanda».


«Signore, ma quali divinità ha in mente il Beato?».


«Ci sono divinità che percepiscono la terra nello spazio. Si stanno
strappando i capelli e piangono, alzano le braccia e piangono, cadono e
rotolano avanti e indietro, gridando: “Così presto il Beato otterrà il
Nibbāna definitivo! Così presto il Sublime otterrà il Nibbāna
definitivo! Così presto l’Occhio svanirà dal mondo!”. E ci sono divinità
che percepiscono la terra nella terra che stanno facendo le stesse cose.
Quelle divinità, però, che sono libere dalla brama si rassegnano,
consapevoli e pienamente presenti: “Le formazioni sono impermanenti.
Come potrebbe avvenire che quel che è nato, giunto all’esistenza,
formato e soggetto alla decadenza non decada? Questo non è possibile”».


«Signore, prima i bhikkhu che trascorrevano la stagione delle piogge in
luoghi differenti erano soliti venire a visitare il Perfetto. In questo
modo erano in grado di vedere e di prestare omaggio a bhikkhu
ammirevoli. Signore, ora, però, quando il Beato sarà andato non saremo
più in grado di farlo».


«Ānanda, ci sono quattro luoghi che possono essere d’ispirazione per un
uomo di rango dotato di fede. Quali quattro? Qui il Perfetto è nato:
questo è un luogo da vedere che può essere d’ispirazione per un uomo di
rango dotato di fede. Qui il Perfetto ha scoperto la suprema e piena
Illuminazione: questo è un luogo da vedere che può essere d’ispirazione
per un uomo di rango dotato di fede. Qui il Perfetto ha messo in moto
l’incomparabile Ruota del Dhamma: questo è un luogo da vedere che può
essere d’ispirazione per un uomo di rango dotato di fede. Qui il
Perfetto ha ottenuto il Nibbāna definitivo privo di residui del passato
attaccamento: questo è un luogo da vedere che può essere d’ispirazione
per un uomo di rango dotato di fede. Bhikkhu e bhikkhuṇī dotati di fede,
seguaci laici e seguaci laiche verranno, dicendo: “Qui il Perfetto è
nato”, “Qui il Perfetto ha scoperto la piena e suprema Illuminazione”,
“Qui il Perfetto ha messo in moto l’incomparabile Ruota del Dhamma”,
“Qui il Perfetto ha ottenuto il Nibbāna definitivo privo di residui del
passato attaccamento”. E tutti coloro che viaggiano per visitare questi
sacrari con cuore fiducioso, alla dissoluzione del corpo riappariranno
in una destinazione felice, perfino in un paradiso celeste».


«Signore, come dobbiamo comportarci con le donne?».


«Non guardatele, Ānanda».


«Signore, se le vediamo, come dobbiamo comportarci?».


«Non rivolgetevi a loro, Ānanda».


«Signore, se ci rivolgiamo a loro, come dobbiamo comportarci?».


«La consapevolezza deve essere mantenuta in essere, Ānanda».


«Signore, come dobbiamo comportarci con i resti del Perfetto?».


«Ānanda, non preoccupatevi di venerare i resti del Perfetto. Per favore,
dedicatevi al vostro scopo, dimorate diligenti, ardenti e
autocontrollati per il vostro bene. Ci sono saggi guerrieri, brāhmaṇa e
capifamiglia che credono nel Perfetto: loro provvederanno a venerare i
resti del Perfetto».


«Signore, come si dovrebbero però trattare i resti del Perfetto?».


«Trattate i resti del Perfetto nello stesso modo in cui sono trattati i
resti di un Monarca Universale\footnote{Il mito indiano del Monarca Universale che gira la Ruota della Giustizia (in pāli: \emph{cakkavattī}; sanscrito: \emph{cakravartin}) è offerto in D. 26 e M. 129} che gira la Ruota della Giustizia».


«Signore, come si dovrebbero però trattare i resti di un Monarca
Universale che gira la Ruota della Giustizia?».


«I suoi resti vengono avvolti in una stoffa nuova, poi vengono avvolti
in panno di cotone ben battuto, e poi vengono avvolti in una stoffa
nuova. E, procedendo in questo modo, vengono avvolti in cinquecento
strati doppi. Poi vengono collocati in un recipiente di olio, fatto di
ferro, che viene chiuso in un altro recipiente [di ferro]. Poi si
accende una pira con tutti i tipi di profumi e i resti vengono bruciati.
Poi gli si erige un monumento a un crocevia. Così è che si trattano i
resti di un Monarca Universale che gira la Ruota della Giustizia. E i
resti del Perfetto devono essere trattati nello stesso modo. Il
monumento del Perfetto deve essere eretto a un crocevia, e chiunque
metterà fiori e profumi su di esso, lo imbiancherà, lo venererà o
proverà in cuor suo fiducia quando si troverà lì, ciò sarà per lungo
tempo a vantaggio del suo benessere e della sua felicità. Costoro sono i
quattro che sono degni di un monumento. Quali quattro? Un Perfetto
realizzato e completamente illuminato, un \emph{Paccekabuddha}, il discepolo
di un Perfetto che è un Arahant e un Monarca Universale che gira la
Ruota della Giustizia. E a quale scopo ognuno di questi quattro è degno
di un monumento? Sono in molti coloro che provano fiducia in cuor loro,
pensando: “Questo è il monumento di quel Beato, realizzato e
completamente illuminato”, “Questo è il monumento di quel Beato, un
\emph{Paccekabuddha}”, “Questo è il monumento di un discepolo di quel Beato”
o “Questo è il monumento di quel retto e legittimo sovrano”. Quando lì
provano fiducia in cuor loro, allora alla dissoluzione del corpo, dopo
la morte, riappariranno in una destinazione felice, perfino in un
paradiso celeste».


Allora il venerabile Ānanda entrò in una dimora e si mise in piedi
appoggiato alla porta e pianse: «Sono ancora solo un allievo il cui
compito non è stato portato a termine. Il mio insegnante sta per
ottenere il Nibbāna definitivo, il mio insegnante che ha compassione di
me!».


Allora il Beato chiese ai bhikkhu: «Bhikkhu, dov’è Ānanda?».


«Signore, è appena entrato in una dimora, e si è messo in piedi
appoggiato alla porta piangendo: “Sono ancora solo un allievo il cui
compito non è stato portato a termine. Il mio insegnante sta per
ottenere il Nibbāna definitivo, il mio insegnante che ha compassione di
me!”».


Il Beato disse a un bhikkhu: «Vieni, bhikkhu, va da Ānanda e digli
queste parole a nome mio: “Il Maestro ti chiama, amico Ānanda”».


«E sia, Signore», rispose il bhikkhu, ed egli andò dal venerabile Ānanda
e gli disse: «Il Maestro ti chiama, amico Ānanda».


«E sia, amico», rispose il venerabile Ānanda, ed egli andò dal Beato e,
dopo avergli prestato omaggio, si mise in piedi da un lato. Il Beato gli
disse: «Basta così, Ānanda, non addolorarti, non lamentarti. Non ti ho
detto molte volte che c’è separazione, distacco e divisione da tutto
quello che ci è caro e che amiamo? Come potrebbe avvenire che quel che è
nato, giunto all’esistenza, formato e soggetto alla decadenza non
decada? Questo non è possibile. Ānanda, tu hai per lungo tempo e
continuamente assistito il Perfetto con atti corporei di gentilezza
amorevole, in modo servizievole, volenteroso, con sincerità e senza
riserve, e altrettanto con atti verbali e mentali. Tu hai ottenuto
meriti, Ānanda. Continua a sforzarti e presto sarai libero dalle
contaminazioni».


Allora il Beato si rivolse ai bhikkhu con queste parole: «Bhikkhu, anche
gli esseri realizzati e completamente illuminati del passato hanno avuto
attendenti che si comportarono con loro come Ānanda ha fatto con me. E
anche gli esseri realizzati e completamente illuminati del futuro
avranno attendenti che si comporteranno con loro come Ānanda ha fatto
con me. Ānanda è saggio, bhikkhu. Egli sa: “Questo è il momento che i
bhikkhu vengano e vedano il Beato. Questo è il momento che le bhikkhuṇī
vengano e vedano il Beato. Questo è il momento che i seguaci laici … che
le seguaci laiche vengano e vedano il Beato. Questo è il momento che i
re, i ministri dei re, i settari e i discepoli dei settari vengano e
vedano il Beato».


\suttaRef{D. 16}


«Quattro sono le cose meravigliose e magnifiche in un Monarca Universale
che gira la Ruota della Giustizia. Quali quattro? Se un’assemblea di
nobili guerrieri, di brāhmaṇa, di capifamiglia o monaci giunge per
vederlo, l’assemblea è contenta di vederlo. Se egli parla, l’assemblea è
lieta di ascoltarlo. Quando egli però torna a stare in silenzio,
l’assemblea non è ancora paga. Allo stesso modo ci sono quattro cose
meravigliose e magnifiche in Ānanda. Quali quattro? Se un’assemblea di
bhikkhu, di bhikkhuṇī, di seguaci laici o di seguaci laiche giunge per
vedere Ānanda, l’assemblea è contenta di vederlo. Se egli parla,
l’assemblea è lieta di ascoltarlo. Quando egli però torna a stare in
silenzio, l’assemblea non è ancora paga».


\suttaRef{D. 16; A. 4:129-30}


Dopo che egli ebbe parlato in questo modo, Ānanda disse: «Signore, che
il Beato non ottenga il Nibbāna definitivo in questa piccola città con i
muri fatti di fango, in questa città isolata, in questa cittadina di
borgata. Ci sono altre grandi città come Campā, Rājagaha, Sāvatthī,
Sāketa, Kosambī e Benares. Che il Beato ottenga il Nibbāna definitivo
dove ci sono molti nobili guerrieri prominenti, brāhmaṇa e capifamiglia
che credono nel Perfetto. Loro venereranno i resti del
Perfetto».\footnote{Secondo il Commentario, a questo punto fu pronunciato il \emph{Mahā Sudassana Sutta} (D. 17).}


«Non dire così, Ānanda, non dire “una piccola città con i muri fatti di
fango, una città isolata, una cittadina di borgata”. Qui una volta c’era
un re chiamato Sudassana il Grande. Egli era un retto e legittimo
Monarca Universale che girava la Ruota della Giustizia, un conquistatore
dei quattro angoli del mondo, che rese stabile il suo regno e che
possedette i sette tesori. La sua capitale era Kusinārā, che allora era
chiamata Kusavatī, ed essa era larga dodici leghe da est a ovest, e
ampia sette leghe da nord a sud. La regia capitale, Kusavatī, era
potente e prosperosa con tanti abitanti, così affollata di gente e colma
di abbondanza come la regia città capitale degli dèi chiamata
Ālakamandā. Nella regia capitale di Kusavatī non mancarono mai i dieci
tipi di suoni, ossia, di elefanti, cavalli, carrozze, tamburi,
tamburelli, liuti, canzoni, cembali, gong, e delle esclamazioni
“Mangia!” “Bevi!” “Assaggia!”, quali dieci suoni».


\suttaRef{D. 16, 17}


«Ora, Ānanda, va a Kusinārā e annuncia ai Malla di Kusinārā: “Questa
notte, Vāseṭṭha, nell’ultima veglia, avrà luogo l’ottenimento del
Nibbāna definitivo del Perfetto. Uscite, Vāseṭṭha, uscite, per non
pentirvi dopo pensando: ‘Ha avuto luogo l’ottenimento del Nibbāna
definitivo del Perfetto nel territorio della nostra città e noi non
abbiamo visto il Perfetto nell’ultima ora’ ”».


«E sia, Signore», rispose il venerabile Ānanda. Egli si vestì, prese la
ciotola e la veste superiore ed entrò a Kusinārā con un altro bhikkhu.
In quel momento i Malla di Kusinārā si trovavano nel loro salone per le
riunioni per alcuni affari e altre cose ancora. Il venerabile Ānanda
andò nel salone per le riunioni e annunciò: «Questa notte, Vāseṭṭha,
nell’ultima veglia, avrà luogo l’ottenimento del Nibbāna definitivo del
Perfetto. Uscite, Vāseṭṭha, uscite, per non pentirvi dopo pensando: “Ha
avuto luogo l’ottenimento del Nibbāna definitivo del Perfetto nel
territorio della nostra città e noi non abbiamo visto il Perfetto
nell’ultima ora”».


Quando loro udirono queste parole dal venerabile Ānanda, i Malla con i
loro giovani, con le loro fanciulle e matrone furono sgomenti e
atterriti. Sopraffatti dal dolore, alcuni si strapparono i capelli e
piansero, altri alzarono le braccia e piansero, altri ancora caddero e
rotolarono avanti e indietro, gridando: «Così presto il Beato otterrà il
Nibbāna definitivo! Così presto il Sublime otterrà il Nibbāna
definitivo! Così presto l’Occhio svanirà dal mondo!».


Sgomenti e atterriti, sopraffatti dal dolore com’erano, i Malla con i
loro giovani, con le loro fanciulle e matrone andarono con il venerabile
Ānanda nel boschetto di alberi \emph{sāla} dei Malla sulla curva dove si
svolta verso Kusinārā. Allora egli pensò: «Se lascio che i Malla di
Kusinārā salutino il Beato singolarmente, la notte sarà terminata prima
che possano finire. E se li facessi salutare il Beato con una
rappresentanza di ogni clan in questo modo: “Signore, il Malla chiamato
così-e-così, con i suoi figli, con sua moglie e con il suo seguito e i
suoi amici, saluta il Beato con il suo capo ai piedi del Beato”?». E
così egli fece. E in questo modo egli fece loro salutare il Beato nella
prima veglia della notte.


Un’asceta itinerante chiamato Subhadda, però, soggiornava in quel
momento a Kusinārā. Egli sentì dire: «Questa notte, nell’ultima veglia,
avrà luogo l’ottenimento del Nibbāna definitivo del monaco Gotama».
Allora egli pensò: «Ho sentito da anziani di rilievo, insegnanti tra gli
asceti itineranti, che i Perfetti appaiono nel mondo di tanto in tanto,
realizzati e completamente illuminati. E questa notte, nell’ultima
veglia, avrà luogo l’ottenimento del Nibbāna definitivo del monaco
Gotama. Benché in me ci sia questo dubbio, ho tuttavia fiducia nel
monaco Gotama, che egli possa insegnarmi il Dhamma in modo che io possa
liberarmi da questo dubbio».


Egli si recò nel boschetto di alberi sāla dei Malla sulla curva dove si
svolta verso Kusinārā, si avvicinò al venerabile Ānanda e gli disse
tutto quello che aveva pensato, aggiungendo: «Se solo potessi vedere il
monaco Gotama, Maestro Ānanda».


Il venerabile Ānanda disse: «Basta così, amico Subhadda, non disturbare
il Perfetto. Il Beato è stanco».


L’asceta itinerante Subhadda fece la stessa richiesta una seconda e una
terza volta, e ricevette la stessa risposta. Il Beato ascoltò la loro
conversazione. Allora egli disse al venerabile Ānanda: «Basta così,
Ānanda, non impedire a Subhadda di avvicinarsi, lascia che veda il
Perfetto. Qualsiasi cosa voglia chiedermi, lo farà solo per ottenere la
conoscenza, non per disturbarmi, e lui capirà velocemente quel che gli
dirò».


Allora il venerabile Ānanda disse all’asceta itinerante Subhadda: «Vai,
amico Subhadda, il Beato ti dà il permesso».


Egli andò dal Beato e scambiò con lui dei saluti e, quando furono
terminati i formali doveri di reciproca cortesia, egli si mise a sedere
da un lato. Allora egli disse al Beato: «Maestro Gotama, ci sono questi
monaci e brāhmaṇa, ognuno con la propria comunità, con il proprio
gruppo, che conducono un gruppo, ognuno è un rinomato e famoso filosofo,
considerato da molti come un santo. Intendo Pūraṇa Kassapa, Makkhali
Gosāla, Ajita Kesakambali, Pakudha Kaccāyana, Sañjaya Belaṭṭhiputta e
Nigaṇṭha Nāthaputta. Hanno avuto conoscenza diretta, come loro stessi
ritengono, oppure alcuni hanno avuto conoscenza diretta e altri no?


«Basta così, Subhadda. Se tutti hanno avuto conoscenza diretta, come
loro ritengono, e altri no, lascia che sia. Ti insegnerò il Dhamma,
Subhadda. Ascolta e presta bene attenzione a quello che dirò».


«E sia, Signore».


«Subhadda, in qualsiasi Dhamma e Disciplina sia assente il Nobile
Ottuplice Sentiero, là è assente il (primo) monaco, è assente il secondo
monaco, è assente il terzo monaco, è assente il quarto
monaco.\footnote{I quattro “monaci” sono coloro che hanno realizzato la condizione di Chi è Entrato nella Corrente, di Chi Torna una Sola Volta, di Chi è Senza Ritorno, di Arahant} In qualsiasi Dhamma e Disciplina sia presente
il Nobile Ottuplice Sentiero, là è presente il (primo) monaco, è
presente il secondo monaco, è presente il terzo monaco, è presente il
quarto monaco. Il Nobile Ottuplice Sentiero è presente in questo Dhamma
e Disciplina, Subhadda, ed è solo qui che è presente il (primo) monaco,
è presente il secondo monaco, è presente il terzo monaco, è presente il
quarto monaco. Le altre dottrine sono prive di monaci. E se questi
bhikkhu vivono rettamente, il mondo non sarà privo di Arahant, di esseri
realizzati».


\begin{quote}
All’età di ventinove anni, Subhadda, me ne andai \\
alla ricerca di quel che è salutare. \\
E ora son passati più di cinquant’anni \\
da quando me ne andai, Subhadda. \\
Al di fuori di questa Dispensazione non c’è monaco \\
che percorra, seppur in parte, la via del Dhamma.
\end{quote}

«Non c’è il secondo monaco, né il terzo monaco, né il quarto monaco. Le
altre dottrine sono prive di monaci. Se questi bhikkhu, però, vivono
rettamente, il mondo non sarà privo di Arahant».


Allora l’asceta itinerante Subhadda disse: «È meraviglioso, Signore, è
magnifico, Signore! Il Dhamma è stato chiarito in molti modi dal Beato,
come se egli avesse raddrizzato quel che era capovolto, rivelato quel
che era nascosto, indicato la via a chi è smarrito, alzato una lampada
nel buio per chi ha occhi per vedere forme visibili. Prendo rifugio nel
Beato, nel Dhamma e nel Saṅgha dei bhikkhu. Vorrei abbracciare la vita
religiosa e ricevere l’ammissione dal Beato».


«Chi fa già parte di una setta religiosa, Subhadda, e vuole abbracciare
la vita religiosa e l’ammissione in questo Dhamma e Disciplina è di
solito messo in prova per quattro mesi. Alla fine dei quattro mesi, se i
bhikkhu sono soddisfatti, gli impartiscono l’ammissione alla vita
religiosa e lo ammettono alla condizione di bhikkhu. So, però, che per
alcune persone sono state fatte delle eccezioni».


«Signore, se è così, che io sia messo in prova per quattro anni e, al
termine dei quattro anni, se i bhikkhu sono soddisfatti, mi impartiranno
l’ammissione alla vita religiosa e mi ammetteranno alla condizione di
bhikkhu».


Il Beato disse però al venerabile Ānanda: «Ora, Ānanda, impartisci a
Subhadda l’ammissione alla vita religiosa».


«E sia, Signore», rispose il venerabile Ānanda.


Allora l’asceta itinerante Subhadda disse al venerabile
Ānanda:\footnote{Il Commentario afferma che Subhadda fece questa osservazione in ragione dell’errata impressione che il Buddha, come alcuni maestri di altre sette, in questi ultimi suoi momenti stesse conferendo al suo discepolo il diritto di succedergli alla guida del Saṅgha. Egli non è la stessa persona del Subhadda menzionato poche pagine dopo.} «È un guadagno per te, amico Ānanda, è un
gran guadagno che tu sia stato consacrato qui, alla presenza del
Maestro, con la consacrazione del discepolo».


E l’asceta itinerante Subhadda ricevette l’ammissione alla vita
religiosa sotto il Beato e ottenne l’ammissione. Non molto tempo dopo la
sua ammissione, dimorando in solitudine, appartato, diligente, ardente e
dotato di autocontrollo, il venerabile Subhadda realizzò la conoscenza
diretta, e qui e ora entrò e dimorò in quella suprema meta della santa
vita per la quale gli uomini di famiglia giustamente lasciano la loro
casa per una vita priva di fissa dimora.
Egli ne ebbe la conoscenza diretta: «La nascita è distrutta, la santa vita è stata vissuta, quel
che doveva essere fatto è stato fatto, non ci sarà altra rinascita». E
il venerabile Subhadda divenne uno degli Arahant. Egli fu l’ultimo dei
discepoli del Beato a testimoniare.


Allora il Beato si rivolse al venerabile Ānanda: «Ānanda, tu potresti
pensare: “La parola del Maestro è una cosa del passato, ora non abbiamo
più un Maestro”. Ma non dovresti pensare in questo modo. Il Dhamma e la
Disciplina insegnati da me e stabiliti per voi sono il vostro Maestro
dopo che me ne sarò andato. Finora i bhikkhu si sono rivolti gli uni
agli altri con la parola “amico”. Questo non lo si deve fare dopo che me
ne sarò andato. Chi è bhikkhu da più tempo deve rivolgersi a un bhikkhu
più giovane usando il suo nome, il suo nome di famiglia o “amico”. Chi è
bhikkhu da meno tempo deve rivolgersi a un bhikkhu più anziano con
“signore” o con “venerabile”. Quando me ne sarò andato, il Saṅgha potrà,
se lo desidera, abolire le regole più minute e minori. Quando me ne sarò
andato, la sanzione maggiore deve essere imposta al bhikkhu
Channa».\footnote{La storia di come il principe Siddhattha Gotama, il Bodhisatta, lasciò la sua casa durante la notte con il suo stalliere, Channa, e il suo cavallo, Kanthaka, non si trova nel Canone; vi è però un accenno a Kanthaka nel canonico \emph{Vimānavatthu} (vv. 7:7). Quel racconto è offerto nella sua completezza nell’Introduzione di Ācariya Buddhaghosa al Commentario ai \emph{Jataka}. Questo bhikkhu Channa (identificato con lo stalliere) appare nel Vinaya (Pār. 4; Sangh. 12, ecc.) come orgoglioso, ostinato e intollerante alle correzioni. Nei sutta si racconta come egli si pentì dopo il \emph{Parinibbāna} e chiese aiuto all’Anziano Ānanda. Il risultato del discorso di quell’Anziano, fu che egli divenne un Arahant (S. 22:90).}


«Signore, qual è, però, la sanzione maggiore?».


«Qualsiasi cosa il bhikkhu Channa voglia, qualsiasi cosa egli dica, non
gli si deve parlare, né deve essere consigliato o istruito dai bhikkhu».


\suttaRef{D. 16}


Allora il Beato si rivolse ai bhikkhu con queste parole: «Bhikkhu, è
possibile che alcuni bhikkhu abbiano un dubbio o un problema riguardo al
Buddha, al Dhamma o al Saṅgha, o al Sentiero e alla via del progresso
spirituale. Domandate, bhikkhu, così da non provare poi rammarico in
questo modo: “Il Maestro era faccia a faccia con noi e noi, in presenza
del Beato, non abbiamo avuto il coraggio di domandare”».


Quando ciò fu detto, i bhikkhu rimasero in silenzio. Una seconda e una
terza volta il Beato pronunciò le stesse parole e, ogni volta, loro
rimasero in silenzio. Allora egli si rivolse a loro con queste parole:
«Bhikkhu, forse non domandate perché in presenza del Maestro vi sentite
in soggezione. Che un amico lo dica a un amico».


Quando ciò fu detto, loro rimasero in silenzio. Allora il venerabile
Ānanda disse al Beato: «È meraviglioso, Signore, è magnifico! Ho una
tale fiducia nel Saṅgha dei bhikkhu da credere che non ci sia un solo
bhikkhu con un dubbio o un problema riguardo al Buddha, al Dhamma o al
Saṅgha, o al Sentiero e alla via del progresso spirituale».


«Ānanda, tu parli in questo modo per fiducia. Il Perfetto ha però
conoscenza che qui, in questo Saṅgha di bhikkhu non c’è un solo bhikkhu
con un dubbio o un problema riguardo al Buddha, al Dhamma o al Saṅgha, o
al Sentiero e alla via del progresso spirituale. Il più indietro di
questi cinquecento bhikkhu è Entrato nella Corrente, non è più soggetto
alla perdizione, certo nella rettitudine e destinato all’Illuminazione».


Allora il Beato si rivolse ai bhikkhu con queste parole: «In verità,
bhikkhu, questo vi dichiaro: è nella natura di tutte le formazioni di
dissolversi. Raggiungete la perfezione mediante la
diligenza».\footnote{Forse la traduzione del prof. T.W. Rhys David «addestratevi per la vostra salvezza con diligenza», che T.S. Eliot ha fatto entrare nei classici della letteratura citandola nella sua \emph{Waste Land} meriterebbe di essere conservata; sembra però un po’ troppo libera. Le ultime parole in lingua pāḷi sono: «Handa ‘dāni bhikkhave āmantayāmi vo: Vaya-dhammā sankhārā; appamādena sampādetha».}


\suttaRef{D. 16; A. 4:76}


(FIXME label pag364)Queste furono le ultime parole del Beato.


Allora il Beato entrò nel primo jhāna. Emergendo da quello, entrò nel
secondo jhāna. Emergendo da quello, entrò nel terzo jhāna. Emergendo da
quello, entrò nel quarto jhāna. Emergendo da quello, entrò nella base
consistente dell’infinità dello spazio. Emergendo da quella, entrò nella
base consistente dell’infinità della coscienza. Emergendo da quella,
entrò nella base consistente del nulla-è. Emergendo da quella, entrò
nella base consistente della né-percezione-né-non-percezione. Emergendo
da quella, entrò nella cessazione della percezione e della sensazione.


Allora il venerabile Ānanda disse al venerabile Anuruddha: «Signore, il
Beato ha ottenuto il Nibbāna definitivo».


«No, amico. Il Beato non ha ottenuto il Nibbāna definitivo, ha ottenuto
la cessazione della percezione e della sensazione».


Allora il Beato, emergendo dalla cessazione della percezione e della
sensazione, entrò nella base consistente della
né-percezione-né-non-percezione. Emergendo da quella, entrò nella base
consistente del nulla-è. Emergendo da quella, entrò nella base
consistente dell’infinità della coscienza. Emergendo da quella, entrò
nella base consistente dell’infinità dello spazio. Emergendo da quella,
entrò nel quarto jhāna. Emergendo da quello, entrò nel terzo jhāna.
Emergendo da quello, entrò nel secondo jhāna. Emergendo da quello, entrò
nel primo jhāna. Emergendo da quello, entrò nel secondo jhāna. Emergendo
da quello, entrò nel terzo jhāna. Emergendo da quello, entrò nel quarto
jhāna. Ed emergendo dal quarto jhāna, il Beato ottenne il Nibbāna
definitivo.


All’ottenimento del Nibbāna definitivo del Beato, ci fu un gran
terremoto, pauroso e orripilante, e i tamburi del cielo risuonarono.


All’ottenimento del Nibbāna definitivo del Beato, Brahmā Sahampati
esclamò questa strofa:


\begin{quote}
Non c’è essere al mondo che non debba deporre \\
il proprio aggregato temporaneo, \\
e perfino un tal Maestro senza pari \\
in tutto il mondo, perfetto, con i poteri, \\
illuminato, ha ottenuto la completa estinzione.
\end{quote}

\suttaRef{D. 16; S. 6:15}


All’ottenimento del Nibbāna definitivo del Beato, Sakka, Sovrano degli
dèi, esclamò questa strofa:


\begin{quote}
Le formazioni sono impermanenti, \\
la loro natura è di sorgere e scomparire, \\
e qualsiasi cosa sorga deve cessare: \\
la vera beatitudine è nel loro acquietarsi.
\end{quote}

\suttaRef{D. 16; S. 6:15}


All’ottenimento del Nibbāna definitivo del Beato, il venerabile
Anuruddha esclamò questa strofa:


\begin{quote}
Perfino uno come lui, con la sua mente in pace, \\
rimase privo del respiro. Senza desideri, \\
il Veggente completò il suo tempo, intento nella pace. \\
Sopportò le sue sensazioni con cuore svincolato: \\
la liberazione del suo cuore fu come l’estinzione di una fiamma.
\end{quote}

All’ottenimento del Nibbāna definitivo del Beato, il venerabile Ānanda
esclamò questa strofa:


\begin{quote}
Oh, allora, paralizzante terrore, \\
oh, allora, capelli ritti per l’orrore – \\
l’Illuminato e supremamente onorato \\
ha ottenuto l’estinzione suprema.
\end{quote}

\suttaRef{D. 16; S. 6:15}


E all’ottenimento del Nibbāna definitivo del Beato, alcuni bhikkhu che
non erano liberi dalla brama alzarono le braccia e piansero, altri
ancora caddero e rotolarono avanti e indietro, gridando: «Così presto il
Beato ha ottenuto il Nibbāna definitivo! Così presto il Sublime ha
ottenuto il Nibbāna definitivo! Così presto l’Occhio è svanito dal
mondo!». Coloro che però erano liberi dalla brama, consapevoli e
pienamente presenti, dissero: «Le formazioni sono impermanenti. Come
potrebbe avvenire che quel che è nato, giunto all’esistenza, formato e
soggetto alla decadenza non decada? Questo non è possibile».


Allora il venerabile Anuruddha si rivolse ai bhikkhu con queste parole:
«Basta così, amici, non addoloratevi, non lamentatevi. Non è già stato
dichiarato dal Beato che c’è separazione, distacco e divisione da tutto
quello che ci è caro e che amiamo? Come potrebbe avvenire che quel che è
nato, giunto all’esistenza, formato e soggetto alla decadenza non
decada? Questo non è possibile. Le divinità stanno protestando, amici».


«Signore, ma quali divinità ha in mente il venerabile Anuruddha?».


«Amici, ci sono divinità che percepiscono la terra nello spazio. Si
stanno strappando i capelli e piangono, alzano le braccia e piangono,
cadono e rotolano avanti e indietro, gridando: “Così presto il Beato ha
ottenuto il Nibbāna definitivo! Così presto il Sublime ha ottenuto il
Nibbāna definitivo! Così presto l’Occhio è svanito dal mondo!”. E ci
sono divinità che percepiscono la terra nella terra che stanno facendo
le stesse cose. Quelle divinità, però, che sono libere dalla brama si
rassegnano, consapevoli e pienamente presenti: “Le formazioni sono
impermanenti. Come potrebbe avvenire che quel che è nato, giunto
all’esistenza, formato e soggetto alla decadenza non decada? Questo non
è possibile”».


Il venerabile Anuruddha e il venerabile Ānanda passarono il resto della
notte in discorsi di Dhamma. Allora il venerabile Anuruddha disse al
venerabile Ānanda: «Andiamo, amico, andiamo a Kusinārā e annunciamo ai
Malla di Kusinārā: “Vāseṭṭha, il Beato ha ottenuto il Nibbāna
definitivo. Ora è tempo di fare quel che reputate opportuno”».


«E sia, Signore», rispose il venerabile Ānanda. Ed essendo mattino, egli
si vestì, prese la ciotola e la veste superiore, ed entrò a Kusinārā con
un altro bhikkhu. In quel momento i Malla di Kusinārā si trovavano nel
loro salone per le riunioni per alcuni affari e altre cose ancora. Il
venerabile Ānanda andò nel salone per le riunioni e annunciò: «Vāseṭṭha,
il Beato ha ottenuto il Nibbāna definitivo».


Quando loro udirono queste parole dal venerabile Ānanda, i Malla con i
loro giovani, con le loro fanciulle e matrone erano sgomenti e
atterriti. Sopraffatti dal dolore, alcuni si strapparono i capelli e
piansero, altri alzarono le braccia e piansero, altri ancora caddero e
rotolarono avanti e indietro, gridando: «Così presto il Beato ha
ottenuto il Nibbāna definitivo! Così presto il Sublime ha ottenuto il
Nibbāna definitivo! Così presto l’Occhio è svanito dal mondo!».


Allora i Malla di Kusinārā impartirono ordini agli uomini: «Raccogliete
profumi, fiori e tutti gli strumenti musicali di Kusinārā». E loro
portarono profumi, fiori e tutti gli strumenti musicali di Kusinārā, e
anche cinquecento lunghezze di stoffa nel boschetto di alberi \emph{sāla} dei
Malla sulla curva dove si svolta verso Kusinārā, ove giaceva il corpo
del Beato. E loro trascorsero quel giorno prestando omaggio, rispetto,
riverenza e venerazione al corpo del Beato con danze, canzoni, musica,
ghirlande e profumi, e facendo tettoie e padiglioni di stoffa. Allora
pensarono: «Oggi è troppo tardi per cremare il corpo del Beato, lo
faremo domani». E così trascorsero il secondo, il terzo, il quarto, il
quinto e il sesto giorno.


Il settimo giorno pensarono: «Portiamo il corpo del Beato fuori dalla
città verso sud, in un posto a sud della città, prestando omaggio,
rispetto, riverenza e venerazione al corpo del Beato con danze, canzoni,
musica, ghirlande e profumi, e là, a sud della città, cremiamo il corpo
del Beato».


Allora otto Malla eminenti lavarono il loro capo e indossarono indumenti
nuovi. Pensando di sollevare il corpo del Beato, non riuscirono a farlo.
Ne chiesero la ragione al venerabile Anuruddha.


«Voi, Vāseṭṭha, avete un’intenzione, mentre le divinità ne hanno
un’altra».


«Allora, Signore, qual è l’intenzione delle divinità?».


«La vostra intenzione, Vāseṭṭha, è questa: “Portiamo il corpo del Beato
fuori dalla città verso sud, in un posto a sud della città, prestando
omaggio, rispetto, riverenza e venerazione al corpo del Beato con danze,
canzoni, musica, ghirlande e profumi, e là, a sud della città, cremiamo
il corpo del Beato”. L’intenzione delle divinità è questa: “Portiamo il
corpo del Beato fuori dalla città verso nord, in un posto a nord della
città, prestando omaggio, rispetto, riverenza e venerazione al corpo del
Beato con danze, canzoni, musica, ghirlande e profumi, e poi, entrando
dalla porta a nord, portiamo il corpo del Beato attraversando il centro
verso il centro della città, dopo di che usciamo dalla porta a est e là,
dove i Malla hanno un sacrario chiamato Makuṭabandhana a est della
città, cremiamo il corpo del Beato».


«Signore, sia come vogliono le divinità».


In quel momento Kusinārā fu tutta cosparsa, perfino i mucchi di rifiuti
e i cumuli d’immondizia, fino all’altezza delle ginocchia con fiori di
\emph{mandārava}.


Così, prestando onore, rispetto, riverenza e venerazione al corpo del
Beato con danze, canzoni, musica, ghirlande e profumi sia divini sia
umani, le divinità con i Malla di Kusinārā portarono il corpo del Beato
fuori dalla città verso nord, in un posto a nord della città, ed
entrando dalla porta a nord, portarono il corpo del Beato attraversando
il centro verso il centro della città, e uscendo dalla porta a est, dove
i Malla hanno un sacrario chiamato Makuṭabandhana a est della città, lo
deposero.


Allora i Malla di Kusinārā dissero al venerabile Ānanda: «Signore,
Ānanda, come dobbiamo trattare i resti del Perfetto?».


«Trattate i resti del Perfetto, Vāseṭṭha, come sono trattati i resti di
un Monarca Universale che gira la Ruota della Giustizia».


«Signore, Ānanda, come lo si fa?».


«I resti di un Monarca Universale che gira la Ruota della Giustizia
vengono avvolti in una stoffa nuova, Vāseṭṭha, poi vengono avvolti in
panno di cotone ben battuto, e poi vengono avvolti in una stoffa nuova.
E, procedendo in questo modo, vengono avvolti in cinquecento strati doppi.
Poi vengono collocati in un recipiente di olio, fatto di ferro, che
viene chiuso in un altro recipiente di ferro. Poi si accende una pira
con tutti i tipi di profumi e i resti vengono bruciati. Poi gli si erige
un monumento a un crocevia. Così è che si trattano i resti di un Monarca
Universale che gira la Ruota della Giustizia. E i resti del Perfetto
devono essere trattati nello stesso modo. Il monumento del Perfetto deve
essere eretto a un crocevia, e chiunque metterà fiori e profumi su di
esso, lo imbiancherà, lo venererà o proverà in cuor suo fiducia quando
si troverà lì, ciò sarà per lungo tempo a vantaggio del suo benessere e
della sua felicità».


A quel punto i Malla di Kusinārā impartirono ordini agli uomini di
raccogliere tutto il cotone battuto dei Malla. Poi avvolsero il corpo
del Beato in una stoffa nuova, e dopo lo avvolsero nel cotone battuto, e
dopo ancora lo avvolsero in una stoffa nuova. E, avendo in quel modo
avvolto il corpo del Beato in cinquecento strati doppi, lo collocarono
in un recipiente di olio, fatto di ferro, che chiusero in un altro
recipiente di ferro. Poi eressero una pira, aggiungendo tutti i tipi di
profumi, e portarono i resti del Beato sulla pira.


\suttaRef{D. 16}


In quel momento il venerabile Mahā-Kassapa stava viaggiando sulla strada
principale che da Pāvā porta a Kusinārā con un largo seguito di bhikkhu,
con cinquecento bhikkhu. Allora egli lasciò la strada e si mise a sedere
ai piedi di un albero. Nel frattempo un asceta mendicante, che aveva
raccolto un fiore di \emph{mandārava} a Kusinārā, stava viaggiando su quella
strada. Il venerabile Mahā-Kassapa lo vide arrivare. Gli chiese:
«Conosci il nostro Maestro, amico?».


«Sì, amico, lo conosco. Il monaco Gotama ha ottenuto il Nibbāna
definitivo sette giorni fa. È così che ho avuto questo fiore di
\emph{mandārava}».


Alcuni bhikkhu che non erano liberi dalla brama alzarono le braccia e
piansero, altri ancora caddero e rotolarono avanti e indietro, gridando:
«Così presto il Beato ha ottenuto il Nibbāna definitivo! Così presto il
Sublime ha ottenuto il Nibbāna definitivo! Così presto l’Occhio è
svanito dal mondo!». Coloro che però erano liberi dalla brama,
consapevoli e pienamente presenti, dissero: «Le formazioni sono
impermamenti. Come potrebbe avvenire che quel che è nato, giunto
all’esistenza, formato e soggetto alla decadenza non decada? Questo non
è possibile».


C’era però un [bhikkhu] seduto nell’assemblea chiamato Subhadda che
aveva abbracciato la vita religiosa quando era anziano. Egli disse a
quei bhikkhu: «Basta così, amici, non addoloratevi, non lamentatevi. Ci
siamo ben liberati dal Grande Monaco. Siamo stati frustrati dalle sue
parole: “Questo vi è consentito, questo non vi è consentito”. Ora faremo
quello che ci piace e non faremo quello che non ci piace».


Allora il venerabile Mahā-Kassapa si rivolse ai bhikkhu con queste
parole: «Basta così, amici, non addoloratevi, non lamentatevi. Non è già
stato dichiarato dal Beato che c’è separazione, distacco e divisione da
tutto quello che ci è caro e che amiamo? Come potrebbe avvenire che quel
che è nato, giunto all’esistenza, formato e soggetto alla decadenza non
decada? Questo non è possibile».


\suttaRef{D. 16; Vin. Cv. 11:1}


Quattro Malla eminenti che si erano lavati il capo e avevano indossato
indumenti nuovi pensarono: «Accendiamo la pira del Beato». Non
riuscirono però a farlo. Ne chiesero la ragione al venerabile Anuruddha.


«Le divinità hanno un’altra intenzione, Vāseṭṭha».


«Allora, Signore, qual è l’intenzione delle divinità?».


«L’intenzione delle divinità è questa, Vāseṭṭha: “C’è il venerabile
Mahā-Kassapa che sta viaggiando sulla strada principale che da Pāvā
porta a Kusinārā con un largo seguito di bhikkhu, con cinquecento
bhikkhu. La pira del Beato non sarà accesa finché il venerabile
Mahā-Kassapa non avrà salutato con il suo capo il Beato”».


«Allora, Signore, sia come vogliono le divinità».


Il venerabile Mahā-Kassapa giunse alla pira del Beato nel Sacrario
Makuṭabandhana dei Malla a Kusinārā. Dopo averlo fatto, sistemò la veste
superiore su una spalla e, alzando le mani giunte, effettuò per tre
volte la circumambulazione della pira girando a destra. Allora i piedi
del Beato furono mostrati, ed egli prostrò il capo ai piedi del Beato. E
i cinquecento bhikkhu sistemarono la veste superiore su una spalla, e
fecero come aveva fatto il venerabile Mahā-Kassapa. Appena ebbero
finito, però, la pira s’incendiò da sé. E proprio come quando il burro o
l’olio bruciano senza produrre brace o cenere, allo stesso modo, quando
il corpo del Beato bruciò, né la pelle esterna, né la pelle interna, né
la carne, né i tendini, né il liquido sinoviale produssero brace o
cenere. Restarono solo le ossa. E dei cinquecento strati doppi solo due
bruciarono: il più interno e il più esterno.


Quando il corpo del Beato fu consumato, una cascata d’acqua cadde dal
cielo ed estinse la pira, e altra acqua sgorgò dalla terra ed estinse la
pira, e i Malla di Kusinārā estinsero la pira con ogni genere di acqua
profumata.


Allora i Malla conservarono le ossa del Beato nel salone per le riunioni
per sette giorni, e fecero un traliccio con delle lance circondato da un
bastione di archi, e prestarono onore, rispetto, riverenza e venerazione
ad esse con danze, canzoni, musica, ghirlande e profumi.


Il re Ajātasattu di Magadha sentì dire: «Sembra che il Beato abbia
ottenuto il Nibbāna definitivo a Kusinārā». Allora egli inviò un messo
ai Malla di Kusinārā con la richiesta: «Il Beato era un guerriero,
anch’io sono un guerriero. Sono degno di condividere le ossa del Beato.
Anche io costruirò un monumento e organizzerò una cerimonia».


E i Licchavi di Vesālī sentirono dire queste stesse cose e anche loro
inviarono un messo con la richiesta: «Il Beato era un guerriero, anche
noi siamo guerrieri. Anche noi siamo degni di condividere le ossa del
Beato. Anche noi costruiremo un monumento e organizzeremo una
cerimonia».


E i Sakya di Kapilavatthu sentirono dire queste stesse cose e anche loro
inviarono un messo con la richiesta: «Il Beato era il più grande dei
nostri consanguinei. Anche noi siamo degni di condividere le ossa del
Beato. Anche noi costruiremo un monumento e organizzeremo una
cerimonia».


E i Bulaya di Allakappaka sentirono dire queste stesse cose e anche loro
inviarono un messo con la richiesta: «Il Beato era un guerriero, anche
noi siamo guerrieri. Anche noi siamo degni di condividere le ossa del
Beato. Anche noi costruiremo un monumento e organizzeremo una
cerimonia».


E i Koliya di Rāmagāma sentirono dire queste stesse cose e anche loro
inviarono un messo con la richiesta: «Il Beato era un guerriero, anche
noi siamo guerrieri. Anche noi siamo degni di condividere le ossa del
Beato. Anche noi costruiremo un monumento e organizzeremo una
cerimonia».


E il brāhmaṇa dell’isola di Veṭha sentì dire queste stesse cose e anche
lui inviò un messo con la richiesta: «Il Beato era un guerriero, io sono
un brāhmaṇa. Anche io sono degno di condividere le ossa del Beato. Anche
io costruirò un monumento e organizzerò una cerimonia».


E i Malla di Pāvā sentirono dire queste stesse cose e anche loro
inviarono un messo con la richiesta: «Il Beato era un guerriero, anche
noi siamo guerrieri. Anche noi siamo degni di condividere le ossa del
Beato. Anche noi costruiremo un monumento e organizzeremo una
cerimonia».


Quando ciò fu detto, i Malla riunirono i messi e risposero loro in
questo modo: «Il Beato ha ottenuto il Nibbāna definitivo nel territorio
della nostra città. Non rinunceremo alle ossa del Beato».


Allora il brāhmaṇa Doṇa si rivolse al gruppo riunitosi con queste
strofe:


\begin{quote}
Signori, ascoltate da me una parola: \\
il nostro Risvegliato predicò la pazienza. \\
Così male ce ne verrà \\
se dovessimo giungere a uno scontro per la divisione delle ossa \\
di quest’eccelso personaggio. \\
Signori, che noi tutti insieme in armonia \\
si concordi una divisione in otto parti. \\
Che monumenti siano eretti ampi e lontani, \\
così che in molti possano ottenere fiducia nel Veggente.
\end{quote}

«Allora, brāhmaṇa, dividi tu stesso e distribuisci equamente in otto
parti le ossa del Beato».


«E sia, signori», egli rispose, e divise e distribuì equamente in otto
parti le ossa del Beato. Allora egli chiese al gruppo riunito in
assemblea: «Datemi questo contenitore, signori. Anche io costruirò un
monumento e organizzerò una cerimonia». E loro gli diedero il
contenitore.


I Moriya di Pipphalivana sentirono dire: «Sembra che il Beato abbia
ottenuto il Nibbāna definitivo a Kusinārā». Allora inviarono un messo
con la richiesta: «Il Beato era un guerriero, anche noi siamo guerrieri.
Anche noi siamo degni di condividere le ossa del Beato. Anche noi
costruiremo un monumento e organizzeremo una cerimonia».


«Non sono rimaste altre ossa del Beato da dividere. Sono state tutte
distribuite. Potete prendere le ceneri da qui». Così loro presero le
ceneri.


Allora Ajātasattu Vedehiputta, re di Magadha, fece costruire un
monumento per le ossa del Beato e organizzò una cerimonia. E tutti gli
altri fecero nello stesso modo. Ci furono così otto monumenti per le
ossa del Beato, uno per il contenitore e uno per le ceneri. Questo è
quel che avvenne.


\suttaRef{D. 16}


