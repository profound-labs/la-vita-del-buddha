\chapter{La disputa di Kosambī}

\narrator{Secondo narratore.}  La tradizione afferma che la sesta stagione delle
piogge fu trascorsa sul monte Makula e che durante l’anno successivo
avvenne nuovamente il miracolo doppio a Sāvatthī, dopo di che il Buddha
ascese nel paradiso delle Trentatré Divinità. Lì egli trascorse la
settima stagione delle piogge, esponendo l’\emph{Abhidhamma} alle divinità,
compresa colei che in precedenza era stata sua madre. Al termine di
quella stagione delle piogge, quando il Buddha tornò sulla terra ebbe
luogo la “discesa degli dèi”. Egli trascorse l’ottava stagione delle
piogge a Suṃsumāragira e la nona a Kosambī.


\narrator{Primo narratore.} Kosambī era la capitale del piccolo regno di Vaṃsa,
incuneato tra il Gange e il Jumna. Il suo re, Udena, è a mala pena
menzionato nel Canone. La maggior parte degli eventi collocati in questi
anni dalla tradizione più tarda, inclusa la visita al paradiso delle
Trentatré Divinità e la discesa degli dèi, non sono menzionati nel
Canone.


\narrator{Secondo narratore.} Per iniziare, questo è il racconto della triviale
circostanza che condusse alla prima grande disputa, che minacciò di
produrre uno scisma nel Saṅgha. Pare che in un monastero vi fossero due
bhikkhu, uno era esperto nella Disciplina e l’altro insegnava i
Discorsi. Quest’ultimo un giorno andò nella latrina e lasciò lì un
catino con dell’acqua, inutilizzata, per lavarsi. L’altro vi andò più
tardi e lo trovò là. Egli chiese all’insegnante dei Discorsi: «Hai
lasciato tu quel catino con dell’acqua dentro?». «Sì». «Sai che si
tratta di un’infrazione?». «No, non lo so». «Si tratta di un’infrazione,
amico». «Allora la confesso». «Se però l’hai fatto non intenzionalmente
e per dimenticanza, non è un’infrazione». L’insegnante di Discorsi andò
via con l’impressione di non aver fatto nulla di sbagliato. L’esperto
nella Disciplina, tuttavia, disse ai suoi allievi: «L’insegnante di
Discorsi non sa quando commette un’infrazione». Loro dissero ai
discepoli dell’altro: «Il vostro precettore ha commesso un’infrazione,
anche se ha l’impressione di non averla commessa». Quando lo
raccontarono al loro precettore, lui disse: «Questo esperto nella
Disciplina ha prima detto che non c’era alcuna infrazione e ora dice
che, invece, ce n’è una. È un bugiardo». Loro allora dissero ai
discepoli dell’esperto nella Disciplina: «Il vostro precettore è un
bugiardo». La sua risposta consistette nel convocare un capitolo e
sospendere l’insegnante di Discorsi.


\narrator{Primo narratore.} Questo è il racconto canonico di quel che successe in
seguito.


\voice{Seconda voce.} Avvenne questo. Mentre il Buddha, il Beato, viveva a
Kosambī nel Parco di Ghosita, un bhikkhu fu coinvolto in un’infrazione.
Egli considerò l’infrazione come un’infrazione, ma l’altro bhikkhu non
considerò l’infrazione come un’infrazione. Poi, lui stesso non considerò
quell’infrazione come un’infrazione, ma l’altro considerò l’infrazione
come un’infrazione. Quei bhikkhu allora gli dissero: «Amico, hai
commesso un’infrazione. La vedi, l’infrazione?».


«Amici, non vedo alcuna infrazione da me commessa».


Quei bhikkhu furono allora d’accordo a sospendere il bhikkhu, anche se
lui non vedeva la sua infrazione. Quel bhikkhu, però, era istruito.
Conosceva il Canone ed era esperto nel Dhamma e nella Disciplina e nei
Codici. Era saggio, sagace, intelligente, modesto, scrupoloso e
desideroso di addestrarsi. Andò dai suoi compagni più intimi e disse:
«Questa non è un’infrazione, questa non è un’infrazione, non ho infranto
nulla … io non sono sospeso, sono stato sospeso mediante un atto
errato, non valido e privo di fondamento. Che i venerabili si schierino
dalla mia parte nel Dhamma e nella Disciplina».


Egli ottenne che fossero dalla sua parte, e mandò dei messaggeri ai suoi
amici e compagni di quel paese. Allora i bhikkhu che appoggiavano il
bhikkhu sospeso si recarono da coloro che lo avevano sospeso e
affermarono la loro tesi. Quando ciò fu fatto, gli altri confermarono la
validità del loro atto di sospensione e dissero: «Che i venerabili non
appoggino e non seguano un bhikkhu sospeso». Però, benché ai bhikkhu che
appoggiavano il bhikkhu sospeso fossero state dette queste parole, loro
continuarono ad appoggiarlo e a seguirlo.


Allora un bhikkhu andò dal Beato e gli raccontò l’accaduto. Il Beato
disse: «Ci sarà uno scisma nel Saṅgha, ci sarà uno scisma nel Saṅgha».
Si alzò dal posto in cui sedeva e andò dai bhikkhu che avevano
organizzato la sospensione. Egli si mise a sedere nel posto
preparatogli, e disse loro: «Bhikkhu, non pensiate che questo o quel
bhikkhu debba essere sospeso semplicemente per questa ragione: “Noi
pensiamo così”. Prendete il caso di un bhikkhu che ha commesso
un’infrazione e, benché egli non la consideri tale, gli altri bhikkhu
tale la considerino. Ora, i bhikkhu che conoscono la gravità di uno
scisma nel Saṅgha non dovrebbero sospendere quel bhikkhu fino a quando
egli non consideri la sua infrazione come un’infrazione. Se loro lo
giudicano in questo modo: “Lui è istruito e desideroso di addestrarsi.
Se lo sospendiamo senza che lui veda la sua infrazione, non saremo in
grado di tenere il santo giorno di osservanza \emph{Uposatha} con lui, né la
cerimonia \emph{Pavāraṇā} – l’Invito alla Critica – alla fine della stagione
delle piogge, né di effettuare gli atti del Saṅgha, né di sedere nello
stesso posto, né di condividere il brodo di riso, né di condividere la
mensa, né di vivere sotto lo stesso tetto, né di compiere atti di
rispetto nei riguardi degli anziani con lui. Dovremo fare tutte queste
cose senza di lui e, a causa di ciò, ci saranno litigi, risse, dispute,
contestazioni e infine atti di scisma, divisione e dissenso nel
Saṅgha”».


Quando ebbe detto queste cose, si alzò e andò dai bhikkhu che seguivano
il bhikkhu sospeso. Egli si mise a sedere nel posto preparatogli, e
disse loro: «Bhikkhu, non pensiate che, essendo stata commessa
un’infrazione, non si debba fare ammenda semplicemente perché pensate:
“Non l’abbiamo commessa”. Prendete il caso di un bhikkhu che ha commesso
un’infrazione e, benché egli non la consideri tale, gli altri bhikkhu
tale la considerino. Ora, un bhikkhu che conosce la gravità di uno
scisma nel Saṅgha dovrebbe riconoscere l’infrazione per fiducia negli
altri, se li giudica in questo modo: “Loro sono istruiti e desiderosi di
addestrarsi. È assurdo andar fuori strada per fervore, odio, illusione e
timore a causa mia oppure a causa degli altri. Se questi bhikkhu mi
sospendono per un’offesa che io non considero tale, loro non saranno in
grado di tenere il santo giorno di osservanza \emph{Uposatha} con me, né la
cerimonia \emph{Pavāraṇā} – l’Invito alla Critica – alla fine della stagione
delle piogge, né di effettuare gli atti del Saṅgha, né di sedere nello
stesso posto, né di condividere il brodo di riso, né di condividere la
mensa, né di vivere sotto lo stesso tetto, né di compiere atti di
rispetto nei riguardi degli anziani con me. Dovranno fare tutte queste
cose senza di me e, a causa di ciò, ci saranno litigi, risse, dispute,
contestazioni e infine atti di scisma, divisione e dissenso nel
Saṅgha”».


Dopo che il Beato ebbe detto queste cose, si alzò e se ne andò.


\suttaRef{Vin. Mv. 10:1}


\voice{Prima voce.} Però, nel Saṅgha scoppiarono litigi, risse e dispute, e i
bhikkhu si ferirono a vicenda con frecce fatte di parole. Non riuscivano
a comporre la loro lite. Allora un bhikkhu andò dal Beato e, dopo
avergli prestato omaggio, si mise in piedi da un lato. Gli raccontò che
cosa stava avvenendo e aggiunse: «Signore, sarebbe bene che il Beato per
compassione si recasse da quei bhikkhu».


Il Beato acconsentì in silenzio. Allora andò da quei bhikkhu e disse
loro: «Basta così, bhikkhu, no ai litigi, no alle risse, no alle
dispute, no alle contestazioni».


Quando ciò fu detto, un bhikkhu replicò: «Signore, che il Beato, il
Maestro del Dhamma, attenda, che il Beato viva dedicandosi a dimorare
piacevolmente nel qui e ora, e non si preoccupi di questa cosa. Siamo
noi che dobbiamo occuparci di questi litigi, di queste risse, dispute e
contestazioni».


Una seconda e una terza volta il Beato disse la stessa cosa e ricevette
la stessa risposta. Poi pensò: «Questi uomini fuorviati paiono degli
ossessi. È impossibile farli ragionare». Si alzò e se ne andò.


Quando si fece mattina si vestì, prese la ciotola e la veste superiore,
e andò a Kosambī per la questua. Quando ebbe fatto il giro per la
questua e tornò dopo il pasto, mise in ordine il posto nel quale
riposava e prese la ciotola e la sua veste superiore. Poi pronunciò
queste strofe:


\suttaRef{M. 128; cf. Vin. Mv. 10:2-3}


\begin{quote}
Quando molte voci urlano contemporaneamente, \\
non vi è chi pensa d’essere un folle, \\
l’Ordine è diviso, nessuno pensa: \\
«Vi ho preso parte, ho contribuito a questo». \\
Hanno dimenticato il saggio parlare, parlano \\
con la mente ossessionata dalle sole parole, \\
sfrenate le loro labbra, sbraitano a volontà, \\
nessuno sa che cosa l’abbia indotto a comportarsi così.
\end{quote}

\emph{M. 128; Jā. 3:488; Ud. 5:9} \\
\suttaRef{Thag. 275; Vin, Mv. 10:3}


\begin{quote}
«Mi ha insultato, mi ha picchiato, \\
mi ha sconfitto, mi ha derubato!» \\
L’odio non si placa mai in chi \\
ha a cuore l’inimicizia. \\
«Mi ha insultato, mi ha picchiato, \\
mi ha sconfitto, mi ha derubato!» \\
L’odio certamente si placa in chi \\
non ha a cuore l’inimicizia. \\
Siccome inimicizia è ricambiata con inimicizia \\
questo mondo non ha mai pace. \\
Trova pace con l’amicizia, \\
questo è un antico principio. \\
Quegli altri non riconoscono \\
che qui dovremmo contenere noi stessi.\footnote{Non vi è accordo sul significato della parola \emph{yamāmase}, se debba essere resa con «dovremmo contenere noi stessi» oppure con «potremmo essere distrutti».} \\
Qui tuttavia alcuni ne sono consapevoli \\
e perciò le loro liti sono sedate.
\end{quote}

\suttaRef{M. 128; Dh. 3-6; Jā. 3:212, 488; Vin. Mv. 10:3}


\begin{quote}
Spaccaossa e assassini, \\
ladri di bestiame, di cavalli, di patrimoni: \\
mentre sono intenti a saccheggiare il reame, \\
perfino costoro possono agire in concordia. \\
Perché allora voi non potete fare altrettanto?
\end{quote}

\suttaRef{M. 128; Jā. 3:488; Vin. Mv. 10:3}


\begin{quote}
Se riuscite a trovare un compagno degno di fede, \\
col quale camminare, virtuoso e risoluto, \\
camminate con lui soddisfatti e consapevoli, \\
vincendo ogni minaccia e pericolo. \\
Se non riuscite a trovare un compagno degno di fede, \\
col quale camminare, virtuoso e risoluto, \\
allora, come un re che abbandona un regno sconfitto, \\
camminate soli come un rinoceronte nella foresta.


È meglio camminare da soli, \\
non c’è amicizia con i folli. \\
Camminate da soli, non ferite nessuno, senza conflitti, \\
siate come un rinoceronte solo nella foresta.
\end{quote}

\emph{M. 128; Jā. 3:488; Vin. Mv. 10:3} \\
\suttaRef{Dh. 328-30; cf. Sn. 45-46}


Dopo aver pronunciato queste strofe, il Beato se ne andò a
Bālakaloṇakāragāma. In quel tempo lì viveva il venerabile Bhagu. Quando
vide in lontananza che il Beato stava arrivando, gli preparò un posto a
sedere e dell’acqua per lavarsi i piedi, uno sgabello e un asciugamano.
Il Beato si mise a sedere nel posto preparatogli e si lavò i piedi. Il
venerabile Bhagu gli prestò omaggio e si mise a sedere da un lato.
Allora il Beato gli disse: «Bhikkhu, spero che tu stia bene, che ti
senta a tuo agio e non abbia problemi a riguardo della questua».


«Sto bene, Beato, mi sento a mio agio e non ho problemi a riguardo della
questua».


Allora il Beato istruì, esortò, risvegliò e incoraggiò il venerabile
Bhagu con un discorso di Dhamma, dopo il quale si alzò dal posto in cui
sedeva e partì per recarsi al Parco Orientale di Bambù. Il venerabile
Anuruddha, il venerabile Nandiya e il venerabile Kimbila in quel tempo
vivevano lì. Il custode del parco vide che il Beato stava arrivando. Gli
disse: «Non entrare in questo parco, monaco. Ci sono tre uomini di rango
che sono alla ricerca del loro bene. Non disturbarli».


Il venerabile Anuruddha sentì il custode del parco che parlava al Beato.
Disse al custode del parco: «Amico custode del parco, non far restare
fuori il Beato. È arrivato il nostro Maestro, il Beato».


Il venerabile Anuruddha andò dal venerabile Nandiya e dal venerabile
Kimbila e disse: «Venite fuori, venerabili signori, venite fuori, è
arrivato il nostro Maestro».


Allora si recarono tutti e tre a incontrare il Beato. Uno prese la
ciotola e la sua veste superiore, uno gli preparò un posto a sedere e
uno dell’acqua per lavarsi i piedi. Il Beato si mise a sedere nel posto
preparatogli e si lavò i piedi. Poi loro gli prestarono omaggio e si
misero a sedere da un lato. Il Beato disse: «Spero che voi stiate bene,
Anuruddha, che vi sentiate a vostro agio e non abbiate problemi a
riguardo della questua».


«Stiamo bene, Beato, ci sentiamo a nostro agio e non abbiamo problemi a
riguardo della questua».


«Spero che viviate tutti in concordia, Anuruddha, in amicizia e senza
discussioni come il latte con l’acqua, guardandovi l’un l’altro con
occhi gentili».


«Certamente ci comportiamo così, Signore».


«Anuruddha, come si fa a vivere così?».


Il venerabile Anuruddha rispose: «Signore, penso che sia un profitto e
una fortuna per me che vivo la santa vita qui, avere compagni come
questi. Mantengo in essere atti, parole e pensieri di gentilezza
amorevole verso questi venerabili sia in pubblico sia in privato. Penso:
“Perché non dovrei mettere da parte quel che io intendo fare, e fare
solo quel che loro intendono fare?”. e mi comporto di conseguenza.
Abbiamo un corpo differente, Signore, ma una sola mente, penso».


Gli altri due dissero la stessa cosa. Aggiunsero: «Signore, è così che
viviamo in amicizia e senza discussioni come il latte con l’acqua,
guardandoci l’un l’altro con occhi gentili».


«Bene, bene, Anuruddha. Spero che dimoriate diligenti, ardenti e
autocontrollati».


«Certamente, Signore».


«Anuruddha, come si fa a dimorare così?».


«Signore, chiunque di noi torni per primo dal villaggio con il cibo
ottenuto dalla questua prepara i posti a sedere, l’acqua da bere e per
lavarsi, e mette al suo posto il secchiello per i rifiuti. Chiunque di
noi torni per ultimo mangia il cibo rimasto, se lo desidera. Altrimenti
lo getta dove non c’è erba o in acqua dove non c’è vita. Ripone i posti
a sedere, l’acqua da bere e per lavarsi. Ripone il secchiello per i
rifiuti dopo averlo lavato, e spazza il refettorio. Chiunque noti che
nei recipienti l’acqua da bere, per lavarsi o per il gabinetto
scarseggia o è finita, se ne occupa. Se è troppo pesante per lui, fa un
cenno a un altro con un gesto della mano e lo spostiamo, aiutandoci. Non
parliamo per tale scopo. Ogni cinque giorni, però, sediamo fuori insieme
nella notte parlando di Dhamma. In questo modo dimoriamo diligenti,
ardenti e autocontrollati».


\suttaRef{M. 128; Vin. Mv. 10:4}


\voice{Seconda voce.} Quando il Beato li ebbe istruiti, esortati, risvegliati e
incoraggiati con un discorso di Dhamma, si alzò dal posto in cui sedeva.
Partì viaggiando per tappe per recarsi a Pārileyyaka. Infine vi giunse e
andò a vivere nella giungla Rakkhita, ai piedi di un fausto albero
\emph{sāla}. Mentre era solo in ritiro, questo pensiero sorse nella sua
mente: «Prima vivevo a disagio, infastidito da quei bhikkhu di Kosambī
che disputavano, discutevano, altercavano, si aggredivano a parole e
litigavano nel bel mezzo del Saṅgha. Ora sono solo e senza compagni,
vivo a mio agio e comodamente, lontano da tutti loro».


C’era un pachiderma che aveva vissuto infastidito da altri elefanti, da
elefantesse, da elefanti giovani ed elefanti cuccioli, aveva dovuto
mangiare erba pestata e rametti spezzati, aveva dovuto bere acqua sporca
e il suo corpo era stato spintonato dalle elefantesse quando usciva dal
luogo in cui aveva fatto il bagno. Considerando tutte queste cose,
pensò: «Perché non dovrei dimorare in solitudine, appartato dalla
folla?». E così aveva abbandonato il branco ed era andato a Pārileyyaka,
nella giungla Rakkhita, ai piedi del fausto albero \emph{sāla} dove si
trovava il Beato. Si prese cura del Beato, procurandogli cibo e acqua, e
con la sua proboscide spazzava via le foglie. Pensò: «Prima vivevo
infastidito da altri elefanti … Ora, solo e ritirato dal branco, vivo a
mio agio e comodamente, lontano da tutti quegli elefanti».


Il Beato, assaporando la sua solitudine, fu consapevole nella sua mente
del pensiero sorto nella mente di quell’elefante. Esclamò queste parole:


\begin{quote}
Qui un pachiderma va d’accordo con un altro pachiderma, \\
l’elefante con zanne lunghe \\
come colonne si delizia a star solo nella foresta: \\
così i loro cuori sono in armonia.
\end{quote}

\suttaRef{Vin. Mv. 10:4; cf. Ud. 4:5}


\voice{Prima voce.} Subito dopo che il Beato aveva lasciato Kosambī, un bhikkhu
andò dal venerabile Ānanda e disse: «Amico Ānanda, il Beato ha messo in
ordine il posto in cui riposava, ha preso la sua ciotola e la sua veste
superiore ed è partito per errare da solo e privo di compagnia senza
informare i suoi attendenti o congedarsi dal Saṅgha dei bhikkhu».


«Amico, quando il Beato fa così, allora vuole vivere solo e non deve
essere seguito da nessuno».


Qualche tempo dopo un certo numero di bhikkhu andò dal venerabile Ānanda
e disse: «Amico Ānanda, da molto tempo non sentiamo un discorso di
Dhamma dalle labbra del Beato. Ci piacerebbe ascoltarlo».


Così il venerabile Ānanda si recò con quei bhikkhu dal Beato ai piedi
del fausto albero \emph{sāla} a Pārileyyaka e, dopo avergli prestato omaggio,
si misero a sedere da un lato. Allora il Beato li incoraggiò con un
discorso di Dhamma.


\suttaRef{S. 22:81}


\voice{Seconda voce.} Quando il Beato era rimasto a Pārileyyaka per tutto il
tempo che volle, partì viaggiando per tappe verso Sāvatthī. Infine vi
arrivò, e andò a vivere nel Boschetto di Jeta, nel Parco di
Anāthapiṇḍika.


Nel frattempo i seguaci laici di Kosambī pensarono: «Questi venerabili
bhikkhu di Kosambī ci stanno arrecando un gran danno. Hanno a tal punto
infastidito il Beato che egli è andato via. Non presteremo più omaggio a
loro, né ci alzeremo per loro, né li saluteremo con riverenza, né li
tratteremo in modo cortese, non li onoreremo, rispetteremo, riveriremo o
venereremo, non daremo più loro cibo in elemosina nemmeno se vengono per
la questua. Così, quando loro non riceveranno onore, rispetto, riverenza
o venerazione da noi, quando saranno costantemente ignorati, se ne
andranno altrove o lasceranno il Saṅgha oppure si recheranno a fare
ammenda dal Beato».


Così si comportarono. Di conseguenza i bhikkhu di Kosambī decisero:
«Andiamo a Sāvatthī, amici, e componiamo questa lite alla presenza del
Beato». Misero perciò in ordine il posto nel quale riposavano, presero
la loro ciotola e la veste superiore e partirono per Sāvatthī.


Il venerabile Sāriputta sentì che stavano arrivando. Andò dal Beato e
gli chiese: «Signore, pare che quei bhikkhu di Kosambī che disputavano,
discutevano, altercavano, si aggredivano a parole e litigavano nel bel
mezzo del Saṅgha stiano arrivando qua a Sāvatthī. Come li devo trattare,
Signore?».


«Attieniti al Dhamma, Sāriputta».


«Signore, come faccio a sapere che cosa è Dhamma oppure che cosa non lo
è?». «Ci sono diciotto modi mediante i quali uno che dice ciò che non è
Dhamma può essere riconosciuto. Un bhikkhu mostra quel che non è Dhamma
come Dhamma e quel che è Dhamma come non Dhamma. Mostra quel che non è
Disciplina come Disciplina e quel che è Disciplina come non Disciplina.
Mostra quel che non è stato affermato dal Beato come se lo fosse stato e
quel che è stato affermato dal Beato come se non lo fosse stato. Mostra
quel che non è stato praticato dal Beato come se lo fosse stato e mostra
quel che è stato praticato dal Beato come se non lo fosse stato. Mostra
quel che non è un’infrazione come un’infrazione e quel che è
un’infrazione come una non-infrazione. Mostra una lieve infrazione come
grande e una grande infrazione come lieve. Mostra un’infrazione con
residuo come senza residuo e una senza residuo come con residuo. Mostra
un’infrazione importante come non importante e una non importante come
importante. Uno che dice ciò che è Dhamma può essere riconosciuto nel
modo opposto».


Il venerabile Mahā-Moggallāna, il venerabile Mahā Kassapa, il venerabile
Mahā Kaccāna, il venerabile Mahā Koṭṭhita, il venerabile Mahā Kappina,
il venerabile Mahā Cunda, il venerabile Anuruddha, il venerabile Revata,
il venerabile Upāli, il venerabile Ānanda e il venerabile Rāhula
sentirono che stavano arrivando. Ognuno di loro si recò dal Beato e
ricevette le stesse istruzioni.


Mahāpajāpatī Gotamī sentì, andò dal Beato e gli chiese come avrebbe
dovuto trattarli.


«Ascolta il Dhamma da entrambe le parti, Gotamī. Dopo averlo fatto,
approva le inclinazioni, le opinioni e i giudizi di coloro che dicono
quel che è Dhamma. Quel che il Saṅgha delle bhikkhuṇī deve attendersi
dal Saṅgha dei bhikkhu deve provenire da coloro che parlano in accordo
con il Dhamma».


Anāthapiṇḍika e Visākhā, la madre di Migāra, sentirono e andarono dal
Beato per ricevere consigli. Egli disse loro: «Offrite doni a entrambe
la parti. Approvate i punti di vista di coloro che parlano in accordo
con il Dhamma».


Infine i bhikkhu di Kosambī giunsero a Sāvatthī. Il venerabile Sāriputta
andò dal Beato e gli chiese: «Signore, pare che quei bhikkhu di Kosambī
siano arrivati a Sāvatthī. Dove dovrebbero dimorare?».


«Alloggiateli separati gli uni dagli altri».


«Se però non ci sono dimore isolate, Signore, che cosa si deve fare?».
«Allora distribuiscile dopo averle rese isolate, Sāriputta. Dico che per
nessuna ragione, tuttavia, deve essere negato un luogo in cui riposare a
un bhikkhu anziano. Chi si comporta così commette un’infrazione di atto
errato».


«Signore, come ci si deve comportare per il cibo e per tutte le altre
cose?».


«Il cibo e tutte le altre cose devono essere distribuite equamente a
tutti».


Ora, mentre il bhikkhu sospeso stava riflettendo sulla Disciplina, gli
venne in mente: «Era un’infrazione, non una non-infrazione, ho commesso
un’infrazione … sono sospeso. Sono stato sospeso mediante un atto legale
che non può essere annullato e che ha validità». Allora andò a
comunicarlo ai suoi sostenitori, e disse loro: «I venerabili possono
reintegrarmi».


I suoi seguaci lo condussero dal Beato e, dopo avergli prestato omaggio,
si misero a sedere da un lato. Raccontarono quel che il bhikkhu sospeso
aveva detto e chiesero: «Signore, come dobbiamo comportarci?».


«Bhikkhu, era un’infrazione, non una non-infrazione, egli ha commesso
un’infrazione … egli è sospeso. È stato sospeso mediante un atto legale
che non può essere annullato e che ha validità. Siccome quel bhikkhu,
che ha commesso quell’infrazione e che è stato sospeso ha visto
l’infrazione, potete reintegrarlo».


Dopo che i seguaci del bhikkhu sospeso lo ebbero reintegrato, andarono
dal bhikkhu che lo aveva sospeso e dissero: «Amici, a proposito del caso
sul quale vi era contrasto e disunione nel Saṅgha, il bhikkhu ha
commesso un’infrazione, è stato sospeso. Ora lui ha visto l’infrazione
ed è stato reintegrato. Celebriamo un atto di composizione al cospetto
del Saṅgha per chiudere la questione».


Allora il bhikkhu che aveva pronunciato la sospensione andò dal Beato e
gli raccontò quel che era avvenuto. L’atto di composizione fu approvato
e la procedura seguita.


\suttaRef{Vin. Mv. 10:5}


