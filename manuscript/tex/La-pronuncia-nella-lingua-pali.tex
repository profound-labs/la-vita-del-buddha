\chapter{La pronuncia nella lingua pāli}

Per quanto riguarda l’accento, si seguono convenzionalmente le regole
del latino: l’accento cade sulla penultima sillaba se è lunga per natura
(una vocale lunga sormontata da un trattino, oppure i dittonghi e e o,
per esempio \emph{Nibbāna, kilèsa}) o per posizione (una vocale che
precede due consonanti, per esempio \emph{avijjā}), oppure se la parola è
bisillaba; se la penultima sillaba è breve, l’accento si ritira sulla
terzultima (per esempio \emph{yàmaka}), salvo in parole di quattro o più
sillabe in cui penultima e terzultima siano brevi: in questo caso
l’accento si ritrae sulla quartultima. Nei composti, ogni parola
conserva il suo accento (per esempio \emph{Bùddha-ghòsa, Visùddhi-màgga}).
Inoltre:

\bigskip

{\setlength{\parindent}{0pt}%
\renewcommand\arraystretch{1.3}%
\fontsize{10}{14}\selectfont

\begin{tabular}{p{10mm} p{90mm}@{}}

  \emph{c} & rappresenta sempre una palatale sorda, anche davanti alle vocali \emph{a, o, u} (p. es., \emph{citta}; ma anche \emph{cakkhu}, pronunciato “ciakkhu”); \\

  \emph{g} & rappresenta sempre una gutturale sonora, anche davanti alle vocali \emph{e, i} (per esempio, \emph{garu}; ma anche \emph{gilāna}, pron. “ghìlana”); \\

  \emph{h} & è una consonante che indica un’aspirazione che deve essere pronunciata (lo \emph{hadaya}); l’aspirazione si deve far sentire anche quando segue una consonante occlusiva (p. es. in Dhamma); \\

  \emph{j} & rappresenta sempre una palatale sonora, anche davanti alle vocali \emph{a, o, u} (p. es., \emph{jana}, pron. “giana”); \\

  \emph{ṃ} & indica una nasalizzazione (p. es., \emph{saṃsāra}); \\

  \emph{ñ} & è il suono \emph{gn} dell’italiano \emph{gnosi} (p. es. in \emph{kañcuka}); \\

  \emph{ṇ} & è la nasale retroflessa (un suono intermedio fra la palatale di gnosi e la dentale di \emph{anta}) (p. es. in \emph{paṇḍu}); \\

  \emph{ṭ, h,\newline ḍ, ḍh} & sono consonanti occlusive retroflesse, pronunciate alla maniera inglese (\emph{t} di \emph{tree}) o siciliana (\emph{d} di \emph{bedda}); \\

  \emph{s} & è la sibilante sorda dell’italiano sarto (p. es. \emph{sutta}); si noti che nella lingua pāli non esiste il suono dell’italiano \emph{caso}, cioè la \emph{s} sonora o “dolce” intervocalica. \\

\end{tabular}

}

\vfill

\noindent
{\footnotesize
  (Quanto segue è ripreso, con lievi modifiche, dalla tabella posta
  all’inizio del volume \emph{Therīgāthā. Canti spirituali della monache
    buddhiste,} a cura di A.S. COMBA, Raleigh 2016.)
\par}

