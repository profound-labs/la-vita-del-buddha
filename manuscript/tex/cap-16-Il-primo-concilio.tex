\chapter{Il primo concilio}

\narrator{Primo narratore.} Dopo l’ottenimento del Nibbāna definitivo da parte
del Buddha, i bhikkhu si allontanarono da Kusinārā e si dispersero. L’Anziano
Mahā-Kassapa emerge quale figura più eminente nel Saṅgha dei bhikkhu.

\narrator{Secondo narratore.} Egli era già stato in precedenza menzionato dal
Buddha come il quarto nella lunga lista dei discepoli di particolare
distinzione. Il primo era l’Anziano Kondañña, il primo convertito; il secondo e
il terzo erano gli Anziani Sāriputta e Moggallāna, i due discepoli eminenti, che
già avevano ottenuto il Nibbāna definitivo. Ci sono molti racconti riguardanti
l’Anziano Mahā-Kassapa nel Canone, nel quale egli appare come una figura austera
ed energica, senza compromessi nella sua dedizione all’ascetismo. Egli impartì
più di un brusco rimprovero all’Anziano Ānanda per il suo indulgere ad atti di
gentile altruismo, invece di dedicarsi a beni più duraturi, portando a
compimento la sua perfezione, quella perfezione già raggiunta dallo stesso
Anziano Mahā-Kassapa.

\narrator{Primo narratore.} Ecco un episodio appartenente a questo periodo di
transizione, che ben illustra il suo carattere.

\voice{Prima voce.}\rangeStart{S-16-11}
Così ho udito. Una volta il venerabile Mahā-Kassapa
soggiornava a Rājagaha, nel Boschetto di Bambù, nel Sacrario degli Scoiattoli.
In quel momento il venerabile Ānanda andava errando nelle Colline Meridionali
con una larga comunità di bhikkhu. Fu allora che trenta fra coloro che
risiedevano con lui, la maggior parte dei quali erano giovani, abbandonarono
l’addestramento monastico e tornarono a quel che avevano lasciato.

Quando il venerabile Ānanda ebbe errato per le Colline Meridionali per tutto il
tempo che volle, andò dal venerabile Mahā-Kassapa nel Boschetto di Bambù, a
Rājagaha. Dopo avergli prestato omaggio, si mise a sedere da un lato. Il
venerabile Mahā-Kassapa disse: «Amico Ānanda, in ragione di quale beneficio il
Beato rese nota la regola che non più di tre bhikkhu insieme possono mangiare
presso delle famiglie?».

«Egli fece così, Venerabile Kassapa, in ragione di tre benefici: per il
contenimento delle persone che pensano in modo errato e per l’agio delle persone
ragionevoli, così che quanti hanno desideri malvagi non formino una fazione nel
Saṅgha, e per compassione nei riguardi delle famiglie».

«Allora, amico Ānanda, perché vai errando con questi nuovi bhikkhu, le cui porte
delle facoltà sensoriali sono incustodite, che non conoscono la giusta misura
nel mangiare, che non sono votati alla vigilanza? Si potrebbe pensare che vai
errando per distruggere raccolti. Si potrebbe pensare che vai errando per
distruggere famiglie. Il tuo seguito va in pezzi. I tuoi nuovi convertiti si
allontanano. E neanche questo ragazzo conosce ancora la sua propria misura!».

«Venerabile Kassapa, in realtà stanno crescendo dei capelli bianchi sulla mia
testa. Che il venerabile Kassapa la smetta di chiamarmi ragazzo».

«Questo è però quel che sei, amico Ānanda. Tu vai errando con questi nuovi
bhikkhu, le cui porte delle facoltà sensoriali sono incustodite, che non
conoscono la giusta misura nel mangiare, che non sono votati alla vigilanza. Si
potrebbe pensare che vai errando per distruggere raccolti. Si potrebbe pensare
che vai errando per distruggere famiglie. Il tuo seguito va in pezzi. I tuoi
nuovi convertiti si allontanano. E neanche questo ragazzo conosce ancora la sua
propria misura!».

La bhikkhuṇī Thullānandā\footnote{La bhikkhunī Thullānandā compare spesso nel
  Vinaya come una donna orgogliosa, intelligente e faziosa, che causò la stesura
  di numerose regole.} udì queste parole. Ella pensò: «Sembra che il Signore
Ānanda, il Veggente Videha, si sia dispiaciuto perché il Signore Kassapa l’ha
chiamato ragazzo». E lei si sentì offesa ed esclamò parole di dispiacere: «Come
può il Signore Mahā-Kassapa, che apparteneva a una setta, sognarsi di dispiacere
il Signore Ānanda, il Veggente Videha, chiamandolo ragazzo?».

Il venerabile Mahā-Kassapa la sentì mentre pronunciava queste parole. Allora
egli disse al venerabile Ānanda: «In verità, amico Ānanda, la bhikkhuṇī
Thullānandā ha parlato frettolosamente, senza riflettere. Da quando ho rasato i
capelli e la barba e indossato l’abito ocra, e rinunciato alla vita in famiglia
per una senza dimora, non ho guardato alcun altro maestro che non fosse il
Beato, realizzato e completamente illuminato. Prima ero un laico che pensava:
“La vita in famiglia è affollata e polverosa, l’abbandono di essa comporta
spaziose aperture. Vivendo in famiglia non è facile condurre una santa vita
assolutamente perfetta e immacolata come una conchiglia ben lucidata. E se mi
rasassi i capelli e la barba, indossassi l’abito ocra, e rinunciassi alla vita
in famiglia per una senza dimora?”. In seguito mi cucii una veste con dei panni
scartati. Poi mi rasai i capelli e la barba e indossai l’abito ocra per lo
stesso scopo di coloro che nel mondo sono degli Arahant, e rinunciai alla vita
in famiglia per una senza dimora».

«Quando abbracciai la vita religiosa, mentre ero in cammino su una strada, tra
Rājagaha e Nālandā vidi il Beato che sedeva nel Sacrario di Bahaputta. Allorché
lo vidi, pensai: “Se io mai riconoscessi un Maestro, che io riconosca come tale
solo il Beato. Se io mai riconoscessi un Sublime, che io riconosca come tale
solo il Beato. Se io mai riconoscessi un Essere Completamente Illuminato, che io
riconosca come tale solo il Beato”. Allora, prostrandomi ai suoi piedi, dissi:
“Signore, il Beato è il mio Maestro, io sono il suo discepolo. Il Beato è il mio
Maestro, io sono il suo discepolo”. Allora il Beato disse: “Kassapa, se qualcuno
dicesse ‘io conosco’ senza conoscere o ‘io vedo’ senza vedere a un discepolo
schietto e leale come te, la sua testa andrebbe a fuoco. È però conoscendo che
dico ‘io conosco’ e vedendo che dico ‘io vedo’. Perciò, Kassapa, devi
addestrarti in questo modo: ‘Una scrupolosa coscienza e un senso di vergogna si
insedierà in me a riguardo dei bhikkhu più anziani, dei nuovi bhikkhu e di
coloro di media anzianità monastica’. E in questo modo: ‘Io ascolterò il Dhamma
con orecchio attento, ascoltando, prestando attenzione e offrendo tutta la mia
mente a qualsiasi cosa sia favorevole a quel che è salutare’. E in questo modo:
‘Non mancherò mai di praticare volentieri la consapevolezza del corpo nel
corpo’. Devi addestrarti in questo modo”. Dopo che il Beato mi ebbe dato questi
consigli, si alzò dal posto in cui sedeva e se ne andò».

«Come un debitore mangiai cibo elemosinato in quel posto per soli sette giorni.
Nell’ottavo giorno sorse la conoscenza finale. Allora il Beato lasciò la strada
e andò ai piedi di un albero. Io piegai in quattro la mia veste fatta di panni
scartati e gli dissi: “Signore, che il Beato si metta a sedere qui, così che ciò
possa essere per lungo tempo a vantaggio del mio benessere e della mia
felicità”. Il Beato si mise a sedere nel posto preparatogli. Poi disse: “La tua
veste fatta di panni scartati è morbida, Kassapa”. – “Che il Beato accetti da me
per compassione la veste fatta di panni scartati”. – “Tu indosserai però la
veste fatta di panni scartati di canapa che io dismetterò, Kassapa?”. –
“Signore, io indosserò la veste fatta di panni scartati di canapa del Beato che
egli dismetterà”».

«Io diedi al Beato la mia veste fatta di panni scartati e, in cambio, presi la
veste fatta di panni scartati di canapa del Beato che egli aveva dismesso. Se di
qualcuno si potrebbe dire: “Egli è il figlio stesso del Beato, nato dalla sua
bocca, nato dal Dhamma, creato dal Dhamma, un erede del Dhamma, che riceve una
veste dismessa fatta di panni scartati di canapa”, è di me che in verità ciò
potrebbe essere detto».

\narrator{Primo narratore.} Egli proseguì raccontando come fosse in grado, ogni
volta che lo desiderava, di entrare e dimorare nei quattro jhāna e anche nei
quattro stati privi di forma, come pure nella cessazione della percezione e
della sensazione, avendo inoltre acquisito i cinque tipi di diretta conoscenza
mondana: ossia i poteri sovrannaturali, l’elemento dell’orecchio divino, la
penetrazione delle menti, il ricordo delle vite passate e l’occhio divino per
mezzo del quale si vede come gli esseri scompaiono e ricompaiono in accordo con
le loro azioni. Egli concluse:

\voice{Prima voce.} «Ogni volta che lo desidero, mediante la realizzazione di me
stesso con la conoscenza diretta qui e ora, io entro e dimoro nella liberazione
della mente e nella liberazione per mezzo della comprensione, che è priva delle
contaminazioni per l’esaurimento delle contaminazioni. Chi pensa che un
pachiderma alto quattordici piedi o più possa essere occultato da una foglia di
palma, può fantasticare di riuscire a superarmi a riguardo di questi sei tipi di
conoscenza diretta».

In seguito la bhikkhuṇī Thullānandā abbandonò la santa vita.

\sourceIndexRange{S-16-11}{S. 16:11}{S}{II}{217-222}{la grandezza di Mahā-Kassapa}

\narrator{Secondo narratore.} Sono passate ancora solo poche settimane dal
Parinibbāna.

\voice{Terza voce.} Il venerabile Mahā-Kassapa disse: «Ora, amici, recitiamo
l’Insegnamento e la Disciplina, il Dhamma e il Vinaya. Già erronei insegnamenti
ed erronea disciplina sono stati corteggiati, e retti insegnamenti e retta
disciplina sono stati derisi. E già sostenitori di erronei insegnamenti ed
erronea disciplina sono diventati forti, e sostenitori di retti insegnamenti e
retta disciplina si sono indeboliti».

«Allora, Signore, che l’Anziano convochi un’assemblea dei bhikkhu».

Così, il venerabile Mahā-Kassapa convocò un’assemblea di cinquecento Arahant
meno uno, perché i bhikkhu avevano detto: «C’è il venerabile Ānanda. Benché egli
sia solo un allievo – nella condizione di Chi è Entrato nella Corrente – è
tuttavia impossibile che vada a finire in una destinazione infelice a causa del
desiderio, dell’ira, dell’illusione o della paura. Egli conosce a fondo una
grande e varia parte del Dhamma e della Disciplina per essere stato in presenza
del Beato. Che l’Anziano convochi anche il venerabile Ānanda».

Così egli convocò anche il venerabile Ānanda. Allora egli chiese ai bhikkhu:
«Dove avrà luogo la ripetizione?».

I bhikkhu anziani pensarono: «Rājagaha è una grande residenza, con abbondanza di
alloggi. Perché non andare a Rājagaha e restare là per la stagione delle
piogge?». Così il venerabile Mahā-Kassapa offrì tale argomento di decisione al
cospetto del Saṅgha:

«Che il Saṅgha mi ascolti, amici. Se al Saṅgha sembra opportuno, che il Saṅgha
autorizzi quanto segue: che questi cinquecento bhikkhu restino a Rājagaha per
questa stagione delle piogge allo scopo di ripetere il Dhamma e la Disciplina, e
che nessun altro bhikkhu resti a Rājagaha per questa stagione delle piogge.
Questo è argomento di decisione. Che il Saṅgha mi ascolti, amici. Il Saṅgha
autorizzi quanto segue: che questi cinquecento bhikkhu restino a Rājagaha per
questa stagione delle piogge allo scopo di ripetere il Dhamma e la Disciplina, e
che nessun altro bhikkhu resti a Rājagaha per questa stagione delle piogge. Chi
è d’accordo resti in silenzio, chi non è d’accordo lo dica. Il Saṅgha è
d’accordo che questa decisione è autorizzata dal Saṅgha, perciò il Saṅgha resta
in silenzio. Così lo memorizzo».

Allora i bhikkhu anziani si incontrarono a Rājagaha per ripetere il Dhamma e la
Disciplina. Loro tuttavia pensarono: «La riparazione di quel che è rotto e
cadente era raccomandata dal Beato. Perciò, amici, occupiamoci di questo durante
il primo mese. Il secondo mese ci riuniremo per la ripetizione».

Intanto giunse il tempo in cui il venerabile Ānanda pensò: «L’assemblea è
domani. Non è decoroso che io mi presenti all’assemblea come un semplice
discente». Egli trascorse la maggior parte della notte nella contemplazione del
corpo nel corpo. Quando si avvicinò l’alba, egli pensò: «Mi metterò a giacere».
Egli, però, mantenne la consapevolezza del corpo nel corpo. Prima che il suo
capo toccasse il cuscino e dopo che i suoi piedi si furono staccati dal suolo,
nel corso di questo intervallo il suo cuore fu liberato dalle contaminazioni per
mezzo del non-attaccamento. Così il venerabile Ānanda si recò all’assemblea come
Arahant.

Allora il venerabile Mahā-Kassapa offrì tale argomento di decisione al cospetto
del Saṅgha: «Che il Saṅgha mi ascolti, amici. Se al Saṅgha sembra opportuno,
interrogherò il venerabile Upāli sulla Disciplina».

Allora il venerabile Upāli offrì tale argomento di decisione al cospetto del
Saṅgha: «Che il Saṅgha mi ascolti, Signori. Se al Saṅgha sembra opportuno, io,
interrogato sulla Disciplina dal venerabile Mahā-Kassapa, risponderò».

Allora il venerabile Mahā-Kassapa disse al venerabile Upāli: «Amico Upāli, dove
fu resa nota la Prima Sconfitta?».

«A Vesālī, Signore».

«A riguardo di chi?».

«A riguardo di Sudinna Kalandaputta».

«Per quale argomento?».

«Sull’argomento del rapporto sessuale».

\narrator{Secondo narratore.} L’Anziano Mahā-Kassapa allora interrogò l’Anziano
Upāli sull’argomento della Prima Sconfitta, sulla sua origine, sulla persona,
sulla proclamazione, sulle modifiche, sull’infrazione e su quel che non
rappresentava un’infrazione. Poi egli lo interrogò nello stesso modo a proposito
delle altre tre Sconfitte: rubare, uccidere degli esseri umani e fare
deliberatamente false dichiarazioni in relazione a conquiste spirituali. In
questo modo egli lo interrogò sui due Codici, ossia il \emph{Pātimokkha} dei
bhikkhu o Codice delle Regole Monastiche e quello delle bhikkhuṇī, come pure su
tutte le altre regole. L’Anziano Upāli rispose a ogni domanda.

\voice{Terza voce.} Allora il venerabile Mahā-Kassapa offrì tale argomento di
decisione al cospetto del Saṅgha: «Che il Saṅgha mi ascolti, amici. Se al Saṅgha
sembra opportuno, interrogherò il venerabile Ānanda sul Dhamma».

Allora il venerabile Ānanda offrì tale argomento di decisione al cospetto del
Saṅgha: «Che il Saṅgha mi ascolti, Signori. Se al Saṅgha sembra opportuno, io,
interrogato sul Dhamma dal venerabile Mahā-Kassapa, risponderò».

Allora il venerabile Mahā-Kassapa disse al venerabile Ānanda: «Amico Ānanda,
dove fu pronunciato il \emph{Brahmajāla Sutta?}».

«Tra Rājagaha e Nālandā, Signore, nella Casa del Re ad Ambalaṭṭhikā».

\narrator{Secondo narratore.} L’anziano lo interrogò poi sull’origine del
\emph{Brahmajāla Sutta}, il primo nella Raccolta dei Discorsi Lunghi, e sulla
persona. Poi egli lo interrogò nello stesso modo a proposito del
\emph{Sāmaññaphala Sutta}. In questa maniera egli lo interrogò su tutti i
discorsi di tutte le quattro Principali Raccolte del \emph{Sutta Piṭaka.}

\voice{Terza voce.} Allora il venerabile Ānanda disse ai bhikkhu anziani:
«Signori, il Beato nel tempo in cui ottenne il Nibbāna definitivo mi disse:
“Quando me ne sarò andato, il Saṅgha potrà, se lo desidera, abolire le regole
più minute e minori”».

«Amico Ānanda, ma tu hai chiesto al Beato quali erano le regole più minute e
minori?».

«No, Signori, non l’ho chiesto».

\narrator{Secondo narratore.} Gli anziani espressero diverse opinioni in
relazioni a quali regole, a parte le Quattro Sconfitte, dovessero essere
considerate minute e minori. Allora il venerabile Mahā-Kassapa offrì un
argomento di decisione al cospetto del Saṅgha.

\voice{Terza voce.} «Che il Saṅgha mi ascolti, amici. Ci sono alcune delle
nostre regole d’addestramento che coinvolgono i laici, mediante le quali i laici
conoscono quello che è permesso ai monaci che sono figli dei Sakya e quello che
non lo è. Se noi aboliamo queste regole più minute e minori, ci sarà chi dirà:
“Le regole d’addestramento proclamate dal monaco Gotama ai suoi discepoli
esistettero solo per il periodo che terminò con la sua cremazione; loro
osservarono le sue regole d’addestramento finché egli fu presente ma, ora che
egli ha ottenuto il Nibbāna definitivo, loro hanno rinunciato a osservare le sue
regole d’addestramento”. Se il Saṅgha lo ritiene opportuno, non permettiamo che
quello che non è stato proclamato sia proclamato e non permettiamo che quello
che è stato proclamato sia abolito. Che il Saṅgha proceda in accordo con le
regole d’addestramento così come esse sono state proclamate». Questo argomento
di decisione fu offerto al cospetto del Saṅgha e approvato.

Allora i bhikkhu anziani dissero al venerabile Ānanda: «Amico Ānanda, questa fu
una mancanza da parte tua: che tu non abbia chiesto al Beato quali fossero le
regole più minute e minori. Riconosci questa mancanza».

«Non fu deliberatamente, Signori, che non lo chiesi al Beato. Non la considero
una mancanza. Tuttavia, per fiducia nei venerabili, la riconosco come mancanza».

«Anche questa fu una mancanza da parte tua: che tu abbia camminato sulla veste
per la pioggia del Beato mentre la stavi cucendo. Riconosci questa mancanza».

«Non lo feci per mancanza di rispetto nei riguardi del Beato, Signori. Non la
considero una mancanza. Tuttavia, per fiducia nei venerabili, la riconosco come
mancanza».

«Anche questa fu una mancanza da parte tua: che tu abbia fatto salutare i resti
del Beato prima dalle donne. Riconosci questa mancanza. Loro stavano piangendo,
e i resti del Beato vennero macchiati dalle loro lacrime. Riconosci questa
mancanza».

«Sono stato costretto a comportarmi così, Signori, affinché l’ora non divenisse
inadatta per loro. Non la considero una mancanza. Tuttavia, per fiducia nei
venerabili, la riconosco come mancanza».

«Anche questa fu una mancanza da parte tua: che pure quando il Beato ti ha
offerto un’allusione così chiara, un’indicazione così evidente, tu non hai
implorato il Beato: “Signore, che il Beato viva per un’era, che il Beato viva
un’era per il benessere e la felicità di molti, per compassione nei riguardi del
mondo, per il bene, il benessere e la felicità di divinità e uomini”. Riconosci
questa mancanza».

«Fu perché la mia mente era sotto l’influsso di Māra, per questo non l’ho
chiesto al Beato. Non la considero una mancanza. Tuttavia, per fiducia nei
venerabili, la riconosco come mancanza».

«Anche questa fu una mancanza da parte tua: che tu ti sia interessato affinché
le donne abbracciassero la vita religiosa nel Dhamma e nella Disciplina
proclamate dal Beato. Riconosci questa mancanza».

«L’ho fatto, Signori, pensando che Mahāpajāpatī Gotamī era la sorella della
madre del Beato, era stata la sua nutrice, la sua madre adottiva, gli aveva dato
il latte, aveva allattato il Beato quando sua madre morì. Non la considero una
mancanza. Tuttavia, per fiducia nei venerabili, la riconosco come mancanza».

\suttaRef{Vin. Cv. 11:1-10}

In quel tempo il venerabile Purāṇa stava errando nelle Colline Meridionali con
una grande comunità di bhikkhu, con cinquecento bhikkhu. Allora, dopo che il
Dhamma e la Disciplina erano state ripetute dagli Anziani, quando l’Anziano
Purāṇa fu rimasto nelle Colline Meridionali per tutto il tempo che volle, egli
andò dagli Anziani nel Boschetto di Bambù a Rājagaha. Loro gli dissero: «Amico
Purāṇa, il Dhamma e la Disciplina sono stati ripetuti dagli Anziani. Tu appoggi
questa ripetizione?».

«Amici, il Dhamma e la Disciplina sono stati ben ripetuti dagli Anziani. Io li
ricorderò tuttavia come li ho uditi dalle labbra stesse del Beato».

\suttaRef{Vin. Cv. 11:11}

\narrator{Primo narratore.} Ecco ora un ultimo episodio, che mostra il giovane
Saṅgha che continua a vivere dopo la scomparsa del fondatore, un organismo
affermato, che è sopravvissuto in modo ininterrotto e fiorente per due millenni
e mezzo, fino a oggi.

\voice{Terza voce.} Così ho udito. Una volta il venerabile Ānanda viveva a
Rājagaha, nel Boschetto di Bambù, nel Sacrario degli Scoiattoli, non molto tempo
dopo che il Beato aveva ottenuto il Nibbāna definitivo.

In quel momento, tuttavia, il re Ajātasattu Vedehiputta di Magadha stava
fortificando Rājagaha, perché era diffidente nei riguardi del re Pajjota di
Avanti.

Al mattino il venerabile Ānanda si vestì, prese la ciotola e la veste superiore,
e andò a Rājagaha per la questua. Allora pensò: «È ancora troppo presto per
errare per la questua a Rājagaha. E se io andassi dove sono in corso i lavori
del ministro della difesa Moggallāna il brāhmaṇa?».

Così fece. Il brāhmaṇa lo vide arrivare. Allora egli disse: «Che il Maestro
Ānanda venga. Benvenuto al Maestro Ānanda. È da molto tempo che il Maestro
Ānanda non passa per questa strada. Che il Maestro Ānanda sieda. C’è un posto
preparato per lui».

Il venerabile Ānanda si mise a sedere nel posto preparatogli, mentre il brāhmaṇa
prese un seggio più basso e si mise a sedere da un lato. Egli disse: «Maestro
Ānanda, c’è un solo bhikkhu che possegga in tutti i modi e in ogni modo le
qualità che possedeva il Maestro Gotama?».

«Non c’è, brāhmaṇa. Perché il Beato fu colui che fece sorgere il sentiero non
sorto, colui che produsse il sentiero non prodotto, colui che dichiarò il
sentiero non dichiarato, il conoscitore del sentiero, il veggente del sentiero,
abile nel sentiero. Ora, però, quando i discepoli dimorano in conformità con
quel sentiero, lo padroneggiano, e fanno così seguendo lui».

Nel frattempo il loro discorso non poté essere condotto a termine, perché il
brāhmaṇa Vassakāra, ministro di Magadha, che stava ispezionando i lavori a
Rājagaha, arrivò dove si trovava il venerabile Ānanda e dove erano in corso i
lavori del ministro della difesa Moggallāna. Scambiò dei saluti e, quando questi
formali doveri di cortesia ebbero termine, si mise a sedere da un lato. Egli
disse: «Per quale discorso vi siete riuniti qui, ora? E nel frattempo quale
discorso non poté essere condotto a termine?».

Il venerabile Ānanda gli raccontò la conversazione che aveva appena avuto luogo.
Egli aggiunse: «Questo era il discorso che nel frattempo non poté essere
condotto a termine, perché tu sei arrivato».

«Maestro Ānanda, c’è un qualche bhikkhu nominato dal Maestro Gotama in questo
modo: “Costui sarà il vostro rifugio quando me ne sarò andato” e al quale potete
ora ricorrere?».

«Nessun bhikkhu fu nominato in questo modo dal Beato che conosce e vede,
realizzato e completamente illuminato».

«Allora, Maestro Ānanda, c’è un qualche bhikkhu che è stato scelto dal Saṅgha,
che è stato eletto dalla maggioranza dei bhikkhu anziani in questo modo: “Costui
sarà il nostro rifugio quando il Beato se ne sarà andato” e al quale potete ora
ricorrere?».

«Non c’è alcun bhikkhu di questo genere, brāhmaṇa. Noi abbiamo un rifugio. Il
Dhamma è il nostro rifugio».

«Maestro Ānanda, in che modo vanno comprese queste affermazioni?».

«Il Beato che conosce e vede, realizzato e completamente illuminato, ha reso
note le regole d’addestramento per i bhikkhu, e ha esposto il \emph{Pātimokkha},
il Codice delle Regole Monastiche. Tutti noi monaci che viviamo nel distretto di
un villaggio ci riuniamo nel giorno di \emph{Uposatha} ogni luna piena e ogni
luna nuova, e quando lo facciamo scegliamo un monaco che abbia familiarità con
il \emph{Pātimokkha}. Se un bhikkhu ha commesso un’infrazione, una
trasgressione, dopo che questo Codice delle Regole Monastiche è recitato, è in
accordo con il Dhamma, in accordo con il precetto, che è da lui agito
[confessando la sua trasgressione]: non sono certamente delle persone che ci
fanno agire, ma è il Dhamma che ci fa agire».

«C’è un qualche bhikkhu, Maestro Ānanda, che voi ora onorate, rispettate,
riverite e venerate, e dal quale dipendete, onorandolo e rispettandolo?».

«C’è un bhikkhu di questo genere, brāhmaṇa».

«Maestro Ānanda, quando però ti è stato chiesto: “C’è un qualche bhikkhu
nominato dal Maestro Gotama in questo modo: ‘Costui sarà il vostro rifugio
quando me ne sarò andato’ e al quale potete ora ricorrere?” tu hai risposto che
non c’è. E quando ti è stato chiesto: “C’è un qualche bhikkhu che è stato scelto
dal Saṅgha, che è stato eletto dalla maggioranza dei bhikkhu anziani in questo
modo: ‘Costui sarà il nostro rifugio quando il Beato se ne sarà andato’ e al
quale potete ora ricorrere?”, tu hai risposto che non c’è. E quando ti è stato
chiesto: “C’è un qualche bhikkhu, Maestro Ānanda, che voi ora onorate,
rispettate, riverite e venerate, e dal quale dipendete, onorandolo e
rispettandolo?” tu hai risposto che c’è. In che modo vanno comprese queste
affermazioni?».

«Brāhmaṇa, dieci cose che ispirano fede e fiducia sono state descritte dal Beato
che conosce e vede, realizzato e completamente illuminato. Noi onoriamo,
rispettiamo, riveriamo e veneriamo colui nel quale queste dieci cose si
evidenziano, e viviamo dipendendo da lui, onorandolo e rispettandolo. Quali
dieci?».

«Un bhikkhu è virtuoso, contenuto con il contenimento del \emph{Pātimokkha},
perfetto nella condotta e nel modo di vivere, egli teme il più piccolo errore,
si addestra portando a effetto i precetti dell’addestramento. Egli ha imparato
molto, e rammenta e ricorda quello che ha udito, gli insegnamenti che sono
salutari al principio, salutari nel mezzo e salutari alla fine, con il
significato e il senso letterale; egli spiega la santa vita che è assolutamente
perfetta e pura, gli insegnamenti che lui ha ben imparato, li ricorda e
consolida per mezzo della parola, li esamina nella sua mente e li penetra a
fondo mediante la retta visione. Egli è contento delle sue vesti monastiche, del
cibo ricevuto in elemosina, del suo alloggio e delle medicine. Egli ottiene a
suo piacimento, senza problemi né riserve, i quattro jhāna che appartengono alle
menti più elevate e procurano un piacevole dimorare qui e ora. Egli è dotato dei
vari tipi di poteri sovrannaturali: essendo uno può diventare molti, essendo
molti può diventare uno; compare e scompare; attraversa senza impedimenti muri,
recinti, montagne, come se fossero spazio; egli sprofonda e sorge dalla terra
come se fosse acqua; seduto a gambe incrociate viaggia nello spazio come un
uccello; con la sua mano tocca e accarezza la luna e il sole, così forte e
potente; egli esercita la padronanza del suo corpo fino al mondo di Brahmā. Con
l’elemento dell’orecchio divino, che è purificato e supera quello umano, egli
sente i quattro tipi di suoni, quelli divini e quelli umani, vicini e lontani.
Egli penetra con la sua mente nella mente degli altri esseri, delle altre
persone; egli comprende la [coscienza] affetta dalla brama come affetta dalla
brama … (si veda il cap.~12, pag.~\pageref{pag272} -- \emph{Un bhikkhu comprende
  la coscienza}) … e la [coscienza] non liberata come non liberata. Egli ricorda
la molteplicità delle sue vite passate … (si veda il cap.~2,
pag.~\pageref{pag27b} -- \emph{Quando la mia mente fu così concentrata}). Con
l’occhio divino, che è purificato e supera quello umano, egli vede gli esseri
morire e rinascere … (si veda il cap.~2, pag.~\pageref{pag28} -- \emph{Quando la
  mia mente fu così concentrata}) … comprende come gli esseri scompaiano in
accordo con le loro azioni. Mediante la realizzazione di se stesso con la
conoscenza diretta, egli qui e ora entra e dimora nella liberazione della mente
e nella liberazione mediante comprensione immacolata per l’esaurimento delle
contaminazioni. Queste sono le dieci cose».

Quando ciò fu detto, il brāhmaṇa Vassakāra si girò verso il generale Upananda e
gli chiese: «Che cosa pensi, generale? Se questo è il modo in cui queste degne
persone onorano chi dovrebbe essere onorato, non lo fanno allora a ragione? Se
non facessero così, chi in verità dovrebbero onorare, rispettare, riverire e
venerare, in dipendenza da chi dovrebbero vivere, onorandolo e rispettandolo?».

Il brāhmaṇa Vassakāra chiese poi al venerabile Ānanda: «Dove vive ora il Maestro
Ānanda?».

«Ora vivo nel Boschetto di Bambù, brāhmaṇa».

«Spero, Maestro Ānanda, che il Boschetto di Bambù sia gradevole e silenzioso,
non disturbato da voci, un luogo con un’atmosfera di separatezza, dove si può
rimanere nascosti dalla gente e favorevole al ritiro».

«In verità, brāhmaṇa, è grazie a guardiani che lo proteggono, come te, che il
Boschetto di Bambù ha tutte quelle qualità».

«In verità, Maestro Ānanda, è grazie alle brave persone che apprendono la
meditazione e la praticano, che il Boschetto di Bambù ha tutte quelle qualità,
perché queste brave persone apprendono la meditazione e la praticano. Una volta
il Maestro Gotama viveva a Vesālī, nel Salone con il Tetto Aguzzo nella Grande
Foresta. Allora mi recai là e mi avvicinai a lui. E là il Maestro Gotama parlò
della meditazione in molti modi. Il Maestro Gotama era uno che praticava la
meditazione ed era avvezzo alla meditazione. Infatti, il Maestro Gotama
raccomandava tutti i tipi di meditazione».

«Il Beato non raccomandava tutti i tipi di meditazione, brāhmaṇa. E nemmeno
condannava tutti i tipi di meditazione. Il Beato quali tipi di meditazione non
raccomandava? Quando qualcuno dimora con il cuore posseduto dalla brama, è una
preda della brama e non comprende rettamente l’abbandono della brama. Egli
impiega ancora il desiderio per tutto, e medita, medita troppo, non medita, e
rimedita di nuovo. E allo stesso modo è posseduto dalla malevolenza, dall’apatia
e dalla sonnolenza, dall’agitazione e dalla preoccupazione, o dal dubbio. Il
Beato non raccomandava questo tipo di meditazione».

«E quali tipi di meditazione raccomandava? Quando qualcuno, del tutto discosto
dai desideri sensoriali, discosto da stati [mentali] non salutari, entra e
dimora nel primo jhāna, che è accompagnato dal pensiero e dall’esplorazione
uniti alla felicità e al piacere nati dall’isolamento. Ed egli entra e dimora
nel secondo, nel terzo e nel quarto jhāna. Il Beato raccomandava questo tipo di
meditazione».

«Allora, Maestro Ānanda, sembra che il Maestro Gotama condannasse il tipo di
meditazione che meritava di essere condannato e raccomandava il tipo di
meditazione che meritava di essere raccomandato. E ora, Maestro Ānanda, noi
andiamo. Siamo impegnati e abbiamo molto da fare».

«È tempo ora, brāhmaṇa, di fare quel che ritieni opportuno».

Allora il brāhmaṇa Vassakāra, il ministro di Magadha, si alzò dal posto in cui
sedeva e, dopo aver approvato e manifestato accordo con le parole del venerabile
Ānanda, se ne andò per la sua strada. Subito dopo che se ne fu andato, il
ministro della difesa, il brāhmaṇa Moggallāna, disse: «Il Maestro Ānanda non ha
risposto alla nostra domanda».

«Non ti ho forse detto, brāhmaṇa: “Non c’è un solo bhikkhu che possegga in tutti
i modi e in ogni modo quelle qualità che il Beato, realizzato e completamente
illuminato, possedeva, perché il Beato fu colui che fece sorgere il sentiero non
sorto, colui che produsse il sentiero non prodotto, colui che dichiarò il
sentiero non dichiarato, il conoscitore del sentiero, il veggente del sentiero,
abile nel sentiero. Ora, però, quando i discepoli dimorano in conformità con
quel sentiero, sono posseduti da esso, e fanno così seguendo lui”?».

\suttaRef{M. 108}

\narrator{Secondo narratore.} Nel frattempo il re Ajātasattu era intento alla
distruzione del suo troppo possente vicino, la confederazione Vajji con capitale
a Vesālī, a nord-est al di là del Gange. Al fine di aiutarlo a raggiungere il
suo scopo, Vassakāra finse di cospirare contro di lui, si fece denunciare come
traditore e fuggì alla volta di Vesālī per chiedere asilo. I successivi tre anni
li impiegò per disseminare con astuzia sfiducia e sospetti reciproci tra i
componenti della confederazione. Quando giudicò che i tempi erano maturi,
informò segretamente il re Ajātasattu. I governanti di Vesālī erano allora
troppo disuniti per difendere il loro territorio, e Ajātasattu fu presto in
grado di riuscire con successo in una invasione e in un ampio massacro della
popolazione. Questa fu la fine dell’indipendenza dei Vajji. Il re Viḍūḍabha di
Kosala seguì velocemente l’esempio di suo cugino, invadendo il territorio dei
Sakya e dei Koliya posti sul suo confine nord-orientale, trattando nello stesso
modo le popolazioni che là vivevano.

\narrator{Primo narratore.} Tutto questo chiude il primo scenario di storia
dell’India. Per il successivo secolo e mezzo, fino alla nascita dell’Impero
Maurya con la sua nuova dinastia, vengono solo menzionati i nomi dei re di
Magadha e il racconto del Secondo Concilio degli Arahant, cento anni dopo il
Parinibbāna. In quel tempo, però, il grande regno settentrionale di Kosala era
sparito (come, non lo sappiamo) e Chandragupta (il “Sandrokottos” del
viaggiatore greco Megastene), in quanto erede dell’antico Magadha, detenne il
comando di tutta la vallata
del Gange, la cui capitale era ora a Patna (Pāṭaliputta). \\
Un racconto del Secondo Concilio fu aggiunto al \emph{Vinaya Piṭaka} senza
dubbio al tempo dello stesso concilio. Il Canone fu nuovamente recitato, e si
può ipotizzare che in tale occasione pochi sutta riguardanti il periodo
successivo al Primo Concilio vennero incorporati nel \emph{Sutta Piṭaka}. In un
terzo concilio, tenuto durante il regno dell’imperatore Asoka (il nipote di
Chandragupta), l’\emph{Abhidhamma Piṭaka} fu completato aggiungendo un libro
sulle eresie e, di fatto, il \emph{Tipiṭaka} venne chiuso.

In questo tempo erano sorte diciotto differenti “scuole”. Il \emph{Theravāda}
(la Dottrina degli Anziani) divenne dominante sotto Asoka, che abbracciò egli
stesso il buddhismo. Suo figlio (o, secondo alcune tradizioni, suo nipote),
l’Arahant Mahinda, portò il \emph{Tipiṭaka} in pāli con il suo Commentario a
Ceylon,\footnote{L’attuale Sri Lanka (BB).} mentre altri anziani si recarono in
altri territori. È questo \emph{Tipiṭaka} in pāli che è stato conservato fino ad
oggi a Ceylon, in Birmania, in Thailandia e in
Cambogia, dove ancora fiorisce il \emph{Theravāda}. \\
Se si accolgono le osservazioni del viaggiatore cinese ITsing, che arrivò in
India (ma non a Ceylon) alla fine del VII secolo, il \emph{Theravāda} prevaleva
in tutti i territori meridionali dell’India, mentre il \emph{Sarvāstivāda} (il
cui Canone, in sanscrito, è ritenuto meno antico di quello in pāli) a
settentrione, benché altre scuole fossero ampiamente diffuse in varie parti. Il
Canone \emph{Sarvāstivāda} si diffuse a nord e a nord-est, e il Canone in pāli a
sud e a sud-est. Il \emph{Mahāyāna}, che I-Tsing (lui stesso era un
Sarvāstivādin) pare suggerire avesse messo radici al suo tempo in tutte o nella
maggior parte delle scuole, sembra sia sorto da una di esse, precisamente il
\emph{Mahāsanghika}. Benché di tanto in tanto fiorente a Ceylon e in Birmania,
in questi territori esso non fu mai in grado di cancellare il suo più antico
rivale. In India, però, il buddhismo in tutte le sue forme si ritiene sia del
tutto scomparso nel XV secolo.

