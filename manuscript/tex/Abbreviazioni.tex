\chapter{Abbreviazioni}

{\setlength{\parindent}{0pt}\setlength{\parskip}{10pt}

\begin{tabular}{@{\hspace{15pt}} l l}
  Vin. & \emph{Vinaya Piṭaka} \\
  Sv. & \emph{Sutta-vibhanga} \\
  Pārā. & \emph{Pārājika} \\
  Saṅgh. & \emph{Saṅghādisesa} \\
  Pāc. & \emph{Pācittiya} \\
  Mv. & \emph{Mahāvagga} \\
  Cv. & \emph{Cullavagga} \\
\end{tabular}

\section{Sutta Piṭaka}

  \begin{tabular}{@{\hspace{15pt}} l l}
  D. & \emph{Dīgha-nikāya} \\
  M. & \emph{Majjhima-nikāya} \\
  S. & \emph{Saṃyutta-nikāya} \\
  A. & \emph{Anguttara-nikāya} \\
\end{tabular}

\section{Khuddaka-Nikāya}

  \begin{tabular}{@{\hspace{15pt}} l l}
  Khp. & \emph{Khuddaka-pāṭha} \\
  Ud. & \emph{Udāna} \\
  Iti. & \emph{Itivuttaka} \\
  Sn. & \emph{Sutta-nipāta} \\
  Dh. & \emph{Dhammapada} \\
  Thag. & \emph{Theragāthā} \\
\end{tabular}

I rinvii sono al capitolo (\emph{khandhaka}) e al numero della sezione per il
\emph{Mahāvagga} e per il \emph{Cullavagga}, al numero della regola per gli
altri libri del \emph{Vinaya Piṭaka,} al discorso mediante il numero o mediante
il gruppo e il numero per i libri principali del \emph{Sutta Piṭaka}, e al
numero del verso per il \emph{Dhammapada} e per le \emph{Theragāthā}.

\begin{tabular}{@{\hspace{15pt}} l l}
NDT & nota del traduttore italiano \\
\end{tabular}

}
