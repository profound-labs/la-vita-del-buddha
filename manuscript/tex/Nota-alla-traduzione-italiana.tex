\chapter{Nota alla traduzione italiana}

Per rendere più agevole la lettura, a differenza dell’edizione inglese
le note sono state collocate al fondo della pagina e non alla fine dei
capitoli. In pochissimi casi si è ritenuto opportuno aggiungere delle
note supplementari che, però, non alterano la numerazione delle note
dell’opera originale, perché anch’esse inserite al fondo della pagina ma
prive di numerazione e contrassegnate dalla sigla NDT.

Nella prima edizione la traduzione è stata
revisionata e corretta da bhikkhu Mahāpañño e poi da Danilo Briarava,
Francesca Fenu, Roberto Bertozzi e Sara Bellettato. In questa seconda edizione
l’impegno di molti ha consentito di eliminare vari refusi presenti nella prima
edizione. Con la consueta attenzione ci ha accompagnati pure Roberto Luongo,
che ha aiutato anche per la versione digitale del libro. Siamo davvero contenti
di aver avuto l’opportunità di mettere a disposizione dei lettori italiani questa
importante – anche perché fedele al Canone in pāli – biografia del Buddha.

«Ogni traduzione è una distorsione», scrisse bhikkhu Ñāṇamoli nella sua
introduzione, riferendosi ai problemi di traduzione dalla lingua pāli
all’inglese. Inevitabilmente, ciò vale anche per questo lavoro, che si è
cercato di realizzare nel modo più coscienzioso possibile. Una
traduzione letterale è stata ovviamente impossibile: sono stati perciò
necessari alcuni ritocchi che, miranti a facilitare la comprensione
del testo da parte di quanti leggeranno questa \emph{Vita del Buddha}, hanno
però inevitabilmente ritoccato anche qualche intendimento dell’autore di
questo splendido libro e, forse, pure il senso di talune parole del
Canone. Sempre per favorire la comprensione del testo, è stato
necessario inserire tra [ ] alcune parole assenti nell’edizione inglese.

\enlargethispage{\baselineskip}

In questa traduzione sono stati indicati in tondo quei termini in lingua
pāli che, come dice bhikkhu Bodhi nella \emph{Nota alla terza edizione}, sono
entrati nella terminologia corrente del Dhamma: “Buddha”, “Dhamma”,
“Saṅgha”, “Nibbāna”, “Tathāgata”. A queste si aggiungano anche
“Arahant”, “bhikkhu”, “bhikkhuṇī”, “bodhi”, “brāhmaṇa”, “deva”, “jhāna”,
“sutta”, “Vinaya” e, ovviamente, tutti i nomi di luogo (inclusi fiumi e
monti), di persona e di stirpe.

{\raggedleft
Roberto Paciocco
\par}

