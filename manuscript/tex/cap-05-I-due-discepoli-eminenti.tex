\chapter{I due discepoli eminenti}

\voice{Seconda voce.} Avvenne questo. L’asceta itinerante Sañjaya si trovava a
Rājagaha con un gran seguito di asceti itineranti, con duecentocinquanta asceti
itineranti. E Sāriputta e Moggallāna stavano vivendo la vita religiosa sotto
l’asceta itinerante Sañjaya. Avevano fatto questo patto: chi di loro avesse per
primo raggiunto Ciò Che Non Muore avrebbe informato l’altro. Ora, essendo
mattino, il venerabile Assaji si vestì, prese la ciotola e la veste superiore, e
si recò a Rājagaha per la questua. Il suo portamento ispirava fiducia e, se
andava avanti o indietro, se guardava davanti o di lato, se si piegava o si
distendeva, teneva gli occhi bassi e si muoveva con grazia. L’asceta itinerante
Sāriputta lo vide mentre faceva la questua a Rājagaha, e pensò: «Nel mondo ci
sono degli Arahant, quelli che posseggono il sentiero degli Arahant, e questo
bhikkhu è uno di loro. E se io mi avvicinassi a lui e chiedessi sotto chi
conduce la vita religiosa, chi è il suo maestro, quale Dhamma professa?». Poi
pensò: «Non è questo il momento di fare domande a questo bhikkhu, mentre fa la
questua camminando fra le case. E se io lo seguissi per risalire a quel che
hanno scoperto i cercatori?».

Quando il venerabile Assaji ebbe terminato il giro per la questua, lasciò
Rājagaha con il cibo ottenuto in elemosina. Allora l’asceta itinerante Sāriputta
si avvicinò a lui e lo salutò. Terminati i formali doveri di cortesia, si mise
in piedi da un lato e gli disse: «Amico, le tue facoltà sono rasserenate, il
colore della tua pelle è chiaro e luminoso. Sotto chi hai praticato la vita
religiosa? Chi è il tuo maestro? Quale Dhamma professi?».

«C’è il Grande Monaco, amico, il figlio dei Sakya, che ha abbandonato un clan
dei Sakya per la vita religiosa. Ho abbracciato la vita religiosa sotto il
Beato. Egli è il mio maestro. È il Dhamma di quel Beato che io professo».

«Che cosa dice il venerabile maestro, che cosa insegna?».

«Ho abbracciato la vita religiosa solo da poco, amico, sono appena giunto a
questo Dhamma e a questa Disciplina, non posso insegnarti il Dhamma nei
dettagli. Ti dirò, però, il suo significato in breve».

Allora Sāriputta disse: «E sia, amico».

\begin{quote}
Di' pure molto o poco, come ti sembra giusto, \\
ma dimmi ora il significato. \\
Perché ho bisogno solo del significato, \\
non mi faccio alcun pensiero dei dettagli, ora.
\end{quote}

Il venerabile Assaji offrì all’asceta itinerante Sāriputta tale schema:

\begin{quote}
Il Perfetto ha detto la causa \\
del sorgere condizionato delle cose, \\
e anche quel che conduce alla loro cessazione: \\
questa è la dottrina predicata dal Grande Monaco.
\end{quote}

Quando l’asceta itinerante Sāriputta ascoltò quest’esposizione di Dhamma, la
pura, immacolata visione del Dhamma sorse in lui: tutto quel che sorge deve
cessare.

\begin{quote}
Questa è la verità: anche se questo fosse tutto, \\
tu hai raggiunto la condizione in cui non c’è dolore \\
che noi per molte volte diecimila ere \\
abbiamo lasciato passare inosservata.
\end{quote}

L’asceta itinerante Sāriputta andò dall’asceta itinerante Moggallāna. L’asceta
itinerante Moggallāna lo vide arrivare. Gli disse: «Le tue facoltà sono
rasserenate, il colore della tua pelle è chiaro e luminoso. Hai forse trovato
Ciò Che Non Muore?». «Sì, amico, ho trovato Ciò Che Non Muore». «Come hai fatto
a trovarlo, amico?».

L’asceta itinerante Sāriputta raccontò quello che era avvenuto. Quando l’asceta
itinerante Moggallāna ascoltò quell’esposizione di Dhamma:

\begin{quote}
Il Perfetto ha detto la causa \\
del sorgere condizionato delle cose, \\
e anche quel che conduce alla loro cessazione: \\
questa è la dottrina predicata dal Grande Monaco.
\end{quote}

La pura, immacolata visione del Dhamma sorse in lui: tutto quel che sorge deve
cessare.

\begin{quote}
Questa è la verità: anche se questo fosse tutto, \\
tu hai raggiunto la condizione in cui non c’è dolore \\
che noi abbiamo lasciato passare inosservata \\
per molte volte diecimila ere.
\end{quote}

Allora Moggallāna disse: «Amico, andiamo dal Beato. Il Beato è il nostro
maestro».

«Però, amico, questi duecentocinquanta asceti itineranti che vivono qui
dipendono da noi, noi siamo il loro punto di riferimento. Prima devono essere
consultati. Loro faranno quello che ritengono giusto».

Insieme si recarono dagli asceti itineranti e dissero loro: «Amici, noi stiamo
andando dal Beato. Il Beato è il nostro maestro».

«Noi viviamo in dipendenza dai venerabili, sono loro il nostro punto di
riferimento. Se loro vanno a vivere la santa vita sotto il Grande Monaco, anche
noi faremo lo stesso».

Così, Sāriputta e Moggallāna andarono da Sañjaya l’asceta itinerante e gli
dissero quello che stavano per fare.

«Va bene così, amici, non andate. Guidiamo tutti e tre insieme questa comunità».

Una seconda e una terza volta loro dissero la stessa cosa e ricevettero la
stessa risposta.

Allora Sāriputta e Moggallāna andarono con i duecentocinquanta asceti itineranti
al Boschetto di Bambù. Sangue bollente sgorgò dalla bocca dell’asceta itinerante
Sañjaya.

Il Beato vide da lontano che arrivavano Sāriputta e Moggallāna. Quando li vide,
disse ai bhikkhu: «Stanno arrivando questi due amici, Kolita e Upatissa. Questi
due saranno i miei discepoli eminenti, una coppia di buon auspicio».

\begin{quote}
Fu allora che il Maestro li annunciò – \\
loro, che erano già liberati \\
nel campo della profonda conoscenza, \\
nella suprema distruzione della materia dell’esistenza, \\
già prima che raggiungessero il Boschetto di Bambù – \\
dicendo: «Stanno arrivando questi due amici, \\
Kolita e Upatissa. \\
Questi due saranno i miei discepoli eminenti, \\
una coppia di buon auspicio».
\end{quote}

Sāriputta e Moggallāna si avvicinarono al Beato e si prostrarono ai suoi piedi.
Gli dissero: «Signore, desideriamo abbracciare la vita religiosa sotto il Beato,
e l’ammissione».

«Venite bhikkhu», disse il Beato. E aggiunse: «Il Dhamma è ben proclamato.
Vivete la santa vita per completare la fine della sofferenza». E quella fu
l’ammissione di quei venerabili.

Ora, in quel tempo, un certo numero di ben noti uomini di rango di Magadha
stavano conducendo la vita santa sotto il Beato. La gente disapprovava,
mormorava e protestava: «Il monaco Gotama sta causando assenza di prole e
vedovanza, sta distruggendo le stirpi. Già un migliaio di asceti dai capelli
intrecciati ha abbracciato la vita religiosa, come pure questi duecentocinquanta
asceti itineranti, e ora anche questi ben noti uomini delle stirpi di Magadha
sono andati a condurre la santa vita sotto il monaco Gotama!». Quando la gente
vedeva i bhikkhu, li sbeffeggiava con queste strofe:

\begin{quote}
Gotama il monaco è arrivato \\
alla Fortezza di Magadha, \\
ha portato via tutto il gruppo di Sañjaya. \\
Chi porterà via oggi?
\end{quote}

I bhikkhu udirono queste parole, andarono dal Beato e gliele riferirono. Egli
disse: «È una cosa che non durerà a lungo. Durerà solo sette giorni. Al termine
di sette giorni cesserà. Così, quando la gente vi sbeffeggia con quelle strofe,
potete rispondere rimproverandola con queste strofe:

\begin{quote}
Sono portati dal Dhamma, loro che sono anche \\
Grandi Eroi ed Esseri Perfetti. \\
E perciò, giacché sono portati dal Dhamma, \\
per quale ragione essere gelosi?
\end{quote}

Così, quando la gente li sbeffeggiava, loro rispondevano rimproverandola. Allora
la gente cominciò a pensare: «I monaci figli del Sakya sono portati dal Dhamma,
sembra, non vanno contro il Dhamma». E la cosa durò sette giorni, e al termine
di sette giorni cessò.

\suttaRef{Vin. Mv. 1:23-24}

\narrator{Secondo narratore.} L’Anziano Moggallāna ottenne la condizione di
Arahant sette giorni dopo essersi recato dal Buddha. L’Anziano Sāriputta
trascorse invece due settimane a passare in rassegna e ad analizzare con la
visione profonda tutti i livelli della coscienza. Come divenne un Arahant è
narrato nel modo seguente.

\voice{Prima voce.} Così ho udito. Mentre il Beato soggiornava a Rājagaha, nella
Caverna Sūkarakhatā, l’asceta itinerante Dīghanakha andò da lui e scambiò i
saluti. Poi disse: «La mia teoria e il mio punto di vista è questo, Maestro
Gotama: “Niente mi piace”».

«Questo è il tuo punto di vista, Aggivessana, “Niente mi piace”: nemmeno questo
punto di vista ti piace?».

«Anche se questo mio punto di vista mi piacesse, tutto sarebbe uguale, Maestro
Gotama, tutto sarebbe uguale».

«Al mondo sono in molti a dire “tutto sarebbe uguale”, e non solo non riescono
ad abbandonare questo punto di vista, ma si attaccano pure ad altri punti di
vista. E al mondo sono pochi a dire “tutto sarebbe uguale”, e abbandonano questo
punto di vista senza attaccarsi ad altri punti di vista».

«Alcuni monaci e brāhmaṇa hanno questa teoria e punto di vista “Tutto mi piace”,
altri “Niente mi piace”, e altri ancora “Qualcosa mi piace” e “Qualcosa non mi
piace”. Ora, il punto di vista di quelli la cui teoria e punto di vista è “Tutto
mi piace” è prossimo alla brama, alla schiavitù, all’assaporare, all’adesione,
all’attaccamento. Il punto di vista di quelli la cui teoria e punto di vista è,
però, “Niente mi piace” è prossimo alla non-brama, alla non-schiavitù, al
non-assaporare, alla non-adesione, al non-attaccamento».

L’asceta itinerante Dīghanakha osservò: «Il Maestro Gotama loda il mio punto di
vista, il Maestro Gotama loda il mio punto di vista».

«E il punto di vista di quelli la cui teoria e punto di vista è “Qualcosa mi
piace” e “Qualcosa non mi piace” è, in ciò che a loro piace, prossimo alla
brama, alla schiavitù, all’assaporare, all’adesione, all’attaccamento, mentre,
in ciò che a loro non piace, è prossimo alla non-brama, alla non-schiavitù, al
non-assaporare, alla non-adesione, al non-attaccamento».

«Un uomo saggio che, tra questi monaci e brāhmaṇa la cui teoria e punto di vista
è “Tutto mi piace”, farebbe questa considerazione: “Il mio punto di vista è che
tutto mi piace”. Se però lo fraintendessi e insistessi dicendo: “Solo questo è
vero, qualsiasi altra cosa è sbagliata”, allora mi scontrerei con entrambi gli
altri gruppi: con i monaci e brāhmaṇa la cui teoria e punto di vista è “Niente
mi piace” e con i monaci e brāhmaṇa la cui teoria e punto di vista è “Qualcosa
mi piace” e “Qualcosa non mi piace”. Mi scontrerei con questi due gruppi. E
quando c’è scontro, ci sono dispute, ci sono discussioni. E quando ci sono
discussioni, c’è danno».

«Quando presagisce questo, egli abbandona quel punto di vista senza attaccarsi a
qualche altro punto di vista. È in questo modo che tali punti di vista vengono
abbandonati, lasciati».

\narrator{Secondo narratore.} Lo stesso è ripetuto per l’“uomo saggio” il cui
punto di vista è “Niente mi piace”, “Qualcosa mi piace” e “Qualcosa non mi
piace”.

\voice{Prima voce.} Ora, Aggivessana, questo corpo che ha una forma materiale
consiste di quattro grandi entità: terra, acqua, fuoco e aria. È procreato da
madre e padre, e cresciuto con riso e pane. Esso è soggetto all’impermanenza, a
essere unto e sfregato, alla dissoluzione e alla disintegrazione. Deve essere
considerato impermanente, come una sofferenza, come una piaga, come una freccia,
come una calamità, come un’afflizione, come un estraneo, come in via di
disintegrazione, come vuoto, come non-sé. Quando è considerato in questo modo,
si abbandona ogni desiderio e amore per esso e l’abitudine di trattarlo come
base necessaria di tutte le sue inferenze».\pagenote{«Abitudine di trattarlo (il
  corpo fisico) come base di tutte le sue inferenze» (\emph{kāyanvayatā}) rinvia
  al modo di pensare secondo il quale il corpo fisico è una realtà basilare, una
  verità empirica, per poi costruire su tale assunto un sistema (il
  materialismo, nei fatti, la visione fisiologica della mente, o la visione
  della coscienza come un “epifenomeno” della materia). Sia questo punto di
  vista sia il suo opposto, che considera la materia come subordinata alla
  mente, sono discusse all’inizio di M. 36.}

«Ci sono tre generi di sensazioni: sensazione piacevole, sensazione dolorosa e
sensazione-né-dolorosa-né-piacevole. Quando un uomo prova una di queste tre, non
prova le altre due. La sensazione piacevole è impermanente, formata, originata
in dipendenza di qualcos’altro, soggetta a esaurirsi, diminuire, svanire e
cessare. E così è pure per la sensazione dolorosa e per la sensazione neutra».

«Quando un ben istruito nobile discepolo vede questo, diventa disincantato nei
riguardi della sensazione piacevole e della sensazione dolorosa e della
sensazione neutra. Diventando disincantato, la sua brama svanisce. Con lo
svanire della brama, il suo cuore è liberato. Quando il suo cuore è liberato,
giunge la conoscenza: “È liberato”. Egli comprende: “La nascita è distrutta, la
santa vita è stata vissuta, quel che doveva essere fatto è stato fatto, non ci
sarà altra rinascita”. Un bhikkhu con il cuore così liberato non parteggia per
nessuno, non disputa con nessuno e utilizza, ma senza fraintendimenti, il
linguaggio corrente del mondo».

Per tutto il tempo, il venerabile Sāriputta era stato in piedi dietro il Beato
per fargli aria con un ventaglio. Allora pensò: «Il Beato, il Sublime, sembra
che parli per diretta conoscenza dell’abbandono e della rinuncia a queste cose».
E quando pensò in questo modo il suo cuore fu liberato dalle contaminazioni
mediante il non-attaccamento.

Nel frattempo la pura, immacolata visione del Dhamma sorse nell’asceta
itinerante Dīghanakha … Egli disse: «… Prendo rifugio nel Maestro Gotama, e nel
Dhamma e nel Saṅgha».

\suttaRef{M. 74}

\narrator{Secondo narratore.} In questo tempo il re Suddhodana mandò Kāludāyī,
il figlio di uno dei suoi ministri, a Rājagaha al fine di persuadere suo figlio,
il Buddha, a visitare Kapilavatthu. Prima di comunicare la sua missione,
Kāludāyī divenne un bhikkhu. Alla fine della stagione fredda – era la prima dopo
l’Illuminazione – egli comunicò tuttavia la sua missione con questi versi,
miranti a persuadere il Buddha a mettersi in viaggio.

\cantor{Cantore}

\begin{quote}

  Signore, ci sono alberi che ora ardono come brace, \\
  sperando nei frutti, hanno lasciato cadere i loro verdi veli \\
  e bruciano audacemente con una fiamma scarlatta: \\
  è l’ora, Grande Eroe, Degustatore della Verità. \\
  Alberi pienamente in fiore che sono una delizia per la mente, \\
  effondono profumi ai quattro venti, \\
  le loro foglie hanno lasciato cadere, in attesa dei frutti: \\
  è l’ora, o Eroe, di partire da qui. \\
  Per i viaggi ora, Signore, la stagione è piacevole \\
  perché non è troppo freddo né troppo caldo. \\
  Consentite ai Sakya e ai Koliya di vedervi \\
  rivolto a occidente, mentre attraversate il fiume Rohiṇī.\pagenote{%
    Secondo il Commentario alle \emph{Theragāthā}, il fiume Rohiṇī scorre verso
    sud e separa, a ovest, il territorio dei Sakya da quello dei Koliya, che è a
    est. Rājagaha si trova molto più a sud, oltre il Gange, così che chi avesse
    viaggiato da questa città attraversando il Vajji e poi il territorio dei
    Koliya, avrebbe attraversato il fiume guardando verso occidente.}

  I campi sono arati con speranza, \\
  i semi sono piantati con speranza, \\
  i commercianti salpano con speranza \\
  attraverso il mare per la ricchezza: \\
  possa la speranza che nutro \\
  avere successo!

  Ancora e poi ancora si piantano i semi, \\
  ancora e poi ancora il Divino Sovrano invia la pioggia, \\
  ancora e poi ancora i contadini arano i campi, \\
  ancora e poi ancora il regno miete il grano, \\
  ancora e poi ancora i mendicanti chiedono l’elemosina, \\
  ancora e poi ancora i generosi offrono i loro doni, \\
  ancora e poi ancora l’offerta dei loro doni \\
  ancora e poi ancora fa trovare loro un posto in paradiso.

  Quale che sia il lignaggio nel quale è nato, \\
  un Eroe, detentore della vera comprensione, \\
  nobilita le sette precedenti generazioni – \\
  Tu, più grande degli dèi, lo sento, puoi fare ben di più, \\
  perché la parola “Perfetto” si è fatta vera in te.

\end{quote}

\suttaRef{Thag. 527-33}

\voice{Seconda voce.} Allorché il Beato era rimasto a Rājagaha per tutto il
tempo che volle, si mise in viaggio per Kapilavatthu. Viaggiando per tappe, alla
fine vi arrivò, e rimase nel Parco di Nigrodha. Ora, quando fu mattino, il Beato
si vestì, prese la ciotola e la veste superiore, si recò alla residenza di
Suddhodana il Sakya, e si mise a sedere nel posto preparatogli.

\suttaRef{Vin. Mv. 1:54}

\narrator{Primo narratore.} Il racconto di questa visita offerto dal Canone è
breve, perfino lapidario. Perciò, prima di continuare con tale racconto, alcuni
dettagli tratti dal Commentario renderanno più chiara la situazione.

\narrator{Secondo narratore.} Quando il Buddha arrivò a Kapilavatthu, gli uomini
del lignaggio Sakya, ben noti per il loro orgoglio, non erano inclini a
prestargli omaggio. A quel punto egli compì il miracolo doppio, causando il
simultaneo comparire di getti di fuoco e di acqua da tutte le sue membra. A ciò
seguì la predicazione della Storia della Nascita di Vessantara. Dopo il primo
pasto cerimoniale offertogli nel palazzo di suo padre, egli predicò la Storia
della Nascita di Dhammapāla, e il re ottenne il terzo, o penultimo, livello di
realizzazione. Egli morì come Arahant circa quattro anni dopo. Nel contempo la
regina, Mahāpajāpati, madre del principe Nanda e zia del Buddha, ottenne il
primo livello di realizzazione. Quello stesso giorno era stato scelto per la
celebrazione dell’imminente matrimonio del principe Nanda, unico figlio della
regina Mahāpajāpati. Ora, quando il Buddha si alzò per andar via, diede al
principe Nanda la sua ciotola e si avviò. Non sapendo che cosa fare, il principe
Nanda lo seguì con la ciotola, e quando si incamminò, la sua futura sposa gli
disse: «Torna presto, principe». Quando arrivarono nel luogo in cui il Buddha
dimorava, il Buddha gli chiese se volesse lasciare la casa famigliare. Più per
venerazione che per propensione, egli accettò. Al settimo giorno il Buddha
consumò di nuovo il suo pasto nel palazzo del padre.

\narrator{Primo narratore.} Ora continua il racconto canonico.

\voice{Seconda voce.} La madre del principe Rāhula disse al principe Rāhula:
«Questo è tuo padre, Rāhula. Vai a chiedergli la tua eredità». Allora il
principe Rāhula andò dal Beato e si mise in piedi di fronte a lui: «Il tuo
aspetto è gradevole, monaco».

Allora il Beato si alzò dal posto in cui sedeva e se ne andò. Il principe Rāhula
andò dietro al Beato, dicendo: «Dammi la mia eredità, monaco, dammi la mia
eredità, monaco».

Allora il Beato disse al venerabile Sāriputta: «Sāriputta, ammettilo alla vita
religiosa».\footnote{\emph{Pabbajjā}: l’ordinazione di un novizio (Nyp.).}

«Come faccio ad ammetterlo alla vita religiosa, Signore?». Il Beato, allora, per
questo motivo e per questa occasione offrì un discorso di Dhamma e si rivolse ai
bhikkhu in questo modo: «Consento che l’ammissione alla vita religiosa sia
impartita mediante i Tre Rifugi. L’ammissione deve però avvenire in questo modo.
Prima devono essere rasati i capelli e la barba, e indossata la veste ocra. Poi,
chi sta per essere ammesso deve ripiegare la veste superiore su una spalla, deve
prestare omaggio ai piedi del bhikkhu, si deve inginocchiare e, con le palme
delle mani giunte, deve dire: “Prendo rifugio nel Buddha, prendo rifugio nel
Dhamma, prendo rifugio nel Saṅgha. Per la seconda volta … Per la terza volta
…”».

Allora il venerabile Sāriputta impartì l’ammissione alla vita religiosa al
principe Rāhula. Suddhodana il Sakya andò dal Beato e, dopo avergli prestato
omaggio, si mise a sedere da un lato. Egli disse: «Chiedo un favore al Beato».

«Gli Esseri Perfetti si sono lasciati alle spalle i favori, Gotama».

«Si tratta di una cosa possibile e non riprovevole, Signore».

«Chiedi, allora, Gotama».

«Signore, ho provato non poco dolore quando il Beato se ne andò di casa per
abbracciare la vita religiosa. Poi fu la volta di Nanda. Rāhula è troppo.
L’amore per i nostri figli taglia la pelle esterna. Dopo aver tagliato la pelle
esterna, taglia la pelle interna. Dopo aver tagliato la pelle interna, taglia le
carni. Dopo aver tagliato le carni, taglia i tendini. Dopo aver tagliato i
tendini, taglia le ossa. Dopo aver tagliato le ossa, raggiunge il midollo e là
resta. Signore, sarebbe bene che i venerabili non impartissero l’ammissione alla
vita religiosa senza il consenso dei genitori».

Il Beato istruì, esortò, risvegliò e incoraggiò Suddhodana il Sakya con un
discorso di Dhamma. Allora Suddhodana il Sakya si alzò dal suo seggio, e dopo
aver prestato omaggio al Beato, se ne andò girandogli a destra.

\margintodo{missing closing »}%
Il Beato, allora, per questo motivo e per questa occasione offrì un discorso di
Dhamma, e si rivolse ai bhikkhu in questo modo: «Bhikkhu, non dovete ammettere
dei bambini alla vita religiosa senza il consenso dei genitori. Se qualcuno lo
fa, commette un’infrazione per atto errato.

\suttaRef{Vin. Mv. Kh. 1:54}

\narrator{Primo narratore.} Secondo la tradizione, la decisione del cugino del
Buddha, Ānanda, e di altri di lasciare la casa famigliare per la vita religiosa
avvenne al tempo di questa visita. Il Buddha era già andato via da Kapilavatthu,
ma si trovava ancora nei territori a nord di Kosala. Essa dovette verificarsi in
corrispondenza dei due seguenti episodi, benché non vi siano precise indicazioni
per collocarla.

\voice{Prima voce.} Così ho udito. Una volta il Beato stava viaggiando
attraverso la regione di Kosala con il venerabile Nāgasamāla, il suo monaco
attendente. Il venerabile Nāgasamāla vide che la strada si biforcava. Egli disse
al Beato: «Signore, questa è la direzione, andiamo in quella direzione».

Quando ciò fu detto, il Beato replicò: «Questa è la direzione, Nāgasamāla.
Andiamo in questa direzione».

Una seconda e una terza volta il venerabile Nāgasamāla disse la stessa cosa e
ricevette la stessa risposta. Poi poggiò la ciotola e la veste superiore del
Beato in terra e se ne andò. Quando percorse quella strada comparvero dei
ladroni che lo percossero con calci e pugni, gli ruppero la ciotola e
strapparono la veste superiore fatta di toppe. In seguito tornò dal Beato con la
ciotola rotta e la veste superiore fatta di toppe strappata, e gli raccontò
quello che era avvenuto. Conoscendo il significato di quest’avvenimento, il
Beato esclamò queste parole:

\begin{quote}
Un saggio e un folle \\
camminavano e vivevano in compagnia. \\
Per bere il latte le gru lasciano le acque paludose: \\
i saggi abbandonano quel che sanno essere male.
\end{quote}

\suttaRef{Ud. 8:7}

Ora, quando il Beato risiedeva nella regione di Kosala, ad Araññakuṭika, alle
pendici dell’Himalaya, mentre era in ritiro da solo sorse in lui questo
pensiero: «È possibile governare senza uccidere e ordinare esecuzioni capitali,
senza confiscare e sequestrare, senza addolorarsi e causare dolori, in altre
parole, governare rettamente?». Allora Māra il Malvagio nella sua mente fu
consapevole del pensiero sorto nella mente del Beato, e andò da lui e disse:
«Che il Beato governi, che il Sublime governi senza uccidere e ordinare
esecuzioni capitali, senza confiscare e sequestrare, senza addolorarsi e causare
dolori, in altre parole, governi rettamente».

«Malvagio, qual è il fine per cui ti rivolgi a me in questo modo?». «Signore le
quattro basi del successo [spirituale]\footnote{Le “quattro basi per il
  successo” (o vie per il potere) sono descritte come «la base per il successo
  che ha concentrazione fondata sul desiderio-di-agire e risolutezza motivata
  dallo sforzo-controllato» (M. 16). Questa è la prima. Per le altre tre,
  sostituire, rispettivamente, “energia”, “(naturale purezza della) mente”, e
  “investigazione” al “desiderio-di-agire”. Esse rappresentano i quattro tipi di
  approccio dello sviluppo, da regolare sulla base delle avversioni
  individuali.} sono state costantemente mantenute in essere e praticate dal
Beato, rese veicolo e base, sono state fondate, consolidate e propriamente
intraprese. E così, Signore, se il Beato decidesse: “Che l’Himalaya, re delle
montagne, diventi d’oro”, esso diventerebbe una montagna d’oro».

\begin{quote}
E se tutta quella montagna fosse di oro giallo, \\
il doppio non basterebbe a soddisfare i desideri di un uomo. \\
Sapere questo è agire di conseguenza. \\
Un uomo che ha visto la sofferenza e la sua fonte \\
come potrebbe volgersi verso i desideri sensoriali? \\
Sapendo che è questa sostanza della rinascita \\
a legarlo al mondo, un uomo \\
non può far altro che addestrarsi per liberarsene.
\end{quote}

Allora Māra il Malvagio seppe: «Il Beato mi conosce, il Sublime mi conosce».
Triste e deluso, subito sparì.

\suttaRef{S. 4:20}

\voice{Seconda voce.} Avvenne questo. Mentre il Beato soggiornava a Anupiyā –
una città dei Malla è chiamata Anupiyā – molti ben noti principi Sakya
abbracciarono la vita religiosa sotto il Beato. C’erano due fratelli, Mahānāma
il Sakya e Anuruddha il Sakya. Anuruddha era stato allevato tra gli agi.
Possedeva tre palazzi, uno per la stagione fredda, uno per la stagione calda e
un altro per quella delle piogge. Per quattro mesi era intrattenuto nel palazzo
per la stagione delle piogge da menestrelli, tra i quali non c’era alcun uomo e
non si recava mai nel piano inferiore del palazzo.

Mahānāma pensò: «Molti ben noti principi Sakya hanno abbracciato la vita
religiosa sotto il Beato. Nella nostra famiglia, però, nessuno ha lasciato la
propria casa per abbracciare la vita religiosa. E se fossi io a farlo, o
Anuruddha?».

Andò allora da Anuruddha e gli disse quel che aveva pensato. Anuruddha disse:
«Io sono stato allevato tra gli agi. Non posso lasciare la nostra casa per
abbracciare la vita religiosa. Sarai tu a farlo».

«Vieni allora Anuruddha, ti istruirò nella vita famigliare. Un campo deve essere
prima arato, poi deve essere seminato, poi in esso si deve condurre l’acqua, poi
l’acqua deve essere drenata, poi bisogna estirpare l’erba, poi deve essere
mietuto il raccolto, poi questo va riunito e ammucchiato, poi deve essere
trebbiato, poi si deve rimuovere la paglia, poi si deve eliminare la pula, poi
si deve setacciarlo e poi lo si deve immagazzinare. Ora, quando si è fatto tutto
questo, bisogna poi farlo di nuovo l’anno successivo, e l’anno dopo ancora. Il
lavoro non finisce mai. Non c’è fine per il lavoro».

«Quand’è che ci sarà una fine per il lavoro? Quando avremo mai modo di
gratificare i cinque lidi dei desideri sensoriali dei quali siamo dotati e
provvisti?».

«Mio caro Anuruddha, il lavoro non finisce mai, non c’è fine per il lavoro.
Nostro padre e nostro nonno sono morti entrambi quando il lavoro non era ancora
finito. Questo è ciò che devi sapere su questa vita famigliare. Io lascerò la
vita famigliare per abbracciare la vita religiosa».

Anuruddha andò dalla madre e le disse: «Madre, desidero lasciare la vita
famigliare per abbracciare la vita religiosa. Per favore, accordami il tuo
permesso».

Quando questo fu detto, lei gli disse: «Voi due, figli miei, mi siete cari e
preziosi, non sgraditi. Qualora moriste, dovremmo perdervi contro i nostri
desideri. Perché allora, giacché siete ancora in vita, dovrei darvi il permesso
di lasciare la vita famigliare per abbracciare la vita religiosa?». Lui lo
chiese una seconda e una terza volta. Allora la madre disse: «Mio caro
Anuruddha, se Bhaddiya il regio Sakya che governa i Sakya abbraccerà la vita
religiosa, potrai farlo anche tu».\pagenote{%
  Non è chiaro se la parola \emph{rājā} (qui resa con “che governa”) applicata a
  Bhaddiya il Sakya significhi “re” (nel qual caso implicita è la morte del re
  Suddhodana) o solo “reggente”. Qui è stata seguita la collocazione dell’evento
  offerta dal Commentario.}

In quel tempo Bhaddiya il regio Sakya che stava governando i Sakya era un amico
di Anuruddha e sua madre aveva pensato: «Bhaddiya è un amico di Anuruddha. Egli
non è ansioso di lasciare la vita famigliare per abbracciare la vita religiosa».
Per questa ragione lei aveva parlato in quel modo.

Allora Anuruddha andò da Bhaddiya e disse: «Che io abbracci o no la vita
religiosa dipende da te».

«Se che tu abbracci la vita religiosa dipende da me, che non sia più così,
allora. Tu ed io lo vogliamo … puoi abbracciare la vita religiosa quando vuoi».

«Vieni, lasciamo insieme la vita famigliare e abbracciamo la vita religiosa».

«Io non posso. Farò qualsiasi altra cosa per te. Sarai tu ad abbracciare la vita
religiosa».

«Mia madre ha detto: “Mio caro Anuruddha, se Bhaddiya il regio Sakya che governa
i Sakya abbraccerà la vita religiosa, potrai farlo anche tu”. E queste sono
state le tue parole: “Se che tu abbracci la vita religiosa dipende da me, che
non sia più così, allora. Tu e io lo vogliamo … puoi abbracciare la vita
religiosa quando vuoi”. Vieni, lasciamo insieme la vita famigliare e abbracciamo
la vita religiosa».

In quel tempo la gente era solita dire la verità, era solita essere di parola.
Bhaddiya disse ad Anuruddha: «Aspetta sette anni. Al termine dei sette anni
abbracceremo entrambi la vita religiosa».

«Sette anni sono troppi. Non posso aspettare sette anni».

«Aspetta sei anni. Al termine dei sei anni abbracceremo entrambi la vita
religiosa».

«Sei anni sono troppi. Non posso aspettare sei anni».

«Aspetta cinque anni … quattro … tre … due anni … un anno … sette mesi … due
mesi … un mese … Aspetta mezzo mese. Al termine di mezzo mese abbracceremo
entrambi la vita religiosa».

«Mezzo mese è troppo. Non posso aspettare mezzo mese».

«Aspetta sette giorni. Al termine di sette giorni abbracceremo entrambi la vita
religiosa. Così io posso tramandare il regno ai miei figli e fratelli».

«Sette giorni non sono troppi. Aspetterò».

Allora Bhaddiya il regio Sakya, Anuruddha, Ānanda, Bhagu, Kimbila e Devadatta,
insieme a Upāli il barbiere, che era il settimo, partirono alla testa di un
quadruplice esercito come se si recassero – così erano soliti fare – per una
parata nel parco.\footnote{La data in cui l’Anziano Ānanda abbracciò la vita
  religiosa non è del tutto certa. I versi da lui pronunciati nelle
  \emph{Theragāthā} indicano un momento successivo.} Allorché ebbero percorso
una certa distanza, abbandonarono l’esercito. Poi attraversarono il confine di
un altro regno e lì lasciarono le loro insegne. Le avvolsero in una veste e
dissero a Upāli il barbiere: «Upāli, faresti meglio a tornare indietro. Qui per
te c’è abbastanza di cui vivere».

Da parte sua, Upāli pensò: «Questi Sakya sono feroci. Per questo potrebbero
anche mettermi a morte, per essere stato complice dell’abbandono della vita
famigliare da parte dei principi. Così, questi principi Sakya stanno
abbandonando la vita famigliare per la vita religiosa. Che fare?». Aprì il
fagotto e appese le cose a un albero, dicendo: «Colui che le trova le prenda in
dono». Poi tornò indietro dai principi Sakya. Quando lo videro arrivare, gli
chiesero: «Perché sei tornato?».

Lui raccontò l’accaduto e aggiunse: «E così sono tornato».

«Hai fatto bene a non tornare a casa, Upāli, perché i Sakya sono feroci. Per
questo avrebbero potuto anche metterti a morte, per essere stato complice
dell’abbandono della vita famigliare da parte dei principi Sakya».

Allora i principi Sakya si recarono dal Beato con Upāli il barbiere e, dopo
avergli prestato omaggio, si misero a sedere da un lato. Dopo averlo fatto,
dissero al Beato: «Signore, siamo Sakya orgogliosi. Upāli, il barbiere, ci ha
assistiti per lungo tempo. Che il Beato lo ammetta per primo alla vita
religiosa, così da potergli prestare omaggio, alzarci in piedi per lui e
offrirgli saluti reverenziali e onori. Così, l’orgoglio dei Sakya sarà umiliato
in noi Sakya». Allora il Beato ammise per primo Upāli il barbiere alla vita
religiosa e poi i principi Sakya.

Fu durante questa stagione delle piogge che il venerabile Bhaddiya conseguì le
tre vere conoscenze. Nel venerabile Anuruddha sorse l’occhio divino. Il
venerabile Ānanda realizzò la fruizione di Chi è Entrato nella Corrente.
Devadatta ottenne i poteri sovrannaturali di un uomo ordinario.

In quel tempo, ogni volta che il venerabile Bhaddiya si recava nella foresta o
ai piedi di un albero o in una stanza vuota, esclamava in continuazione: «Oh
beatitudine! Oh beatitudine!».

Alcuni bhikkhu andarono dal Beato e glielo riferirono, aggiungendo: «Non pare ci
siano dubbi, Signore, che il venerabile Bhaddhiya sia insoddisfatto della santa
vita. Forse sta ricordando la sua precedente condizione di governante».

Allora il Beato lo mandò a chiamare e gli chiese se fosse vero.

«È così, Signore».

«Bhaddiya, che cosa ci trovi di buono, però, nel farlo?». «Prima, Signore,
quando la mia condizione era quella di un sovrano, c’erano guardie ben appostate
sia all’interno sia all’esterno del palazzo, sia all’interno sia all’esterno
della città e sia all’interno sia all’esterno del distretto. Sebbene io fossi
così custodito e protetto, avevo paura, ero ansioso, sospettoso e preoccupato.
Ora, però, Signore, quando vado nella foresta o ai piedi di un albero o in una
stanza vuota, non ho più paura, non sono ansioso o sospettoso o preoccupato.
Vivo a mio agio, in tranquillità, dipendo dai doni altrui, con una mente simile
a quella di un cervo selvatico. Questo ci trovo di buono nel farlo».

Conoscendo il significato di ciò, il Beato esclamò queste parole:

\begin{quote}
Colui che dentro di sé non ha più conflitti in agguato \\
ha superato ogni genere di esistenza, \\
perché egli è senza paura, beato, libero dal dolore. \\
Nessuna divinità può gareggiare con la sua gloria.
\end{quote}

\suttaRef{Vin. Cv. 7:1; cf. Ud. 2:10}

\voice{Prima voce.} Il venerabile Nanda, il fratellastro del Beato, indossò
degli abiti variopinti e ben stirati, si truccò gli occhi e prese una ciotola
lucente. Poi andò dal Beato e, dopo avergli prestato omaggio, si mise a sedere
da un lato. Quando lo ebbe fatto, il Beato gli disse: «Nanda, non è opportuno
che tu, un uomo di rango che ha lasciato la sua casa e la vita famigliare per la
vita religiosa, abbia indossato degli abiti variopinti e ben stirati, ti sia
truccato gli occhi e abbia preso una ciotola lucente. Quel che è opportuno per
te, un uomo di rango che ha lasciato la sua casa e la vita famigliare per la
vita religiosa, è dimorare nella foresta, mangiare solo cibo ottenuto in
elemosina, indossare vesti cucite con panni scartati, e dimorare senza alcun
interesse per i desideri sensoriali».

\suttaRef{S. 21:8}

\narrator{Secondo narratore.} Nel frattempo il novizio Rāhula, che ora aveva
dieci anni, viveva sotto le cure dell’Anziano Sāriputta ad Ambalaṭṭhikā, nei
pressi di Rājagaha, dove il Buddha tornò a tempo debito.

\voice{Prima voce.} Così ho udito. Una volta il Beato soggiornava a Rājagaha,
nel Boschetto di Bambù, nel Sacrario degli Scoiattoli e il venerabile Rāhula
viveva ad Ambalaṭṭhikā. Il venerabile Rāhula lo vide arrivare, preparò per lui
un posto a sedere e dell’acqua per lavarsi i piedi. Il Beato si mise a sedere
nel posto preparatogli e si lavò i piedi. Poi il venerabile Rāhula gli prestò
omaggio e si mise a sedere da un lato. Il Beato versò una piccola quantità
d’acqua nel mestolo e rivolse al venerabile Rāhula queste parole: «Rāhula, vedi
questo po’ d’acqua nel mestolo?».

«Sì, Signore».

«Se le persone non fanno attenzione a evitare di mentire intenzionalmente,
altrettanto poco di buono vi è in loro».

Allora il Beato gettò via quella piccola quantità d’acqua, e chiese: «Rāhula,
vedi quel po’ d’acqua che ho gettato via?».

«Sì, Signore».

«Se le persone non fanno attenzione a evitare di mentire intenzionalmente, quel
che di buono che c’è in loro è gettato via in questo modo».

Allora il Beato capovolse il mestolo e chiese: «Rāhula, vedi questo mestolo
capovolto?».

«Sì, Signore».

«Se le persone non fanno attenzione a evitare di mentire intenzionalmente, quel
che di buono v’è in loro è trattato in questo modo».

Poi il Beato rimise il mestolo dritto e chiese: «Rāhula, vedi questo mestolo
completamente vuoto?».

«Sì, Signore».

«Se le persone non fanno attenzione a evitare di mentire intenzionalmente, loro
sono allo stesso modo vuoti di bene. Ora, Rāhula, supponiamo che ci sia un
elefante reale con le zanne lunghe come le aste di un carro, del tutto cresciuto
in statura, molto addestrato e ben abituato a combattere, e che in battaglia usi
le sue zampe anteriori e le sue zampe posteriori, la parte anteriore del suo
corpo e la parte posteriore del suo corpo, la sua testa e i suoi orecchi e le
sue zanne, e tuttavia tenga indietro la proboscide. L’uomo che sta dietro di lui
penserebbe: “Benché faccia uso di tutte le sue membra, tiene indietro la sua
proboscide, e perciò non ha ancora offerto la sua vita al re”. Però, se
l’elefante usasse tutte le sue membra e anche la sua proboscide, l’uomo che sta
dietro di lui penserebbe: “Usa tutte le sue membra e anche la sua proboscide, e
perciò ha offerto la sua vita al re, non ha più bisogno di essere addestrato”.
Allo stesso modo, Rāhula, se le persone non fanno attenzione a evitare di
mentire intenzionalmente, di loro non dico che non hanno più bisogno di essere
addestrate. Perciò, Rāhula, devi addestrarti a non affermare mai il falso,
neanche per scherzo. A che cosa pensi che serva uno specchio, Rāhula?».

«Per vedere se stessi, Signore».

«Proprio nello stesso modo devi continuare a osservare le tue azioni, le tue
parole e i tuoi pensieri».

\suttaRef{M. 61}

\narrator{Secondo narratore.} Il Buddha continuò impartendogli istruzioni
dettagliate su come esaminare ogni azione prima, durante e dopo che sia stata
compiuta, giudicandola non salutare qualora essa sia a danno proprio o degli
altri e di entrambi, oppure giudicandola salutare se non lo è, modellando di
conseguenza le azioni future.

