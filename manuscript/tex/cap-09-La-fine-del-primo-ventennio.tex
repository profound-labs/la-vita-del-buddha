\chapter{La fine del primo ventennio}

\narrator{Secondo narratore.} La decima stagione delle piogge successiva
all’Illuminazione fu trascorsa a Pārileyyaka, mentre la disputa di Kosambī era
al culmine. La stessa tradizione riporta che l’undicesima stagione delle piogge
fu trascorsa nelle colline del sud – le colline a sud di Rājagaha – e che allora
accadde il seguente evento.

\voice{Prima voce.} Così ho udito. Una volta il Beato soggiornava nel regno di
Magadha, nel villaggio di Ekanālā. Era il tempo della semina e Kasi
(l’“aratore”) Bhāradvāja della casta dei brāhmaṇa aveva ben cinquecento aratri
al lavoro. Di mattino presto il Beato si vestì, prese la ciotola e la veste
superiore, e andò dove si stava svolgendo il lavoro di Kasi Bhāradvāja. Allora
il brāhmaṇa stava distribuendo il cibo. Il Beato andò dove era distribuito il
cibo e si mise in piedi da un lato. Il brāhmaṇa, vedendolo attendere per la
questua, disse:

«Io aro, monaco, e semino, e siccome ho arato e seminato, mangio. Anche tu,
monaco, dovresti arare e seminare e, dopo aver arato e seminato, mangerai».

«Anche io, brāhmaṇa, aro e semino e, dopo aver arato e seminato, mangio».

«Non vediamo alcun giogo, né aratro, né ganascia, né pungolo e neanche i buoi
del Maestro Gotama. Tuttavia il Maestro Gotama ha detto: “Anche io, brāhmaṇa,
aro e semino e, dopo aver arato e seminato, mangio”». Egli allora si rivolse al
Beato in strofe:

\begin{quote}
Tu pretendi di essere un aratore, tuttavia \\
noi non vediamo la tua aratura. \\
Rispondi perciò, signore, così che \\
si possa vedere la tua aratura.

Il seme è la fede, la mia pioggia è l’autocontrollo, \\
il mio aratro e il mio giogo sono la comprensione, \\
il mio palo è la coscienza, la mente è la mia cinghia, \\
e la consapevolezza è la mia ganascia e il mio pungolo. \\
Controllato nel corpo e nella parola \\
e modesto nell’assunzione del cibo, \\
la Verità è il mio raccolto, \\
la rinuncia è la mia libertà dal giogo. \\
Il mio bue imbrigliato è l’energia \\
che conduce alla cessazione dei legami \\
per andare dove non c’è alcun dolore, \\
senza che si torni di nuovo indietro. \\
Questa è la mia aratura, \\
Ciò Che Non Muore è il suo frutto. \\
Chi ara in questo modo sarà libero \\
Da ogni genere di sofferenza.
\end{quote}

Allora Kasi Bhāradvāja riempì una grande ciotola di bronzo con riso cotto nel
latte e la portò dal Beato: «Che il Maestro Gotama mangi il riso cotto nel
latte. Il Maestro Gotama è un aratore perché la sua aratura ha come frutto Ciò
Che Non Muore».

\begin{quote}
Non posso ottenere ricompense per i miei versi, \\
questa è la legge dei profeti. Gli Esseri Illuminati \\
non accettano ricompense per i loro versi, \\
questo è il loro costume finché prevale la loro legge. \\
Quando un profeta è libero dalle contaminazioni, \\
placati i conflitti – come voi direste: \\
“ha raggiunto l’Assoluto”\footnote{Il termine \emph{kevalī} (“ha raggiunto l’Assoluto”) pare fosse usato dal Buddha quando si rivolgeva ai brāhmaṇa.} – \\
allora offrite a lui doni con altri pensieri nella mente: \\
egli è il campo per coloro che mieteranno meriti.
\end{quote}

«A chi darò questo riso cotto nel latte, Maestro Gotama?».

«Brāhmaṇa, in questo mondo con i suoi deva, con i suoi Māra e con le sue
divinità, in questa generazione con i suoi monaci e brāhmaṇa, con i suoi
principi e uomini, non vedo nessuno che possa ben digerire questo riso cotto nel
latte dopo averlo mangiato, salvo che non sia un Essere Perfetto o un suo
discepolo. Perciò, brāhmaṇa, getta via quel riso cotto nel latte ove non c’è
erba o versalo in acqua ove non ci sia vita».

Kasi Bhāradvāja il brāhmaṇa versò il riso cotto nel latte in acqua ove non c’era
vita. Appena lo ebbe versato nell’acqua, esso sibilò e bollì ed emise fumo e
vapore. Proprio come una ganascia per arare riscaldata per un giorno intero e
gettata nell’acqua avrebbe sibilato e bollito ed emesso fumo e vapore, così
avvenne al riso cotto nel latte.

Allora il brāhmaṇa fu colto dallo sgomento e i suoi capelli si rizzarono. Andò
dal Maestro Gotama, si prostrò ai suoi piedi e disse: «Magnifico, Maestro
Gotama! Desidero abbracciare la vita religiosa e ricevere l’ammissione dal
Maestro Gotama». E non molto tempo dopo il venerabile Bhāradvāja divenne uno
degli Arahant.

\suttaRef{Sn. 1:4; S. 7:11}

\narrator{Secondo narratore.} Il novizio Rāhula, il figlio del Buddha, aveva ora
diciotto anni. Il Buddha dimorava nel Boschetto di Jeta e un mattino era andato
in città per la questua. Suo figlio lo seguiva da vicino e, mentre lo seguiva, i
suoi pensieri vagarono e cominciò a chiedersi quale avrebbe potuto essere il suo
futuro se suo padre fosse diventato un Monarca Universale, com’era stato
predetto, e non avesse rinunciato alla vita famigliare.

\voice{Prima voce.} Mentre il venerabile Rāhula lo stava seguendo da vicino, il
Beato si girò e lo guardò, e si rivolse a lui in questo modo: «Rāhula, qualsiasi
forma materiale, sia essa passata, futura o presente, interna o esterna,
grossolana o sottile, inferiore o superiore, vicina o lontana, dovrebbe essere
considerata per quello che realmente è, con retta comprensione, in questo modo:
“Questo non è mio, questo non è ciò che io sono, questo non è il mio sé”».

«Solo la forma materiale, Beato? Solo la forma materiale, Sublime?».

«La forma materiale, Rāhula, la sensazione, la percezione, le formazioni mentali
e la coscienza».

Allora il venerabile Rāhula pensò: «Chi mai può entrare in città per la questua
dopo essere stato pubblicamente ammonito dal Beato?». Ed egli tornò indietro e
si mise a sedere sotto un albero a gambe incrociate, con il corpo eretto e con
la consapevolezza fissa davanti a lui. Il venerabile Sāriputta lo vide mentre
stava così e gli disse: «Rāhula, mantieni in essere la consapevolezza del
respiro. Se la mantieni in essere e la sviluppi bene, essa reca grande frutto e
molte benedizioni».

Quando fu sera, il venerabile Rāhula si alzò dal ritiro e andò dal Beato. Dopo
avergli prestato omaggio, si mise a sedere da un lato. Allora egli disse:
«Signore, come dovrebbe essere mantenuta in essere e ben sviluppata la
consapevolezza del respiro affinché essa possa recare grande frutto e molte
benedizioni?».

\suttaRef{M. 62}

\narrator{Secondo narratore.} Il Buddha gli descrisse prima in dettaglio i
quattro elementi primari della forma materiale – terra o solidità, acqua o
coesione, fuoco o temperatura e maturazione, e aria o dilatazione e movimento –
e dello spazio, e come ognuno di essi dovrebbe essere considerato allo stesso
modo della forma materiale. Allora egli disse:

\voice{Prima voce.} «Cerca di essere come la terra, Rāhula. Così facendo, quando
sorgono i contatti piacevoli o spiacevoli, essi non invaderanno il tuo cuore e
non resteranno lì, proprio come quando la gente butta cose pulite o cose sporche
o escrementi o urina o sputo o pus o sangue sulla terra, e la terra non prova
vergogna, umiliazione o disgusto. Cerca di essere come l’acqua, Rāhula. Quando
la gente lava via queste cose con l’acqua, l’acqua non prova vergogna,
umiliazione o disgusto. Cerca di essere come il fuoco Rāhula. Quando il fuoco
brucia queste cose, il fuoco non prova vergogna, umiliazione o disgusto. Cerca
di essere come l’aria, Rāhula. Così facendo, quando sorgono i contatti piacevoli
o spiacevoli, essi non invaderanno il tuo cuore e non resteranno lì, proprio
come quando il vento porta via cose pulite o sporche, o escrementi o urina o
sputo o pus o sangue sulla terra, e l’aria non prova vergogna, umiliazione o
disgusto. Cerca di essere come lo spazio, Rāhula. Così facendo, quando sorgono i
contatti piacevoli o spiacevoli, essi non invaderanno il tuo cuore e non
resteranno lì, perché lo spazio non ha luogo in cui restare».

«Pratica la gentilezza amorevole per vincere la malevolenza. Pratica la
compassione per vincere la crudeltà. Pratica la gioia empatica per vincere
l’apatia. Pratica l’equanimità per vincere il risentimento. Pratica la
contemplazione della repulsività del corpo per vincere la lussuria. Pratica la
contemplazione dell’impermanenza per vincere la presunzione dell’“io sono”.
Pratica la consapevolezza del respiro, perché quando essa è mantenuta in essere
e ben sviluppata, reca gran frutto e molte benedizioni».

\suttaRef{M. 62}

\narrator{Secondo narratore.} Il Buddha descrisse allora i sedici modi in cui
può essere praticata la consapevolezza del respiro.

\narrator{Primo narratore.} Il raggiungimento della condizione di Arahant da
parte del novizio Rāhula è narrato in seguito.

\narrator{Secondo narratore.} Nella successiva stagione delle piogge, la
dodicesima, il Buddha si trovò a Verañjā.

\label{pag137}%
\voice{Seconda voce.} Avvenne questo. Il Buddha, il Beato, soggiornava a
Verañjā, ai piedi dell’albero \emph{nimba} di Naḷeru con una grande comunità di
bhikkhu, di cinquecento bhikkhu, quando un brāhmaṇa di Verañjā sentì parlare del
Beato e decise di andare ad incontrarlo. Andò da lui e scambiò con lui dei
saluti e, quando furono terminati i formali doveri di reciproca cortesia, egli
si mise a sedere da un lato. Disse: «Maestro Gotama, ho sentito che il monaco
Gotama non presta omaggio ai brāhmaṇa che sono vecchi, anziani, gravati dagli
anni, avanti nella vita e giunti allo stadio finale, che egli non si alza in
piedi per loro né li invita a sedersi. E, inoltre, vedo che, infatti, così è,
perché il Maestro Gotama in realtà non fa queste cose. Questo non va bene,
Maestro Gotama».

«Brāhmaṇa, in questo mondo con i suoi deva, con i suoi Māra e con le sue
divinità, in questa generazione con i suoi monaci e brāhmaṇa, con i suoi
principi e uomini, non vedo nessuno al quale io possa prestare omaggio o per il
quale alzarmi in piedi o invitarlo a sedersi, perché se un Essere Perfetto
prestasse omaggio o si alzasse per qualcuno o lo invitasse a sedersi, a costui
gli si spaccherebbe la testa».

«Il Maestro Gotama è privo di gusto».

«C’è una ragione per la quale si potrebbe giustamente dire che il monaco Gotama
è privo di gusto: gusto per le forme visibili, gusto per i suoni, odori, sapori
e oggetti tangibili. Queste cose sono rigettate da un Essere Perfetto, tagliate
alla radice, rese come ceppi di palma, eliminate e non più soggette a sorgere in
futuro. È però sicuro, brāhmaṇa, che tu intenda questo?».

«Il Maestro Gotama non ha il senso dei valori».

«C’è una ragione per la quale si potrebbe giustamente dire che il monaco Gotama
non ha il senso dei valori: il senso del valore delle forme visibili, il senso
del valore dei suoni, degli odori, dei sapori e degli oggetti tangibili. Queste
cose sono rigettate da un Essere Perfetto … e non più soggette a sorgere in
futuro. È però sicuro, brāhmaṇa, che tu intenda questo?».

«Il Maestro Gotama insegna che non si dovrebbe fare nulla».\pagenote{%
  Alcuni dei giochi di parole presenti in questo passo mettono a dura prova le
  qualità di un traduttore. «Insegna che non ci sono cose da fare»
  (\emph{akiriyavādī}) indica colui il quale afferma che le azioni sono amorali
  e non fanno maturare effetti, né buoni né cattivi. «Insegna il nichilismo»
  (\emph{ucchedavādī}) indica colui il quale crede che alcuni tipi di anima o di
  sé abbiano una permanenza temporanea, che a un certo punto viene però
  interrotta. Essa presuppone l’esistenza di un’anima temporanea. «Uno da
  portare via» (\emph{venayika}) è l’espressione più difficile. La parola
  \emph{vineti} (letteralmente “portare via”) significa sia portare via sia,
  metaforicamente, disciplinare. “Portare via” è pure utilizzato dal Buddha nel
  senso di condurre i discepoli lontano dalla sofferenza e, dai suoi oppositori,
  per insultarlo come uno che porta la gente fino alla distruzione, procurata
  dal nichilismo, l’“abisso del nulla”, e, di conseguenza, per loro egli è uno
  “da portare via”, ossia di cui sbarazzarsi.}

«C’è una ragione per la quale si potrebbe giustamente dire che il monaco Gotama
insegna che non si dovrebbe fare nulla: io insegno che non si dovrebbero
compiere atti corporei o verbali errati o alimentare pensieri malsani e molti
altri generi di cose malvagie e non salutari. È però sicuro, brāhmaṇa, che tu
intenda questo?».

«Il Maestro Gotama insegna il nichilismo».

«C’è una ragione per la quale si potrebbe giustamente dire che il monaco Gotama
insegna il nichilismo: io insegno l’annichilimento della brama, dell’odio e
dell’illusione, e di molti generi di cose malvagie e non salutari. È però
sicuro, brāhmaṇa, che tu intenda questo?».

«Il Maestro Gotama è fastidioso».

«C’è una ragione per la quale si potrebbe giustamente dire che il monaco Gotama
è fastidioso: io sono fastidioso in relazione ad atti corporei o verbali errati
o pensieri malsani e molti altri generi di cose non salutari. È però sicuro,
brāhmaṇa, che tu intenda questo?».

«Il monaco Gotama è uno da portare via».

«C’è una ragione per la quale si potrebbe giustamente dire che il monaco Gotama
è uno da portare via: io insegno il Dhamma che porta via dalla brama, dall’odio
e dall’illusione, e da molti generi di cose malvagie e non salutari. È però
sicuro, brāhmaṇa, che tu intenda questo?».

«Il monaco Gotama è un mortificatore».

«C’è una ragione per la quale si potrebbe giustamente dire che il monaco Gotama
è un mortificatore: dico che gli atti corporei o verbali errati o pensieri
malsani sono cose malvagie e non salutari da mortificare, e chiamo mortificatore
colui nel quale le cose malvagie e non salutari da mortificare sono rifiutate,
tagliate alla radice, rese come ceppi di palma, eliminate e non più soggette a
sorgere nel futuro, e in un Essere Perfetto queste cose sono rifiutate … e non
più soggette a sorgere nel futuro. È però sicuro, brāhmaṇa, che tu intenda
questo?».

«Il monaco Gotama ha mancato la sua rinascita».

«C’è una ragione per la quale si potrebbe giustamente dire che il monaco Gotama
ha mancato la sua rinascita. Quando il rientro di una persona in un utero e il
suo pervenire alla nascita sono rifiutati … e non sono più soggetti a sorgere
nel futuro, allora di tale persona dico che ha mancato la sua rinascita, e
nell’Essere Perfetto il rientro in un utero e una futura rinascita sono
rifiutati … e non sono più soggetti a sorgere nel futuro. È però sicuro,
brāhmaṇa, che tu intenda questo?».

«Supponiamo che una chioccia stia covando otto, dieci o dodici uova, che le covi
e le faccia schiudere con cura: il primo di quei pulcini a forare il guscio con
la punta del suo becco e gli artigli delle sue zampe, il primo a uscir fuori
sano, dovrebbe essere chiamato il più anziano o il più giovane?».

«Dovrebbe essere chiamato il più anziano, Maestro Gotama, perché è il più
anziano di quei pulcini».

«Allo stesso modo, brāhmaṇa, in questa generazione dominata dall’ignoranza,
racchiusa in un uovo d’ignoranza, sigillata dall’ignoranza, sono io l’unico al
mondo ad aver scoperto la suprema e piena Illuminazione forando il guscio
dell’ignoranza, della nescienza. Sono perciò io il più anziano ed eminente nel
mondo».

\suttaRef{Vin. Sv. Pārā. 1; A. 8:11}

\narrator{Secondo narratore.} Il Buddha poi descrisse come, mediante
l’ottenimento dei quattro jhāna e delle tre vere conoscenze, pervenne a
conoscere direttamente che non vi era più nascita per lui. Il brāhmaṇa si
convinse e prese i Tre Rifugi. Egli allora offrì ricovero e sostegno al Buddha
per la successiva stagione delle piogge, e il Buddha accettò.

\voice{Seconda voce.} A Verañjā ottenere cibo in elemosina era difficile. C’era
carestia ed erano stati emessi dei buoni per ottenere il cibo. Non era facile
sopravvivere neanche spigolando strenuamente. Tuttavia, alcuni commercianti del
nord del paese con cinquecento cavalli avevano allora preso alloggio per la
stagione delle piogge a Verañjā. Avevano fatto sapere che per ogni bhikkhu ci
sarebbe stata una misura di crusca presso i recinti dei cavalli.

Un mattino i bhikkhu si vestirono, presero le loro ciotole e la veste superiore,
e si avviarono per la questua a Verañjā. Quando non ottennero alcun cibo, si
recarono presso i recinti dei cavalli e ognuno di loro portò una misura di
crusca in monastero, ove la pestarono in un mortaio e la mangiarono. Il
venerabile Ānanda macinò una misura di crusca su una pietra e la portò al Beato.
Il Beato la mangiò.

Egli aveva sentito il rumore di un mortaio. Gli Esseri Perfetti sanno e
chiedono, ma, anche, sanno e non chiedono. Chiedono quando lo reputano opportuno
e si astengono dal chiedere quando lo reputano inopportuno. Gli Esseri Perfetti
chiedono al fine di promuovere il bene, per nessun’altra ragione. Nel caso degli
Esseri Perfetti il ponte verso il male è demolito. Gli Esseri Illuminati, gli
Esseri Perfetti, interrogano i bhikkhu per due ragioni: per insegnare il Dhamma
o per rendere noto un precetto d’addestramento ai discepoli. Per quell’occasione
il Beato chiese al venerabile Ānanda: «Ānanda, che cos’è quel rumore di
mortaio?». Il venerabile Ānanda glielo spiegò.

«Bene, bene, Ānanda. Ci siete riusciti, come brave persone. Nelle future
generazioni, però, ci saranno alcuni che guarderanno dall’alto in basso perfino
pasti di riso fino cotto con la carne».

Il venerabile Mahā-Moggallāna andò dal Beato. Egli disse: «Signore, è ora
difficile procurarsi cibo in elemosina a Verañjā. C’è carestia e sono stati
emessi dei buoni per ottenere il cibo. Non è facile sopravvivere neanche
spigolando strenuamente. Signore, sotto la superficie di questa terra vi è un
humus ricco e dolce come il miele. Sarebbe bene che io rivoltassi la terra. Così
i bhikkhu sarebbero in grado di cibarsi dell’humus sul quale vivono le piante
acquatiche».

«Moggallāna, che cosa ne sarebbe, però, delle creature che dipendono
dall’humus?».

«Signore, renderò una mia mano larga come la grande terra e prenderò le creature
che dipendono dall’humus e le metterò lì. Rivolterò la terra con l’altra mano».

«Basta così, Moggallāna, non suggerire di rivoltare la terra. Le creature
saranno confuse».

«Signore, sarebbe bene che il Saṅgha dei bhikkhu andasse nel Continente
Settentrionale di Uttarakuru per la questua».

«Basta così, Moggallāna, non suggerire che il Saṅgha dei bhikkhu vada nel
Continente Settentrionale di Uttarakuru per la questua».

Mentre il venerabile Sāriputta era in ritiro da solo sorse in lui questo
pensiero: «La santa vita di quale Buddha non durò a lungo? La santa vita di
quale Buddha durò a lungo?».

«Al tempo dei Beati Vipassī, Sikhī e Vessabhū la santa vita non durò a lungo,
Sāriputta. Al tempo dei Beati Kakusandha, Koṇāgamana e Kassapa la santa vita
durò a lungo».

«Signore, per quale ragione al tempo dei Beati Vipassī, Sikhī e Vessabhū la
santa vita non durò a lungo?».

«Quei Beati non furono solleciti a insegnare il Dhamma ai loro discepoli
dettagliatamente e pronunciarono pochi Fili di Discorsi (sutta),\footnote{C’è un
  gioco di parole sul termine sutta, letteralmente “filo” e metaforicamente
  “filo di discorsi” o insieme di idee connesse. È in quest’ultimo senso che i
  discorsi del Buddha sono chiamati “sutta”, perché in essi l’insegnamento è
  tenuto assieme nella forma di un filo di argomenti legati l’uno con l’altro.}
Canti, Esposizioni, Strofe, Esclamazioni, Detti, Storie di Nascite, Meraviglie e
Domande. Non fu resa nota alcuna regola di addestramento per i discepoli. Il
\emph{Pātimokkha}, il Codice Monastico, non fu esposto. Proprio come quando vari
fiori sono posti su un tavolo senza essere tenuti assieme da fili possono venire
facilmente sparpagliati, spazzati via e andare perduti – perché? Perché non sono
tenuti assieme da fili – allo stesso modo, quando quei Buddha, quei Beati e i
loro discepoli da loro personalmente illuminati scomparvero, allora i discepoli
che in seguito abbracciarono la vita religiosa, chiamati in vari modi,
appartenenti a varie razze e varie stirpi, fecero estinguere la vita religiosa.
Quei Beati leggevano di norma la mente dei loro discepoli e li consigliavano di
conseguenza. Una volta, il Beato Vessabhū, realizzato e completamente
illuminato, in una boscaglia d’una giungla che ispirava timore lesse la mente di
un Saṅgha forte di un migliaio di bhikkhu, e così li esortò e istruì: “Pensate
così, non pensate così. Prestate attenzione così, non prestate attenzione così.
Abbandonate questo, entrate e dimorate in questo”. Poi, seguendo le sue
istruzioni, i loro cuori furono liberati dalle contaminazioni per mezzo del
non-attaccamento. E la boscaglia di quella giungla ispirava a tal punto timore
che di solito avrebbe fatto rizzare i capelli a un uomo se egli non fosse stato
libero dalla brama. Questa fu la ragione per cui la vita santa di quei beati non
durò a lungo».

«Signore, per quale ragione al tempo dei Beati Kakusandha, Koṇāgamana e Kassapa
la santa vita durò a lungo?».

«Quei Beati furono solleciti a insegnare il Dhamma ai loro discepoli
dettagliatamente e pronunciarono molti Fili di Discorsi, Canti, Esposizioni,
Strofe, Esclamazioni, Detti, Storie di Nascite, Meraviglie e Domande. Furono
rese note regole di addestramento per i discepoli. Il Pātimokkha, il codice
monastico, fu esposto. Proprio come quando vari fiori sono posti su un tavolo
tenuti ben legati assieme da fili, e non possono venire sparpagliati, spazzati
via e andare perduti – perché? Perché sono tenuti ben legati assieme da fili –
allo stesso modo, quando quei Buddha, quei Beati e i loro discepoli da loro
personalmente illuminati scomparvero, allora i discepoli che in seguito
abbracciarono la vita religiosa, chiamati in vari modi, appartenenti a varie
razze e varie stirpi, fecero continuare la vita religiosa per lungo tempo.
Questa fu la ragione per cui la vita santa di quei beati durò a lungo».

Allora il venerabile Sāriputta si alzò dal posto in cui sedeva e, sistemandosi
la sua veste su una spalla, levò le palme giunte delle sue mani verso il Beato e
disse: «Questo è il tempo, Beato, questo è il tempo che il Beato renda note le
regole di addestramento, che esponga il \emph{Pātimokkha}, in modo che la santa
vita possa durare a lungo».

«Aspetta Sāriputta, aspetta! L’Essere Perfetto saprà quando è il momento di
farlo. Il Maestro non renderà note le regole di addestramento per i discepoli né
esporrà il \emph{Pātimokkha} fino a quando non si manifesteranno alcune cose che
generano contaminazioni qui nel Saṅgha. Appena questo avverrà, allora il Maestro
si occuperà di entrambe queste cose, al fine di allontanare queste cose che
generano contaminazioni. Alcune cose che generano contaminazioni non si
manifesteranno finché il Saṅgha non si sarà ingrandito in quanto fondato da
tempo, e sarà cresciuto [quanto al numero dei bhikkhu]: sarà allora che esse si
manifesteranno e sarà allora che il Maestro renderà note le regole di
addestramento per i discepoli, e esporrà il \emph{Pātimokkha} al fine di
allontanare queste cose che generano contaminazioni. Alcune cose che generano
contaminazioni non si manifesteranno finché il Saṅgha non si sarà ingrandito
mediante completezza … non si sarà ingrandito mediante beni eccessivi … non si
sarà ingrandito mediante erudizione … Al momento, però, il Saṅgha è libero da
infezioni, libero da pericoli, è immacolato, puro ed è fatto di durame. Perché
di questi cinquecento bhikkhu chi si trova più indietro è nella condizione di
Chi è Entrato nella Corrente, non è più soggetto alla perdizione, certo nella
rettitudine e destinato all’Illuminazione».

Allora il Beato si rivolse al venerabile Ānanda: «Ānanda, è costume degli Esseri
Perfetti di non avviarsi a errare per il paese senza essersi congedati da coloro
che li hanno invitati per la stagione delle piogge. Andiamo e congediamoci dal
brāhmaṇa di Verañjā».

«E sia, Signore», rispose il venerabile Ānanda.

Allora il Beato si vestì, prese la ciotola e la veste superiore, e andò con il
venerabile Ānanda quale suo attendente nella casa del brāhmaṇa di Verañjā, ove
si mise a sedere nel posto preparatogli.

Il brāhmaṇa arrivò e gli prestò omaggio. Il Beato disse: «Abbiamo trascorso la
stagione delle piogge qui, invitati da te, brāhmaṇa, e ora ci congediamo.
Desideriamo avviarci a errare per il paese».

«È vero, Maestro Gotama. Siete stati invitati da me a trascorrere qui la
stagione delle piogge. Quel che avrebbe dovuto essere dato non è stato dato.
Ciò, però, non è avvenuto perché non avevamo capito o perché non fossimo
disposti a dare. Come potevamo fare? La vita laica è piena di impegni, molte
sono le cose da fare. Che il Maestro Gotama assieme al Saṅgha dei bhikkhu
accetti il pasto di domani da me».

Il Beato accettò in silenzio. Poi, dopo aver istruito il brāhmaṇa con un
discorso di Dhamma, si alzò e andò via.

Il giorno seguente, quando il pasto fu terminato, il brāhmaṇa di Verañjā offrì
al Beato la stoffa per una veste e a ogni bhikkhu due pezzi di stoffa. E il
Beato, dopo averlo istruito con un discorso di Dhamma, se ne andò.

\suttaRef{Vin. Sv. Pārā. 1}

\narrator{Secondo narratore.} Il seguente episodio si verificò mentre la
tredicesima stagione delle piogge veniva trascorsa a Cālikā.

\voice{Prima voce.} Così ho udito. Mentre il Beato soggiornava a Cālikā, sulla
Rupe Cālikā, il suo attendente era allora il venerabile Meghiya. Egli andò dal
Beato e gli disse: «Signore, voglio entrare a Jantugāma per la questua».

«È tempo, Meghiya, di fare quel che reputi opportuno».

Allora era mattino e così il venerabile Meghiya si vestì, prese la ciotola e la
veste superiore ed entrò a Jantugāma per la questua. Allorché ebbe fatto il giro
per la questua e stava tornando dopo il pasto, giunse sulla riva del fiume
Kimikālā. Mentre stava camminando ed errando lungo la riva del fiume per
muoversi un po’, vide un grazioso e invitante boschetto di alberi di mango.
Pensò: «Questo grazioso e invitante boschetto di alberi di mango sarà utile per
lo sforzo di un uomo di rango che cerca un tale sforzo. Se il Beato lo consente,
verrò in questo boschetto di alberi di mango per lo sforzo».

Egli allora si recò dal Beato e gliene parlò. Il Beato disse: «Aspetta, Meghiya,
siamo ancora soli. Aspetta che arrivino altri bhikkhu».

Una seconda volta il venerabile Meghiya disse: «Il Beato non ha molto altro da
fare, Signore. Non v’è bisogno di confermare ciò che egli ha già fatto. Noi,
però, abbiamo ancora qualcosa da fare. Abbiamo bisogno di confermare ciò che
abbiamo già fatto. Se il Beato lo consente, Signore, vorrei andare in quel
boschetto di alberi di mango per lo sforzo».

Una seconda volta il Beato disse: «Aspetta, Meghiya, siamo ancora soli. Aspetta
che arrivino altri bhikkhu».

Una terza volta il venerabile Meghiya ripeté la sua richiesta.

«Dal momento che tu parli di “sforzo”, Meghiya, che cosa posso dirti? È tempo
che tu faccia quel che reputi opportuno».

Allora il venerabile Meghiya si alzò dal posto in cui sedeva e, dopo aver
prestato omaggio al Beato, girandogli a destra, si avviò verso il boschetto di
alberi di mango, ove si mise a sedere ai piedi di un albero, sua dimora diurna.
Allora, per quasi tutto il tempo che egli rimase nel boschetto di alberi di
mango, tre generi di pensieri non salutari occuparono la sua mente, ossia
pensieri di desideri sensoriali, pensieri di malevolenza e pensieri di crudeltà.
Gli capitò così di pensare: «È meraviglioso, è stupefacente! Eccomi qui, ho
abbandonato la vita famigliare per fede e ora sono tormentato da questi tre
generi di pensieri malvagi e non salutari».

Quando fu sera, si alzò dal ritiro e andò dal Beato. Gli disse quel che era
avvenuto.

«Meghiya, quando la liberazione del cuore è ancora immatura, cinque cose la
conducono a maturazione. Quali cinque? Primo, un bhikkhu con buoni amici e buoni
compagni. Secondo, un bhikkhu è perfetto nella virtù, contenuto con il
contenimento del \emph{Pātimokkha}, perfetto per condotta e per modo di vivere,
vede il pericolo nella più piccola colpa, si addestra portando a effetto i
precetti dell’addestramento. Terzo, ascolta volentieri senza problemi o riserve
discorsi che riguardano l’annientamento, che favoriscono la liberazione del
cuore, che conducono al totale disincanto, allo svanire, al cessare, alla
pacificazione, alla conoscenza diretta, all’Illuminazione, al Nibbāna, ossia a
volere poco, ad accontentarsi, all’isolamento, al dissociarsi dalla società,
all’energia, alla virtù, alla concentrazione, alla comprensione, alla
liberazione, alla conoscenza e alla visione della liberazione. Quarto, un
bhikkhu è energico nell’abbandonare cose non salutari e a portare a effetto le
cose salutari, è risoluto, costante e instancabile riguardo alle cose salutari.
Quinto, un bhikkhu ha comprensione, ha la penetrante comprensione propria degli
Esseri Nobili a riguardo del sorgere e dello svanire che conduce alla cessazione
completa della sofferenza».

«Ora, quando un bhikkhu ha buoni amici e buoni compagni, da lui ci si può
attendere che sarà virtuoso … che ascolterà volentieri … discorsi che riguardano
l’annientamento … che sarà energico nell’abbandonare cose non salutari e a
portare a effetto le cose salutari … che egli avrà la penetrante comprensione
propria degli Esseri Nobili a riguardo del sorgere e dello svanire che conduce
alla completa cessazione della sofferenza».

«Per fondare dentro di sé queste cinque cose, però, un bhikkhu dovrebbe, per di
più, mantenere in essere queste quattro cose. La ripugnanza (in relazione
all’aspetto repellente del corpo)\footnote{%
  “Ripugnanza” è un termine che indica l’oggetto di contemplazione consistente
  sia nelle “trentuno parti del corpo” (trentadue nei Commentari) sia la
  decomposizione dei cadaveri (cap.~12, pag.~\pageref{pag270} -- \emph{Ancora,
    un bhikkhu considera questo corpo come se stesse guardando\ldots}). Lo scopo
  è ridurre l’attaccamento al corpo fisico dimostrando che è non attraente ma
  transitorio.}
dovrebbe essere mantenuta in essere al fine di abbandonare la lussuria. La
gentilezza amorevole al fine di abbandonare la malevolenza. La consapevolezza
del respiro al fine di interrompere i pensieri discorsivi. La percezione
dell’impermanenza al fine di eliminare la presunzione dell’“io sono”. Perché
quando si percepisce l’impermanenza, la percezione del non-sé si fonda, e quando
si percepisce il non-sé, si giunge all’eliminazione della presunzione dell’“io
sono” e questo è il Nibbāna qui e ora».

Conoscendo il significato di ciò, il Beato esclamò queste parole:

\begin{quote}
Pensieri meschini, pensieri triviali \\
arrivano a tentare la mente e poi volano via. \\
Non comprendendo questi pensieri nella mente, \\
il cuore vaga avanti e indietro rincorrendoli. \\
Un uomo che comprende questi pensieri nella sua mente \\
li espelle con consapevolezza vigorosa. \\
E un Essere Illuminato se n’è sbarazzato \\
perché le tentazioni non agitano più la sua mente.
\end{quote}

\suttaRef{Ud. 4:1; A. 9:3}

\narrator{Secondo narratore.} Il figlio del Buddha aveva ora vent’anni. Gli fu
di conseguenza impartita la piena ammissione (in quanto non conferibile prima di
tale età). E la tradizione riporta che fu in questo stesso anno che il Buddha
pronunciò il discorso che fu per lui la causa per ottenere la condizione di
Arahant.

\voice{Prima voce.} Così ho udito. Allora il Beato soggiornava a Sāvatthī, nel
Boschetto di Jeta, nel Parco di Anāthapiṇḍika. Ora, mentre egli era solo in
meditazione questo pensiero sorse nella sua mente: «Le cose che giungono a
maturazione nella Liberazione sono mature nella mente di Rāhula. E se io lo
conducessi al definitivo esaurimento delle contaminazioni?».

Quando fu mattino il Beato si vestì, prese la ciotola e la veste superiore, e si
recò a Sāvatthī per la questua. Quando ebbe fatto il giro per la questua a
Sāvatthī, tornò dopo il pasto e disse al venerabile Rāhula: «Rāhula, prendi con
te una stuoia su cui sedere e andiamo a trascorrere la giornata nel Boschetto
del Cieco».

«Così sia, Signore», rispose il venerabile Rāhula e, dopo aver preso con sé una
stuoia, seguì il Beato. In quella circostanza, però, anche molte migliaia di
divinità seguirono il Beato, pensando: «Oggi il Beato sta per condurre il
venerabile Rāhula al definitivo esaurimento delle contaminazioni».

Allora il Beato entrò nel Boschetto del Cieco e si mise a sedere ai piedi di un
albero. E il venerabile Rāhula prestò omaggio al Beato e si mise a sedere da un
lato. Dopo che lo ebbe fatto, il Beato disse:

(1a) «Cosa ne pensi, Rāhula, l’occhio è permanente o impermanente?».

«Impermanente, Signore».

«Quel che è impermanente è però spiacevole o piacevole?».

«Spiacevole, Signore».

«A riguardo di ciò che è impermanente, spiacevole e soggetto al cambiamento, è
giusto dire: “Questo è mio, questo è quel che io sono, questo è il mio sé?”».

«No, Signore».

(1b) «Cosa ne pensi, Rāhula, le forme visibili sono permanenti o
impermanenti?». …

(1c) «Cosa ne pensi, Rāhula, la coscienza visiva è permanente o
impermanente?». …

(1d) «Cosa ne pensi, Rāhula, il contatto visivo è permanente o
impermanente?». …

(1e) «Cosa ne pensi, Rāhula, è permanente o impermanente una sensazione,
una percezione, una formazione [mentale], una coscienza che sorge avendo
come condizione il contatto visivo?». …

\narrator{Secondo narratore.} Le cinque stesse proposizioni da (a) a (e) furono
ripetute per (2) orecchio e suoni, (3) naso e odori, (4) lingua e sapori, (5)
corpo e oggetti tangibili, (6) mente e oggetti mentali.

\voice{Prima voce.} «Con questa comprensione, Rāhula, il saggio nobile discepolo
diventa disincantato nei riguardi dell’occhio, delle forme visibili, della
coscienza visiva e del contatto visivo, ed egli diventa disincantato nei
riguardi della sensazione, della percezione, delle formazioni mentali e della
coscienza che sorge avendo come condizione il contatto visivo».

«Diventa disincantato nei riguardi dell’orecchio e dei suoni … nei riguardi del
naso e degli odori … nei riguardi della lingua e dei sapori … nei riguardi del
corpo e degli oggetti tangibili … nei riguardi della mente e degli oggetti
mentali …».

«Diventando disincantato, la sua brama svanisce. Con lo svanire della brama, il
suo cuore è liberato. Quando il suo cuore è liberato, giunge la conoscenza: “È
liberato”. Egli comprende: “La nascita è distrutta, la santa vita è stata
vissuta, quel che doveva essere fatto è stato fatto, non ci sarà altra
rinascita”».

Questo è ciò che il Beato disse. Il venerabile Rāhula si rallegrò per queste
parole. E, quando questo discorso fu terminato, il cuore del venerabile Rāhula
fu liberato dalle contaminazioni mediante il non-attaccamento. E in quelle molte
migliaia di divinità sorse la pura, immacolata visione del Dhamma: tutto quel
che sorge deve cessare.

\suttaRef{M. 147}

\narrator{Secondo narratore.} Le sei successive stagioni delle piogge – ossia
dalla quattordicesima alla diciannovesima – furono trascorse in luoghi
differenti. La ventesima a Sāvatthī, nel Boschetto di Jeta. Secondo la
tradizione dei Commentari, il Buddha decise allora di trascorrere regolarmente
ogni stagione delle piogge a Sāvatthī, e scelse in modo permanente come suo
attendente l’anziano Ānanda. Due eventi di rilievo narrati nei Piṭaka sono
collocati dalla tradizione in questo anno. Si tratta della conversione del
bandito Aṅgulimāla e di un tentativo di screditare il Buddha messo in atto da
alcuni suoi oppositori.

\voice{Prima voce.} Così ho udito. Una volta, quando il Beato soggiornava a
Sāvatthī, comparve un bandito nel regno del re Pasenadi di Kosala. Era chiamato
Aṅgulimāla, ossia “Collana di Dita”, ed era un assassino, un sanguinario, dedito
alle percosse e alla violenza, crudele con tutti gli esseri viventi. Devastava
villaggi, città e distretti. Continuava a uccidere le persone, e indossava una
collana fatta con le loro dita.

Un mattino il Beato prese la ciotola e la veste superiore, e andò a Sāvatthī per
la questua. Quando ebbe fatto il giro per la questua a Sāvatthī e fu ritornato
dopo il pasto, mise in ordine il posto nel quale riposava e, poi, portando con
sé la ciotola e la veste superiore, si incamminò verso il luogo in cui si
trovava Aṅgulimāla. Bovari, pastori, agricoltori e viaggiatori\pagenote{%
  La parola \emph{padhāvino} (viaggiatori) compare nella stessa frase in M. 50,
  ma è pronunciata \emph{pathāvino} (P.T.S. ed.). È stato seguito il Commentario
  a M. 50. Il Dizionario della P.T.S. offre entrambi i termini, ma con
  significati differenti, benché l’inclusione di \emph{padhāvin} sia un
  errore.}
videro il Beato e dissero: «Non incamminarti per quella strada, monaco. Su
quella strada c’è il bandito Aṅgulimāla. Uomini hanno percorso quella strada in
bande di dieci, venti, trenta e anche quaranta di tanto in tanto, ma sono tutti
caduti nelle mani di Aṅgulimāla».

Quando ciò fu detto, il Beato proseguì in silenzio. Una seconda volta avvenne la
stessa cosa, e il Beato proseguì in silenzio. Una terza volta avvenne la stessa
cosa, e il Beato proseguì in silenzio.

Vedendolo arrivare da lontano, il bandito Aṅgulimāla pensò: «È meraviglioso, è
davvero stupefacente! Uomini hanno percorso questa strada perfino in bande di
quaranta di tanto in tanto. E ora questo monaco arriva da solo, non
accompagnato. Si potrebbe pensare che era destino che venisse. Perché non dovrei
prendere la vita di questo monaco?».

Prese spada e scudo, allacciò l’arco e la faretra, e andò alla ricerca del
Beato. Allora il Beato compì un atto miracoloso, così che Aṅgulimāla, per quanto
corresse, non fu in grado di raggiungere il Beato che, invece, camminava a passo
normale. Allora Aṅgulimāla pensò: «È meraviglioso, è stupefacente! Ero solito
raggiungere e catturare un elefante al galoppo, allo stesso modo di un cavallo
al galoppo, di un carro al galoppo o di un daino al galoppo. Per quanto stia
correndo più velocemente che posso, però, non riesco a raggiungere questo monaco
che sta camminando a passo normale».

Si fermò e gridò: «Fermati, monaco! Fermati, monaco!».

«Io mi sono fermato, Aṅgulimāla, fermati anche tu».

Il bandito pensò: «Questi monaci, figli dei Sakya, dicono la verità, affermano
la verità. Questo monaco però sta camminando e, tuttavia egli dice: “Io mi sono
fermato, Aṅgulimāla, fermati anche tu”. E se rivolgessi delle domande a questo
monaco?». Allora si rivolse al Beato in strofe:

\begin{quote}
Mentre stai camminando, monaco, \\
mi dici di esserti fermato, \\
ma ora che mi sono fermato, \\
mi dici che non mi sono fermato. \\
Ti chiedo, o monaco, qual è di questo il significato? \\
Com’è che tu ti sei fermato, e io no?

Aṅgulimāla, io mi sono fermato per sempre, \\
giurando di rinunciare a compiere violenza \\
verso ogni essere vivente, \\
tu, invece, non conosci contenimento verso nulla. \\
Per questo io mi sono fermato e tu no.

Oh, che viva a lungo un saggio che io posso riverire, \\
questo monaco è ora apparso in questa grande foresta. \\
Certamente io rinuncerò per molto tempo a ogni malvagità \\
ascoltando la tua esposizione in strofe del Dhamma.

Così dicendo, il bandito prese spada e armi \\
e le gettò in una fossa, in una voragine. \\
Il bandito si prostrò ai piedi del Sublime, venerandolo, \\
e poi gli chiese l’ammissione alla vita religiosa.

L’Illuminato, il Saggio di grande compassione, \\
l’insegnante del mondo con le sue divinità, \\
si rivolse a lui con queste parole: «Vieni, bhikkhu» \\
e fu così che lui divenne un bhikkhu.
\end{quote}

Il Beato si mise poi in viaggio per tappe per Sāvatthī con Aṅgulimāla come suo
monaco attendente. Infine arrivarono a Sāvatthī e il Beato si fermò nel
Boschetto di Jeta. Allora molta folla era riunita nei pressi del cancello del
palazzo del re Pasenadi, chiassosa e turbolenta, per chiedere che il bandito
fosse eliminato. A mezzogiorno il re si avviò verso il parco, accompagnato da
cinquecento cavalieri. Procedette finché la strada lo consentì alle carrozze e
poi scese e si avvicinò a piedi al Beato. Poi gli prestò omaggio e si mise a
sedere da un lato. Il Beato gli chiese: «Che cosa succede, gran re? Seniya
Bimbisāra, re di Magadha, ti sta attaccando? Oppure i Licchavi di Vesālī, o
qualche altro governante ostile?».

«No, Signore. Un bandito è apparso nel mio regno. Egli continua a uccidere le
persone, e indossa una collana fatta con le loro dita. Non riuscirò mai a
eliminarlo, Signore».

«Gran re, se però tu vedessi che Aṅgulimāla si è rasato barba e capelli, ha
indossato la veste ocra e ha rinunciato alla vita famigliare per la vita
religiosa, e che si astiene dall’uccidere e dal rubare, che mangia solo una
volta e prima di mezzogiorno, che vive la santa vita, virtuoso, con la bontà
quale suo ideale, che cosa ne faresti di lui?».

«Signore, dovremmo prestargli omaggio, oppure dovremmo alzarci, o invitarlo a
sedersi, oppure chiedergli di accettare vesti, cibo in elemosina, alloggio e
medicinali o organizzarci per proteggerlo, dargli asilo e difenderlo. Signore,
lui è però un miscredente che ha il male quale suo ideale. Come potrebbe avere
una tale virtù e un tale contenimento?».

Proprio allora, tuttavia, il venerabile Aṅgulimāla era lì seduto, non lontano.
Il Beato allungò il suo braccio destro e disse: «Gran re, ecco Aṅgulimāla».

Il re fu sconvolto e impaurito, e gli si rizzarono i capelli. Il Beato vide
tutto questo e disse: «Non temere, gran re, non temere. Non c’è nulla di cui
aver paura».

Allora lo sconvolgimento e la paura del re si placarono. Egli si avvicinò al
venerabile Aṅgulimāla e disse: «Signore, Aṅgulimāla era un nobile, o no?».

«Sì, gran re».

«Qual era la famiglia del padre del nobile? Qual era la famiglia della madre?».

«Mio padre, gran re, era un Gagga. Mia madre era una Mantāṇī».

«Che il nobile signore Gagga Mantāṇīputta mi consenta di provvedere alle sue
vesti, al cibo in elemosina, all’alloggio e ai medicinali».

In quel tempo, tuttavia, il venerabile Aṅgulimāla era un monaco che dimorava
nella foresta, mangiava solo cibo ottenuto dalla questua, indossava solo vesti
cucite di panni scartati e si limitava a tre sole vesti. Egli rispose: «Non ce
n’è bisogno gran re, il mio abito, composto dalle tre vesti, è al completo».

Il re Pasenadi tornò dal Beato e, dopo avergli prestato omaggio, si mise a
sedere da un lato. Egli disse: «È meraviglioso, Signore, è stupefacente come il
Beato domi gli indomiti, acquieti gli inquieti, porti l’estinzione in ciò che
non è estinto. Uno che non poté essere domato con punizioni e armi, il Beato lo
ha domato senza punizioni o armi. E ora, Signore, noi andiamo, siamo impegnati e
abbiamo molto da fare».

«È tempo ora, gran re, di fare quel che ritieni opportuno».

Allora il re Pasenadi si alzò dal posto in cui sedeva e, dopo aver prestato
omaggio, se ne andò, girando alla destra del Beato.

Un mattino il venerabile Aṅgulimāla prese la ciotola e la veste superiore e
entrò in Sāvatthī per la questua. Quando stava vagando di casa in casa a
Sāvatthī per la questua, vide una donna che stava partorendo un bimbo deforme.
Pensò: «Di quali contaminazioni soffrono le creature! Oh, di quali
contaminazioni soffrono le creature!». Poi andò dal Beato e gli raccontò
l’accaduto.

«Allora, Aṅgulimāla, vai a Sāvatthī e di' a quella donna: “Sorella, da quando
sono nato non ho mai preso di proposito la vita a un essere vivente. Grazie a
questa verità, che tu e il bimbo possiate ottenere la pace”».

«Signore, ma io non dovrei evitare di mentire in piena consapevolezza? Io ho
preso di proposito la vita a molti esseri viventi».

«Allora, Aṅgulimāla, vai a Sāvatthī e di' a quella donna: “Sorella, da quando
sono nato con questa nobile nascita non ho mai preso di proposito la vita a un
essere vivente. Grazie a questa verità, che tu e il bimbo possiate ottenere la
pace”».

«Così sia, Signore», egli rispose, e andò a Sāvatthī e disse a quella donna:
“Sorella, da quando sono nato con questa nobile nascita non ho mai preso di
proposito la vita a un essere vivente. Grazie a questa verità, che tu e il bimbo
possiate avere la pace”». E la donna e il bimbo ottennero la pace.

Allora, dimorando in solitudine, ritirato, diligente, ardente e autocontrollato,
il venerabile Aṅgulimāla, realizzandolo da se stesso mediante conoscenza
diretta, qui e ora entrò e dimorò in quella suprema meta della santa vita per la
quale gli uomini di famiglia giustamente lasciano la loro casa per una vita
priva di fissa dimora. Comprese direttamente: “La nascita è distrutta, la santa
vita è stata vissuta, quel che doveva essere fatto è stato fatto, non ci sarà
altra rinascita”». E il venerabile Aṅgulimāla divenne uno degli Arahant.

Un mattino il venerabile Aṅgulimāla si vestì, prese la ciotola e la veste
superiore e entrò a Sāvatthī per la questua. In quell’occasione, una zolla
tiratagli da qualcuno colpì il suo corpo, e un bastone tiratogli da qualcuno
colpì il suo corpo, un coccio tiratogli da qualcuno colpì il suo corpo. Allora,
con la testa rotta e con il sangue che ne fuoriusciva, con la ciotola in pezzi e
la rappezzata veste superiore strappata, andò dal Beato. Vedendolo arrivare, il
Beato disse: «Sopporta, brāhmaṇa, sopporta. Tu hai sperimentato qui e ora, in
questa vita, la maturazione delle azioni che potresti aver sperimentato
all’inferno per molti anni, per molti secoli, per molti millenni».

Quando il venerabile Aṅgulimāla era solo in ritiro assaporando la beatitudine
della Liberazione, esclamò queste parole:

\begin{quote}
Chi ha in precedenza vissuto con avventatezza \\
e poi così più non vive \\
illumina il mondo come la luna piena \\
quando le nuvole non la mascherano. \\
Chi esamina alla luce delle azioni salutari \\
le malvage azioni già compiute \\
illumina il mondo come la luna piena \\
quando le nuvole non la mascherano. \\
Chi, giovane bhikkhu, mostra \\
devozione al Dhamma del Buddha \\
illumina il mondo come la luna piena \\
quando le nuvole non la mascherano.

Oh, fate che i miei nemici ascoltino discorsi di Dhamma, \\
oh, fate che i miei nemici giungano all’insegnamento del Buddha, \\
oh, fate che i miei nemici si mettano al servizio di queste persone \\
per servire il Dhamma ed essere in pace. \\
Oh, fate che i miei nemici prestino orecchio di tanto in tanto \\
e ascoltino il Dhamma da chi predica pazienza e tolleranza, \\
da chi parla lodando pure la gentilezza, \\
e fanno sì che le loro azioni siano adeguate alle loro parole. \\
Certamente non desidereranno allora nuocermi, \\
né cercheranno di recare danno ad altri esseri viventi. \\
Così, chi tutti gli esseri protegge, deboli o forti che siano, \\
possa ottenere la pace suprema.

I costruttori di canali convogliano l’acqua, \\
i costruttori di archi addrizzano le frecce, \\
i falegnami raddrizzano le travi, \\
i saggi cercano di domare se stessi. \\
Alcuni domano con le percosse, \\
altri con pungoli e altri ancora con la sferza. \\
Chi non ha bacchetta né armi: \\
da costui io sono domato.

Innocente\footnote{NDT. Il nome attribuito ad Aṅgulimāla dal padre, un brāhmaṇa, fu Ahiṃsaka, che significa appunto “innocente”, “non violento”, “innocuo”.} è il mio nome, \\
fui nocivo agli altri in passato. \\
Il mio nome oggi è vero: \\
non faccio male ad alcuno. \\
Benché io sia vissuto da bandito \\
con il nome “Collana di Dita”, \\
guardate ora quale rifugio ho trovato: \\
non esiste più ciò che conduce alla rinascita. \\
Benché abbia compiuto molte azioni che promettevano \\
una nascita in infelici destinazioni, \\
i loro risultati mi hanno raggiunto ora, \\
e così mangio senza essere più in debito.

Oh, è folle e privo di intelligenza \\
chi si consegna all’avventatezza, \\
ma chi è diligente nel contenimento sensoriale \\
e lo considera come il bene più grande, \\
oh, non dà spazio all’avventatezza, \\
né nutre amore per i desideri sensoriali, \\
ma pratica la meditazione diligentemente \\
per raggiungere la più alta beatitudine.

Sia allora benvenuta questa mia scelta \\
la si lasci così com’è, non fu cosa mal fatta, \\
la triplice conoscenza è stata ottenuta \\
e quel che il Veggente ha ordinato è stato fatto.
\end{quote}

\suttaRef{M. 86}

\narrator{Secondo narratore.} Questa è la storia di un tentativo di screditare
il Buddha.

\voice{Prima voce.} Così ho udito. Una volta, quando il Beato soggiornava a
Sāvatthī, era onorato, rispettato, riverito, venerato e lodato. Otteneva vesti,
cibo in elemosina, alloggio e medicinali, e così pure il Saṅgha dei bhikkhu. Per
gli asceti itineranti di altre sette, però, le cose andavano diversamente. Non
potevano sopportare il rispetto dimostrato al Beato e al Saṅgha dei bhikkhu, e
perciò si recarono dalla monaca errante Sundarī e dissero: «Sorella, cerca di
aiutare i tuoi cugini».

«Che cosa devo fare, signori? Che cosa posso fare? La mia stessa vita è promessa
per il bene dei miei cugini».

«Allora, sorella, recati regolarmente nel Boschetto di Jeta».

«Così sia, signori», lei rispose. E si recò regolarmente nel Boschetto di Jeta.

Quando gli asceti itineranti seppero che lei era stata vista da molte persone
recarsi regolarmente nel Boschetto di Jeta, la uccisero e la seppellirono in una
buca scavata in un fossato del Boschetto di Jeta. Poi si recarono dal re
Pasenadi di Kosala e dissero: «Gran re, non riusciamo a trovare la monaca
itinerante Sundarī».

«Dove sospettate che sia?».

«Nel Boschetto di Jeta, gran re».

«Allora perlustrate il Boschetto di Jeta».

Gli asceti itineranti perlustrarono il Boschetto di Jeta e la dissotterrarono
dalla buca nel fossato in cui l’avevano sepolta. La collocarono su un letto e,
dopo essere entrati a Sāvatthī, si recarono di via in via, di crocicchio in
crocicchio, dichiarando alla gente: «Guardate, signori, guardate che cosa hanno
fatto questi figli dei Sakya! Questi figli dei Sakya sono svergognati,
sfacciati, malvagi, bugiardi e pure lussuriosi! Loro, che pretendono di
procedere nel Dhamma con equità e purezza, di dire il vero, di essere virtuosi e
buoni, loro non hanno nulla dei monaci, non hanno nulla dei brāhmaṇa. Sono solo
travestiti da monaci e da brāhmaṇa. In loro dov’è il monaco e il brāhmaṇa? Sono
molto lontani dall’essere monaci e brāhmaṇa. Com’è che un uomo può fare quello
che l’uomo fa con una donna, e poi ucciderla?».

Quando la gente vide i bhikkhu, li maltrattò, li maledisse, li insultò e li
rimproverò con parole scortesi e dure: «Questi figli dei Sakya sono svergognati,
sfacciati, malvagi, bugiardi e pure lussuriosi!» E ripeterono l’intera accusa. I
bhikkhu, sentendo queste cose, le riferirono al Beato.

«Questo clamore non durerà a lungo, bhikkhu. Durerà solo sette giorni. Al
termine di sette giorni cesserà. Così, quando la gente vi insulta in questo
modo, ammonitela con questa strofa:»

\begin{quote}
Il bugiardo va all’inferno, come colui che agisce \\
e poi dichiara: «Non sono stato io», \\
quando muoiono entrambi viaggiano allo stesso modo \\
nella vita successiva, come uomini dal comportamento abietto.
\end{quote}

I bhikkhu impararono questa strofa dal Beato. Quando la gente li insultò, loro
la ammonirono con essa. La gente pensò: «Questi monaci, questi figli dei Sakya,
non l’hanno fatto. Non sono stati loro a farlo. Lo giurano».

Questo clamore non durò a lungo. Durò solo sette giorni. Al termine di sette
giorni cessò. Allora un certo numero di bhikkhu andò dal Beato e disse: «È
meraviglioso, Signore, è magnifico quanto esatta sia stata la predizione del
Beato!».

Conoscendo il significato di ciò, il Beato esclamò allora queste parole:

\begin{quote}
Uomini incauti provocano con parole come frecce \\
fatte volare contro un elefante in battaglia. \\
Ma quando parole dure sono rivolte a un bhikkhu, \\
che egli sopporti con mente imperturbata.
\end{quote}

\suttaRef{Ud. 4:8}

\narrator{Primo narratore.} Non sappiamo quando gli eventi di seguito narrati si
verificarono, ma con essi possiamo chiudere i primi venti anni.

\voice{Prima voce.} Così ho udito. Una volta il Beato soggiornava a Cātumā in un
boschetto di mirabolano. In quell’occasione cinquecento bhikkhu guidati dal
venerabile Sāriputta e dal venerabile Mahā-Moggallāna erano giunti a Cātumā per
vedere il Beato. Mentre i bhikkhu in visita scambiavano saluti con i bhikkhu che
lì risiedevano e stavano preparando i giacigli, mettendo via le ciotole e le
vesti superiori, avvenne che fecero molto tumulto e rumore. Allora il Beato si
rivolse al venerabile Ānanda: «Ānanda, chi sono queste persone che fanno tanto
tumulto e rumore? Si potrebbe pensare che siano pescatori che cercano di vendere
il pesce pescato».

Quando il venerabile Ānanda glielo disse, egli rispose: «Allora, Ānanda, vai a
dire a questi bhikkhu da parte mia: “Il Maestro vi chiama, venerabili”». E il
venerabile così fece. Loro si recarono dal Beato e, dopo avergli prestato
omaggio, si misero a sedere da un lato. Dopo che lo ebbero fatto, il Beato
chiese loro: «Bhikkhu, perché fate tanto tumulto e rumore? Si potrebbe pensare
che siate pescatori che cercano di vendere il pesce pescato».

«Signore, questi sono cinquecento bhikkhu guidati dal venerabile Sāriputta e dal
venerabile Mahā-Moggallāna che sono venuti a vedere il Beato. Mentre stavano
scambiando saluti con i bhikkhu che lì risiedevano e stavano preparando i
giacigli, mettendo via le ciotole e le vesti superiori, fecero molto tumulto e
rumore». «Andate, bhikkhu. Io vi congedo. Non potete vivere con me».

«Sì, Signore», replicarono, si alzarono dal posto in cui sedevano e, dopo aver
prestato omaggio al Beato, se ne andarono girandogli a destra, ravvolsero i loro
giacigli, presero la loro ciotola e la veste superiore, e se ne andarono.

In quell’occasione i Sakya di Cātumā si trovavano nel loro salone per le
riunioni per alcuni affari e altre cose ancora. Videro da lontano i bhikkhu che
arrivavano. Uscirono a incontrarli e chiesero loro: «Dove state andando,
Signori?».

«Amici, il Saṅgha dei bhikkhu è stato congedato dal Beato».

«Allora che i venerabili restino seduti per un po’. Forse saremo in grado di far
tornare la fiducia nel Beato».

Così, i Sakya di Cātumā andarono dal Beato e, dopo avergli prestato omaggio, si
misero a sedere da un lato. Dopo averlo fatto, dissero: «Signore, che il Beato
perdoni il Saṅgha dei bhikkhu, che il Beato dia a loro il benvenuto e li aiuti,
come era solito fare in passato. Signore, ci sono nuovi bhikkhu che hanno appena
abbracciato la vita religiosa, che da poco sono giunti a questo Dhamma e
Disciplina. Se non hanno l’opportunità di vedere il Beato, nei loro cuori può
avvenire qualche cambiamento, qualche alterazione. Signore, proprio come quando
delle giovani piantine non ricevono acqua, in esse può avvenire qualche
cambiamento, qualche alterazione, oppure proprio come quando un giovane vitello
non vede la madre, nel suo cuore può avvenire qualche cambiamento, qualche
alterazione, altrettanto potrebbe avvenire a loro. Signore, che il Beato dia il
benvenuto al Saṅgha dei bhikkhu e lo aiuti, come era solito fare in passato».

E Brahmā Sahampati scomparve dal mondo di Brahmā, apparve di fronte al Beato e
fece la stessa richiesta.

Tutti insieme furono in grado di far tornare la fiducia nel Beato con le
immagini delle piantine e del giovane vitello.

Allora il venerabile Mahā-Moggallāna si rivolse ai bhikkhu in questo modo:
«Alzatevi, amici, prendete la vostra ciotola e la veste. I Sakya di Cātumā e
Brahmā Sahampati hanno fatto tornare la fiducia nel Beato con le immagini delle
piantine e del giovane vitello».

Quando furono tornati alla presenza del Beato, egli chiese al venerabile
Sāriputta: «Che cosa hai pensato, Sāriputta, quando il Saṅgha dei bhikkhu è
stato da me congedato?».

«Signore, ho pensato: “Adesso il Beato dimorerà inoperoso, si voterà a dimorare
piacevolmente nel qui e ora, e anche noi adesso dimoreremo inoperosi, ci
voteremo a dimorare piacevolmente nel qui e ora”».

«Basta così, Sāriputta, basta così! Pensieri come questi non devono più venirti
in mente». Allora il Beato chiese al venerabile Mahā-Moggallāna: «Che cosa hai
pensato, Mahā-Moggallāna, quando il Saṅgha dei bhikkhu è stato da me
congedato?».

«Signore, ho pensato: “Adesso il Beato dimorerà inoperoso, si voterà a dimorare
piacevolmente nel qui e ora, mentre io e il venerabile Sāriputta continueremo a
guidare il Saṅgha dei bhikkhu”».

«Bene, bene, Moggallāna. O sarò io a continuare a guidare il Saṅgha dei bhikkhu
oppure lo faranno Sāriputta e Moggallāna».

\suttaRef{M. 67}

\narrator{Secondo narratore.} Il Buddha raccontò ai bhikkhu di essere stato
negli alti paradisi del mondo di Brahmā.

\voice{Prima voce.} «Bhikkhu, una volta, quando vivevo a Ukkaṭṭhā nel Boschetto
di Subhaga ai piedi di un reale albero \emph{sāla}, in Brahmā Baka era sorto un
pernicioso modo di vedere (in relazione alla sua stessa permanenza e
assolutezza). Io nella mia mente fui consapevole del pensiero sorto nella mente
di Brahmā, e … comparvi in quel mondo. Brahmā Baka mi vide arrivare e disse:
“Vieni, buon signore! Benvenuto, buon signore! È da molto tempo, buon signore,
che non hai avuto occasione di venire qui. Ora, buon signore, questo è
permanente, questo dura per sempre, questo è eterno, questo è il tutto, questo
non è soggetto a svanire, perché questo non è né nato, né invecchia, né muore,
né svanisce e neanche ricompare, e oltre a questo non c’è altra via di fuga”».

«Allora Māra il Malvagio entrò in uno di coloro che componevano l’assemblea di
Brahmā e mi disse: “Bhikkhu, bhikkhu, non pensare che non dica il vero, non
pensare che non dica il vero, perché questo Brahmā è il Gran Brahmā, Essere
Trascendente Intrasceso, Lungimirante Branditore della Maestria, Signore
Artefice e Creatore, Altissima Provvidenza, Maestro e Padre di coloro che sono e
potranno essere. In un periodo a te precedente, bhikkhu, nel mondo c’erano
monaci e brāhmaṇa che condannavano la terra provando disgusto per la terra, che
condannavano l’acqua … il fuoco … l’aria … gli esseri … gli dèi … Pajāpati,
Signore della Creazione … che condannavano Brahmā provando disgusto per Brahmā.
Alla dissoluzione del corpo, quando il loro respiro si interruppe, rinacquero in
un corpo inferiore. In un periodo a te precedente, bhikkhu, nel mondo c’erano
monaci e brāhmaṇa che lodavano tutte queste cose provando amore per esse. Alla
dissoluzione del corpo, quando il loro respiro si interruppe, rinacquero in un
corpo superiore. Perciò, bhikkhu, questo ti dico: ‘Mettiti al sicuro, buon
signore, fai solo quel che dice Brahmā. Non trasgredire mai la parola di Brahmā.
Se lo farai, bhikkhu, tu sarai come un uomo che, raggiunto da un raggio di luce,
cerca di deviarlo con una bacchetta, oppure come un uomo che perde la presa
della terra con le mani e con i piedi e scivola in un abisso profondo. Sii
certo, buon signore, fai solo quel che dice Brahmā. Non trasgredire mai la
parola di Brahmā. Non vedi la Divina Assemblea che è qui seduta, bhikkhu?’ ”. E
Māra il Malvagio chiamò a testimonianza la Divina Assemblea».

«Quando ciò fu detto, io mi rivolsi a Māra il Malvagio: “Io ti conosco,
Malvagio, non immaginare: ‘Lui non mi conosce’. Tu sei Māra il Malvagio, e
Brahmā e la Divina Assemblea con tutti i suoi membri sono tutti caduti nelle tue
mani, sono tutti caduti in tuo potere. Tu, Malvagio, pensi che pure io sia
caduto in tuo potere, ma non è così”».

«Quando ciò fu detto, Brahmā Baka mi disse: “Buon signore, del permanente dico
che è permanente, di quel che dura per sempre che dura per sempre, dell’eterno
che è eterno, del tutto che è il tutto, di quel che non è soggetto a svanire che
non è soggetto a svanire, di quel che non è nato, né invecchia, né muore, né
svanisce e neanche ricompare che non è nato, né invecchia, né muore, né svanisce
e neanche ricompare, e di quello al di là del quale non c’è via di fuga, che non
c’è via di fuga al di là di quello. In un periodo a te precedente, bhikkhu, nel
mondo c’erano monaci e brāhmaṇa il cui ascetismo durò tanto a lungo quanto la
tua vita stessa. Loro sapevano che quando al di là c’era una via di fuga, che al
di là c’era una via di fuga, e che quando al di là non c’era una via di fuga,
che al di là non c’era una via di fuga. Perciò, bhikkhu, questo io ti dico: ‘Al
di là di questo non troverai via di fuga, e se cercherai di farlo alla fine
otterrai stanchezza e delusione. Se crederai nella\footnote{“Se crederai nella”:
  letteralmente \emph{sace … ajjhosissasi} significa “se accetterai” oppure,
  come dice il Commentario: “Se, per mezzo della fiducia (ossia
  dell’accettazione), della deglutizione, dell’assimilazione, presupporrai
  mediante bramosia, presunzione e opinioni”.} terra … nell’acqua … nel fuoco …
nell’aria … negli esseri … negli dèi … in Pajāpati … Se crederai in Brahmā, tu
sarai uno di quelli che stanno al mio fianco, risiederai nel mio dominio, quando
sarà giunto per me il momento di esercitare la mia volontà e di punire’ ”».

«“Io conosco anche te, Brahmā. Comprendo così in tal modo fin dove puoi arrivare
e la tua influenza: ‘Il potere di Brahmā Baka, la sua potenza, il suo seguito,
si estende fino a questo punto e non oltre’ ”».

«“Ora, buon signore, com’è che intendi l’estensione di fin dove posso arrivare e
il mio influsso?”».

\begin{quote}
Quant’è ampio il tragitto circolare di luna e sole, \\
il loro splendore e luminosità nelle quattro direzioni, \\
più di mille volte l’ampiezza di un mondo, \\
il tuo potere può esercitare il suo influsso. \\
E colà tu conosci sia l’alto sia il basso, \\
e coloro che sono governati dalla lussuria e da essi liberi, \\
la condizione di ciò che è così e altrimenti, \\
e la provenienza delle creature e la loro destinazione.
\end{quote}

«“Così intendo l’estensione di fin dove puoi arrivare e il tuo influsso. Ci sono
tuttavia altri tre corpi principali di dèi Brahmā che tu non conosci e neanche
vedi, ma io lo conosco e vedo. C’è il corpo chiamato Ābhassara (della Fluente
Radianza), dalla quale sei scomparso per ricomparire qui. Il tuo lungo dimorare
qui, però, lo ha fatto cancellare dalla tua memoria, e così tu non lo conosci e
neanche vedi, ma che io conosco e vedo. Io che sto qui, non sono allo stesso tuo
livello di conoscenza diretta, io non so meno di te, ma di più. E lo stesso
dicasi per gli altri ancor più alti corpi di Subhakiṇṇa (della Rifulgente
Gloria) e di Vehapphala (del Grande Frutto)”».

«“Ora, Brahmā, avendo avuto conoscenza diretta della terra in quanto terra, e
avendo avuto conoscenza diretta di quel che non è coessenziale rispetto
all’essenza della terra, io non pretendo di essere terra,\pagenote{%
  L’enfasi è sulla nozione dell’essere (“essere o non essere”). L’attribuzione
  di espressioni e letture è tratta dall’edizione birmana, che qui è più
  affidabile di qualsiasi altra e ha \emph{nāpahosiṃ} invece di \emph{nāhosi}.
  Così si dovrebbe ad esempio leggere: \emph{sabbaṃ kho ahaṃ brahme sabbato
    abhiññāya yāvatā sabbassa sabbattena ananubhūtaṃ, tad abhiññāya sabbaṃ
    nāpahosiṃ, sabbasmiṃ nāpahosiṃ, sabbato nāpahosiṃ, sabbam me ti nāpahosiṃ,
    sabbaṃ nābhivadiṃ} (“Avendo avuto conoscenza del tutto in quanto tutto …” ).
  Sia in questo sutta sia in D. 11 la riga \emph{Viññāṇam anidassanam anantaṃ
    sabbatopabhaṃ} (“La coscienza che non si mostra …”) è menzionata dal Buddha
  (pag.~\pageref{pag162} e anche pag.~\pageref{pag167}). Questa frase è stata un
  problema per molti. Il Commentario al \emph{Majjhima} ha un’ampiezza molto
  maggiore del Commentario al \emph{Dīgha} e propone una derivazione dalla
  radice \emph{bhū} (essere) per \emph{pabhaṃ} (o \emph{pahaṃ}). Seguendo questo
  suggerimento, sebbene non del tutto in linea con quanto suggerito dal
  Commentario, possiamo ritenere che \emph{sabbatopabhaṃ} sia costituito da
  \emph{sabbato} e da una forma contratta del participio presente di
  \emph{pahoti} (= \emph{pabhavati}), ossia \emph{pahaṃ} (= \emph{pabhaṃ}).
  Questo si lega con il precedente \emph{sabbato abhiññāya … sabbaṃ nāpahosiṃ =
    sabbato apabhaṃ} (“non pretendo di essere separato dal tutto”). Le lettere
  \emph{h} e \emph{bh} vengono facilmente confuse in Singalese. In D. 11, nel
  quale ricorre la stessa frase, il Buddha cita probabilmente da questo
  discorso. Abbiamo qui materiale per un interessante punto per uno studio
  ontologico.}
non pretendo che la terra sia mia, non affermo nulla a riguardo della terra.
Avendo avuto conoscenza diretta dell’acqua in quanto acqua … del fuoco …
dell’aria … degli esseri … degli dèi … di Pajāpati … di Brahmā … di Ābhassara …
di Subhakiṇṇa … di Vehapphala … dell’Essere Trascendente (Abhudhū) … Avendo
avuto conoscenza diretta del tutto in quanto tutto, e avendo avuto conoscenza
diretta di quel che non è coessenziale con la totalità del tutto, io non
pretendo di essere tutto, io non pretendo di essere nel tutto, io non pretendo
di essere separato dal tutto, io non pretendo che il tutto sia mio, non affermo
nulla a riguardo del tutto. Io che sto qui, inoltre, io non so meno di te, ma di
più”».

«“Buon signore, se tu pretendi d’aver acceduto a quel che non è coessenziale
alla totalità del tutto, che non si possa affermare che tu sia vano e vuoto!”».

\begin{quote}

\label{pag162}%
La coscienza che non si mostra \\
e che nemmeno ha a che fare con la finitezza, \\
pretendendo di non essere separata dal tutto

\end{quote}

non è coessenziale all’essenza della terra, all’essenza dell’acqua … all’essenza
del tutto.

«“Allora, buon signore, io sparirò dal tuo cospetto”».

«“Allora, Brahmā, sparisci dal mio cospetto, se puoi”».

«Brahmā Baka, pensando: “Io sparirò dal cospetto del monaco Gotama, io sparirò
dal cospetto del monaco Gotama”, non fu in grado di farlo. Io dissi: “Allora,
Brahmā, io sparirò dal tuo cospetto”».

«“Allora, buon signore, sparisci dal mio cospetto, se puoi”».

«Definii il potere sovrannaturale in questo modo: “Solo in relazione a Brahmā e
all’Assemblea, che loro sentano il suono della mia voce senza vedermi”, e dopo
essere scomparso, esclamai questa strofa:

\begin{quote}
Ho visto la paura in ogni tipo di esistenza \\
inclusi gli esseri che cercano la non-esistenza; \\
non c’è tipo di esistenza, affermo, \\
che non provi diletto per ciò a cui si attacca.
\end{quote}

«Allora Brahmā e l’Assemblea e tutti i suoi componenti si stupirono e si
meravigliarono, e dissero: “È meraviglioso, signori, è stupefacente! Questo
monaco Gotama che ha abbandonato la stirpe dei Sakya ha una forza e un potere
talmente grandi che noi mai abbiamo visto in qualsiasi altro monaco o brāhmaṇa!
Signori, benché viva in una generazione che si delizia nell’esistenza, che ama
l’esistenza, che prova contentezza nell’esistenza, egli ha estirpato l’esistenza
e le sue radici!”».

«Allora Māra il Malvagio entrò in uno di coloro che componevano l’assemblea di
Brahmā e disse: “Buon signore, se questo è quel che conosci, se questo è quel
che hai scoperto, non condurre a questo i tuoi discepoli laici o coloro che
hanno lasciato la propria casa per la vita religiosa, non insegnare a loro il
tuo Dhamma, né fai sorgere in loro il desiderio per esso. In un periodo a te
precedente, bhikkhu, nel mondo c’erano monaci e brāhmaṇa che pretendevano di
essere realizzati e completamente illuminati, e lo fecero. Alla dissoluzione del
corpo, però, quando il loro respiro si interruppe, rinacquero in un corpo
inferiore. In un periodo a te precedente, bhikkhu, nel mondo c’erano pure monaci
e brāhmaṇa che questo pretendevano, e non lo fecero. Alla dissoluzione del
corpo, quando il loro respiro si interruppe, rinacquero in un corpo superiore.
Perciò, bhikkhu, questo ti dico: ‘Mettiti al sicuro, buon signore, dimorando
inattivo, dedicati a dimorare piacevolmente nel qui e ora. È meglio che queste
cose non vengano dichiarate, buon signore, e perciò non informarne nessun altro’
”».

«Quando ciò fu detto, io risposi: “Io ti conosco, Malvagio. Non è per
compassione o per il desiderio del mio bene che tu parli in questo modo. Tu stai
pensando che coloro ai quali insegnerò questo Dhamma andranno al di là delle tue
possibilità di raggiungerli. Questi tuoi monaci e brāhmaṇa che pretendevano di
essere realizzati e completamente illuminati, in realtà non lo furono. Io però
lo sono, realizzato e completamente illuminato. Un Essere Perfetto è tale sia
che insegni il suo Dhamma ai discepoli sia che non lo faccia, sia che guidi i
suoi discepoli sia che non lo faccia. Perché? Perché quegli inquinanti che
contaminano, portano a rinnovate esistenze, recano ansietà, maturano nella
sofferenza, producono rinascita, invecchiamento e morte, sono in lui recisi alla
radice, resi come ceppi di palma, eliminati, così che non sono più soggetti a
sorgere nel futuro, proprio come una palma non può più crescere quando la sua
corona è tagliata”. Così, poiché Māra non aveva più nulla da dire, e in ragione
dell’invito a me fatto da Brahmā (di sparire), questo discorso può essere
intitolato “Dietro invito di un Brahmā”».

\suttaRef{M. 49}

Una volta il Beato soggiornava a Nālandā nel Boschetto di Pāvārikā. Allora il
figlio del capofamiglia Kevaḍḍha si recò da lui e, dopo avergli prestato
omaggio, si mise a sedere da un lato. Egli disse: «Signore, Nālandā ha successo,
è prosperosa, popolosa, affollata da esseri umani e ha fiducia nel Beato.
Signore, sarebbe cosa buona se il Beato incaricasse un bhikkhu di operare un
miracolo con poteri sovrannaturali maggiori di quelli propri della condizione
umana, così che Nālandā possa avere una fiducia ancora maggiore nel Beato».

Il Beato rispose: «Kevaḍḍha, non insegno il Dhamma ai bhikkhu in questo modo:
“Venite, bhikkhu, operate un miracolo con poteri sovrannaturali maggiori di
quelli propri della condizione umana per i laici vestiti di bianco”».

\narrator{Secondo narratore.} Il Buddha diede la stessa risposta quando tale
domanda fu ripetuta una seconda volta. Quando fu ripetuta ancora una volta, egli
rispose di conoscere per esperienza tre tipi di miracoli: il miracolo del potere
sovrannaturale che consiste nell’abilità di moltiplicarsi e di passare
attraverso i muri, di volare nell’aria e di camminare sull’acqua, perfino di
recarsi nel mondo di Brahmā (si veda il capitolo 16); il miracolo di divinazione
che consiste nell’abilità di leggere le menti; e il miracolo della guida che
consiste nell’istruire la gente, in breve o dettagliatamente, a proposito di che
cosa fare per il proprio bene. I primi due tipi di miracoli, se operati per
impressionare le persone, non sono diversi dalle arti magiche dette
rispettivamente \emph{gandhārī} e \emph{maṇikā}, e si potrebbe ben dire che se
un bhikkhu si comportasse in questo modo, praticherebbe tali arti. Questa è la
ragione per cui Egli, il Buddha, considerava questi miracoli come fonte di
vergogna, di umiliazione e di disgusto. Il terzo tipo di miracolo, quello della
guida, consisteva nell’insegnamento così com’era da lui impartito, il quale,
benché includesse proprio queste manifestazioni [miracolose], aveva come scopo
l’esaurimento delle contaminazioni e la fine della sofferenza. Al fine di
sottolineare l’inadeguatezza dei primi due conseguimenti, il Buddha raccontò la
vicenda di un bhikkhu che possedeva questi poteri magici, e come questi non gli
fossero serviti a nulla nella sua ricerca per una via d’uscita dalla sofferenza.

\voice{Prima voce.} «C’era un bhikkhu in questo Saṅgha di bhikkhu che ebbe
questo pensiero: “Dov’è che queste quattro entità cessano senza residuo, ossia
l’elemento terra, l’elemento acqua, l’elemento fuoco e l’elemento aria?”. Egli
entrò in uno stato tale di concentrazione che, quando la sua mente fu
concentrata, gli si manifestò il sentiero verso gli dèi. Allora si recò dalle
divinità del Regno dei Quattro Divini Sovrani e chiese loro: “Amici, dov’è che
queste quattro entità cessano senza residuo?”. Esse risposero: “Non lo sappiamo,
bhikkhu. Ci sono però questi stessi Quattro Divini Sovrani che sono più grandi
di noi e a noi superiori. Loro dovrebbero saperlo”. Così egli andò da loro».

\narrator{Secondo narratore.} Essi gli diedero la stessa risposta e lo inviarono
nel paradiso Tāvatiṃsa, e così egli andò attraverso tutti i cieli dell’esistenza
sensoriale fino a che fu inviato al di là di essi, nel mondo di Brahmā, il mondo
delle supreme divinità. Egli pose agli dèi dell’Assemblea di Brahmā la stessa
domanda. Loro gli dissero:

\voice{Prima voce.} «“Non lo sappiamo, bhikkhu. C’è però Brahmā, il Gran Brahmā,
Essere Trascendente Intrasceso, Lungimirante Branditore della Maestria, Signore
Artefice e Creatore, Altissima Provvidenza, Maestro e Padre di coloro che sono e
potranno essere, che è più grande di noi e a noi superiore. Lui dovrebbe
saperlo”. “Dov’è ora questo Brahmā, amici?”. “Bhikkhu, noi non sappiamo il dove,
il come e il quando del Gran Brahmā. Solo che Brahmā si manifesterà quando si
percepiranno dei segni, quando apparirà una luce, quando si manifesterà una
radiosità, perché tutto questo precorre alla manifestazione di Brahmā”».

«Subito dopo il Gran Brahmā si manifestò. Il bhikkhu si avvicinò e pose la sua
domanda. Quando essa fu formulata, Brahmā rispose: “Bhikkhu, io sono Brahmā, il
Gran Brahmā, Essere Trascendente Intrasceso, Lungimirante Branditore della
Maestria, Signore Artefice e Creatore, Altissima Provvidenza, Maestro e Padre di
coloro che sono e potranno essere”. Il bhikkhu chiese una seconda volta: “Amico,
non ti ho domandato questo. Ti ho chiesto: ‘Dov’è che queste quattro entità
cessano senza residuo?’. Il Gran Brahmā diede la stessa risposta di prima.
Quando la domanda fu posta per la terza volta, il Gran Brahmā prese il bhikkhu
per un braccio e lo condusse in disparte. Egli disse: “Bhikkhu, gli dèi
dell’Assemblea di Brahmā pensano in questo modo: ‘Non c’è nulla che Brahmā non
abbia visto, conosciuto e realizzato’. Per questa ragione non ti ho risposto
alla loro presenza. Amico, io non so dov’è che queste quattro entità cessano
senza residuo. Così tu hai sbagliato, hai trasgredito, a questo proposito hai
trascurato il Beato e cercato una risposta alla tua domanda lontano da lui. Vai
e poni al Beato la tua domanda e, quando ti risponderà, dovresti ricordare
quella sua risposta”».

«Allora il bhikkhu sparì da quel mondo e venne a farmi quella stessa domanda. Io
gli dissi: “Bhikkhu, i commercianti che vanno per mare, salpano portando con sé
un uccello in grado di trovare la costa e, quando dalla loro nave non si vede la
terra, liberano l’uccello. Va a est, a sud, a ovest e a nord, in alto e nel
mezzo. Se vede la terra da una parte, va in quella direzione, ma se non la vede
torna indietro sulla nave. Allo stesso modo, bhikkhu, ovunque tu abbia cercato,
perfino nel mondo di Brahmā non hai trovato una risposta alla tua domanda, e sei
tornato da me. La domanda, però, non dovrebbe essere posta in quel modo,
dovrebbe essere posta così»:

\begin{quote}
Dimmi, allora, dov’è che non trovano appoggio \\
acqua, terra, fuoco e aria? \\
Come pure il lungo e il corto, \\
il piccolo e il grande, il giusto e il disonesto? \\
Dov’è che nome-e-forma \\
cessano senza residuo?
\end{quote}

Questa è la risposta:

\begin{quote}

\label{pag167}%
La coscienza che non si mostra \\
né ha a che fare con la finitezza, \\
senza ritenere di essere separata dal tutto: \\
là è che acqua, terra, \\
fuoco e aria non trovano appoggio, \\
come pure il lungo e il corto, \\
il piccolo e il grande, il giusto e il disonesto. \\
Là è che nome-e-forma \\
cessano senza residuo.

\end{quote}

\suttaRef{D. 11}

