\chapter{Voci}

\narrator{Primo narratore.} È un commentatore del nostro tempo che offre cognizioni
introduttive, un osservatore imparziale con una generale conoscenza
degli eventi.

\narrator{Secondo narratore.} È un commentatore che propone informazioni storiche e
culturali presenti solo nei Commentari in lingua pāli, soprattutto in
quelli di Buddhaghosa, che risalgono al V secolo d.C. Ha la funzione di
offrire il materiale strettamente indispensabile per una
contestualizzazione storica e, occasionalmente, sintetizza parti del
Canone stesso.

\voice{Prima voce.} È la voce di Ānanda, discepolo e assistente personale del
Buddha, che recitò i Discorsi (o sutta) durante il Primo Concilio,
tenutosi a Rājagaha tre mesi dopo la morte e il \emph{Parinibbāna} del
Buddha.

\voice{Seconda voce.} È la voce di Upāli, discepolo del Buddha, che durante il
Primo Concilio recitò la Disciplina (o Vinaya).

\voice{Terza voce.} È un partecipante al Secondo Concilio, tenutosi cento anni
dopo il \emph{Parinibbāna} del Buddha, che nel XVI capitolo narra gli eventi
verificatisi prima e dopo il Primo Concilio.

\cantor{Cantore}

\begin{quote}
Questa voce recita alcuni versi \\
in forma di brevi poemi o inni \\
che nel Canone in lingua pāli \\
non sono introdotti \\
dalle tradizionali parole \\
pronunciate da Ānanda, «Così ho udito», \\
né sono presenti nella Disciplina.
\end{quote}

