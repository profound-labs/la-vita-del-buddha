\chapter{La dottrina}

\section*{Questioni varie}

\narrator{Primo narratore.} Che cos’è il “Dhamma” che è “ben proclamato” dal
“Medico Supremo”? È un tentativo di realizzare una completa descrizione del
mondo? È un sistema metafisico?

\voice{Prima voce.} Una volta il Beato soggiornava a Sāvatthī, nel Boschetto di
Jeta. Una divinità di nome Rohitassa andò da lui a notte tarda, gli prestò
omaggio e chiese: «Signore, la fine del mondo, nella quale non si nasce, né si
invecchia, né si muore, né si scompare e neanche si ricompare: è possibile
conoscerla o vederla viaggiando fin là?».

«Amico, che ci sia una fine del mondo, nella quale non si nasce, né si
invecchia, né si muore, né si scompare e neanche si ricompare, che possa essere
conosciuta o vista arrivando fin là: questo non lo dico. Non dico tuttavia che
ci sia una fine della sofferenza senza raggiungere la fine del mondo. Piuttosto
è in questa carcassa lunga un braccio, con le sue percezioni e la sua mente che
io descrivo il mondo, l’origine del mondo, la cessazione del mondo, e la via che
conduce alla cessazione del mondo».

\begin{quote}
È assolutamente impossibile \\
raggiungere la fine del mondo camminando. \\
Nessuno però sfugge alla sofferenza \\
a meno che non sia stata raggiunta la fine del mondo.

È un Saggio, un conoscitore del mondo, colui \\
che raggiunge la fine del mondo, ed è da lui \\
che è stata vissuta la santa vita. \\
Conoscendo la fine del mondo egli è in pace \\
e non ripone speranza né in questo mondo né nel successivo.
\end{quote}

\suttaRef{S. 2:36; A. 4:46}

Una volta il Beato soggiornava a Kosambī, in una foresta di alberi
\emph{siṃsapa}. Prese poche foglie in mano e chiese ai bhikkhu: «Cosa ne pensate
bhikkhu, sono di più le poche foglie che ho in mano o quelle degli alberi nella
foresta?».

«Le foglie che il Beato ha preso in mano sono poche, Signore. Quelle nella
foresta sono molte di più».

«Allo stesso modo, bhikkhu, le cose che ho conosciuto per conoscenza diretta
sono molte, le cose che vi ho detto sono solo poche. Perché non ve le ho dette?
Perché esse non recano beneficio, progresso nella santa vita e perché non
conducono al distacco, al disincanto, alla cessazione, all’acquietamento, alla
conoscenza diretta, all’Illuminazione, al Nibbāna. Per questa ragione non ve le
ho dette. E quali sono le cose che vi ho detto? “Questa è la sofferenza, questa
è l’origine della sofferenza, questa è la cessazione della sofferenza, questo è
il Sentiero che conduce alla cessazione della sofferenza”. Queste sono le cose
che vi ho detto. Perché ve le ho dette? Perché esse recano beneficio, progresso
nella santa vita e perché conducono al distacco, al disincanto, alla cessazione,
all’acquietamento, alla conoscenza diretta, all’Illuminazione, al Nibbāna. Così,
bhikkhu, che questo sia il vostro compito: Questa è la sofferenza, questa è
l’origine della sofferenza, questa è la cessazione della sofferenza, questo è il
Sentiero che conduce alla cessazione della sofferenza».

\suttaRef{S. 56:31}

\narrator{Primo narratore.} Non si tratta, allora, di un tentativo di realizzare
una completa descrizione del mondo, sia interiore sia esteriore. È un sistema
metafisico, una costruzione logica coerente? E, se è così, su quale fondamento
poggia?

\label{pag230}%
\voice{Prima voce.} Una volta, quando il Beato era entrato a Rājagaha per la
questua, l’asceta itinerante nudo Kassapa andò da lui e, dopo averlo salutato,
disse: «Qualora il Maestro Gotama acconsentisse a darci una risposta, vorremmo
chiedere una cosa al Maestro Gotama». – «Non è il momento di fare domande,
Kassapa, ci troviamo in mezzo ad abitazioni». Egli chiese una seconda e una
terza volta e ricevette la stessa risposta. Poi disse: «Non è molto quel che
vogliamo chiedere, Maestro Gotama». – «Domanda, allora, Kassapa, quel che vuoi
domandare».

«Com’è che stanno le cose, Maestro Gotama, la sofferenza è una propria
creazione?». – «Non metterla in questi termini, Kassapa». – Allora la sofferenza
è la creazione di un altro?». – «Non metterla in questi termini, Kassapa». –
«Allora la sofferenza è una creazione sia propria sia di un altro?». «Non
metterla in questi termini, Kassapa». – «Allora la sofferenza non è una
creazione propria né di un altro, bensì accidentale?». – «Non metterla in questi
termini, Kassapa». – «Allora non c’è sofferenza?». – «Non è un dato di fatto che
non ci sia la sofferenza, la sofferenza c’è, Kassapa». – «Allora il Maestro
Gotama non conosce né vede la sofferenza?». – «Non è un dato di fatto che io non
conosca né veda la sofferenza: io sia conosco sia vedo la sofferenza, Kassapa».

\suttaRef{S. 12:17}

Una volta anche l’asceta itinerante Uttiya andò dal Beato e, dopo averlo
salutato, si mise a sedere da un lato. Poi gli chiese: «Com’è, Maestro Gotama,
il mondo è eterno: questa è la sola verità e tutto il resto è errato?». – «A
questo non ho dato risposta, Uttiya». – «Allora il mondo non è eterno: questa è
la sola verità e tutto il resto è errato?». – «Nemmeno a questo ho dato
risposta, Uttiya». – «Allora il mondo è finito: questa è la sola verità e tutto
il resto è errato?». – «Nemmeno a questo ho dato risposta, Uttiya». – «Allora il
mondo è infinito: questa è la sola verità e tutto il resto è errato?». –
«Nemmeno a questo ho dato risposta, Uttiya». – «Allora, l’anima è la stessa cosa
del corpo: questa è la sola verità e tutto il resto è errato?». – «Nemmeno a
questo ho dato risposta, Uttiya». – «Allora l’anima è una cosa e il corpo
un’altra: questa è la sola verità e tutto il resto è errato?». – «Nemmeno a
questo ho dato risposta, Uttiya». – «Dopo la morte un Perfetto esiste: questa è
la sola verità e tutto il resto è errato?». – «Nemmeno a questo ho dato
risposta, Uttiya». – «Dopo la morte un Perfetto non esiste: questa è la sola
verità e tutto il resto è errato?». – «Nemmeno a questo ho dato risposta,
Uttiya». – «Allora dopo la morte un Perfetto sia esiste sia non esiste: questa è
la sola verità e tutto il resto è errato?». – «Nemmeno a questo ho dato
risposta, Uttiya». – «Allora dopo la morte un Perfetto né esiste né non esiste:
questa è la sola verità e tutto il resto è errato?». – «Nemmeno a questo ho dato
risposta, Uttiya».

«Perché il Maestro Gotama rifiuta di rispondere quando gli pongo queste domande?
A quali domande risponde allora il Maestro Gotama?».

«Io insegno il Dhamma ai discepoli per mezzo della conoscenza diretta, Uttiya,
per la purificazione degli esseri, per il superamento dell’afflizione e del
lamento, per la fine del dolore e del dispiacere, per il raggiungimento della
meta suprema, per l’ottenimento del Nibbāna».

«Maestro Gotama, questo Dhamma fornisce una via d’uscita dalla sofferenza per
tutto il mondo, oppure per la metà o per un terzo?».

Quando questo fu detto, il Beato rimase in silenzio.

Allora il venerabile Ānanda pensò: «L’asceta itinerante Uttiya non deve
concepire un punto di vista pernicioso quale “Quando al monaco Gotama è posta
una domanda insolita che mi è tipica e non è di nessun altro, si trova in
difficoltà e non risponde, ciò avviene perché non è in grado di farlo?”. Ciò gli
procurerebbe danno e sofferenza». Così egli disse: «Amico Uttiya, ti offrirò una
similitudine, perché qui alcuni saggi comprendono il significato di ciò che
viene detto tramite la conoscenza mediante una similitudine. Supponiamo che un
re abbia una città con profondi fossati, forti terrapieni e bastioni, e una sola
porta, e abbia un guardiano saggio, intelligente, sagace che blocchi alla porta
chi non conosce e faccia entrare solamente chi conosce. E siccome lui stesso ha
fatto un giro intorno alla città e non ha visto varchi nei terrapieni né alcun
foro abbastanza grande per farci passare un gatto, può giungere alla conclusione
che esseri viventi più grandi d’una certa dimensione debbano entrare e uscire
usando la porta. Così, amico Uttiya, interesse di un Perfetto non è che
“Mediante questo [Dhamma] trovi una via d’uscita tutto il mondo, oppure la metà
o un terzo”, bensì che “Chiunque abbia trovato o trovi o troverà una via
d’uscita dal mondo della sofferenza, questo avviene sempre abbandonando i cinque
impedimenti – ossia desiderio di sensorialità, malevolenza,
torpore-e-sonnolenza, agitazione-e-preoccupazione, dubbio – contaminazioni che
indeboliscono la comprensione, e mantenendo in essere i sette fattori
dell’Illuminazione con la mente ben fondata sui quattro fondamenti della
consapevolezza”. La domanda che hai posto al Beato era presentata in modo
sbagliato, per questa ragione il Beato non ha risposto».

\suttaRef{A. 10:95}

Un’altra volta l’asceta itinerante Vacchagotta andò dal Beato e scambiò con lui
saluti. Poi gli chiese: «Com’è che stanno le cose, Maestro Gotama, il sé
esiste?». Quando ciò fu detto, il Beato rimase in silenzio. «Com’è che stanno le
cose, Maestro Gotama, il sé non esiste?». E per la seconda volta il Beato rimase
in silenzio. Allora l’asceta itinerante Vacchagotta si alzò dal luogo in cui
sedeva e se ne andò. Non molto tempo dopo che era andato via, il venerabile
Ānanda chiese al Beato: «Signore, com’è che al Beato sono state poste delle
domande e lui non ha risposto?».

«Se quando mi è stato chiesto “Il sé esiste?”. io avessi risposto “Il sé
esiste”, ciò avrebbe implicato la credenza di coloro che sostengono la teoria
dell’eternalismo. E se quando mi è stato chiesto “Il sé non esiste?”. io avessi
risposto “Il sé non esiste”, ciò avrebbe implicato la credenza di coloro che
sostengono la teoria del nichilismo. Inoltre, se quando mi è stato chiesto “Il
sé esiste?” io avessi risposto “Il sé esiste”, ciò sarebbe stato conforme con la
mia conoscenza che tutte le cose sono non-sé? E se quando mi è stato chiesto “Il
sé non esiste? Io avessi risposto “Il sé non esiste” allora, confuso com’è,
l’asceta itinerante Vacchagotta sarebbe diventato ancor più confuso, pensando:
“Certamente prima avevo un sé e ora non ce l’ho”».

\suttaRef{S. 44:10}

Una volta il Beato soggiornava a Sāvatthī e a quel tempo un gran numero di
asceti itineranti e brāhmaṇa di varie sette si erano recati a Sāvatthī per la
questua. Avevano differenti punti di vista, opinioni e nozioni, e per ottenere
supporto contavano sui loro differenti punti di vista. C’erano alcuni asceti e
brāhmaṇa che asserivano e credevano che «Il mondo è eterno: questa è la sola
verità e tutto il resto è errato», e altri che asserivano e credevano in ognuno
degli altri nove punti di vista. Litigavano, bisticciavano, disputavano e si
ferivano a vicenda con frecce fatte di parole: «Il Dhamma è così, il Dhamma non
è così! Il Dhamma non è così, il Dhamma è così!».

Allora un gruppo di bhikkhu di ritorno dal giro per la questua lo raccontò al
Beato. Il Beato disse:

«Bhikkhu, una volta a Sāvatthī c’era un re. Egli disse a un uomo: “Vieni, uomo,
riunisci tutti gli uomini che a Sāvatthī sono nati ciechi”. – “Sì, Signore”,
egli rispose. E quando lo ebbe fatto, lo comunicò al re, il quale disse: “Mostra
loro un elefante”. Lo fece dicendo: “Voi, uomini che siete ciechi fin dalla
nascita, così è un elefante”, e ad alcuni fece toccare la testa dell’elefante,
ad altri un orecchio, ad altri una zanna, ad altri la proboscide, ad altri il
corpo, ad altri una zampa, ad altri la parte posteriore, ad altri la coda e ad
altri ancora il ciuffo di peli alla fine della coda. Poi andò dal re e gli disse
quel che aveva fatto. Il re allora si recò dagli uomini ciechi fin dalla nascita
e chiese loro: “Vi è stato mostrato un elefante?”. – “Sì, sovrano”. –
“Descrivetemi allora com’è un elefante”. Coloro ai quali era stata fatta toccare
la testa dissero “Sovrano, l’elefante è come una giara”, coloro ai quali era
stato fatto toccare un orecchio dissero “È come un setaccio”, coloro ai quali
era stata fatta toccare una zanna dissero “È come un palo”, coloro ai quali era
stata fatta toccare la proboscide dissero “È come l’asta di un aratro”, coloro
ai quali era stato fatto toccare il corpo dissero “È come un granaio”, coloro ai
quali era stata fatta toccare una zampa dissero “È come la base di una colonna”,
coloro ai quali era stata fatta toccare la parte posteriore dissero “È come un
mortaio”, coloro ai quali era stata fatta toccare la coda dissero “È come un
pestello” e coloro ai quali era stato fatto toccare il ciuffo di peli alla fine
della coda dissero “È come una scopa”. Si prendevano a pugni, urlando “Un
elefante è così, non è così. Un elefante non è così, è così!”. Il re, però, era
compiaciuto. Allo stesso modo, anche gli asceti itineranti di altre sette sono
ciechi e privi di occhi. Per questa ragione litigano, bisticciano, disputano e
si feriscono a vicenda con frecce fatte di parole: “Il Dhamma è così, il Dhamma
non è così! Il Dhamma non è così, il Dhamma è così!”».

\suttaRef{Ud. 6:4}

\narrator{Primo narratore.} Sarebbe perciò un errore definire l’insegnamento del
Buddha sia un tentativo di realizzare una completa descrizione del mondo sia un
sistema metafisico costruito mediante la logica. Esso è allora un comandamento
etico, una religione di fede rivelata o, semplicemente, un codice
comportamentale stoico? Prima di tentare di trovare delle risposte a queste
domande, è necessario un sommario delle dottrine insegnate. Il materiale
contenuto nei Discorsi sembra, nei fatti, avere piuttosto le caratteristiche del
materiale necessario all’elaborazione di una mappa, per consentire a ognuno di
realizzarne una propria, ma che conduca tutti verso una sola direzione. Queste
descrizioni orientate di sfaccettature dell’esperienza, infatti, consentono a
una persona di valutare la propria posizione e di giudicare da sé cosa sia
meglio fare. I Discorsi offrono non tanto una descrizione quanto, piuttosto, una
serie di descrizioni sovrapposte. In un esame condotto da vicino, dell’esistenza
si rinviene sempre un qualcosa che ha le qualità d’un miraggio e, dietro
l’apparenza, d’un paradosso, ma delle conclusioni non è mai possibile
individuarle. Le numerosissime diverse sfaccettature offerte nei sutta con
innumerevoli ripetizioni di alcune di tali sfaccettature in varie combinazioni e
contesti, ricorda un insieme di fotografie aeree mediante le quali si debbano
realizzare delle mappe. Le sfaccettature presenti nei Discorsi sono tutte
orientate verso la cessazione della sofferenza, grazie a una bussola i cui
quattro punti cardinali sono le Quattro Nobili Verità. Proviamo a realizzare una
mappa campione da una parte di questi materiali. Siccome da qualche parte pur si
deve cominciare, possiamo farlo prendendo la nascita come punto di partenza,
che, assieme alla morte, rappresenta per l’uomo comune un evento quotidiano e,
nello stesso tempo, un mistero irrisolvibile.

\section*{Non c’è un Primo Inizio}

\narrator{Secondo narratore.} La coscienza è concepibile senza un passato? Si
può dire che abbia un inizio?

\voice{Prima voce.} «Bhikkhu, il cerchio non ha inizio. Degli esseri che
viaggiano e arrancano in questo cerchio, rinserrati come sono nell’ignoranza e
incatenati dalla brama, non si può descrivere alcun inizio».

\suttaRef{S. 15:1}

«Che sia io sia voi abbiamo dovuto viaggiare e arrancare in questo lungo cerchio
è dovuto al fatto che non abbiamo scoperto, non abbiamo penetrato quattro
verità. Quali quattro? Esse sono: (I) la Nobile Verità della Sofferenza, (II) la
Nobile Verità dell’Origine della Sofferenza, (III) la Nobile Verità della
Cessazione della Sofferenza, e (IV) la Nobile Verità del Sentiero che conduce
alla Cessazione della Sofferenza».

\suttaRef{D. 16}

\section*{Le Quattro Nobili Verità}

\narrator{Secondo narratore.} Ecco una descrizione delle Quattro Nobili Verità.

\voice{Prima voce.} I. «Qual è la Nobile Verità della Sofferenza? La nascita è
sofferenza, la vecchiaia è sofferenza, la malattia è sofferenza, la morte è
sofferenza. L’afflizione, il lamento, il dolore, il dispiacere e la disperazione
sono sofferenza. Associarsi con quel che si detesta è sofferenza, separarsi da
quel che si ama è sofferenza, non ottenere ciò che si vuole è sofferenza. In
breve, i cinque aggregati affetti dall’attaccamento sono sofferenza».\footnote{I
  “cinque aggregati affetti dall’attaccamento” (\emph{upādāna-kkhanda}) possono
  essere considerati come le cinque apposite “classi” o categorie sotto le quali
  ogni componente dell’esperienza (nel senso più ampio del termine) che si trova
  a sorgere può essere raggruppato per l’analisi e la discussione. Esse non
  hanno esistenza separata dai componenti che li rappresentano. Quel che
  rappresentano non si verifica separatamente. Essi sono d’altra parte
  interdipendenti, come un bicchiere di vetro implica contemporaneamente
  materiale (il vetro), affettività (attraente, non attraente o indifferente),
  caratteristiche individuali (forma, colore, ecc.), determinatezza (essere
  formato) quanto all’utilità (tutte cose che sono costitutive di
  “nome-e-forma”), e la coscienza di tutto ciò che non è.}

\suttaRef{S. 56:11}

II. «Qual è la Nobile Verità dell’Origine della Sofferenza? È la brama, che
rinnova l’esistenza e che è accompagnata dal diletto e dal desiderio,
dall’assaporare questo e quello: in altre parole, brama per i desideri
sensoriali, brama per l’esistenza, brama per la non-esistenza. Su cosa sorge e
fiorisce, però, questa brama? Ovunque ci sia qualcosa che sembra amabile e
gratificante, su questo sorge e fiorisce».

\suttaRef{D. 22}

«È con l’ignoranza quale condizione che le formazioni [mentali] giungono a
esistere; con le formazioni [mentali] quale condizione, la coscienza; con la
coscienza quale condizione, nome-e-forma; con nome-e-forma quale condizione, la
sestuplice base per il contatto; con la sestuplice base quale condizione, il
contatto; con il contatto quale condizione, la sensazione; con la sensazione
quale condizione, la brama; con la brama quale condizione, l’attaccamento; con
l’attaccamento quale condizione, l’esistenza; con l’esistenza quale condizione,
la nascita; con la nascita quale condizione, giungono all’esistenza la vecchiaia
e la morte, e anche l’afflizione, il lamento, il dolore, il dispiacere e la
disperazione. Così ha origine tutto questo aggregato di sofferenza. Questa è
detta Nobile Verità dell’Origine della Sofferenza».

\suttaRef{A. 3:61}

III. «Qual è la Nobile Verità della Cessazione della sofferenza? È lo svanire
senza residuo e la cessazione di quella stessa brama, il rifiuto, l’abbandono,
la rinuncia a essa. Ma dove questa brama è abbandonata e fatta cessare? Ovunque
ci sia qualcosa che sembra amabile e gratificante, è qui che essa è abbandonata
e condotta a cessazione».

\suttaRef{D. 22}

Con lo svanire senza residuo e la cessazione dell’ignoranza, c’è la cessazione
delle formazioni [mentali]; con la cessazione delle formazioni [mentali], la
cessazione della coscienza … con la cessazione della nascita, la vecchiaia e la
morte cessano, e anche l’afflizione, il lamento, il dolore, il dispiacere e la
disperazione. Così c’è la cessazione di tutto questo aggregato di sofferenza.
Questa è detta Nobile Verità della Cessazione della Sofferenza».

\suttaRef{A. 3:61}

IV. «Qual è la Nobile Verità del Sentiero che conduce alla Cessazione della
Sofferenza? È il Nobile Ottuplice Sentiero, ossia: retta visione, retta
intenzione, retta parola, retta azione, retto modo di vivere, retto sforzo,
retta consapevolezza, retta concentrazione».

\suttaRef{D. 22}

«Di queste Quattro Nobili Verità, la Nobile Verità della Sofferenza deve essere
penetrata con piena comprensione della sofferenza; la Nobile Verità dell’Origine
della Sofferenza deve essere penetrata mediante l’abbandono della brama; la
Nobile Verità della Cessazione della Sofferenza deve essere penetrata
realizzando la cessazione della brama; la Nobile Verità del Sentiero che conduce
alla Cessazione della Sofferenza deve essere penetrata mantenendo in essere il
Nobile Ottuplice Sentiero».

\suttaRef{S. 56:11 e 29 (adattati)}

«Queste Quattro Nobili Verità (Realtà) sono reali, non irreali, non diverse da
quello che sembrano».

\suttaRef{S. 56:27}

\narrator{Primo narratore.} Ognuna delle Quattro Nobili Verità è analizzata e
definita dettagliatamente.

\section*{La Verità della Sofferenza}

\narrator{Secondo narratore.} È stato detto che la Verità della Sofferenza era
«in breve, i cinque aggregati affetti dall’attaccamento». Ecco una definizione
di essi.

\voice{Prima voce.} I. «Quali sono i cinque aggregati affetti dall’attaccamento?
Essi sono l’aggregato della forma (materiale) affetto dall’attaccamento,
l’aggregato della sensazione affetto dall’attaccamento, l’aggregato della
percezione affetto dall’attaccamento, l’aggregato delle formazioni [mentali]
affetto dall’attaccamento e l’aggregato della coscienza affetto
dall’attaccamento».

\suttaRef{D. 22}

«Perché si dice “forma”? Essa è deformata (\emph{ruppati}), ecco perché è
chiamata “forma” (\emph{rūpa}). Deformata da che cosa? Dal freddo e dal caldo,
dalla fame e dalle sete, dal contatto con i tafani, le zanzare, il vento, le
scottature del sole e le cose striscianti».

\suttaRef{S. 22:79}

«Che cos’è la forma? Le quattro grandi entità e ogni forma ricavata da esse per
mezzo dell’attaccamento sono chiamate forma».

\suttaRef{S. 22:56}

«Ogni cosa in un essere, appartenente a un essere, che sia solida, solidificata
e attaccata [a qualcosa di organico], come capelli, peli, unghie, denti, pelle,
carne, muscoli, ossa, midollo osseo, reni, cuore, fegato, diaframma, milza,
polmoni, intestino, viscere, cibo non digerito, feci, o qualsiasi altra cosa in
un essere, che appartiene a un essere, che sia solida, solidificata e attaccata:
ciò è chiamato elemento terra\footnote{La “terra” rappresenta la solidità,
  l’“acqua” la coesione, il “fuoco” sia la temperatura sia la maturazione,
  l’“aria” sia l’estensione (distensione) sia il moto.} in un essere. Ora,
l’elemento terra in un essere e l’elemento terra esteriore sono solo elemento
terra».

«Ogni cosa in un essere … che sia acqua, acquosa e attaccata, come bile, flegma,
pus, sangue, sudore, grasso, lacrime, materia oleosa, saliva, muco, liquido
sinoviale, urina, o qualsiasi altra cosa in un essere … che sia acqua, acquosa e
attaccata: ciò è chiamato elemento acqua in un essere. Ora, l’elemento acqua in
un essere e l’elemento acqua esteriore sono solo elemento acqua».

«Ogni cosa in un essere … che sia fuoco, infuocata e attaccata, come ciò per
mezzo del quale ci si scalda, si invecchia e ci si consuma, e per mezzo del
quale ciò che è mangiato, bevuto, masticato e gustato viene digerito e
assimilato, o qualsiasi altra cosa in un essere … che sia fuoco, infuocata e
attaccata: ciò è chiamato elemento fuoco in un essere. Ora, l’elemento fuoco in
un essere e l’elemento fuoco esteriore sono solo elemento fuoco».

«Ogni cosa in un essere … che sia aria, ariosa e attaccata, come i venti (forze)
che vanno verso l’alto, i venti (forze) che vanno verso il basso, i venti
(forze) nella pancia e nelle viscere, i venti (forze) che pervadono tutte le
membra, l’inspirazione e l’espirazione, o qualsiasi altra cosa in un essere …
che sia aria, ariosa e attaccata: ciò è chiamato elemento aria in un essere.
Ora, l’elemento aria in un essere e l’elemento aria esteriore sono solo elemento
aria».

«Ogni cosa in un essere … che sia spazio, spaziosa e attaccata, come il foro
dell’orecchio, il foro della bocca, la porta della bocca, e ciò (l’apertura)
mediante cui si deglutisce quel che si mangia, beve, mastica e assapora, e ciò
in cui questo è contenuto, e ciò mediante cui questo passa verso il basso, o
qualsiasi altra cosa in un essere … che sia spazio, spazioso e attaccato: ciò è
chiamato elemento spazio [in un essere]. Ora, l’elemento spazio in un essere e
l’elemento spazio esteriore sono solo elemento spazio … E l’elemento spazio non
ha alcun luogo nel quale può esistere di per sé».

\suttaRef{M. 62}

«Qualsiasi forma, passata, futura o presente, in un essere oppure esteriore,
grossolana o sottile, inferiore o superiore, lontana o vicina, che sia affetta
da contaminazioni e provochi l’attaccamento: essa è chiamata aggregato della
forma affetto da attaccamento».

\suttaRef{S. 22:48}

«Perché si dice “sensazione”? È sentita, ecco perché è chiamata “sensazione”.
Sentita come che cosa? Sentita come piacere, come dolore, oppure come
né-dolore-né-piacere».

\suttaRef{S. 22:79; cf. M. 43}

«Qualsiasi cosa sia sentita con il corpo o con la mente come piacevole e
gratificante è sensazione piacevole. Qualsiasi cosa sia sentita con il corpo o
con la mente come dolorosa e lesiva è sensazione dolorosa. Qualsiasi cosa sia
sentita con il corpo o con la mente come né gratificante né lesiva è sensazione
né-dolorosa-né-piacevole … La sensazione piacevole è piacevole in ragione della
presenza e dolorosa in ragione del cambiamento. La sensazione dolorosa è
dolorosa in ragione della presenza e piacevole in ragione del cambiamento. La
sensazione né-dolorosa-né-piacevole è piacevole in ragione della conoscenza e
dolorosa in ragione della mancanza di conoscenza».

\suttaRef{M. 44}

«Ci sono questi sei corpi di sensazione: la sensazione nata dal contatto con
l’occhio, dal contatto con l’orecchio, dal contatto con il naso, dal contatto
con la lingua, dal contatto con il corpo e dal contatto con la mente».

\suttaRef{S. 22:56}

«Qualsiasi sensazione … che sia affetta da contaminazioni e provochi
l’attaccamento: essa è chiamata aggregato della sensazione affetto da
attaccamento».

\suttaRef{S. 22:48}

«Perché si dice “percezione”? È percepita, ecco perché è chiamata “percezione”.
Percepita come che cosa? Percepita, ad esempio, blu e gialla e rossa e bianca».

\suttaRef{S. 22:79}

«Ci sono questi sei corpi della percezione: percezione delle forme (visibili),
dei suoni, degli odori, dei sapori, degli oggetti tangibili e delle idee».

\suttaRef{S. 22:56}

«Qualsiasi percezione … che sia affetta da contaminazioni e provochi
l’attaccamento: essa è chiamata aggregato della percezione affetto da
attaccamento».

\suttaRef{S. 22:48}

«Perché si dice “formazioni”? Danno forma al formato, ecco perché si chiamano
“formazioni”. Che cos’è il formato al quale danno forma? La forma (materiale),
in quanto stato (essenza) della forma, è il formato (composto) al quale esse
danno forma (il composto). La sensazione, in quanto stato della sensazione, è il
formato al quale esse danno forma. La percezione, in quanto stato della
percezione, è il formato al quale esse danno forma. Le formazioni, in quanto
stato delle formazioni, è il formato al quale esse danno forma. La coscienza, in
quanto stato della coscienza, è il formato al quale esse danno
forma».\footnote{«Qualsiasi cosa abbia la caratteristica di dare forma dovrebbe
  essere compresa, tutt’insieme, come aggregato delle formazioni … ha la
  caratteristica di agglomerare … (e) ha la funzione di accumulare»; cf.
  \emph{The Path of Purification (Visuddhimagga)}, tr. da Ñāṇamoli, XIV, 131
  (Nyp.).}

\suttaRef{S. 22:79}

«Tre tipi di formazioni: formazione del merito (in quanto azione che matura in
piacere), formazione del demerito (in quanto azione che matura in dolore), e
formazione dell’imperturbabilità (in quanto azione, ossia, la meditazione, che
matura in stati privi di forma che, per il tempo che durano, non sono perturbati
dalla percezione della forma, della resistenza o della differenza)».

\suttaRef{D. 33}

«Tre formazioni: inspirazione ed espirazione appartengono a un corpo, queste
sono cose legate a un corpo, per questa ragione sono formazioni corporee. Dopo
aver pensato ed esplorato, si irrompe nel parlare, per questa ragione pensare ed
esplorare sono formazioni verbali. Percezione e sensazione appartengono alla
coscienza, queste sono cose legate alla coscienza, per questa ragione esse sono
formazioni mentali».

\suttaRef{M. 44; cf. M. 9}

«Che cosa sono le formazioni? Ci sono sei corpi di
scelta:\footnote{\emph{Cetanā}, di solito tradotto con “volizione”, volontà
  (Nyp.).} scelta tra le forme visibili, tra i suoni, tra gli odori, tra i
sapori, tra gli oggetti tangibili e tra gli oggetti mentali».

\suttaRef{S. 22:56}

«Chiamo azione la scelta».

\suttaRef{A. 6:63}

«Qualsiasi formazione … che sia affetta da contaminazioni e provochi
l’attaccamento: essa è chiamata aggregato delle formazioni affetto da
attaccamento».

\suttaRef{S. 22:48}

«Perché si dice “coscienza”? Essa ha cognizione, ecco perché si chiama
“coscienza”. Di che cosa ha cognizione? Essa ha cognizione, ad esempio,
dell’aspro, dell’amaro, del pungente, del dolce, dell’alcalino, del non
alcalino, del salato e del non salato».

\suttaRef{S. 22:79}

«Di che cosa ha cognizione la coscienza? Essa ha cognizione, ad esempio, che c’è
il piacere, che c’è il dolore, che c’è né-dolore-né-piacere».

\suttaRef{M. 43, 140}

«Ci sono questi sei corpi della coscienza: coscienza visiva, coscienza uditiva,
coscienza olfattiva, coscienza gustativa, coscienza corporea e coscienza
mentale».

\suttaRef{S. 22:56}

«La coscienza ha un nome in base alle condizioni che la fanno sorgere. Quando la
coscienza sorge a causa dell’occhio e delle forme, è chiamata coscienza visiva.
Se sorge a causa dell’orecchio e dei suoni, coscienza uditiva … Se sorge a causa
della mente e delle idee, coscienza mentale».

\suttaRef{M. 38}

«Sensazione, percezione e coscienza sono congiunte, non disgiunte, ed è
impossibile separarle una dall’altra al fine di descrivere le loro differenti
potenzialità. Perché quando uno ha una sensazione, è quello stesso a percepire,
e quando uno ha una percezione, è quello stesso ad averne cognizione. Mediante
la mera coscienza mentale disgiunta dalle cinque facoltà sensoriali, la base
(esterna) che consiste nell’infinitezza dello spazio può essere conosciuta come
“spazio infinito”. La base (esterna) che consiste nella infinitezza della
coscienza può essere conosciuta come “coscienza infinita”. E la base (esterna)
che consiste nel nulla può essere conosciuta come “nulla-è”. Un’idea conoscibile
è compresa mediante l’occhio della comprensione».

\suttaRef{M. 43}

% FIXME missing »

«La coscienza per la sua esistenza poggia su un dualismo (il dualismo
dell’interiorità e le basi esterne per il contatto).

\suttaRef{S. 35:93}

«Qualsiasi coscienza, passata, futura o presente, in un essere oppure esteriore,
grossolana o sottile, inferiore o superiore, lontana o vicina, che sia affetta
da contaminazioni e provochi l’attaccamento: essa è chiamata aggregato della
coscienza affetto da attaccamento».

\suttaRef{S. 22:48}

«Questi cinque aggregati affetti da attaccamento hanno il desiderio per la loro
radice … Le quattro grandi entità (di terra, acqua, fuoco e aria) sono la causa
e la condizione per descrivere l’aggregato della forma. Il contatto è la causa e
la condizione per descrivere gli aggregati della sensazione, della percezione e
delle formazioni [mentali]. Nome-e-forma è la causa e la condizione per
descrivere l’aggregato della coscienza».

\suttaRef{M. 109}

«Qualsiasi monaco o brāhmaṇa ricordi la sua vita passata nei suoi vari modi,
ricorda i cinque aggregati affetti da attaccamento o uno o l’altro di essi».

\suttaRef{S. 22:79}

\section*{La Verità dell’Origine della Sofferenza}

\narrator{Secondo narratore.} Ecco alcune definizioni dettagliate della Seconda
Nobile Verità.

\voice{Prima voce.} «Questi cinque aggregati affetti dall’attaccamento provano
desiderio per la loro radice … L’attaccamento non è la stessa cosa dei cinque
aggregati affetti dall’attaccamento, né è qualcosa di separato da essi. È il
desiderio e la brama in essi contenuto che è l’attaccamento».

\suttaRef{M. 109}

«Quello giunge all’esistenza quando c’è questo, quello sorge con il sorgere di
questo».\footnote{Nel senso di condizione necessaria.}

\suttaRef{M. 38}

«(Nell’esposizione della genesi interdipendente:)\footnote{Sulla genesi
  interdipendente, o originazione interdipendente e coproduzione condizionata,
  si veda \emph{The Path of Purification}, cap. XVII.} Che cos’è
l’invecchiamento? Nei vari generi di esseri è l’invecchiare, la vecchiaia, i
denti che si rompono, il grigiore dei capelli e la rugosità, il declino della
vita e l’indebolimento delle facoltà sensoriali.

Che cos’è la morte? Nei vari generi di esseri è la scomparsa, il trapasso, la
dissoluzione, lo scomparire, il morire, il completamento del tempo, la
dissoluzione degli aggregati, il giacere della carcassa.

Che cos’è la nascita? Nei vari generi di esseri è la nascita, il venire alla
nascita, il depositarsi in un utero, la generazione, la manifestazione degli
aggregati, l’acquisizione delle basi di contatto.

Che cos’è l’esistenza? Tre sono i tipi di esistenza: l’esistenza nella modalità
del desiderio sensoriale, l’esistenza nella modalità della forma, l’esistenza
nella modalità del senza forma. Che cos’è l’attaccamento? Quattro sono le
varietà di attaccamento: l’attaccamento come abitudine al desiderio sensoriale,
l’attaccamento come abitudine all’errata visione, l’attaccamento come abitudine
(al fraintendimento) della virtù e del dovere,\footnote{\emph{Sīlabbatupādāna},
  l’attaccamento a riti e rituali (Nyp.).} e l’attaccamento come abitudine alla
teoria del sé.

Che cos’è la brama? Sei sono i corpi della brama: la brama per le forme
visibili, per i suoni, per gli odori, per i sapori, per gli oggetti tangibili e
per le idee.

Che cos’è la sensazione? Sei sono i corpi (delle tre specie) della sensazione:
sensazione nata dal contatto con l’occhio, dal contatto con l’orecchio, dal
contatto con il naso, dal contatto con la lingua, dal contatto con il corpo e
dal contatto con la mente.

Che cos’è il contatto?\footnote{Il “contatto” è contatto tra l’“in-sé” e
  l’“esterno” (ad esempio, la vista insieme a ciò che è visto), il quale è reso
  possibile solo dalla presenza della coscienza (ad esempio, coscienza visiva).
  È perciò un fattore basilare nell’essenziale complessità di qualsiasi cosa
  sorga, sia percepita e formata, tanto dai cinque sensi quanto dalla mente, sia
  dai sensi e dalla mente insieme.} Sei sono i corpi del contatto: il contatto
con l’occhio, il contatto con l’orecchio, il contatto con il naso, il contatto
con la lingua, il contatto con il corpo e il contatto con la mente.

Che cos’è la sestuplice base? È la base dell’occhio, la base dell’orecchio, la
base del naso, la base della lingua, la base del corpo e la base della mente.

Che cos’è nome-e-forma?\footnote{“Nome-e-forma” è contemporaneamente il
  percepire e quel che è percepito, esperito e riconosciuto (“nominato”). È
  l’immaginato insieme alla materia, che insieme costituiscono la forma
  individualizzata e soggettivamente determinata di un oggetto. Nei sutta, però,
  esso non include la coscienza, grazie alla quale ciò è reso possibile. La
  successiva letteratura include la coscienza all’interno del “nome”, creando
  così le basi per un’opposizione tra mente e materia priva di riscontri nel
  Canone.} Quel che è chiamato nome comprende la sensazione, la percezione, la
scelta,\footnote{Altre traduzioni di \emph{cetanā} (qui reso con “scelta”) sono
  “volizione” e “intenzione”.} il contatto e l’attenzione; quel che è chiamato
forma comprende i quattro grandi elementi e qualsiasi forma da essi derivata
mediante l’attaccamento, perciò questo nome e questa forma sono ciò che viene
chiamato nome-e-forma.

Che cos’è la coscienza? Sei sono i corpi della coscienza: coscienza visiva,
coscienza uditiva, coscienza olfattiva, coscienza gustativa, coscienza corporea
e coscienza mentale.

Che cosa sono le formazioni? Tre sono le formazioni: formazioni corporee,
formazioni verbali e formazioni mentali.

Che cos’è l’ignoranza? È la nescienza in relazione alla sofferenza, all’origine
della sofferenza, alla cessazione della sofferenza e al sentiero che conduce
alla cessazione della sofferenza».

\suttaRef{S. 12:2}

«In dipendenza dall’occhio e dalle forme visibili, sorge la coscienza visiva. La
coincidenza dei tre è data dal contatto. Con il contatto quale condizione, la
sensazione. Con la sensazione quale condizione, la brama. Ecco come ha origine
la sofferenza (e così con l’orecchio … la mente)».

\suttaRef{S. 12:43}

«Infiammato dalla brama, reso furente dall’odio, confuso dall’illusione, da essi
trasceso e con la mente ossessionata, un uomo sceglie per la propria afflizione,
per l’afflizione degli altri, per l’afflizione propria e per quella degli altri,
e sperimenta dolore e afflizione».

\suttaRef{A. 3:55}

«Gli esseri sono possessori delle loro azioni, eredi delle loro azioni, hanno le
loro azioni come progenitori, le azioni come loro congiunti (e responsabilità),
le azioni come loro rifugio, sono le azioni che differenziano gli esseri in
inferiori e superiori».

\suttaRef{M. 135}

«Che cosa sono vecchie azioni? Occhio, orecchio, naso, lingua, corpo sono
vecchie azioni (già) determinate e scelte che devono essere sperimentate per
essere viste. Che cosa sono le nuove azioni? È qualsiasi azione che si compia
ora, sia per mezzo del corpo, della parola o della mente».

\suttaRef{S. 35:145}

«Questo corpo non appartiene a voi o ad altri, ma è azione passata (già)
determinata e scelta che deve essere sperimentata per essere vista».

\suttaRef{S. 12:37}

«Chiamo azione la scelta. È scegliendo che un uomo agisce con il corpo, con la
parola e con la mente. Ci sono azioni la cui maturazione sarà sperimentata
nell’inferno, nel regno degli spiriti, in un utero animale, tra gli esseri umani
e nei mondi paradisiaci. Le azioni maturano in tre modi; possono maturare qui e
ora, ricomparendo, oppure, al di là di questo, in un qualche altro processo
vitale».

\suttaRef{A. 6:63}

«Le azioni compiute dietro spinta della brama, dell’odio o dell’illusione
maturano ovunque sia generato un sé individuale, e ovunque queste azioni
maturino, là viene sperimentata la loro maturazione, sia qui e ora o in un
successivo ricomparire oppure in un qualche altro processo vitale».

\suttaRef{A. 3:33}

«Ci sono quattro cose incommensurabili, che non possono essere misurate, e un
tentativo di concepirle condurrebbe a frustrazione e follia. Quali quattro? Esse
sono la sfera d’influsso dei Buddha, la sfera d’influsso di chi ha raggiunto i
jhāna, la maturazione delle azioni e la stima del mondo».

\suttaRef{A. 4:77}

«Il mondo è condotto dalla mente».

\suttaRef{S. 1:72}

\section*{La Verità della Cessazione della Sofferenza}

\narrator{Secondo narratore.} Ecco alcune definizioni dettagliate della Terza
Nobile Verità.

\voice{Prima voce.} «Quello non giunge all’esistenza quando non c’è questo,
quello cessa con la cessazione di questo».

\suttaRef{M. 38}

«In dipendenza dall’occhio e dalle forme visibili, sorge la coscienza visiva. La
coincidenza dei tre è data dal contatto. Con il contatto quale condizione, là
sorge quel che è sentito come piacevole, o doloroso, oppure
né-doloroso-né-piacevole. Se, sperimentando il contatto con una sensazione
piacevole, non la si assapora, né le si dà il benvenuto e nemmeno la si
accoglie, e se non vi è più la soggiacente tendenza di fondo a provare desiderio
per essa. – Se, sperimentando il contatto con una sensazione dolorosa, non si
prova dispiacere, né ci si lamenta e nemmeno ci si batte il petto, si piange e
ci si sconvolge, e se non vi è più la soggiacente tendenza di fondo a resistere
a essa. – Se, sperimentando il contatto con una sensazione né-dolorosa-né-
piacevole, si comprende, così com’essa è in realtà, il sorgere, lo scomparire,
la gratificazione, la pericolosa inadeguatezza e la via di fuga nel caso di
quella sensazione, e se non vi è più la soggiacente tendenza di fondo a
ignorarla. – È allora in verità che si può porre fine alla sofferenza mediante
l’abbandono della soggiacente tendenza di fondo a provare desiderio per la
sensazione piacevole, mediante l’eliminazione della soggiacente tendenza di
fondo a resistere alla sensazione dolorosa e mediante l’abolizione della
soggiacente tendenza di fondo a ignorare la sensazione né-dolorosa-né-piacevole:
tutto questo è possibile».

\suttaRef{M. 148}

% FIXME missing »

«Quando la brama, l’odio e l’illusione sono abbandonate, un uomo non sceglie per
la propria afflizione, per l’afflizione degli altri, per l’afflizione propria e
per quella degli altri. In questo modo giunge in essere l’estinzione qui e ora
che, senza indugio, invita all’investigazione e conduce verso l’interiorità, e
che è [direttamente] sperimentabile dal saggio.

\suttaRef{A. 3:55}

«Le azioni compiute sulla base della non-brama, del nonodio e della
non-illusione, sono compiute quando la brama, l’odio e l’illusione sono
scomparse, sono state abbandonate, [eliminate,] recise alla radice, rese come un
ceppo di palma, abolite e non più soggette a sorgere in futuro».

\suttaRef{A. 3:33}

«Gli stati privi di forma sono più sereni degli stati dotati di forma, la
cessazione è più serena degli stati privi di forma».\footnote{È necessario
  evitare di confondere il “privo di forma” (\emph{arūpa}), che è un tipo di
  esistenza (\emph{bhava}), con “non-formato” (o “incondizionato”,
  \emph{asaṅkhata}), che è ciò che non ha formazioni (o condizioni,
  \emph{saṅkhāra}). Quest’ultimo è un termine per il Nibbāna. Il “privo di
  forma” è sempre condizionato.}

\suttaRef{Iti. 73}

«C’è quella base (esterna) ove non (c’è) terra, acqua, fuoco, aria, e neanche
una base consistente dell’infinità dello spazio, una base consistente
dell’infinità della coscienza, una base consistente del nulla-è, una base
consistente della né-percezione-né-non-percezione, e neanche questo mondo, un
altro mondo, la luna o il sole. E questo io lo chiamo né venire, né andare, né
stare, né morire, né ricomparire. Non ha base, non ha evoluzione, non ha
supporto. È la fine della sofferenza».

\begin{quote}
Il Non-Condizionato è difficile da vedere, \\
non è facile vedere la Verità. \\
Per conoscere bisogna togliere il velo alla brama, \\
per vedere bisogna essersi affrancati dal possesso.
\end{quote}

«C’è un non-nato, un non-condotto-all’esistenza, un non-fatto, un non-formato.
Se non ci fosse, non si potrebbe far conoscere una via d’uscita a chi è nato,
condotto all’esistenza, fatto, formato. Siccome c’è un non-nato, un
non-condotto-all’esistenza, un non-fatto, un non-formato è perciò possibile
descrivere una via d’uscita a chi è nato, condotto all’esistenza, fatto,
formato».

\suttaRef{Ud. 8:1-3}

«Ci sono due elementi del Nibbāna. Quali due? C’è un elemento del Nibbāna con
residuo del passato attaccamento e l’elemento del Nibbāna senza residuo del
passato attaccamento. Qual è l’elemento del Nibbāna con residuo del passato
attaccamento? Ecco un bhikkhu che è un Arahant con le contaminazioni esaurite,
che ha vissuto la vita [santa], che ha fatto quel che doveva essere fatto, che
ha poggiato il fardello, che ha raggiunto lo scopo supremo, che ha distrutto le
catene dell’esistenza e che si è completamente liberato mediante la conoscenza
finale. Restano le sue cinque facoltà sensoriali, in ragione della cui presenza
egli ancora incontra il piacevole e lo spiacevole, ancora sperimenta il
piacevole e il doloroso. È in lui l’esaurimento della brama, dell’odio e
dell’illusione che è chiamato elemento del Nibbāna con residuo del passato
attaccamento. E qual è l’elemento del Nibbāna senza residuo del passato
attaccamento? Ecco un bhikkhu che è un Arahant [con le contaminazioni esaurite,]
che ha vissuto la vita [santa] … che si è completamente liberato mediante la
conoscenza finale. Tutte le sensazioni che in lui sono provate, poiché egli non
le assapora, si raffreddano qui, proprio in questa vita: questo è chiamato
elemento del Nibbāna senza residuo del passato attaccamento».

\suttaRef{Iti. 44}

«Quel che è l’esaurimento della brama, dell’odio e dell’illusione è chiamato
Nibbāna».

\suttaRef{S. 38:1}

\begin{quote}
«Proprio come una fiamma soffiata via dalla forza del vento, \\
Upasīva», disse il Beato, \\
«si spegne, e come tale non può più essere designata, \\
così pure il Saggio Silenzioso, essendosi liberato dal nome-corpo, \\
si spegne, e come tale non può più essere designato».

«Quando allora egli se n’è così andato, non esiste più? \\
Oppure egli è reso immortale per l’eternità? \\
Piaccia al Saggio chiarirmi questo punto, \\
poiché si tratta d’una condizione che egli ha compreso».

«Non c’è modo di definire chi se n’è così andato, \\
Upasīva», disse il Beato, \\
«e nulla di lui si può dire, \\
perché quando tutte le idee sono state abolite, \\
sono stati aboliti anche tutti i modi di dire».
\end{quote}

\suttaRef{Sn. 5:7}

\section*{La Verità del Sentiero}

\narrator{Secondo narratore.} La Quarta Nobile Verità è il Nobile Ottuplice
Sentiero. Ognuna delle sue otto componenti necessita di essere definita
separatamente.

\section*{Retta visione}

\voice{Prima voce.} «Proprio come l’alba annuncia e prevede il sorgere del sole,
così la retta visione annuncia e prevede la penetrazione delle Quattro Nobili
Verità in accordo con quel che esse in realtà sono».

\suttaRef{S. 56:37}

\narrator{Secondo narratore.} La retta visione ha molte sfaccettature.
Osserviamole una per una, iniziando con la “maturazione dell’azione” che, in
certe forme e con alcune riserve, è pure condivisa con altri insegnamenti.

\voice{Prima voce.} «Viene prima la retta visione.\footnote{Fino ad ora sono
  stati offerti solo dettagli analitici delle prime tre Nobili Verità. Qui
  incontreremo mere descrizioni che ci aiutano a comprenderle.} Come? Si
comprende l’errata visione come errata visione e si comprende la retta visione
come retta visione. Che cos’è l’errata visione? La visione che non c’è niente di
dato, offerto o sacrificato,\footnote{Ciò significa che in queste azioni non c’è
  significato morale (Nyp.).} che non c’è frutto o maturazione delle buone e
delle cattive azioni, non c’è questo mondo né un altro mondo, non c’è madre né
padre, non ci sono esseri che compaiono, non ci sono monaci buoni e virtuosi e
brāhmaṇa che hanno realizzato se stessi mediante conoscenza diretta e dichiarato
[com’è] questo mondo e l’altro mondo: questa è errata visione».

«Che cos’è la retta visione? Ci sono due tipi di retta visione: c’è quella
affetta da contaminazioni, che porta meriti e matura negli essenziali
dell’esistenza. E c’è la retta visione degli Esseri Nobili priva di
contaminazioni, che è sovramondana ed è un fattore del Sentiero. Che cos’è la
retta visione affetta da contaminazioni? La visione che c’è quel che è dato,
offerto o sacrificato, che c’è frutto e maturazione delle buone e delle cattive
azioni, e che c’è questo mondo e un altro mondo, madre e padre, ed esseri che
compaiono, e monaci buoni e virtuosi e brāhmaṇa che hanno realizzato se stessi
mediante conoscenza diretta e dichiarato [com’è] questo mondo e l’altro mondo:
questa è retta visione affetta da contaminazioni che porta meriti e matura negli
essenziali dell’esistenza. E che cos’è la retta visione degli Esseri Nobili?
Ogni comprensione, facoltà di comprensione, potere di comprensione, fattore
dell’Illuminazione d’investigazione degli stati, retta visione come fattore del
Sentiero, in chi ha la mente nobilitata e pura, possiede il Sentiero e lo
mantiene in essere: questa è la retta visione degli Esseri Nobili priva di
contaminazioni, che è sovramondana ed è un fattore del Sentiero».

\suttaRef{M. 117}

\narrator{Secondo narratore.} Ancora, è la retta visione della genesi
interdipendente – la struttura basilare dell’“insegnamento peculiare ai Buddha”
e la prima delle nuove scoperte fatte dal Buddha. Niente può sorgere da sé,
senza il supporto di altre cose dalle quali l’esistenza di una cosa dipende.

\voice{Seconda voce.}

\begin{quote}
Il Perfetto ha dichiarato la causa \\
del sorgere delle cose condizionate, \\
e anche quel che conduce alla loro cessazione: \\
questa è la dottrina predicata dal Grande Monaco.
\end{quote}

«La pura, immacolata visione del Dhamma sorse in lui: tutto quel che sorge deve
cessare».

\suttaRef{Vin. Mv. 1:23}

\voice{Prima voce.} «Quello giunge all’esistenza quando c’è questo, quello sorge
con il sorgere di questo. Quello non giunge all’esistenza quando non c’è questo,
quello cessa con la cessazione di questo».

\suttaRef{M. 38}

«Chi vede la genesi interdipendente vede il Dhamma, chi vede il Dhamma vede la
genesi interdipendente».

\suttaRef{M. 28}

«Che gli Esseri Perfetti compaiano o no, questo elemento resta, questa struttura
delle cose (dei fenomeni), questa certezza nelle cose, ossia: una specifica
condizionalità. Un Perfetto l’ha scoperta».

\suttaRef{S. 12:20}

«Se non ci fosse affatto nascita, di nulla, da nessuna parte … non essendoci
nascita, con la cessazione della nascita, potrebbero essere descritte la
vecchiaia e la morte?». – «No, Signore». – «Di conseguenza, questa è una
ragione, una fonte, un’origine, una condizione per la vecchiaia e la morte». (E
così via, con le altre coppie della formula della genesi interdipendente.)

\suttaRef{D. 15}

«Signore, “retta visione, retta visione” è stato detto. A che cosa si riferisce
la “retta visione”?». – «Di solito, Kaccāyana, questo mondo dipende dal dualismo
dell’esistenza e della non-esistenza. Quando però uno vede l’origine del mondo
com’è nella realtà con retta comprensione, per lui non c’è niente della
(cosiddetta) non-esistenza nel mondo, e quando egli vede la cessazione del mondo
com’è nella realtà con retta comprensione, per lui non c’è niente della
(cosiddetta) esistenza nel mondo».

«Di solito il mondo è incatenato da pregiudizi, attaccamenti e ostinazioni, ma
per uno come costui (che ha retta visione) – il quale, invece di accogliere
pregiudizi, invece di aggrapparsi e invece di decidere in relazione a “me
stesso” con questi pregiudizi, con quest’aggrapparsi e con queste decisioni
legati alla soggiacente tendenza di fondo a ostinarsi – non ci sono dubbi o
incertezze sul fatto che quel che sorge è solo sofferenza che sorge, e che quel
che cessa è solo sofferenza che cessa, e in questo la sua conoscenza è
indipendente dagli altri. A questo si riferisce “retta visione”. “(Un) tutto
esiste è un estremo”, “(un) tutto non esiste” è l’altro estremo. Invece di
ricorrere a uno di questi due estremi, un Perfetto espone il Dhamma mediante la
Via di Mezzo: “È con l’ignoranza quale condizione che le formazioni [mentali]
giungono all’esistenza; con le formazioni [mentali] quale condizione, la
coscienza; con la coscienza…” (e così via sia con il sorgere sia con il
cessare)».

\suttaRef{S. 12:15}

«Se si afferma: “Chi produce (sofferenza), (la) prova: essendo egli fin
dall’inizio, è lui stesso a produrre la sua sofferenza”, allora si giunge
all’eternalismo. Se però si afferma: “Uno produce (sofferenza), un altro (la)
prova: essendo egli schiacciato dalla sensazione, la sua sofferenza è prodotta
da un altro”, allora si giunge al nichilismo. Invece di ricorrere a uno di
questi due estremi, un Perfetto espone il Dhamma mediante la Via di Mezzo: …
(ossia, mediante la genesi interdipendente e la cessazione)».

\suttaRef{S. 12:17}

«Tutti gli esseri sono mantenuti dal nutrimento».

\suttaRef{D. 33; A. 10:27, 28; Khp. 2}

«Che cos’è il nutrimento? Ci sono questi quattro generi di nutrimento per
mantenere gli esseri che già esistono, e per soccorrere quelli che cercano di
tornare a esistere: essi sono il cibo fisico come nutrimento grossolano o
sottile, il secondo è il contatto, la scelta è il terzo e la coscienza è il
quarto».

\suttaRef{S. 12:63; M. 38}

\narrator{Secondo narratore.} La stessa essenza della retta visione è, tuttavia,
la comprensione delle Quattro Nobili Verità, la quale abbraccia la genesi
interdipendente e costituisce l’“insegnamento peculiare dei Buddha”. Esse
costituiscono l’oggetto del Primo Sermone.

\voice{Prima voce.} «Che cos’è la retta visione? È la conoscenza della
sofferenza, dell’origine della sofferenza, della cessazione della sofferenza e
del Sentiero che conduce alla cessazione della sofferenza: questa è detta retta
visione».

\suttaRef{S. 45:8; D. 22}

(I) «“Quattro serpenti velenosi” è un nome per i quattro grandi elementi (terra,
acqua, fuoco e aria)».

\suttaRef{S. 35:197}

\begin{quote}
La forma è come un grumo di schiuma, \\
la sensazione è come una bolla d’acqua, \\
la percezione anche è come un miraggio, \\
le formazioni [mentali] come il tronco di un banano.\footnote{Il tronco di un banano è fatto solo d’un involucro, è privo di nucleo.} \\
E la coscienza, manifestazione dei figli di Āditi,\footnote{NDT. L’espressione rinvia ai figli del Dio Sole (Āditi), dodici come i mesi dell’anno nel \emph{Bhāgavata Purāṇa}, che si manifestano appunto in modo via via differente.} \\
altro non è che un gioco di prestigio.
\end{quote}

\suttaRef{S. 22:95}

\label{pag254}%
«Le sei basi, di per se stesse, possono essere definite come un villaggio vuoto,
perché se un uomo saggio le investiga quali occhio, orecchio, naso, lingua,
corpo o mente, esse appaiono come cavità, vuote e vacue. Le sei basi esterne
possono essere definite come briganti che fanno incursioni in un villaggio,
perché l’occhio è assillato da forme gradevoli e sgradevoli, l’orecchio da suoni
siffatti, il naso da odori siffatti, la lingua da sapori siffatti, il corpo da
oggetti tangibili siffatti e la mente da oggetti mentali siffatti».

\suttaRef{S. 35:197}

\begin{quote}
(II) Nel mondo vedo questa generazione tormentata \\
dalla brama per l’esistenza, \\
miserevoli uomini che farfugliano di fronte alla Morte, \\
ancora bramosi, speranzosi per un qualche tipo di esistenza. \\
Guardate come fremono per quel che pretendono essere “mio”, \\
come pesci in una pozzanghera che si sta prosciugando.
\end{quote}

\suttaRef{Sn. 4:2}

(III) «Questa è (la più alta) serenità, questa è (la meta) superiore (a tutto),
ossia è la pacificazione di tutte le formazioni [mentali], l’abbandono di tutti
gli essenziali dell’esistenza, l’esaurimento della brama, la cessazione, il
Nibbāna».

\suttaRef{A. 10:60}

\begin{quote}
(IV) La più grande delle acquisizioni (mondane) è la ricchezza, \\
il Nibbāna è la più grande beatitudine. \\
Il Nobile Ottuplice Sentiero è il sentiero migliore, \\
per arrivare al sicuro a Ciò che Non Muore.
\end{quote}

\suttaRef{M. 75}

\narrator{Secondo narratore.} È di nuovo la retta visione delle tre
caratteristiche universali dell’impermanenza, della sofferenza (o insicurezza) e
del non-sé, che esprime globalmente quel che la genesi interdipendente esprime
strutturalmente. Esse costituiscono l’oggetto del Secondo Sermone.

\voice{Prima voce.} «Tre sono le caratteristiche formate di ciò che è
formato:\footnote{“Formato” è \emph{saṅkhata}, tradotto anche con “composto” o
  “condizionato”; “non-formato” è \emph{asaṅkhata}, tradotto anche con
  “non-composto” o “incondizionato”. Quest’ultimo è identificato come Nibbāna
  (Nyp.)} il sorgere è evidente, il declino è evidente e l’alterazione di ciò
che è presente è evidente. Tre sono le caratteristiche non-formate di ciò che è
non-formato: il non-sorgere è evidente, il nondeclino è evidente e la
non-alterazione è evidente».

\suttaRef{A. 3:47}

«Allorché si comprende come forma, sensazione, percezione, formazioni [mentali]
e coscienza (e come l’occhio, ecc.) sono impermanenti, in ciò si possiede retta
visione».

\suttaRef{S. 22:51; 35:155}

«Tutto è impermanente. E che cos’è il tutto che è impermanente? L’occhio è
impermanente, le forme sono impermanenti, la coscienza visiva è impermanente …
il contatto con l’occhio, qualsiasi cosa sia sentita come piacevole, dolorosa o
né-dolorosa-né-piacevole nata dal contatto con l’occhio è impermanente.
L’orecchio, ecc. … Il naso, ecc. … La lingua, ecc. … Il corpo, ecc. … La mente è
impermanente, gli oggetti mentali … la coscienza mentale … il contatto mentale …
qualsiasi cosa sia sentita … nata dal contatto mentale è impermanente».

\suttaRef{S. 35:43}

«Quel che è impermanente è sofferenza, quel che è sofferenza è non-sé».

\suttaRef{S. 35:1; 22:46}

«Che un Perfetto compaia o no, questo elemento resta, questa struttura delle
cose (dei fenomeni), questa certezza nelle cose: tutte le formazioni sono
impermanenti, tutte le formazioni sono sofferenza, tutte le cose sono non-sé».

\suttaRef{A. 3:134}

«Bhikkhu, io non disputo con il mondo: il mondo disputa con me. Chi proclama il
Dhamma non disputa con nessuno nel mondo. Quello che gli uomini saggi del mondo
dicono non esserci, anche io dico non esserci. E quel che gli uomini saggi del
mondo dicono esserci, anche io dico esserci. Gli uomini saggi del mondo dicono
che non c’è forma permanente, durevole, eterna che non sia soggetta al
cambiamento, e anche io dico che non ce n’è alcuna. (E così anche degli altri
quattro aggregati.) Gli uomini saggi del mondo dicono che c’è una forma
impermanente, che è sofferenza e soggetta al cambiamento, e anche io dico che
c’è. (E così con gli altri quattro.)».

\suttaRef{S. 22:94}

«Questo corpo è impermanente, è formato ed è sorto in dipendenza».

\suttaRef{S. 36:7}

«Per un uomo ignorante e ordinario sarebbe meglio trattare come se fosse un sé
questo corpo, che è costruito sulla base di quattro grandi elementi, invece che
la mente.\footnote{\emph{citta}: mente, mentalità, cognizione} Perché? Perché
questo corpo può durare un anno, due anni … cento anni. Quel che però è chiamato
“mente” e “coscienza” sorge e cessa in vari modi notte e giorno, proprio come
una scimmia che attraversa una foresta passando di ramo in ramo e, lasciandone
uno, ne afferra un altro».

\suttaRef{S. 12:61}

«L’atto del donare è fruttuoso … tuttavia è ancor più fruttuoso prendere rifugio
con cuore fiducioso nel Buddha, nel Dhamma e nel Saṅgha, e prendere i cinque
precetti della virtù … Questo è fruttuoso … tuttavia è ancor più fruttuoso
mantenere in essere la gentilezza amorevole anche solo per il tempo di mungere
una mucca … Questo è fruttuoso … tuttavia è ancor più fruttuoso mantenere in
essere la percezione dell’impermanenza anche solo per il tempo di far schioccare
le dita».

\suttaRef{A. 9:20 (condensato)}

«Chiunque apprezza l’occhio, apprezza la sofferenza, e non sarà libero dalla
sofferenza, questo dico».

\suttaRef{S. 35:19}

«Che cos’è la maturazione della sofferenza? Quando qualcuno è sopraffatto e la
sua mente è ossessionata dalla sofferenza, o si addolora e si lamenta e,
battendosi il petto, piange e diviene sconvolto, oppure intraprende una ricerca
esteriormente: “C’è qualcuno che sa una parola, due parole, per la cessazione
della sofferenza?”. Dico che la sofferenza matura o nella confusione o nella
ricerca».

\suttaRef{A. 6:63}

«Che qualcuno possa vedere le formazioni come piacere … oppure il Nibbāna come
sofferenza, e abbia una predilezione conforme [alla Verità], questo non è
possibile. (L’opposto però) è possibile».

\suttaRef{A. 6:99}

«Qualsiasi forma, sensazione, percezione, formazione e coscienza, di qualsiasi
genere, passata, futura o presente, interna o esterna, grossolana o sottile,
inferiore o superiore, lontana o vicina, dovrebbe essere considerata come
realmente è in questo modo: “Questo non è mio, questo non è quel che io sono,
questo non è il mio sé”».

\suttaRef{S. 22:59}

«Nel mondo mediante cui si percepisce il mondo e si concepiscono concetti a
proposito del mondo, ciò è chiamato “il mondo” nella Disciplina degli Esseri
Nobili. E con che cosa si fa tutto questo nel mondo? Con l’occhio, l’orecchio,
il naso, la lingua, il corpo e la mente».

\suttaRef{S. 35:116}

«Si va logorando (\emph{lujjati}), ecco perché è chiamato “il mondo”
(\emph{loka})».

\suttaRef{S. 35:82}

% FIXME missing »

«“Mondo vuoto, mondo vuoto” si dice, Signore. In quale modo si dice “mondo
vuoto”? – «È perché è vuoto del sé e della proprietà del sé che si dice “mondo
vuoto” Ānanda. E che cosa è vuoto del sé e della proprietà del sé? L’occhio … le
forme … la coscienza visiva … il contatto visivo … qualsiasi sensazione … nata
dal contatto visivo … L’orecchio, ecc. … Il naso, ecc. … La lingua, ecc. … Il
corpo, ecc. … La mente, ecc. … qualsiasi sensazione piacevole o dolorosa oppure
né-piacevole-né-dolorosa nata dal contatto mentale è vuota del sé e della
proprietà del sé».

\suttaRef{S. 35:85}

«Quando un bhikkhu dimora molto con la sua mente fortificata dalla percezione
dell’impermanenza, la sua mente retrocede, si ritrae e indietreggia dal
guadagno, dall’onore e dalla fama invece di avvicinarsi ad essi, come la piuma
di un gallo o un brandello di tendine gettati su un fuoco retrocedono, si
ritraggono e indietreggiano invece di avvicinarsi ad esso … Quando egli dimora
molto con la sua mente fortificata dalla percezione della sofferenza
nell’impermanenza, si stabilisce in lui una vivida percezione di timore verso la
rilassatezza, l’indolenza, la pigrizia, la negligenza, la mancanza di dedizione
e di riflessione, come se si trovasse al cospetto di un assassino con un’arma
pronta a colpirlo … Quando egli dimora molto con la sua mente fortificata dalla
percezione del non-sé nella sofferenza, la sua mente si libera di quelle
presunzioni che considerano come “io” e “mio” questo corpo con la sua coscienza
e tutti i segni esteriori».

\suttaRef{A. 7:46}

\narrator{Secondo narratore.} La razionalizzata “teoria del sé” che,
indipendentemente dalla forma che assume, è chiamata «sia un’opinione sia una
catena», si fonda su una sottile distorsione di fondo nell’atto del percepire,
la «presunzione “io sono”», che è «una catena, ma non un’opinione». Le teorie
del sé possono o non possono essere formulate, ma se lo sono, non è possibile
descriverle in modo specifico senza far riferimento ai cinque aggregati. Per
questa ragione esse possono essere ricondotte, quando descritte, a uno dei tipi
di quel che è chiamata “opinione della
personificazione”,\footnote{“Personificazione”: \emph{sakkāya} = \emph{sa}
  (“esistente” o “proprio”) più \emph{kāya} (corpo). L’identificazione del sé
  (\emph{attā}) con uno o più dei cinque aggregati costituisce perciò una
  “personificazione” di quel sé, e ciò fonda un’errata visione. Si noti che
  \emph{sakkāyadiṭṭhi è} di solito più tradotto con “opinione dell’io” (Nyp.).}
che è esposta schematicamente. Tutto ciò è abbandonato da Chi è Entrato nella
Corrente, anche se la presunzione “io sono” non lo è.

\label{pag259}%
\voice{Prima voce.} «Com’è che perviene a esistere l’opinione della
personificazione?». – «Un uomo ignorante e ordinario che non ha considerazione
per gli Esseri Nobili e non è versato con il loro Dhamma e Disciplina … vede la
forma come sé o il sé come dotato di una forma, o la forma come nel sé o il sé
come nella forma. (E così via con ognuno degli altri quattro aggregati:
sensazione, percezione, formazioni [mentali] e coscienza.) Un ben istruito
nobile discepolo non lo fa».

\suttaRef{M. 44; M. 109}

«L’uomo ignorante e ordinario che non ha considerazione per gli Esseri Nobili …
presta un’irragionevole (acritica) attenzione a queste cose: “In passato io
esistevo? Non esistevo io in passato? Che cos’ero io in passato? Com’ero io in
passato? Essendo stato quello, che cos’ero io in passato? Esisterò io in futuro?
Non esisterò io in futuro? Che cosa sarò io in futuro? Come sarò io in futuro?
Essendo stato quello, che cosa sarò io in futuro?”. Oppure così si domanda in
relazione a se stesso, ora, in quanto sorto nel presente: “Io sono? Io non sono?
Che cosa sono io? Come sono io? Da dove è venuto questo essere? Dov’è
diretto?”».

«Allorché egli presta un’irragionevole attenzione a queste cose, allora uno dei
sei tipi di opinione del sé sorge in lui come vera e fondata: “il mio sé esiste”
o “il mio sé non esiste”, “io percepisco il sé con il sé” o “io percepisco il
non-sé con il sé”, “io percepisco il sé con il non-sé” oppure altre opinioni
quali “questo è il mio sé che parla, ha sensazioni e sperimenta qui o là la
maturazione delle buone e delle cattive azioni, ma questo mio sé è permanente,
durevole, non soggetto al cambiamento, e durerà in eterno”. Questo ambito di
opinioni è chiamato cespuglio di opinioni, bosco di opinioni, contorsione di
opinioni, tentennamento di opinioni, catena di opinioni. L’uomo ignorante e
ordinario legato dalla catena di opinioni non è libero dalla nascita, dalla
vecchiaia e dalla morte, dall’afflizione, dal lamento, dal dolore, dal
dispiacere e dalla disperazione: egli non si è liberato dalla sofferenza, dico».

\suttaRef{M. 2}

«Bhikkhu, ci sono due tipi di (errata) visione, e quando le divinità e gli
esseri umani sono nella loro morsa, alcuni restano indietro e altri vanno troppo
oltre. Sono solo quelli con [retta] visione che vedono. Com’è che alcuni restano
indietro? Divinità ed esseri umani amano l’esistenza, si deliziano
dell’esistenza, apprezzano l’esistenza. Quando il Dhamma viene loro esposto per
la fine dell’esistenza, il loro cuore non viene raggiunto né acquisisce fiducia,
fermezza e decisione. È così che alcuni restano indietro. E com’è che alcuni
vanno troppo oltre? Alcuni si vergognano, si sentono umiliati e disgustati da
questa stessa esistenza, e guardano più oltre in direzione della non-esistenza
in questo modo: “Signori, quando alla dissoluzione del corpo questo sé è
eliminato, annullato e perciò dopo la morte non esiste più, quella è la serenità
maggiore, la meta superiore a tutte le altre, questa è la realtà”. È così che
alcuni vanno troppo oltre. E com’è che quelli con [retta] visione vedono? Un
bhikkhu vede qualsiasi cosa giunta all’esistenza come giunta all’esistenza.
Vedendo in questo modo egli si è messo sulla strada del distacco per essa, del
disincanto e della cessazione della brama per essa. È così che uno con la
[retta] visione vede».

\suttaRef{Iti. 49}

«Bhikkhu, i possedimenti che uno può possedere che siano permanenti, perenni …
Vedete possedimenti di questo genere?». – «No. Signore». – «… Una teoria del sé,
alla quale ci si attacca ovunque ci si possa attaccare, senza che faccia mai
sorgere afflizione e … disperazione in chi ad essa si attacca. Vedete una teoria
del sé di questo genere?». – «No, Signore». – «Un’opinione che sia di supporto,
che si possa prendere quale supporto senza che faccia mai sorgere afflizione e …
disperazione in chi la sceglie quale supporto. Vedete un’opinione di supporto di
questo genere?». – «No, Signore». – «… Bhikkhu, esistendo un sé, esisterebbe
anche una proprietà del sé?». – «Sì, Signore». – «Ed esistendo una proprietà del
sé, esisterebbe anche un sé?». – «Sì, Signore». – «Bhikkhu, essendo sé e
proprietà del sé inafferrabili come veri e fondati, non sarebbe allora questa
opinione: “Questo è il mondo, questo è il sé, dopo la morte io sarò permanente,
perenne, eterno, non soggetto al cambiamento, durerò per l’eternità”
[nient’altro che] la pura perfezione dell’idea di un folle?». – «Come potrebbe
non essere così, Signore? Sarebbe la pura perfezione dell’idea di un folle».

\suttaRef{M. 22}

«Ogni qual volta monaci o brāhmaṇa vedono il sé nelle sue varie forme, tutti
loro vedono i cinque aggregati affetti dall’attaccamento, o uno o l’altro di
essi. L’uomo ignorante e ordinario che non ha considerazione per gli Esseri
Nobili … vede la forma come sé o il sé come dotato di forma, la forma come nel
sé o il sé come nella forma (oppure egli fa la stessa cosa con gli altri quattro
aggregati). Egli ha perciò questo (razionalizzato) modo di pensare ed ha anche
l’attitudine (di fondo) “io sono”. Fino a quando, però, c’è l’attitudine “io
sono” c’è organizzazione delle cinque facoltà sensoriali dell’occhio,
dell’orecchio, del naso, della lingua e del corpo. Poi c’è la mente e ci sono le
idee, e c’è l’elemento dell’ignoranza. Quando un uomo ignorante e ordinario è
toccato dalla sensazione nata dal contatto con l’ignoranza, gli capita di
pensare “io sono” e “io sono questo”, “io sarò” e “io non sarò”, “io sarò dotato
di forma” e “io sarò privo di forma”, “io sarò percettivo” e “io sarò
impercettivo” e “io sarò né percettivo né impercettivo”. Nel caso però di un ben
istruito nobile discepolo, mentre le cinque facoltà sensoriali restano così come
sono, l’ignoranza a riguardo di esse è abbandonata ed è sorta la vera
conoscenza. Con essa non gli capita di pensare “io sono” o … “io sarò né
percettivo né impercettivo”».

\suttaRef{S. 22:47}

\narrator{Secondo narratore.} L’uomo ignorante e ordinario è ignaro della
sottile attitudine di fondo, della soggiacente tendenza o presunzione “io sono”.
Essa, nella percezione di un percetto, lo fa automaticamente e simultaneamente
presumere in termini di “io”, presupponendo una relazione dell’io con il
percetto, come identica con esso o come contenuta all’interno di esso, o come
separata da esso oppure in termini di possesso. Quest’attitudine, questa
concezione, è abbandonata solo con il raggiungimento della condizione di
Arahant, non prima (si veda ad es. M. 1 e M. 49).

\voice{Prima voce.} «“Io sono” è una derivazione, non una non-derivazione. Una
derivazione da che cosa? È una derivazione da forma, sensazione, percezione,
formazioni [mentali] e coscienza».

\suttaRef{S. 22:83}

«Quando ogni monaco o brāhmaṇa con la forma (e il resto) quale mezzo, che è
impermanente, è sofferenza e soggetta al cambiamento, pensa “io sono superiore”,
“io sono uguale” o “io sono inferiore”, che cos’è questo se non cecità rispetto
a quello che in realtà è?».

\suttaRef{S. 22:49}

(Interrogato dagli Anziani, l’Anziano Khemada disse:) «In questi cinque
aggregati affetti dall’attaccamento non vedo alcun sé o proprietà del sé …
tuttavia non sono un Arahant con le contaminazioni esaurite. Al contrario, ho
ancora l’attitudine “io sono” riguardo a questi cinque aggregati affetti
dall’attaccamento sebbene io non pensi “io sono questo” rispetto ad essi … Non
dico “io sono forma”, “io sono sensazione”, “io sono percezione”, “io sono
formazioni [mentali] o “io sono coscienza”, e nemmeno dico “io sono separato
dalla forma … separato dalla coscienza”. Tuttavia ho ancora l’attitudine “io
sono” rispetto ai cinque aggregati affetti dall’attaccamento sebbene io non
pensi “io sono questo” rispetto ad essi. Benché un nobile discepolo possa aver
abbondonato le cinque catene inferiori (si veda sotto), la sua presunzione “io
sono”, il desiderio “io sono”, la soggiacente tendenza “io sono” rispetto ai
cinque aggregati affetti dall’attaccamento non è ancora abolita. In seguito egli
dimora contemplando il sorgere e lo scomparire in questo modo: “Questa è la
forma, questa è la sua origine, questo è il suo scomparire” (e così con gli
altri quattro [aggregati], finché, così facendo, alla fine la sua presunzione
“io sono” giunge a essere abolita».

\suttaRef{S. 22:89}

\narrator{Secondo narratore.} Siamo infine giunti alle dieci catene, che sono
progressivamente spezzate dai quattro stadi della realizzazione.

\voice{Prima voce.} «L’uomo ignorante e ordinario che non ha considerazione per
gli Esseri Nobili … vive con il suo cuore posseduto e reso schiavo dall’opinione
della personificazione, dal dubbio, dall’errata comprensione della virtù e del
dovere,\footnote{Oppure “attaccamento a riti e rituali”
  (\emph{sīlabbata-parāmāsa}) (Nyp.).} dal desiderio sensoriale e dalla
malevolenza, ed egli non vede come sfuggire ad essi quando sorgono. Questi,
quando sono abituali e permangono non sradicati in lui, sono chiamati catene
inferiori».

\suttaRef{M. 64}

«Le cinque catene superiori sono: desiderio per la forma, desiderio per i
fenomeni privi di forma, presunzione (la presunzione “io sono”), agitazione e
ignoranza».

\suttaRef{D. 33}

«Ci sono bhikkhu che, con l’esaurimento delle (prime) tre catene, sono Entrati
nella Corrente, e non sono più soggetti alla perdizione, sono certi della
rettitudine e destinati all’Illuminazione. Ci sono bhikkhu che, con
l’esaurimento delle tre catene e l’attenuazione della brama, dell’odio e
dell’illusione, Tornano una Sola Volta: tornando una sola volta in questo mondo,
porranno fine alla sofferenza. \label{pag263}Ci sono bhikkhu che, con la
distruzione delle cinque catene inferiori, sono [Senza Ritorno, sono] destinati
a ricomparire spontaneamente altrove e lì otterranno il Nibbāna definitivo,
senza tornare nel frattempo da quel mondo. Ci sono bhikkhu che sono Arahant con
le contaminazioni esaurite, che hanno vissuto la vita [santa], che hanno fatto
quel che doveva essere fatto, che hanno poggiato il fardello, che hanno
raggiunto lo scopo supremo, che hanno distrutto le catene dell’esistenza e che
si sono completamente liberati mediante la conoscenza finale».

\suttaRef{M. 118}

«L’esaurimento della brama, dell’odio e dell’illusione è chiamato condizione di
Arahant».

\suttaRef{S. 38:2}

«Quando un bhikkhu viaggia in molti paesi, gente colta di ogni condizione
sociale gli pone delle domande. Persone colte e indagatrici gli chiederanno:
“Che cosa dice il Maestro degli esseri venerabili, che cosa predica?”. Per
rispondere rettamente, potete dire: “Il nostro Maestro predica la rimozione del
desiderio e della brama”. E se vi chiedono: “Rimozione del desiderio e della
brama per che cosa?”, potete rispondere: “Rimozione del desiderio e della brama
per la forma (e così via)”. E se poi vi chiedono: “Quale inadeguatezza
(pericolo) vedete in queste cose?”, potete rispondere: “Quando uno non è privo
di brama, desiderio, amore, sete, febbre e avidità per queste cose, poi, con il
loro cambiamento e alterazione, sorgono in lui l’afflizione, il lamento, il
dolore, il dispiacere e la disperazione”. E se poi vi chiedono: “E quale
vantaggio vedete nel fare in questo modo?”, potete rispondere: “Quando uno è
libero da brama, desiderio, amore, sete, febbre e avidità per forma, sensazione,
percezione, formazioni [mentali] e coscienza, poi, con il loro cambiamento e
alterazione, non sorgono in lui l’afflizione, il lamento, il dolore, il
dispiacere e la disperazione”».

\suttaRef{S. 22:2}

\section*{Retta intenzione}

\narrator{Secondo narratore.} Il riassunto della retta visione è concluso. Il
successivo fattore del Nobile Ottuplice Sentiero è la retta intenzione.

\voice{Prima voce.} «Che cos’è la retta intenzione? È l’intenzione della
rinuncia, l’intenzione della non-malevolenza, l’intenzione della non-crudeltà:
questa è chiamata retta intenzione».

\suttaRef{S. 45:8; D. 22}

«Quando un nobile discepolo ha chiaramente visto con retta comprensione come in
realtà stanno le cose, quanto sia piccola la gratificazione offerta dai desideri
sensoriali e quanto dolore e disperazione essa comporti, e quanto grande sia la
loro inadeguatezza, e consegue la felicità e il piacere dissociati dai desideri
sensoriali e dagli stati non salutari, o qualcosa di ancor più alto di questo,
allora egli non è più interessato ai desideri sensoriali».

\suttaRef{M. 14}

«Anche se dei banditi lo tagliassero a pezzi con una sega da boscaiolo, se nel
suo cuore concepisse odio nei loro riguardi, costui non potrebbe essere
considerato uno che segue il mio insegnamento».

\suttaRef{M. 21}

«Egli non sceglie per la propria afflizione, per l’afflizione degli altri o per
l’afflizione propria e per quella degli altri».

\suttaRef{M. 13}

\section*{Retta parola}

\label{pag265}%
\narrator{Secondo narratore.} Questi due fattori della retta visione e della
retta intenzione insieme costituiscono (quel gruppo dei fattori del Sentiero
chiamato) “saggezza” (\emph{paññā}). Passiamo ora al terzo fattore, la retta
parola.

\voice{Prima voce.} «Che cos’è la retta parola? Astenersi dalla menzogna, dalla
calunnia, dall’insulto e dal pettegolezzo. Questo è la retta parola».

\suttaRef{S. 45:8; D. 22}

«Qualcuno abbandona la menzogna: quando è convocato in giudizio, in una riunione
e alla presenza dei suoi parenti o dell’associazione della quale fa parte o al
cospetto della famiglia reale, se richiesto come testimone in questo modo
“Allora, buon uomo, dicci quello che sai”, se egli non sa dice “io non so”, se
egli sa dice “io so”, se non ha visto dice “io non ho visto”, se ha visto dice
“io ho visto”. Egli non afferma il falso in piena consapevolezza a suo
vantaggio, a vantaggio di un altro o di un qualche meschino fine terreno. Egli
abbandona la calunnia: come chi non ripete altrove quel che ha sentito qui allo
scopo di causare divisioni da questi, né ripete a questi ciò che ha udito
altrove allo scopo di causare divisioni da quelli, ed egli così riunisce ciò che
è diviso, è promotore dell’amicizia, gioisce della concordia, si rallegra nella
concordia, si delizia nella concordia, pronuncia parole che promuovono la
concordia. Egli abbandona l’insulto: pronuncia parole che non suscitano
sofferenza, che sono piacevoli da ascoltare e amabili, che vanno [dritte] al
cuore, che sono educate, desiderate da molti e a molti care. Egli abbandona il
pettegolezzo: come chi dice quel che è opportuno, concreto, buono, e il Dhamma e
la Disciplina, parla con un linguaggio giusto che merita di essere ricordato,
che è motivato, preciso e connesso al bene».

\suttaRef{M. 41}

\section*{Retta azione}

\narrator{Secondo narratore.} Il quarto fattore, retta azione.

\voice{Prima voce.} «Che cos’è la retta azione? Astenersi dall’uccidere esseri
viventi, dal rubare, da una cattiva condotta sessuale. Questo è la retta
azione».

\suttaRef{S. 45:8; D. 22}

«Quando un seguace laico possiede cinque cose, egli vive fiducioso nella propria
casa, e si troverà in paradiso tanto certamente come se fosse stato trascinato
via e messo là». Quali cinque? Si astiene dall’uccidere esseri viventi, dal
prendere ciò che non gli è stato dato, dalla cattiva condotta sessuale, dal dire
il falso e dall’indulgere ai liquori, al vino e alle bevande fermentate».

\suttaRef{A. 5:172-73}

\section*{Retti mezzi di sostentamento}

\narrator{Secondo narratore.} Il quinto fattore, retti mezzi di sostentamento.

\voice{Prima voce.} «Che cosa sono i retti mezzi di sostentamento? Un nobile
discepolo abbandona gli errati mezzi di sostentamento e si guadagna da vivere
mediante retti mezzi di sostentamento».

\suttaRef{S. 45:8; D. 22}

«Manovrare (ingannare), persuadere, alludere, sminuire, mercanteggiare. Questi
sono errati mezzi di sostentamento (per i bhikkhu)».

\suttaRef{M. 117}

«Cinque sono i tipi di commercio che un seguace laico non dovrebbe esercitare.
Quali cinque? Commerciare armi, esseri viventi, carne, liquori e veleni».

\suttaRef{A. 5:177}

\section*{Retto sforzo}

\narrator{Secondo narratore.} Gli ultimi tre fattori, retta parola, retta azione
e retti mezzi di sostentamento, costituiscono (quel gruppo dei fattori del
Sentiero chiamato) “virtù” (\emph{sīla}). Sono noti in quanto stadio preliminare
del Sentiero. Ora si giunge al sesto fattore, il retto sforzo.

\voice{Prima voce.} «Che cos’è il retto sforzo? Un bhikkhu risveglia il
desiderio per il non-sorgere degli stati non salutari non sorti, per cui egli si
sforza, suscita energia, esercita la sua mente, si applica intensamente … Egli
risveglia il desiderio per l’abbandono degli stati non salutari già sorti, per
cui egli si sforza … Egli risveglia il desiderio per il sorgere degli stati
salutari non sorti, per cui egli si sforza … Egli risveglia il desiderio per la
continuazione, la non-corruzione, il rafforzamento, il mantenimento in essere e
il perfezionamento degli stati salutari già sorti, per cui egli si sforza,
suscita energia, esercita la sua mente, si applica intensamente. Questo è
chiamato retto sforzo».

\suttaRef{S. 45:8; D. 22}

\section*{Retta presenza mentale}

\label{pag267}%
\narrator{Secondo narratore.} Siamo giunti al settimo fattore, la retta presenza
mentale.

\voice{Prima voce.} «Che cos’è la retta presenza mentale? Un bhikkhu dimora
contemplando il corpo come corpo, ardente, con piena consapevolezza e presenza
mentale, avendo messo da parte la cupidigia e il rimpianto per il mondo. Dimora
contemplando le sensazioni come sensazioni, ardente … Dimora contemplando la
coscienza come coscienza, ardente … Dimora contemplando gli oggetti mentali come
oggetti mentali, ardente, con piena consapevolezza e presenza mentale, avendo
messo da parte la cupidigia e il rimpianto per il mondo. Questa è la retta
presenza mentale».

\suttaRef{S. 45:8; D. 22}

«Come dimora un bhikkhu contemplando il corpo come corpo? Un bhikkhu, recatosi
nella foresta o ai piedi di un albero o in una stanza vuota, siede a terra. Dopo
aver incrociato le gambe, siede con il corpo eretto e fissa la consapevolezza di
fronte a sé, consapevole inspira, consapevole espira.\footnote{L’esercizio qui
  descritto è l’osservazione mentale, non lo sviluppo corporeo mediante
  controllo del respiro come nell’\emph{hathayoga}. Questo sutta, il
  \emph{Satipaṭṭhāna Sutta,} è attualmente molto noto quale fondamento della
  pratica meditativa. L’argomento di cui tratta, la costituzione della
  consapevolezza, è la pietra angolare dell’insegnamento del Buddha.} Come un
tornitore esperto o come un suo esperto apprendista quando fa una tornitura
lunga sa “io sto facendo una tornitura lunga”, o quando fa una tornitura corta
sa “io sto facendo una tornitura corta”, allo stesso modo quando sta facendo
un’inspirazione lunga un bhikkhu sa “sto facendo un’inspirazione lunga”, o
quando fa un’espirazione lunga sa “sto facendo un’espirazione lunga”; quando sta
facendo un’inspirazione corta, egli sa “sto facendo un’inspirazione corta”, o
quando fa un’espirazione corta, egli sa “sto facendo un’espirazione corta”. Egli
si addestra in questo modo: “Inspirerò sperimentando l’intero corpo (del
respiro)”. Egli si addestra in questo modo: “Espirerò sperimentando l’intero
corpo (del respiro)”. Egli si addestra in questo modo: “Inspirerò
tranquillizzando la formazione corporea (le funzioni corporee)”. Egli si
addestra in questo modo: “Espirerò tranquillizzando la formazione corporea (le
funzioni corporee)”».\footnote{Secondo il Commentario, “sperimentare l’intero
  corpo (del respiro)” significa essere del tutto consapevole dell’intera
  inspirazione e dell’intera espirazione. “Tranquillizzare la formazione
  corporea” significa rendere il respiro sempre più sottile e calmo (BB).}

«Egli dimora contemplando il corpo come corpo in questo modo in se stesso, o
esternamente, o in se stesso ed esternamente».\footnote{Secondo il Commentario,
  “esternamente” significa il corpo di qualcun altro, ecc. (ma potrebbe
  riferirsi anche alla pura oggettività vista nel proprio corpo); questo primo
  paragrafo enfatizza la concentrazione. Il secondo, sul sorgere e sul cessare
  (decadimento) si riferisce alla visione profonda (retta visione). Il terzo
  descrive la piena consapevolezza in chi ha raggiunto il traguardo finale.}

«Oppure egli contempla nel corpo i fattori della sua origine, o i fattori del
suo decadimento, o i fattori della sua origine e del suo decadimento».

«Oppure la consapevolezza che “c’è un corpo” si fonda in lui nella misura di
mera conoscenza e rammemorazione di essa mentre egli dimora indipendente, senza
attaccarsi a nulla nel mondo».

«Così un bhikkhu dimora contemplando il corpo come corpo».

«Ancora, quando cammina, un bhikkhu sa “sto camminando” o, quando è in piedi, sa
“sto in piedi” o, quando è seduto, sa “sto seduto” oppure, quando giace, sa “sto
giacendo”. In qualsiasi posizione si trovi il suo corpo, egli sa che è in quella
posizione».

«Egli dimora contemplando il corpo come corpo … esternamente».

«Oppure, anche, egli contempla … i fattori della sua origine e i fattori del suo
decadimento».

«Oppure, anche, la consapevolezza … senza attaccarsi a nulla nel mondo».

«È pure così che un bhikkhu dimora contemplando il corpo come corpo».

«Ancora, un bhikkhu è del tutto consapevole quando si muove avanti e indietro,
quando guarda avanti e lontano, quando piega ed estende gli arti, quando indossa
la veste superiore fatta di toppe, la ciotola e le altre vesti, quando mangia,
quando beve, quando mastica, quando assapora, quando evacua l’intestino e urina,
ed ha piena consapevolezza e presenza mentale quando cammina, quando sta in
piedi, quando sta seduto, quando va a dormire, quando si sveglia, parla e
mantiene il silenzio».

«Egli dimora contemplando …».

«È pure così che un bhikkhu dimora contemplando il corpo come corpo».

% FIXME missing »

«Ancora, come se ci fosse una borsa con due aperture, piena di molti tipi di
granaglie, come riso delle alture, riso rosso, fagioli, piselli, miglio e riso
bianco, e un uomo dotato di buona vista l’avesse aperta e la stesse passando in
rassegna: “Questo è riso delle alture, questo è riso rosso, questi sono fagioli,
questi sono piselli, questo è miglio, questo è riso bianco”. Allo stesso modo un
bhikkhu passa in rassegna questo corpo, dalle punte dei piedi in su e dalla cima
dei capelli in giù, in quanto pieno di molte cose sudicie: “In questo corpo ci
sono capelli, peli, unghie, denti, pelle, carne, tendini, ossa, midollo osseo,
reni, cuore, fegato, diaframma, milza, polmoni, intestino, viscere, cibo non
digerito, feci, bile, flegma, pus, sangue, sudore, grasso, lacrime, materia
oleosa, saliva, muco, liquido sinoviale e urina”.

«Egli dimora contemplando …».

«È pure così che un bhikkhu dimora contemplando il corpo come corpo».

«Ancora, come se un macellaio esperto o un suo esperto apprendista avessero
macellato una mucca e stessero seduti a un crocevia con l’animale fatto a pezzi.
Allo stesso modo, un bhikkhu, in qualsiasi posizione sia il suo corpo, lo passa
in rassegna in base ai [quattro] elementi: “In questo corpo ci sono l’elemento
terra, l’elemento acqua, l’elemento fuoco e l’elemento aria”».

«Egli dimora contemplando …».

«È pure così che un bhikkhu dimora contemplando il corpo come corpo».

\label{pag270}%
«Ancora, un bhikkhu considera questo corpo come se stesse guardando un cadavere
gettato in un carnaio, morto da un giorno, morto da due giorni, morto da tre
giorni, gonfio, livido, e che trasuda materia: “Anche questo corpo ha tale
natura, sarà così, non è esente da questo”».

«Egli dimora contemplando …».

«È pure così che un bhikkhu dimora contemplando il corpo come corpo».

«Ancora, un bhikkhu considera questo corpo come se stesse guardando un cadavere
gettato in un carnaio, mentre viene divorato da corvi, nibbi, avvoltoi, cani,
sciacalli e da una molteplice varietà di vermi: … come se stesse guardando un
cadavere gettato in un carnaio, uno scheletro con carne e sangue, e tenuto
assieme da tendini … uno scheletro scarno e macchiato di sangue, e tenuto
assieme da tendini … uno scheletro senza carne e sangue, e tenuto assieme da
tendini … ossa prive di tendini, sparpagliate in tutte le direzioni, qui le ossa
di una mano, là le ossa di un piede, là le ossa di uno stinco, là un femore, là
il bacino, là la colonna vertebrale, là un teschio … ossa sbiancate, del colore
delle conchiglie … ossa ammucchiate, vecchie più di un anno … ossa decomposte e
sminuzzate fino a divenire polvere: “Anche questo corpo ha tale natura, sarà
così, non è esente da questo”».

«Egli dimora contemplando …».

«È pure così che un bhikkhu dimora contemplando il corpo come corpo».

«E come dimora un bhikkhu contemplando le sensazioni come sensazioni?».

«Un bhikkhu, quando prova una sensazione piacevole, sa “provo una sensazione
piacevole”. Quando prova una sensazione dolorosa sa “provo una sensazione
dolorosa”. Quando prova una sensazione né-dolorosa-né-piacevole, sa “provo una
sensazione né-dolorosa-né-piacevole”. Quando prova una sensazione piacevole
materiale, sa “provo una sensazione piacevole materiale”.\footnote{“Materiale”
  (\emph{āmisa}) si riferisce a cose fisiche come il cibo, il vestirsi, ecc., e
  con le sensazioni a ciò connesse.} … (E così via con le altre due.) Quando
prova una sensazione piacevole non materiale, sa “provo una sensazione piacevole
non materiale”. … (E così via con le altre due)».

«Egli dimora contemplando le sensazioni come sensazioni in questo modo in se
stesso, o esternamente, o in se stesso ed esternamente».

«Oppure egli contempla nelle sensazioni i fattori della loro origine, o i
fattori del loro decadimento, o i fattori della loro origine e del loro
decadimento».

«Oppure la consapevolezza che “ci sono sensazioni” si fonda in lui nella misura
di mera conoscenza e rammemorazione di essa mentre egli dimora indipendente,
senza attaccarsi a nulla nel mondo».

«Ecco come dimora un bhikkhu contemplando le sensazioni come sensazioni».

«E come dimora un bhikkhu contemplando la coscienza come coscienza?».

\label{pag272}%
«Un bhikkhu comprende la coscienza affetta dalla brama come affetta dalla brama,
e quella non affetta dalla brama come non affetta dalla brama. Egli comprende la
coscienza affetta dall’odio come affetta dall’odio, e quella non affetta
dall’odio come non affetta dall’odio. Egli comprende la coscienza affetta
dall’illusione come affetta dall’illusione, e quella non affetta dall’illusione
come non affetta dall’illusione. Egli comprende la coscienza contratta come
contratta, e quella distratta come distratta. Egli comprende la coscienza
esaltata come esaltata, e quella non esaltata come non esaltata. Egli comprende
la coscienza superata come superata, e quella non superata come non
superata.\footnote{“Contratta” dal torpore; “esaltata” dallo stato sensoriale a
  quello meditativo; “superata” nella meditazione o nella realizzazione.} Egli
comprende la coscienza concentrata come concentrata, e quella non concentrata
come non concentrata. Egli comprende la coscienza liberata come liberata, e
quella non liberata come non liberata».

«Egli dimora contemplando la coscienza come coscienza in questo modo in se
stesso, o esternamente, o in se stesso ed esternamente».

«Oppure egli contempla nella coscienza i fattori della sua origine, o i fattori
del suo decadimento, o i fattori della sua origine e del suo decadimento».

«Oppure la consapevolezza che “c’è la coscienza” si fonda in lui nella misura di
mera conoscenza e rammemorazione di essa mentre egli dimora indipendente, senza
attaccarsi a nulla nel mondo».

«Ecco come dimora un bhikkhu contemplando la coscienza come coscienza».

«E come dimora un bhikkhu contemplando gli oggetti mentali come oggetti
mentali?».

«Un bhikkhu dimora contemplando gli oggetti mentali come oggetti mentali nei
termini dei cinque impedimenti.\footnote{“Impedimento” dovrebbe essere inteso
  nel senso, per così dire, di barriera che fa restare nella corrente della
  brama, dell’odio e dell’illusione, piuttosto che come un ostacolo che blocca
  la strada.} E come lo si fa? Quando in lui c’è desiderio sensoriale, egli sa
“in me c’è desiderio sensoriale”, o quando in lui non c’è desiderio sensoriale,
egli sa “in me non c’è desiderio sensoriale”. Ed egli comprende anche come
giunge in essere il sorgere del non sorto desiderio sensoriale, e come giunge in
essere l’abbandono del sorto desiderio sensoriale, e come giunge in essere il
futuro non-sorgere dell’abbandonato desiderio sensoriale. Quando in lui c’è
malevolenza … Quando in lui c’è torpore e sonnolenza … Quando in lui c’è
agitazione e preoccupazione … Quando in lui c’è dubbio … egli comprende come
giunge in essere il futuro non-sorgere dell’abbandonato dubbio».

«Egli dimora contemplando gli oggetti mentali come oggetti mentali in questo
modo in se stesso, o esternamente, o in se stesso ed esternamente».

«Oppure egli contempla negli oggetti mentali i fattori della loro origine, o i
fattori del loro decadimento, o i fattori della loro origine e del loro
decadimento».

«Oppure la consapevolezza che “ci sono gli oggetti mentali” si fonda in lui
nella misura di mera conoscenza e rammemorazione di essa, mentre egli dimora
indipendente, senza attaccarsi a nulla nel mondo».

«Ecco come dimora un bhikkhu contemplando gli oggetti mentali negli oggetti
mentali nei termini dei cinque impedimenti».

«Ancora, un bhikkhu dimora contemplando gli oggetti mentali come oggetti mentali
nei termini dei cinque aggregati affetti dall’attaccamento. E come lo si fa? Un
bhikkhu comprende: “Questa è la forma, questa è la sua origine, questo è il suo
scomparire; questa è la sensazione, questa è la sua origine, questo è il suo
scomparire; questa è la percezione, questa è la sua origine, questo è il suo
scomparire; queste sono le formazioni [mentali], questa è la loro origine,
questo è il loro scomparire; questa è la coscienza, questa è la sua origine,
questo è il suo scomparire”».

«Egli dimora contemplando …».

«Ecco come dimora un bhikkhu contemplando gli oggetti mentali come oggetti
mentali nei termini dei cinque aggregati affetti dall’attaccamento».

«Ancora, un bhikkhu dimora contemplando gli oggetti mentali come oggetti mentali
nei termini delle sei basi in se stesso ed esternamente. E come lo si fa? Un
bhikkhu comprende l’occhio e le forme visibili e le catene che sorgono a causa
di entrambi. Comprende come giunge in essere il sorgere di catene non sorte, e
come giunge in essere l’abbandono delle catene sorte, e come giunge in essere il
futuro non sorgere delle catene abbandonate. Egli comprende l’orecchio e i suoni
… il naso e gli odori … la lingua e i sapori … il corpo e gli oggetti tangibili
… la mente e gli oggetti mentali e le catene che sorgono a causa di entrambi …
ed egli comprende come giunge in essere il futuro non sorgere delle catene
abbandonate».

«Egli dimora contemplando …».

«Ecco come dimora un bhikkhu contemplando gli oggetti mentali come oggetti
mentali nei termini delle sei basi in se stesso ed esternamente».

«Ancora, un bhikkhu dimora contemplando gli oggetti mentali come oggetti mentali
nei termini dei sette fattori dell’Illuminazione. E come lo si fa? Quando in lui
c’è la consapevolezza quale fattore dell’Illuminazione, un bhikkhu sa “in me c’è
la consapevolezza quale fattore dell’Illuminazione”, e quando non c’è la
consapevolezza quale fattore dell’Illuminazione, egli sa “in me non c’è la
consapevolezza quale fattore dell’Illuminazione”. Ed egli sa come giunge in
essere il sorgere della non sorta consapevolezza quale fattore
dell’Illuminazione e come giunge in essere lo sviluppo e il perfezionamento
della sorta consapevolezza quale fattore dell’illuminazione. Quando in lui c’è
l’investigazione degli stati [mentali] quale fattore dell’Illuminazione … in lui
c’è l’energia quale fattore dell’Illuminazione … in lui c’è la felicità quale
fattore dell’Illuminazione … in lui c’è la tranquillità quale fattore
dell’Illuminazione … in lui c’è la concentrazione quale fattore
dell’Illuminazione … in lui c’è l’equanimità quale fattore dell’Illuminazione …
Ed egli sa come giunge in essere il sorgere della non sorta equanimità quale
fattore dell’Illuminazione e come giunge in essere lo sviluppo e il
perfezionamento della sorta equanimità quale fattore dell’Illuminazione».

«Egli dimora contemplando …».

«Ecco come dimora un bhikkhu contemplando gli oggetti mentali come oggetti
mentali nei termini dei sette fattori dell’Illuminazione».

«Ancora, un bhikkhu dimora contemplando gli oggetti mentali come oggetti mentali
nei termini delle Quattro Nobili Verità. E come lo si fa? Un bhikkhu comprende
in accordo con ciò che nei fatti è: “Questa è la sofferenza” e “Questa è
l’origine della sofferenza” e “Questa è la cessazione della sofferenza” e
“Questo è il Sentiero che conduce alla cessazione della sofferenza”».

«Egli dimora contemplando gli oggetti mentali come oggetti mentali in questo
modo in se stesso, o esternamente, o in se stesso ed esternamente».

«Oppure egli contempla negli oggetti mentali i fattori della loro origine, o i
fattori del loro decadimento, o i fattori della loro origine e del loro
decadimento».

«Oppure la consapevolezza che “ci sono gli oggetti mentali” si fonda in lui
nella misura di mera conoscenza e rammemorazione di essa, mentre egli dimora
indipendente, senza attaccarsi a nulla nel mondo».

«Ecco come dimora un bhikkhu contemplando gli oggetti mentali come oggetti
mentali nei termini delle Quattro Nobili Verità».

«Bhikkhu, se qualcuno mantenesse in essere questi quattro fondamenti della
consapevolezza per sette anni … i sette anni a parte … per sette giorni, allora
egli si potrebbe attendere uno di questi due frutti: la conoscenza finale qui e
ora, oppure il non-ritorno».

\suttaRef{D. 22; M. 10}

«Bhikkhu, vi esporrò l’origine e lo scomparire dei quattro fondamenti della
consapevolezza: il corpo ha [bisogno del] nutrimento per la sua origine, e
scompare con la cessazione del nutrimento; le sensazioni hanno [bisogno del]
contatto per la loro origine, e scompaiono con la cessazione del contatto; la
coscienza ha [bisogno di] nome-e-forma per la sua origine, e scompare con la
cessazione di nome-e-forma; gli oggetti mentali hanno [bisogno dell’]attenzione
per la loro origine, e scompaiono con la cessazione dell’attenzione».

\suttaRef{S. 47:42}

«Tutte le cose hanno come loro radice il desiderio, l’attenzione provvede alla
loro esistenza, il contatto alla loro origine, la sensazione al loro luogo
d’incontro, la concentrazione al confronto con esse, la consapevolezza al loro
controllo, la comprensione è la più alta di esse e la liberazione il loro
nucleo».

\suttaRef{A. 8:83}

«Se s’intende custodire se stessi, sono i fondamenti della consapevolezza a
dover essere coltivati. Se s’intende custodire gli altri, sono i fondamenti
della consapevolezza a dover essere coltivati. Chi custodisce se stesso,
custodisce gli altri. Chi custodisce gli altri, custodisce se stesso».

\suttaRef{S. 47:19}

\section*{Retta concentrazione}

\narrator{Secondo narratore.} Siamo giunti all’ottavo e ultimo fattore, la retta
concentrazione.

\voice{Prima voce.} «Che cos’è la retta concentrazione? Un bhikkhu, del tutto
isolato dai desideri sensoriali, isolato dagli stati [mentali] non salutari,
entra e dimora nel primo jhāna, che è accompagnato dal pensiero e
dall’esplorazione uniti alla felicità e al piacere nati dall’isolamento».

\suttaRef{D. 2; D. 22; M. 39; S. 45:8}

«Proprio come un esperto addetto al lavacro o come un suo esperto apprendista
accumula polvere di sapone in una bacinella di metallo e, irrorandola
gradualmente d’acqua, la impasta fino a che la mistura non diventa una palla di
sapone, la impregna dentro e fuori senza che la palla diventi liquida, allo
stesso modo un bhikkhu imbeve il corpo della felicità e del piacere nati
dall’isolamento, lo immerge in essi, di essi lo ricolma fino a che non vi è
alcuna parte di tutto il corpo nella quale tale felicità e tale piacere non
giungano».

\suttaRef{D. 2; M. 39}

«Con l’acquietarsi del pensiero e dell’esplorazione egli entra e dimora nel
secondo jhāna, privo di pensiero ed esplorazione, che è accompagnato da fiducia
interiore e unificazione della mente unite alla felicità e al piacere nati dalla
concentrazione».

\suttaRef{D. 2; D. 22; M. 39; S. 45:8}

«Proprio come un lago, la cui acqua sgorga dal basso e che non ha afflussi da
est, da ovest, nord o sud, né viene riempito di tanto in tanto dalla pioggia del
cielo, e la fresca fonte d’acqua che zampilla dal lago imbeve, impregna, riempie
il lago, estendendosi del tutto in esso, e non c’è parte dell’intero lago nel
quale l’acqua fresca non si estenda, allo stesso modo un bhikkhu imbeve,
impregna, riempie questo corpo con la felicità e il piacere nati dalla
concentrazione, estendendola del tutto in esso, così che non c’è parte
dell’intero corpo nel quale tale felicità e tale piacere non giungano».

\suttaRef{D. 2; M. 39}

«Con lo svanire anche di questa felicità egli dimora nell’equanimità e,
consapevole e pienamente presente, provando ancora piacere nel corpo, entra e
dimora nel terzo jhāna, in relazione al quale gli Esseri Nobili affermano:
“Dimora piacevolmente chi osserva con equanimità e consapevolezza”».

\suttaRef{D. 2; D. 22; M. 39; S. 45:8}

«Proprio come in un laghetto alcuni fiori di loto blu, bianchi o rossi, nati
sotto la superficie dell’acqua, cresciuti sott’acqua, non fuoriescono dall’acqua
ma fioriscono immersi nell’acqua, e l’acqua li imbeve, impregna, riempie,
estendendosi dalla loro sommità fino alle loro radici, e non c’è parte di tutti
questi fiori di loto sui quale l’acqua non si estenda, allo stesso modo un
bhikkhu imbeve, impregna, riempie questo corpo con il piacere spoglio della
felicità, estendendolo del tutto in esso, così che non c’è parte dell’intero
corpo nel quale tale piacere spoglio della felicità non giunga».

\suttaRef{D. 2; M. 39}

«Con l’abbandono del piacere e del dolore, e con la precedente scomparsa della
gioia e dell’afflizione mentale, egli entra e dimora nel quarto jhāna, nel quale
non c’è né piacere né dolore, e la purezza della consapevolezza è dovuta
all’equanimità».

\suttaRef{D. 2; D. 22; M. 39; S. 45:8}

«Proprio come un uomo seduto vestito di bianco da capo a piedi, senza che ci sia
una sola parte di tutto il suo corpo sulla quale il bianco non giunga, allo
stesso modo un bhikkhu siede con una pura e luminosa cognizione che si estende
su tutto il suo corpo, senza che ci sia una sola parte di esso nella quale tale
pura e luminosa cognizione non giunga».

\suttaRef{D. 2; M. 39}

«Qual è quella retta concentrazione degli Esseri Nobili con le sue cause e il
suo corredo? È qualsiasi unificazione della mente che sia corredata dagli altri
sette fattori del Sentiero. Prima viene la retta visione: si comprendono
l’errata visione, l’errata intenzione, l’errata parola, l’errata azione, gli
errati mezzi di sostentamento come errati. Si comprendono la retta visione, la
retta intenzione, la retta parola, la retta azione, i retti mezzi di
sostentamento come retti, ossia ognuno dei due tipi come associati alle
contaminazioni e che maturano negli essenziali dell’esistenza, oppure come
sovra-mondani e come fattori del Sentiero. Si fanno sforzi per abbandonare
l’errata visione e gli altri quattro, e per acquisire la retta visione e gli
altri quattro: questo è il retto sforzo. Consapevolmente si abbandona ciò che è
errato e si entra sul Sentiero del giusto: questa è la retta presenza mentale».

\suttaRef{M. 117 (condensato)}

\narrator{Secondo narratore.} Questi ultimi tre fattori, retto sforzo, retta
presenza mentale e retta concentrazione, tutti insieme costituiscono la
“concentrazione”. Gli otto fattori, insieme alla retta conoscenza e alla retta
liberazione, sono chiamati le “dieci rettitudini”, che costituiscono la
“certezza della rettitudine” raggiunta con il Sentiero di Chi è Entrato nella
Corrente. Prima di abbandonare il tema della concentrazione, però, ci sono
quattro ulteriori stadi raggiungibili, chiamati i quattro “stati privi di
forma”. Essi sono un aggiuntivo alla “retta concentrazione”, solo degli
affinamenti del quarto jhāna.

\voice{Prima voce.} «Con il completo superamento della percezione della forma,
con la scomparsa delle percezioni della resistenza, mediante il non prestare
attenzione alle percezioni della differenza, (consapevole dello) “spazio
infinito”, un bhikkhu entra e dimora nella base consistente nell’infinità dello
spazio».

«Ancora, mediante il completo superamento della base consistente nell’infinità
dello spazio, (consapevole della) “coscienza infinita”, egli entra e dimora
nella base consistente nell’infinità della coscienza».

«Ancora, mediante il completo superamento della base consistente nell’infinità
della coscienza, (consapevole del) “non c’è nulla”, egli entra e dimora nella
base consistente nel nulla-è».

«Ancora, mediante il completo superamento della base consistente nel nulla è,
egli entra e dimora nella base consistente nella
né-percezione-né-non-percezione».

«I quattro jhāna nella Disciplina degli Esseri Nobili non sono chiamati
annullamento, nella Disciplina degli Esseri Nobili sono chiamati piacevole
dimorare qui e ora. I quattro stati privi di forma non sono chiamati
annullamento, nella Disciplina degli Esseri Nobili sono chiamati sereno
dimorare».

\suttaRef{M. 8}

«Il bhikkhu (che pratica tali otto conseguimenti) si dice che ha bendato Māra,
che ha (temporaneamente) privato Māra della vista dei suoi oggetti e che è
diventato invisibile al Malvagio».

\suttaRef{M. 25}

\narrator{Secondo narratore.} Nessuno di questi otto conseguimenti (né le
quattro Divine Dimore, \hyperlink{cap-10-Il-periodo-di-mezzo#pag200b}{}) sono
rivendicate come peculiari dell’insegnamento dei Buddha. La loro pratica priva
di retta visione conduce solo al paradiso, ma non al Nibbāna. L’insegnamento
peculiare dei Buddha consiste nelle Quattro Nobili Verità. Un nono
conseguimento, il “conseguimento della cessazione” è descritto come raggiunto
nei due più alti stadi di realizzazione ed è perciò peculiare dei Buddha e dei
loro discepoli.

\voice{Prima voce.} «Con il completo superamento della base consistente nella
né-percezione-né-non-percezione, un bhikkhu entra e dimora nella cessazione
della percezione e della sensazione, e le sue contaminazioni si esauriscono
allorché egli le vede con comprensione. Allora si dice che un bhikkhu ha bendato
Māra, che ha privato Māra della vista dei suoi oggetti e che è diventato
invisibile al Malvagio e, quel che più conta, che è andato al di là di ogni
attaccamento per il mondo».

\suttaRef{M. 25}

\begin{quote}
Quando un saggio, ben fondato nella virtù, \\
sviluppa consapevolezza e comprensione, \\
allora come un bhikkhu, ardente e sagace, \\
riesce a dipanare il groviglio.
\end{quote}

\suttaRef{S. 1:23}

«Bhikkhu, se un uomo dovesse viaggiare e arrancare attraverso un’era, allora il
cumulo, la pila, la massa delle sue ossa, se riunite né distrutte, sarebbe tanto
alto quanto il Monte Vepulla».

\suttaRef{Iti. 24}

«Supponiamo che un uomo gettasse nell’oceano un giogo con un foro in esso e,
poi, che il vento dell’est lo sospingesse a ovest, il vento dell’ovest lo
sospingesse a est, il vento del nord lo sospingesse a sud e il vento del sud lo
sospingesse a nord. E che ci fosse una tartaruga cieca che sale in superficie
solo una volta ogni cento anni. Che cosa ne pensate, bhikkhu, quella tartaruga
cieca potrebbe infilare la testa in quel giogo con un foro in esso?».

«Potrebbe, Signore, solo dopo un lungo periodo».

«Bhikkhu, la tartaruga cieca metterebbe la sua testa in quel giogo con un solo
foro in esso prima che un folle, destinato alla perdizione, possa trovare la via
per tornare alla condizione umana».

\suttaRef{M. 129}

«Bhikkhu, il Dhamma da me ben proclamato è schietto, aperto, evidente e spoglio
di parole inutili. In questo Dhamma così da me ben proclamato chiunque abbia
semplice fiducia in me, semplice amore per me, è destinato al paradiso».

\suttaRef{M. 22}

«Ciò che dovrebbe essere fatto per i discepoli per compassione da un Maestro che
cerca il loro benessere ed è compassionevole, questo ho fatto io per voi. Ci
sono questi spazi ai piedi degli alberi, queste stanze vuote: praticate la
meditazione, bhikkhu, evitate di rimandare per rimpiangerlo in seguito. Questo è
il mio insegnamento per voi».

\suttaRef{M. 8; M. 152}

\narrator{Secondo narratore.} Con questo si conclude tale esposizione. Ma come è
in realtà percorso il Sentiero?

\section*{Il Nobile Ottuplice Sentiero in pratica}

\label{pag281}%
\voice{Prima voce.} Un mattino il venerabile Ānanda si vestì, prese la ciotola e
la veste superiore e si recò a Sāvatthī per la questua. Egli vide il brāhmaṇa
Jānussoni che usciva da Sāvatthī alla guida di un carro trainato da quattro
giumente, tutto era bianco: bianchi i cavalli, bianche le imbracature, bianco il
carro, bianca la tappezzeria, bianchi i sandali, e lui stesso era rinfrescato da
un bianco ventaglio. Quando la gente vide tutto questo, disse: «Che divino
veicolo! Ecco com’è un veicolo divino!».

Quando tornò, il venerabile Ānanda lo raccontò al Beato e gli chiese: «Signore,
in questo Dhamma e Disciplina può essere individuato un divino veicolo?».

«Sì, Ānanda, disse il Beato: “Divino veicolo” è un nome per il Nobile Ottuplice
Sentiero, ed è “veicolo del Dhamma” e “impareggiabile vittoria nella battaglia”.
Per tutti i componenti del Nobile Ottuplice Sentiero esso culmina
nell’espulsione della brama, dell’odio e dell’illusione».

\suttaRef{S. 45:4}

«(Un bambino viene concepito e con la nascita e la crescita) le sue facoltà
sensoriali maturano, viene dotato e investito dei cinque componenti dei desideri
sensoriali e li utilizza. Le forme conoscibili mediante l’occhio sono ambite e
desiderate, sono gradevoli e piacevoli, connesse con il desiderio sensoriale e
inducono la brama. Lo stesso è per i suoni conoscibili mediante l’orecchio, gli
odori conoscibili mediante il naso, i sapori conoscibili mediante la lingua e
gli oggetti tangibili mediante il corpo».

«Vedendo una forma visibile con l’occhio, ascoltando un suono con l’orecchio,
sentendo un odore con il naso, gustando un sapore con la lingua, toccando un
oggetto tangibile con il corpo, conoscendo un’idea con la mente, egli li brama
se sono attraenti, oppure prova malevolenza nei loro riguardi se sono
spiacevoli. Egli dimora senza la consapevolezza del corpo fondata e con la mente
ristretta, mentre non comprende come in realtà sono la liberazione della mente e
la liberazione per mezzo della comprensione, ove questi stati non salutari
cessano senza residuo. Impegnato com’è nell’indulgere e nell’opporsi quando
prova ogni sensazione, piacevole o dolorosa oppure né-piacevole-né-dolorosa,
egli assapora quella sensazione, la conferma e l’accetta. L’assaporamento sorge
in lui quando lo fa. Ogni assaporamento di quelle sensazioni è attaccamento. Con
l’attaccamento quale condizione, l’esistenza; con l’esistenza quale condizione,
la nascita; con la nascita quale condizione, giungono all’esistenza
l’invecchiamento e la morte, e anche l’afflizione, il lamento, il dolore, il
dispiacere e la disperazione. Così ha origine tutto questo aggregato di
sofferenza».

«Un Perfetto appare nel mondo, realizzato e completamente illuminato, perfetto
per vera conoscenza e condotta, conoscitore dei mondi, incomparabile guida degli
uomini che devono essere addestrati, insegnante di dèi e uomini, illuminato,
beato. Egli dichiara a questo mondo con i suoi deva, con i suoi Māra e con le
sue divinità, a questa generazione con i suoi monaci e brāhmaṇa, con i suoi
principi e uomini che egli ha realizzato se stesso mediante conoscenza diretta.
Egli insegna un Dhamma salutare al principio, nel mezzo e alla fine, con il
significato e il senso letterale, e spiega la santa vita che è assolutamente
perfetta e pura».

«Un capofamiglia, o suo figlio, oppure uno nato nello stesso clan, ascolta
questo Dhamma. Ascoltandolo, egli ha fiducia nel Perfetto. Avendo questa
fiducia, egli pensa: “La vita in famiglia è affollata e polverosa, l’abbandono
di essa comporta spaziose aperture. Vivendo in famiglia non è facile condurre
una santa vita assolutamente perfetta e immacolata come una conchiglia ben
lucidata. E se mi rasassi i capelli e la barba, indossassi l’abito ocra, e
rinunciassi alla vita in famiglia per una senza dimora?”».

«E in un’ulteriore circostanza, abbandonando forse una piccola fortuna, forse
una grande fortuna, abbandonando forse una piccola, forse una grande cerchia di
parenti, si rasa i capelli e la barba, indossa l’abito ocra e rinuncia alla vita
in famiglia per una senza dimora».

«Avendo così abbracciato la vita religiosa e in possesso dell’addestramento e
del modo di vivere di un bhikkhu, egli abbandona l’uccidere esseri viventi,
astenendosi da ciò mettendo da parte bastoni e armi; in modo gentile e benevolo,
egli dimora nella compassione per tutti gli esseri. Egli abbandona il prendere
quel che non è dato, astenendosi da ciò prendendo solo quel che è dato;
aspettandosi solo quel che è dato, dimora puro in se stesso mediante il non
rubare. Egli abbandona il non celibato, vivendo la vita celibataria come uno che
vive appartato, astenendosi dalla volgare lascivia. Egli abbandona l’errata
parola, astenendosi da ciò dicendo il vero; aderendo al vero quando parla, egli
è affidabile, attendibile e non inganna il mondo. Egli abbandona la calunnia …
Egli abbandona l’insulto … Egli abbandona il pettegolezzo … parla con un
linguaggio giusto che merita di essere ricordato, che è motivato, preciso e
connesso al bene».\footnote{Si veda sopra, la “\hyperlink{pag265}{}” per il
  testo completo.}

«Egli si astiene dal danneggiare semi e piante. Egli mangia solo in una parte
del giorno, astenendosi dal cibo di notte e dai pasti tardivi. Egli si astiene
dal ballo, dal canto, dalla musica e dagli spettacoli teatrali; dall’indossare
ghirlande, dall’abbellirsi con profumi e dall’adornarsi con unguenti; da letti
alti e grandi; dall’accettare oro e argento, grano, carne cruda, donne e
ragazze, schiave e schiavi, pecore e capre, pollame e maiali, elefanti, bovini,
cavalli e cavalle, campi e terreni; dall’andare a fare commissioni; da acquisti
e vendite; da falsi pesi, falsi metalli e false misure; da truffe, raggiri,
frodi e inganni; da mutilazioni, esecuzioni, imprigionamenti, rapine, saccheggi
e violenze».

«Si accontenta dell’abito [monastico] per proteggere il corpo, del cibo ricevuto
in elemosina per sostentarsi, così che ovunque vada porta tutto con sé, proprio
come un uccello dotato di ali vola usando le proprie ali. Possedendo l’insieme
delle virtù degli Esseri Nobili, egli prova in se stesso un’irreprensibile
beatitudine».

«Egli diviene uno che, vedendo una forma con l’occhio, non afferra segni e
caratteristiche mediante i quali possa essere invaso da non salutari stati di
cupidigia e afflizione, come se egli avesse lasciato la facoltà visiva
incustodita. Pratica la via del contenimento, custodisce la facoltà visiva,
porta a effetto il contenimento della facoltà visiva. (Allo stesso modo, quando
ascolta un suono con l’orecchio, quando sente un odore con il naso, quando
assapora un sapore con la lingua, quando tocca un oggetto tangibile con il corpo
e quando concepisce un’idea con la mente.) Possedendo questa facoltà del
contenimento degli Esseri Nobili, egli prova in se stesso un’incontaminata
beatitudine».

«È del tutto consapevole quando si muove avanti e indietro … e mantiene il
silenzio».\footnote{Si veda sopra, la “\hyperlink{pag267}{}” per il testo.}

«Possedendo l’insieme delle virtù degli Esseri Nobili, e questa facoltà del
contenimento degli Esseri Nobili, e questa consapevolezza e piena presenza
mentale degli Esseri Nobili, egli si avvale di un posto isolato per riposare –
una foresta, uno spazio ai piedi di un albero, un precipizio, una caverna d’una
montagna, un carnaio, una boscaglia d’una giungla, uno spiazzo, un mucchio di
paglia. Di ritorno dal giro per l’elemosina dopo il pasto, egli si mette seduto,
a gambe incrociate, con il corpo eretto e con la consapevolezza fissa davanti a
lui».

«Abbandonando la bramosia per il mondo, egli dimora con una mente priva di
bramosia; egli purifica la sua mente dalla bramosia. Abbandonando la malevolenza
e l’odio, egli dimora privo di pensieri di malevolenza, compassionevole per il
benessere di tutti gli esseri viventi; egli purifica la sua mente dalla
malevolenza e dall’odio. Abbandonando il torpore e la sonnolenza, egli dimora
con una mente libera dal torpore e dalla sonnolenza, percettivo della luce,
consapevole e pienamente presente; egli purifica la sua mente dal torpore e
dalla sonnolenza. Abbandonando l’agitazione e la preoccupazione, egli dimora
privo di agitazione, con la mente pacificata in se stesso; egli purifica la sua
mente dall’agitazione e dalla preoccupazione. Abbandonando il dubbio, egli
dimora con una mente che ha superato il dubbio, senza più interrogarsi sugli
stati [mentali] non salutari; egli purifica la sua mente dal dubbio».

\suttaRef{M. 38}

«Supponiamo che un uomo prenda un prestito, che intraprenda delle attività e che
tali attività abbiano successo, così che egli sia in grado di restituire tutto
il denaro del vecchio prestito e gliene rimanga altro per la moglie e i figli, e
che, poi, considerando queste cose, egli si senta soddisfatto e felice. Oppure,
supponiamo che un uomo sia afflitto, sofferente e gravemente malato e che il suo
cibo non gli sia di sostegno, che il suo corpo sia privo di vigore, ma che in
seguito egli guarisca da questa afflizione e che il suo corpo riottenga vigore.
Oppure, supponiamo che un uomo sia imprigionato in una prigione, ma che in
seguito egli sia liberato dalla prigione sano e salvo, senza perdita alcuna dei
suoi beni. Oppure, supponiamo che un uomo sia in schiavitù, che non sia autonomo
ma che dipenda dagli altri e che non sia in grado di andare dove vuole, ma che
in seguito egli venga liberato da quelle catene e diventi autonomo, indipendente
dagli altri, un uomo libero in grado di andare dove vuole. Oppure, supponiamo
che un uomo il quale rechi con sé beni e possessi entri in una strada che
attraversa un deserto, ma che in seguito attraversi sano e salvo il deserto,
senza perdita alcuna dei suoi beni, e che poi, considerando queste cose, egli si
senta soddisfatto e felice. Allo stesso modo, quando i cinque impedimenti non
sono in lui abbandonati, un bhikkhu li vede rispettivamente come un debito, una
malattia, una prigione, delle catene e una strada che attraversa un deserto, e
quando sono in lui abbandonati, egli li vede come un non debito, la salute,
l’uscita dalla prigione, la libertà dalle catene e una terra sicura».

\suttaRef{M. 39}

«Avendo abbandonato i cinque impedimenti, le imperfezioni mentali che
indeboliscono la comprensione, allora, del tutto isolato dai desideri
sensoriali, isolato dagli stati [mentali] non salutari, entra e dimora nel primo
jhāna … nel secondo jhāna … nel terzo jhāna … nel quarto jhāna».

«Vedendo una forma visibile con l’occhio, ascoltando un suono con l’orecchio,
sentendo un odore con il naso, assaporando un sapore con la lingua, toccando un
oggetto tangibile con il corpo, conoscendo un’idea con la mente, egli non li
brama se sono attraenti, e non prova malevolenza nei loro riguardi se sono
spiacevoli. Egli dimora con la consapevolezza del corpo fondata e con un
incommensurabile stato mentale, mentre comprende come in realtà sono la
liberazione della mente e la liberazione per mezzo della comprensione, ove
questi stati non salutari cessano senza residuo. Avendo così abbandonato
l’indulgere e l’opporsi quando prova ogni sensazione, piacevole o dolorosa
oppure né-piacevole-né-dolorosa, egli non assapora quella sensazione, né la
conferma e nemmeno l’accetta. Quando non lo fa, l’assaporamento di quelle
sensazioni cessa. Con la cessazione del suo assaporamento, cessa l’attaccamento.
Con la cessazione dell’attaccamento, cessa l’esistenza; con la cessazione
dell’esistenza, cessa la nascita; con la cessazione della nascita, cessano
l’invecchiamento e la morte, e anche l’afflizione, il lamento, il dolore, il
dispiacere e la disperazione. Così c’è la cessazione di tutto questo aggregato
di sofferenza».

\suttaRef{M. 38}

\section*{I mezzi}

«Supponiamo che un uomo, volendo un serpente, ne veda uno grande e in seguito lo
afferri in modo sbagliato, per le spire o per la coda, così che il serpente si
giri e lo morda, e a causa di questo egli muoia o patisca sofferenze mortali. –
Perché? Perché ha afferrato il serpente in modo sbagliato. – Allo stesso modo,
alcuni uomini fuorviati imparano il Dhamma senza esaminare il significato degli
insegnamenti con comprensione, così che non acquisiscono alcun diletto nel
meditare su di essi. Imparandolo invece allo scopo di cavillare e confutare, non
riescono ad apprezzare il fine per cui il Dhamma viene imparato, ma sperimentano
che gli insegnamenti, essendo da loro stati erroneamente compresi, li conducono
per lungo tempo verso danno e sofferenza. Supponiamo invece che un uomo, volendo
un serpente, ne veda uno grande e in seguito lo catturi in modo giusto, con un
bastone a forcella per il collo, ma, benché il serpente possa avvolgere le sue
spire intorno alla sua mano o al suo braccio o ai suoi arti, a causa di questo
tuttavia egli né muoia né patisca sofferenze mortali. Allo stesso modo, alcuni
uomini di rango che imparano il Dhamma esaminando il significato degli
insegnamenti con comprensione, acquisiscono una preferenza nel meditare su di
essi. Non imparandolo allo scopo di cavillare e confutare, apprezzano il fine
per cui il Dhamma viene imparato, e sperimentano che quegli insegnamenti,
essendo da loro stati rettamente compresi, li conducono per lungo tempo verso
benessere e felicità».

«Bhikkhu, supponiamo che un viaggiatore veda una grande distesa d’acqua, la cui
riva più vicina è pericolosa e temibile e quella più lontana sicura e innocua,
ma che non ci sia né un traghetto né un ponte. Dopo aver preso in considerazione
tutto questo, che egli raccolga erba, rami, ramoscelli, foglie e che li leghi
assieme facendone una zattera, aiutato dalla quale, e sforzandosi con mani e
piedi, riesca sano e salvo nella traversata. Poi quando è riuscito, pensa:
“Questa zattera mi è stata molto utile, perché grazie ad essa sono riuscito sano
e salvo nella traversata. E se io me la caricassi sulla testa o sulle spalle e
andassi dove intendo andare?”. Farebbe in questo caso quel che deve essere fatto
con una zattera?».

«No, Signore». – «Che cosa dovrebbe fare con la zattera? Se, quando è riuscito
nella traversata, pensasse: “Questa zattera mi è stata molto utile, perché
grazie ad essa sono riuscito sano e salvo nella traversata. E se io la tirassi
in secco o la mandassi alla deriva e andassi dove intendo andare?”, allora
farebbe quel che deve essere fatto con la zattera. Vi ho così mostrato come il
Dhamma somigli a una zattera che serve alla traversata, non ad attaccarsi.
Bhikkhu, quando conoscete la Similitudine della Zattera, (allora perfino i
buoni) insegnamenti dovrebbero essere da voi abbandonati, e a maggior ragione
quelli cattivi».

\suttaRef{M. 22 (condensato)}

\section*{La Meta}

% FIXME missing » 

«La cessazione della brama, dell’odio e dell’illusione è il Non-Formato
(Incondizionato), la Meta, l’Incontaminato, la Verità, l’Altra Sponda, il
Sottile, il Molto Difficile da Vedere, l’Indeclinabile, il Perenne,
l’Indisintegrabile, l’Invisibile, l’Indiversificato, la Pace, Ciò che Non Muore,
Il Fine Supremo, la Beatitudine, la Sicurezza, l’Esaurimento della Brama, il
Magnifico, il Meraviglioso, la Non Afflizione, il Naturalmente Non Angustiato,
il Nibbāna, la Non Sofferenza (la Non Ostilità), lo Svanire del Desiderio, la
Purezza, la Libertà, l’Indipendenza dal dipendente, l’Isola, il Ricovero, il
Porto, il Rifugio, l’Oltre.

\suttaRef{S. 43:1-44}

